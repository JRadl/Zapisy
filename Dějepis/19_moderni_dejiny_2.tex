\title{Moderní dějiny 2}
\documentclass[10pt,a4paper]{article}
\usepackage[utf8]{inputenc}
\usepackage[czech]{babel}
\usepackage{amsmath}
\usepackage{amsfonts}
\usepackage{amssymb}
\usepackage{chemfig}
\usepackage{geometry}
\usepackage{wrapfig}
\usepackage{graphicx}
\usepackage{floatflt}
\usepackage{hyperref}
\usepackage{fancyhdr}
\usepackage{tabularx}
\usepackage{makecell}
\usepackage{csquotes}
\usepackage{footnote}

\MakeOuterQuote{"}

\renewcommand{\labelitemii}{$\circ$}
\renewcommand{\labelitemiii}{--}
\newcommand{\ra}{$\rightarrow$ }
\newcommand{\x}{$\times$ }
\newcommand{\lp}[2]{#1 -- #2}
\newcommand{\timeline}{\input{timeline}}


\geometry{lmargin = 0.8in, rmargin = 0.8in, tmargin = 0.8in, bmargin = 0.8in}
\date{\today}
\author{Jakub Rádl}

\makeatletter
\let\thetitle\@title
\let\theauthor\@author
\makeatother

\hypersetup{
colorlinks=true,
linkcolor=black,
urlcolor=cyan,
}



\begin{document}
\maketitle
\tableofcontents
\begin{figure}[b]
Toto dílo \textit{\thetitle} podléhá licenci Creative Commons \href{https://creativecommons.org/licenses/by-nc/4.0/}{CC BY-NC 4.0}.\\ (creativecommons.org/licenses/by-nc/4.0/)
\end{figure}
\newpage
\newpage

\section{Občanská válka v USA}

\paragraph{Rozsah USA od poč. 19. stol.}
\begin{itemize}
\item Louisiana (od Francie)
\item Florida (od Španělska)
\item Kalifornie, Texas, Nové Mexiko (zisk z války s Mexikem)
\item 1800 \ra 1830 -- rozšíření obyvatelstva  z 5 na 13 milionů (zlatá horečka, levná půda)
\end{itemize}

\paragraph{Hospodářství}
\begin{itemize}
\item jih -- plantáže
\item sever -- průmysl, železnice
\item zlatá horečka
\item od 1815 -- průmyslová revoluce
\item 1820 -- zákaz šíření otrokářství do nově ovládnutých států
	\begin{itemize}
	\item převládalo především na jihu
	\item na severu obchodníci, otrok nemá peníze \ra obchodník nemá zisk
	\end{itemize}
\item Abolicisté (Louis Garrison)
	\begin{itemize}
	\item 2000 spolků na pomoc otroků
	\item Underground Raleway (ilegální, ne podzemní)
	\item \lp{1859}{Johrn Brown}
		\begin{itemize}
		\item chtěl vyvolat hromadné povstání otroků, přepadl zbrojnici, pověšen
		\end{itemize}
	\end{itemize}
\end{itemize}

\paragraph{Politické strany}
\begin{itemize}
\item 1828 -- Demokratická strana
\item 1854 -- Republikánská strana
	\begin{itemize}
	\item hospodáská jednota
	\item zrušení otroctví
	\item bezplatné příděly půdy na Západě
	\end{itemize}
\item 1860 -- zvolen prezidentem Abraham Lincoln
	\begin{itemize}
	\item umírněné názory na otroctví, nesmí se zrušit hromadně, Jih by se naštval
	\item Jih se stejně naštval 
	\item[\ra] 20. 12. 1860 -- zástupci J. Karolíny odvoláni z Washingtonu
	\item[\ra] 4. 2. 1861 -- 11 států vystoupilo z unie
		\begin{itemize}
		\item Konfederované státy americké (hl. m. Richmond)
		\item příprava na vojenský střet
		\end{itemize}
	\end{itemize}
\end{itemize}

\paragraph{Občanská válka severu proti jihu}
\subsection{1. fáze (1861--1865)}
\begin{itemize}
\item \lp{12. 4. 1861}{útok na Fort Sumter} (úspěch jižanů -- R. E. Lee)
\item \lp{15. 4. 18}{počátek války}
\item
\item
\item
\item
\item 1862 -- zákon o bezplatých přídělech půdy (za malý registrašní poplatek, po 5 letech přejde půda do vlastnictví)
\item 1863 -- zákon o zrušení otroctví
\item 
\end{itemize}

\subsection{2. fáze - období rekonstrukce (1865--1877)}
\begin{itemize}
\item vyrovnání hospodářských rozdílů
\item jih obsazen uniiní armádou, aby se zajistilo dodržování nových zákonů
\item dodatek ústavy
	\begin{itemize}
	\item zrušeno otroctví bez náhrady
	\item rovnoprávnost černochů
	\item volební právo
	\end{itemize}
\item \lp{1866}{Založen Ku-klux-klan v Tenessee}
	\begin{itemize}
	\item terorizovali černošké obyvatelstvo v noci v kápích
	\item název není zkratka ale onomatopoické vyjádření zvuku pušky používané členy
	\end{itemize}
\end{itemize}

\paragraph{Důsledky války}
\begin{itemize}
\item proměna struktury venkova
	\begin{itemize}
	\item plantážníci ztratili otroky \ra půda rozparcelována, pronajímána svobodným (hlavně černochům)
	\end{itemize}
\item \lp{70. léta}{hospodářská krize} -- ceny obilí klesají, ceny průmyslových výrobků stagnují či rostou
\end{itemize}

\paragraph{Ekonomická situace}
\begin{itemize}
\item příliv inteligence z Evropy, levná pracovní síla
\item průmyslový rozvoj -- taylorismus -- pásová sériová výroba
	\begin{itemize}
	\item špatně placení pracovníci, 12-14 hodin pracovní doba
	\item časté pracovní úrazy
	\end{itemize}
\item vznik monopolů, koncernů (H. Ford)

\end{itemize}

\paragraph{Sociální otázky}
\begin{itemize}
\item amrická federace práce
\item 1886 - Chicago (1. 5., 4. 5.)
	\begin{itemize}
	\item obrovská demonstrace -- 350 000 lidí -> 3 lidé zastřelení, později 7 zabito, 70 zraněno, spousta zatčeno, 4 odsouzeno k smrti
	\end{itemize}
\item progresivisté
	\begin{itemize}
	\item proti monopolům, pro férovější jednání
	\item Theodor Roosevelt
	\item Woodrow Wilson
	\end{itemize}
\end{itemize}

\paragraph{Expanze}
\begin{itemize}
\item tichý oceán
\item 1898 anexe Havajských ostrovů
\item Filipíny (1898)
\item část souostroví Samoa
\item sféra vlivu -- španělské kolonie -- Kuba
\item Čína
\item kanál v Panamské šíji, umělý stát Panama, dodnes ovládáno
\end{itemize}

\paragraph{Latinská Amerika}
\begin{itemize}
\item 1810 -- kreolové začali válku za nezávislost
\item 1826 -- Španělé vyhnáni zéměř ze všech kolonií
\item 1823 -- pres. James Monroe -- Monroeov dokrína "Amerika Američanům"
	\begin{itemize}
	\item přednes nového prezidenta o co se bude snažit
	\end{itemize}
\end{itemize}

\section{Koloniální velmoci (19./20. století)}
\paragraph{Imperialismus}
\begin{itemize}
\item označení z dokby kolem 1900
\item snaha podrobit si území a jeho obyvatele
\item hospodářská podstata -- vývoz kapitálu (investice do zahraniční výroby), odbytiště zboží, za cíl zisk
\item nacionalismus, šíření evropské kultury, přezíravost vůči koloniální kultuře
\end{itemize}

\paragraph{Misijní činnost}
\begin{itemize}
\item odpor domorodců \ra mnoho misionářů zabito
\item pozitivní -- stavění nemocnic 
\item vzdělání 
\end{itemize}

\paragraph{Konec 19. století}
\begin{itemize}
\item boj o kolonie mezi velmocemi
\item pořadí vyspělosti velmocí od roku 1870: USA $>$ Německo $>$ Japonsko $>$ VB $>$ Francie $>$ Belgie
\end{itemize}

\subsection{Velká Británie}
\begin{itemize}
\item Londýn -- City
\item 1882 -- ovládnut Egypt (samostatná vláda, ale ekonomicky i politicky ovládaný VB)
\item 1898 -- Súdán x Francie
	\begin{itemize}
	\item Fašodský mír
	\end{itemize}
\item 1898--1902 -- búrská válka
\item strategické body -- Gibraltar, Malta, Suet, Kapsko, Aden, Cejlon, Singapur
\item 
\end{itemize}

\paragraph{Dominia}
\begin{itemize}
\item aby se nechtěly kolonie trhat, je jim dán statut dominia
\item Kanada, Austrálie, Nový Zéland, Jihoafrická unie (30 mil km2 , 360 mil obyvatel
\end{itemize}

\subsection{Francie}
\begin{itemize}
\item kolonie rozšířeny o Alžír, Tunis, Maroko, Kongo
\item Francouzská Indočína (Vietnam, Laos, Kambodža)
\item \ra 10mil km, 56mil obyvatel
\end{itemize}

\subsection{Německo}
\begin{itemize}
\item Togo, Kamerun, Německá jihozápadní Afrika, Německá východní Afrika
\item Tichomoří (souostroví císaře Viléma)
\item 
\end{itemize}


\subsection{Itálie}
\begin{itemize}
\item Eritrea
\item část Somálska
\item pokus ovládnout Etiopii (Habeš)
\item Libye
\end{itemize}

\subsection{Japonsko}
\begin{itemize}
\item samo předmětem zájmu velmocí
\item císař Meidži (osvěta)
	\begin{itemize}
	\item reformy, vzdělaný, cestoval po světě
	\item reformy, stroje inspirované německem \ra průmyslová velmoc
	\item porazil povstání samurajů, kteří nechtěli reformy
	\end{itemize}
\item 1889 -- vypracována ústava podle německé
\item 3. místo hospodáčské úrovně na světě (<N<USA) \ra výboje 
\item 1894--95 -- válka s Čínou o Tchaj-wan, Pescadorské ostrovy
\item 1904--1905 -- rusko japonská válka o Mandžusko, Jižní Sachalin
\item 1910 -- ovládnuta Korea do konce 2. sv. v.
\end{itemize}

\subsection{Spojené státy}
\begin{itemize}
\item první místo na světě 
\item 
\item
\end{itemize}

\subsection{Kolonie a polokolonie}
\paragraph{Čína}
\begin{itemize}
\item Velká Británie, Francie, Německo, Rusko, USA, Belgie
\item 1900 povstání boxerů -- potlačeno společným vojskem velmocí
\item 1912 -- Kuomintang -- sdružovala politické strany v boji proti velmocem
\item 1911 jih Číny \ra republika
\item 1912 se císař Pchu-ji vzdal vlády
\end{itemize}

\paragraph{Indie}
\begin{itemize}
\item od 1818 britská kolonie
\item 1857--1859 povstání Sipáhiů
\item 1858 = Indie součástí Britského impéria \ra pozice místokrále
\item 1876 -- Viktorie císařovnou indickou
\item 1885 -- Indický národní kongres (pod. Kuomintang)
	\begin{itemize}
	\item hlavní aktivista -- Máhátma Gandhí (Máhátma -- "jeho duše je veliká", titul)
	\end{itemize}
\end{itemize}

\subsection{Mezinárodní vztahy před první světovou válkou}
\paragraph{Konec systému rovnováhy sil}
\begin{itemize}
\item nastolen na Vídeňském kongresu
\item narušen Krymskou válkou (1853--56) -- VB+Fr x Rusko
\item 1870/71 -- změny v mezinárodních vztazích
	\begin{itemize}
	\item Francie poražena 
	\item Německo velmocí
	\end{itemize}
\item hledají spojence
	\begin{itemize}
	\item bismarck -- snaha o izolaci Francie \ra usmíření s Habsburky, zlepšení vztahů s Ruskem
	\end{itemize}
\end{itemize}

\paragraph{Balkán}
\begin{itemize}
\item 1875 -- protiturecké povstání v Bosně a Hercegovině
\item 1876 -- v Bulharsku \ra turecký teror (30 000 mrtvých)
\item 1876 Srbosko + Černá Hora -- válka s Tureckem, turecké represe
\item postoj velmocí
\item VB chce zachovat Turecko, Rusko ńa straně slovanských národů
\end{itemize}

\paragraph{Rusko-Turecká válka}
\begin{itemize}
\item 1877 -- ruské vojsko + dobrovolníci z Rumunska a Bulharska
\item přechod průsymskem Šipka (1878)
\item 1878 -- mír v San Stefanu \ra nezávislost
	\begin{itemize}
	\item velké Bulharsko
	\item Srbsko
	\item Rumunsko
	\item Černá Hora
	\item RUsko zisk: Besarábie, Kars
	\end{itemize}
\end{itemize}

\paragraph{Vyhrocení vztahů mezi velmocemi}
\begin{itemize}
\item \lp{1879}{R-U + N \ra dvojspolek proti Rusku}
\item \lp{1882}{Dvojspolek + Itálie \ra trojspolek}
\item smlouvy se srbskem a RUmunskem
\item \lp{1887}{Ferdinand Koburský bulharským králem}
\item Německo vytlačovalo ruský vliv na balkáně 
\end{itemize}

\paragraph{1887 Nová krize v mezinárodních vztazích}
\begin{itemize}
\item neobratná politika Bismarcka v Rusku
	\begin{itemize}
	\item 
	\end{itemize}
\item 1893 tajná dohoda Ruska + Francie
\item dvojdohoda
\item VB -- splendid isolation nevýhodná
\item 1904 srdečná dohoda (VB+Fr)
\item 1907 smlouva 1893 + VB \ra Trojdohoda
\item vznik dvou vojenskopolitických bloků
\item 1908 -- R-U anektovalo Bosnu a Hercegovinu
\item 1911 Francie obsadila Maroko
\item 1912 -- Itálie okupuje Libyi
\item 1912--1913 -- 1. Balkánská válka
	\begin{itemize}
	\item Bulharsko, Černá hora, Řecko, Srbsko x Turecko
	\item Turecko poraženo, osvobouena Makedonie, Thrákie, Albánie
	\end{itemize}
\item 1913 -- Druhá balkánská válka
	\begin{itemize}
	\item Bulharsko x Srbsko, Č. Hora, Řecko, Rumunsko
	\item Bulharsko ztrácí území, sblížení s německem, znepřátelení Ruska
	\end{itemize}
\end{itemize}







\end{document}