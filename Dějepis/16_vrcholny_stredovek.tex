\title{Vrcholný středověk}
\documentclass[10pt,a4paper]{article}
\usepackage[utf8]{inputenc}
\usepackage[czech]{babel}
\usepackage{amsmath}
\usepackage{amsfonts}
\usepackage{amssymb}
\usepackage{chemfig}
\usepackage{geometry}
\usepackage{wrapfig}
\usepackage{graphicx}
\usepackage{floatflt}
\usepackage{hyperref}
\usepackage{fancyhdr}
\usepackage{tabularx}
\usepackage{makecell}
\usepackage{csquotes}
\usepackage{footnote}

\MakeOuterQuote{"}

\renewcommand{\labelitemii}{$\circ$}
\renewcommand{\labelitemiii}{--}
\newcommand{\ra}{$\rightarrow$ }
\newcommand{\x}{$\times$ }
\newcommand{\lp}[2]{#1 -- #2}
\newcommand{\timeline}{\input{timeline}}


\geometry{lmargin = 0.8in, rmargin = 0.8in, tmargin = 0.8in, bmargin = 0.8in}
\date{\today}
\author{Jakub Rádl}

\makeatletter
\let\thetitle\@title
\let\theauthor\@author
\makeatother

\hypersetup{
colorlinks=true,
linkcolor=black,
urlcolor=cyan,
}



\begin{document}
\maketitle
\tableofcontents
\begin{figure}[b]
Toto dílo \textit{\thetitle} podléhá licenci Creative Commons \href{https://creativecommons.org/licenses/by-nc/4.0/}{CC BY-NC 4.0}.\\ (creativecommons.org/licenses/by-nc/4.0/)
\end{figure}
\newpage



\section{Krize feudálního systému}
\subsection{Ekonomické poměry v Evropě}
\begin{itemize}
\item systém se nedokáže více rozvíjet, ale ekonomický vývoj jde dál
\item stálé zvyšování daní \ra poddaní nemají peníze \ra stagnování obchodu mezi městy a vesnicemi
\item[\ra] \textbf{sociální nepokoje}
	\begin{itemize}
	\item \lp{1348}{povstání Etiena Marcella}
	\item \lp{1378}{povstání v manufakturách}
	\item \lp{1381}{povstání Watta Taylora}
	\end{itemize}
\item kritika církve měšťany
	\begin{itemize}
	\item odpustky, obročí (nekněží si mohli zakoupit titul církevního hodnostáře), schizma
	\item církev získávala peníze za nic
	\end{itemize}
\end{itemize}

\subsection{Český stát}
\begin{itemize}
\item \textbf{kritika církve}
\item sekty rozšířené z Francie (Pikarti)
\item Karlem IV. pozván \textbf{Konrád Waldhauser}
\item první čeští reformátoři
	\begin{itemize}
	\item Jan Milíč z Kroměříže (kazatel)
	\item Matěj z (českého) Janova (profesor univerzity)
	\item Jeroným Pražský (překládal spisy Johna Wycllifa)
	\end{itemize}
\end{itemize}

\subsection{Jan Hus}
\begin{itemize}
\item *1371, vystudoval teologii, stal se kantorem a \textbf{rektorem UK}
\item koncem 14. století začal kázat (česky) v \textbf{Betlémské kapli} (navštěvováno i Václavem IV.)
\item \lp{1409}{\textbf{Dekret Kutnohorský}} -- Češi mají trojitou váhu hlasu na univerzitě
\item \lp{1410}{konflikt s arcibiskupem ohledně Wycllifových spisů \ra uvalena \textbf{klatba}}
\item \lp{1412}{\textbf{odpustkové bouře v Praze}}
	\begin{itemize}
	\item kritizoval prodej odpustků \ra uvalen \textbf{interdikt}
	\item během bouří byli popraveni tři tovaryšové
	\end{itemize}
\item \lp{1414}{další vlna odpustků}
\item interdiktem byl \textbf{vyhnán z Prahy} \ra kázal a psal díla na hradech svých příznivců
\item dílo \textbf{Knihy o svatokupectví}, \textbf{Postila}, \textbf{Dcerka}
\end{itemize}

\paragraph{\lp{1415}{Kostnický koncil}}
\begin{itemize}
\item svolán na popud Zikmunda s cílem vyřešit církevní problémy (schizma, \ldots)
\item Hus pozván, aby hájil své teologické myšlenky
\item byl odsouzen církví a předán kostnickému soudu, který ho \textbf{odsoudil k upálení} (6.7.1415)
\end{itemize}

\paragraph{Jakoubek ze Stříbra}
\begin{itemize}
\item Husův přítel, jako jeden z prvních kázal, že by lidem mělo být povoleno příjímání podobojí
\end{itemize}



\section{1. období husitství (1419--1422)} 
\subsection{Reakce na smrt Jana Husa}
\paragraph{Rozdělení společnosti}
\begin{itemize}
\item \textbf{katolíci}
\item \textbf{strana pražská} (umírnění) -- chtěli chudou církev, aby nemuseli platit vysoké daně
\item \textbf{radikální} 
	\begin{itemize}
	\item požadují rovnost nejen před bohem, ale i v životě
	\item soustředěni v Plzni, Žatci a Praze
	\item vůdcem kněz \textbf{Jan Želivský}
	\end{itemize}
\item \textbf{chiliasmus} -- vykonávali poutě na hory, věřili v tisícileté království
\end{itemize}

\paragraph{Protestní listina} -- 452 šlechtických pečetí 

\paragraph{\lp{30. 7. 1419}{1. pražská defenestrace}}
\begin{itemize}
\item průvod s Janem Želivským v čele procházel Prahou k radnici
\item cílem bylo osvobození tovaryšů
\item konšelé byly \textbf{vyházeni z okna na kopí} a všichni zemřeli
\item \textbf{Václav IV. dostal mrtvici}, když se o tom dozvěděl
\end{itemize}

\paragraph{\lp{1420}{založen Tábor}}
\begin{itemize}
\item založen husity, kteří chtěli žít podle ideálních společenských pravidel
\item \textbf{strategická poloha} na vysokém skalnatém kopci, chráněn řekou Lužicí
\end{itemize}

\paragraph{\lp{1420}{bitva u Sudoměře}}
\begin{itemize}
\item \textbf{Jan Žižka zvítězil} nad šlechtou
\item z jedné strany chráněni rozbahněným rybníkem, z druhé napuštěným, z třetí vozovou hradbou
\end{itemize}

\subsection{Čtyři artikuly pražské}
\begin{itemize}
\item \lp{červen 1420}{sepsány Janem Žižkou}
\item cílem je sjednocení programu umírněných a radikálních husitů, aby mohli bojovat se Zikmundem
\item nejdokonalejší program -- \textbf{sjednotil všechny společenské vrstvy}
\end{itemize}
\begin{enumerate}
\item \textbf{svobodné kázání slova božího} (každý má právo vykládat Bibli svým způsobem)
\item \textbf{povolení příjímání podobojí}
\item \textbf{konec světského panování kněžstva} (panovník má moc nad církví)
\item \textbf{trestání smrtelných hříchů nezávisle na stavu}
\end{enumerate}

\subsection{Boj se Zikmundem}
\paragraph{\lp{1420}{1. křížová výprava (Zikmund útočí na Vítkov)}}
\begin{itemize}
\item květen -- Jan Žižka správně předpokládal místo útoku, Praha byla jednotná \ra\textbf{ útok odražen}
\begin{itemize}
\item Zikmund nelegálně korunován českým Králem
\end{itemize}
\item \lp{listopad 1420}{Zikmund prohrává pod Vyšehradem}
\end{itemize}

\paragraph{\lp{1422}{Čáslavský zemský sněm}}
\begin{itemize}
\item \textbf{Zikmund sesazen z trůnu}, protože nebyl legálně korunován
\item ustanovena \textbf{rada 20 správců}, kteří vládnou v období bez krále (5 pánů, 7 rytířů, 8 měšťanů)
\item královská města mají "\textbf{právo třetího hlasu}" (hlasovat na zemském sněmu)
\item Čtyři artikuly přijaty jako zemský zákon
\item[\ra] český stát je stavovskou monarchií
\end{itemize}


\section{2. období husitství (1422--1424)}
\subsection{Křížové výpravy}
\paragraph{\lp{1422}{2. křížová výprava}}
\begin{itemize}
\item \lp{6. 1. 1422}{Zikmund poražen u Kutné hory} (směr útoku kvůli stříbru)
\item \lp{8. 1. 1422}{Zikmund poražen u Habrů}
\item \lp{10. 1. 1422}{zbytek Zikmundova vojska poražen u Německého brodu} (dnes Havlíčkův)
\end{itemize}

\paragraph{\lp{9. 3. 1422}{převrat v Praze}}
\begin{itemize}
\item Jan Želivský a jeho spolupracovníci nalákáni na staroměstskou radnici a popraveni
\end{itemize}

\paragraph{Diferenciace názorů v táboře}
\begin{itemize}
\item \textbf{Orebské bratrstvo} -- odpojená část husitů, založili Menší Tábor na hoře Oreb
\item sepsán \textbf{Žižkův vojenský řád}
	\begin{itemize}
	\item popisuje vojenské taktiky (vozová hradba, \ldots)
	\item stanovuje pevná pravidla o drancování vesnic po boji (zákaz u spřátelených)
	\item popisuje používané zbraně (halapartna, řemdich, palcát, \ldots)
	\end{itemize}
\end{itemize}



\paragraph{\lp{1424}{Vznik panské jednoty}}
\begin{itemize}
\item spojení vyšší a nižší šlechty proti radikálním husitům (aby donutili rolníky pracovat)
\end{itemize}

\paragraph{\lp{1424}{bitva u Malešova}}
\begin{itemize}
\item Žižka slepý bitvu zorganizoval a vyhrál
\item[\ra] mír s panskou jednotou \ra tažení na Moravu
\item Jan Žižka při tažení umírá \ra nástupci: Zikmund Korybut \ra Prokop Holý Veliký (bitva u Ústí, Tachova, Domažlic)
\end{itemize}

\paragraph{\lp{1426}{3. křížová výprava -- bitva u Ústí}} (křižáci poraženi)
\paragraph{\lp{1427}{4. křížová výprava -- bitva u Tachova}} (křižáci poraženi)
\paragraph{\lp{1431}{5. křížová výprava -- bitva u Domažlic}} (křižáci poraženi)
\begin{itemize}
\item vedena kardinálem Juliem Cesarinim -- ztratil klobouk \ra husité ho získal
\item křižáci: 40000 jízda + 90000 pěšáků $\times$ husité: 40000
\item křižáci začali prchat odstrašeni husitským chorálem
\item[\ra] Zikmund si uvědomil, že není možné Husity vojensky porazit
\end{itemize}

\subsection{Změna charakteru Husitských válek}
\paragraph{\lp{1428--1433}{Rejsy -- "spanilé jízdy"}}
\begin{itemize}
\item směřovali do Uherska, Rakouska, Německa, Polska (až k Baltu)
\item cílem bylo oslabení nepřátel, získání kořisti a šíření myšlenek husitství
\item ikonoklasmus
\end{itemize}

\paragraph{Martin Luther} -- považoval Husa za svého předchůdce

\paragraph{\lp{1431}{Basilejský koncil}}
\begin{itemize}
\item díky Zikmundovi pozváni \textbf{Prokop Holý} a \textbf{Jan Rokycana} za cílem vyřešení otázek husitských požadavků
\item \textbf{Soudce chebský} -- listina zajišťující bezpečnost jejich cesty a pobitu v Basileji
\begin{itemize}
\item koncil je považuje za sobě rovné
\item udělena díky Zikmundovi (chtěl urychlit konec válek)
\end{itemize}
\item koncil nabídl povolení přijímání podobojí a rozdělení majetku církve mezi šlechtu (jak katolickou, tak kališnickou) a královská města 
\item Holý a Rokycana na podmínky nepřistoupily, šlechta s nimi, ale souhlasila \ra obnovení panské jednoty za cílem ukončení husitských válek
\end{itemize}

\paragraph{\lp{30. 5. 1434}{bitva u Lipan}}
\begin{itemize}
\item umírnění (+ panská jednota) \x radikální
\item umírnění znali husitskou bojovou strategii \ra provedli falešný útok, stáhli se, radikálové otevřeli vozovou hradbu, umírnění provedli druhý útok \ra zvítězili
\item padlo 300 umírněných a 1300 radikálů
\item Luděk Marlod: obraz Bitva u Lipan
\end{itemize}

\subsection{Po bitvě u Lipan}
\paragraph{\lp{červenec 1436}{Basilejská kompaktáta}} (Jihlava)
\begin{itemize}
\item povoleno přijímání podobojí
\item majetek církve rozdělen mezi šlechtu a královská města
\item Zikmund oficiálně přijat za českého krále
\end{itemize}

\paragraph{Jan Roháč z Dubé}
\begin{itemize}
\item Zikmundův soupeř, obléhal hrad Kalich
\item \lp{1437}{Roháč zajat a v Praze popraven}
\end{itemize}

\paragraph{\lp{1437}{Zikmund se vrací do Uherska}}
\begin{itemize}
\item po cestě zemřel
\item české království odkázáno Albrechtu Habsburskému, manželovi jeho dcery
\end{itemize}

\subsection{Výsledky a význam husitství}
\begin{itemize}
\item vznik nové (poprvé oficiálně uznané papežem) křesťanské církve
\item majetek církve rozdělen mezi šlechtu a královská města
\item české království se stává stavovskou monarchií
\end{itemize}



\section{Český stát v letech 1436-1440}
\begin{itemize}
\item \textbf{Zikmund Lucemburský} (1419--1437)
\item \textbf{Albrecht Habsburský} (1437--1439)
\item \textbf{Ladislav "Pohrobek" Habsburský} (1453 -- 1457)
\end{itemize}

\subsection{Petr Chelčický}
\begin{itemize}
\item kritika společnosti, církve, státní formy
\item zaměřil se na učení o trojím lidu (byl proti rozdělení společnosti)
\item domníval se, že Bůh každého člověka odmění podle toho, jak se chová na zemi
\item dílo: \textbf{Postila}, \textbf{O trojím lidu}, \textbf{Sieť viery pravé}
\item zakladatel \textbf{jednoty bratrské}
	\begin{itemize}
	\item \lp{1457}{založena jednota bratrská}
	\item \lp{1467}{jednota bratrská uznána jako samostatná církev}
	\item stoupenci nejdříve z řad měštanů, později šlechty a české inteligence (J. A. Komenský, Jan Blahoslav)
	\item z počátku pronásledována
	\end{itemize}
\end{itemize}



\section{Doba Poděbradská}
\paragraph{Bezvládí (1439--1453)} 
\begin{itemize}
\item Ladislav příliš mladý na to aby nastoupil na trůn
\item obnova landfrýdů (vojensko-politický svaz měšťanů královských měst, vyšší a nižší šlechty)
\end{itemize}

\paragraph{\lp{1444}{Jiří z Poděbrad zvolen do čela východočeského landfrýdu}}
\begin{itemize}
\item v Praze jsou zmatky kvůli bezvládí \ra chtěl ovládnout Prahu, aby Čechy lépe prosperovali
\item \lp{1448}{ovládl Prahu nočním obchvatem}
\end{itemize}
\paragraph{\lp{1449}{vznik Jednoty strakonické}} -- katolický spolek českých pánů proti Jiřímu

\paragraph{\lp{1452}{Jiří zvolen správcem Království českého}}
\begin{itemize}
\item \textbf{revize pozemkové držby} -- 1454 (dokumentace přerozdělení majetku církve)
\item obnovení zemského soudu
\end{itemize}

\paragraph{\lp{1453}{Ladislav Pohrobek se ujal vlády}} (dosáhl 12 let)

\paragraph{\lp{1457}{Ladislav pohrobek umírá}}
\begin{itemize}
\item chytaný sňatek Ladislava s Magdalénou Francouzskou
\item Jiří z Poděbrad obviňován z otrávení, později nalezeny lékařské spisy a zjištěno, že šlo o leukémii
\end{itemize}

\paragraph{\lp{1458}{Jiří zvolen králem}}
\begin{itemize}
\item "král dvojího lidu" -- vládl katolíkům i kališníkům
\item pozval do Čech \textbf{Antonia Mariniho} jako ekonomického konzultanta \ra zákaz vývozu Au, Ag
\item spory s Rožmberky a s papežem
	\begin{itemize}
	\item požadoval od papeže znovupotvrzení basilejských kompaktát
	\item \lp{1462}{papež prohlásil kompaktáta za zrušená}
	\end{itemize}
\end{itemize}

\paragraph{\lp{1464}{Spolek křesťanských panovníků}}
\begin{itemize}
\item listina s cílem sjednocení Evropy v boji proti Turkům a intrikám papeže
\item Lev z Rožmitálu, Václav Šašek z Bříkova s listinou cestovali 
	\begin{itemize}
	\item ŘŘ \ra Londýn \ra Francie \ra Španělsko (\ra Alois Jirásek: Z Čech až na konec světa)
	\end{itemize}
\item navázáno mnoho spojenectví, některé státy byly ale odrazeny od podepsání papežem
\end{itemize}

\paragraph{\lp{1466}{křížová výprava proti Jiřímu}}
\begin{itemize}
\item papež Pavel II. na Jiřího uvalil klatbu a vyhlásil výpravu
\item většina panovníků se odmítla výpravy zúčastnit
\item do čela výpravy se postavil \textbf{Matyáš Hunyady Korvín}
\item \lp{1469}{bitva u Vilémova} 
	\begin{itemize}
	\item Matyáš zajat, propuštěn pod podmínkou vzdání se nároků na České království
	\item v Olomouci zvolen katolickými stavy za krále 
	\end{itemize}
\end{itemize}

\paragraph{\lp{1471}{Jiří z Poděbrad umírá}}
\begin{itemize}
\item během příprav na válku s Matyášem
\item nechtěl aby jeho synové byli jeho nástupci, protože není z královského rodu, což by mohlo ohrozit jejich postavení a vést k válce \ra 1469 schválen sněmem jako nástupce \textbf{Vladislav Jagelonský}
\end{itemize}



\section{Uhersko ve 2. pol. 15. stol}
\subsection{Uhersko po smrti Zikmunda}
\begin{itemize}
\item sílí pozice magnátů
\item rozdělení šlechty po roce 1440
	\begin{itemize}
	\item pro Ladislava Pohrobka: \textbf{Jan Jiskra z Brandýsa} -- sestavil křižáckou armádu \textbf{bratříci}
	\item pro Jano Hunyadyho: střední šlechta
	\end{itemize}
\end{itemize}

\paragraph{\lp{1440}{Jan Hunyady zvolen gubernátorem}}
\begin{itemize}
\item 1446 -- zemřel \ra funkci převzal \textbf{Ladislav Hunyady}
\end{itemize}

\paragraph{\lp{1451}{bitva u Lučence}}
\begin{itemize}
\item bratříci zvítězili nad Hunyadym 
\item \lp{1452}{Ladislav korunován uherským králem}
\item bratříci se stávají živelným hnutím (na Slovesnku), Petr Aksamit z Košova
\end{itemize}

\subsection{Uhersko za Matyáše}
\begin{itemize}
\item \lp{1458}{Matyáš Hunyady Korvín zvolen králem}
\item \lp{1458}{bitva u Šarišského Potoka}
\item \lp{1467}{bitva u Velkých Kostolan}
	\begin{itemize}
	\item Matyáš porazil bratříky \ra vytvořil z nich vlastní armádu
	\end{itemize}
\item pocházel z Čech \ra čeština stanovena za úřední jazyk
	\begin{itemize}
	\item považována za vyspělostí blízkou latině, k níž se Uhry po jeho smrti vrátily
	\end{itemize}
\item \lp{1465}{založena \textbf{Academia Istropolitana}}
\item podporoval renezanční umění
\end{itemize}



\section{Doba Jagellonská (1471--1526)}
\subsection{Vladislav II. Jagellonský (1471--1516)}
\begin{itemize}
\item král český, ztráta vedlejších zemí Koruny české
\item \lp{1490}{král uherský} \ra nucen k přesídlení do Budína
	\begin{itemize}
	\item z hospodářských důvodů -- soustředění královského dvora
	\item z politických důvodů -- aby ho mohla uherská šlechta lépe ovládat ("král Bene")
	\end{itemize}
\item[\ra] úpadek královské moci
\end{itemize}

\paragraph{Hospodářská situace}
\begin{itemize}
\item šlechta cestuje, nakupuje, poznává nové životní styly \ra potřebuje více peněz
\item[\ra] vznik \textbf{feudálních velkostatků}
	\begin{itemize}
	\item zabírání půdy poddaným
	\item pěstování obilí, chov ovcí, výroba, rybníkářství (Jakub Krčín z Jelčan, Štěpánek Netolický -- Zlatá stoka)
	\item[\ra] panské výrobky jdou na trh
	\item spory šlechty s královskými městy
	\end{itemize}
\item zvyšování nároků na poddané \ra nevolnictví
	\begin{itemize}
	\item poddaný je vázán na vrchnost pouze ekonomicky
	\item nevolník se nesmí odstěhovat, poslat děti studovat, vdávat bez souhlasu vrchnosti
	\item pro odchod potřebuje "výhostní list"
	\end{itemize}
\end{itemize}

\paragraph{Povstání}
\begin{itemize}
\item Praha po smrti Jana Želivského znovu rozdělena na 3 města
\item \lp{1483}{2. pražská defenestrace}
\item \lp{1494}{povstání v Zábřehu na Moravě}
\item \lp{1496}{povstání v Ploskovicích} -- Dalibor z Kozojed (inspirace pro B. Smetanu)
\item \lp{1496}{povstání v Kutné hoře} -- potlačeno a havíři naházeni do šachet (inspirace pro J. K. Tyla: Kutnohorští Havíři)
\end{itemize}

\paragraph{\lp{1485}{Kutnohorský náboženský mír}} (rovnoprávnost kališnictví a katolictví)

\paragraph{\lp{1500}{\textbf{Vladislavské zřízení zemské}}}
\begin{itemize}
\item sepsal Albrecht Rédl z Oušavy ve prospěch šlechty (na úkor měst a krále)
\item dává šlechtě právo podnikat, nedává šlechtě právo 3. hlasu
\item král zřízení podepsal (1502), města ho nikdy neuznala
\end{itemize}

\paragraph{\lp{1517}{povstání na Křivoklátsku}}
\begin{itemize}
\item[\ra] \lp{24.10.1517}{\textbf{Svatováclavská smlouva}}
\item šlechta může podnikat, města mají právo třetího hlasu
\end{itemize}

\subsection{Ludvík "dítě" Jagellonský (1516--1526)}
\begin{itemize}
\item \lp{1519}{příchod lutheránů z Německa} (nové reformované náboženství \ra nebezpečí pro katolíky a kališníky)
\item \lp{1525}{obnovení kompaktát} (legální je katolictví a kališnictví)
\item smlouvy o nástupnictví s Habsburky (1515)
	\begin{itemize}
	\item sňatek Ludvíka Jagelloonského s Marií Habsburskou
	\item sňatek Ferdinanda Habsburkého s Annou Jagellonskou
	\item v případě smrti jednoho z mužů přebírá vládu druhý
	\end{itemize}
\end{itemize}

\paragraph{\lp{1514}{křížová výprava proti Turkům}}
\begin{itemize}
\item před nástupem Ludvíka útoky Turků \ra křížová výprava
\item verbováni sedláci, řemeslníci a nižší šlechta (křižáci v Uhrách = "kuruci")
\item výpravu vedl Jiří Dóža (Dózsa György)
\item nedostatek jídla a financí \ra výprava se změnila v \textbf{Dóžovo povstání}
\item povstání bylo potlačeno
\item Uherský zemský sněm -- uzákoněno nevolnictví -- \textit{"perpetuum silentium"} (věčné mlčení)
\end{itemize}

\paragraph{\lp{1526}{křížová výprava}}
\begin{itemize}
\item většina Uherska byla pod tureckou nadvládou
\item \lp{29. 8. 1526}{bitva u Moháče}

\end{itemize}


\subsection{Vladislavská gotika}
\subsubsection{Architektura}
\begin{itemize}
\item značka "W" na stavbách
\item jinde už renesance, ale Češi ji nechtěli přijmout
\item žebra, ozdoba stěn ve formě větví
\item královská oratoř v chrámu sv. Víta -- král odtud sledoval mši
\end{itemize}

\paragraph{Matyáš Rejsek}
\begin{itemize}
\item chrám svaté Barbory (Kutná hora) -- diamantová klenba se znaky české šlechty 
\item Prašná brána -- (novodobě pseudogoticky dostavěna)
\end{itemize}

\paragraph{Benedikt Reid} -- Vladislavský sál (zasedání stavů)

\subsubsection{Sochařství}
\paragraph{Anton Pilgram}
\begin{itemize}
\item portál Staré radnice v Brně (křivá věžička symbolizující spravedlnost)
\item kazatelna v chrámu sv. Štěpána ve Vídni
\end{itemize}

\paragraph{Matěj Rejsek} -- Kutná hora


\subsubsection{Malířství}
\begin{itemize}
\item chrámy zdobené freskami (chrám sv. Barbory)
\item \textbf{Mistr Litoměřického oltáře}
\end{itemize}

\begin{itemize}
\item[]
\end{itemize}

\timeline

\end{document}