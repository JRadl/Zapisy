\title{Střední novověk}
\documentclass[10pt,a4paper]{article}
\usepackage[utf8]{inputenc}
\usepackage[czech]{babel}
\usepackage{amsmath}
\usepackage{amsfonts}
\usepackage{amssymb}
\usepackage{chemfig}
\usepackage{geometry}
\usepackage{wrapfig}
\usepackage{graphicx}
\usepackage{floatflt}
\usepackage{hyperref}
\usepackage{fancyhdr}
\usepackage{tabularx}
\usepackage{makecell}
\usepackage{csquotes}
\usepackage{footnote}

\MakeOuterQuote{"}

\renewcommand{\labelitemii}{$\circ$}
\renewcommand{\labelitemiii}{--}
\newcommand{\ra}{$\rightarrow$ }
\newcommand{\x}{$\times$ }
\newcommand{\lp}[2]{#1 -- #2}
\newcommand{\timeline}{\input{timeline}}


\geometry{lmargin = 0.8in, rmargin = 0.8in, tmargin = 0.8in, bmargin = 0.8in}
\date{\today}
\author{Jakub Rádl}

\makeatletter
\let\thetitle\@title
\let\theauthor\@author
\makeatother

\hypersetup{
colorlinks=true,
linkcolor=black,
urlcolor=cyan,
}



\begin{document}
\maketitle
\tableofcontents
\begin{figure}[b]
Toto dílo \textit{\thetitle} podléhá licenci Creative Commons \href{https://creativecommons.org/licenses/by-nc/4.0/}{CC BY-NC 4.0}.\\ (creativecommons.org/licenses/by-nc/4.0/)
\end{figure}
\newpage

\section{Osvícenský absolutismus v habsburské monarchii (1740--1790)}
\subsection{Války o španělské dědictví (1701--1714)}
\begin{itemize}
\item Středoevropští Habsburkové ztrácí Španělsko
\item zůstává jim Belgie a území v Itálii
\item \lp{1713}{\textbf{Pragmatická sankce} Karla VI.} -- trůn se dědí i přes ženskou linii
\item synové Karla VI. zemřeli jako děti \ra dynastie vymírá po meči \ra nastupuje Marie Terezie
\end{itemize}

\subsection{Marie Terezie (1740--1780)}
\begin{itemize}
\item provdána za \textbf{Františka Štěpána Lotrinského} (nevýhodné, Lotrinsko vzdáno Francii \ra velkovévoda Toskánský)
\item současníkem Fridrich II. Veliký (pruský král 1740--1786)
\end{itemize}

\paragraph{Slezské války}
\begin{itemize}
\item Marie Terezie vs. Prusko + Bavorsko, Sasko, Francie
\item 1. 1740--1742
\item 2. 1744--1745
\item ztráta Slezska 
\item korunována královnou českou
\item František Štěpán (manžel) císařem římským (Marie Terezie nebyla nikdy korunována císařovnou)
\end{itemize}

\paragraph{Sedmiletá válka}
\begin{itemize}
\item \textbf{Rakousko} + Francie, Rusko, Švédsko, Sasko vs. \textbf{Prusko} + Anglie
\item vpád Prusů do Saska, Čech (1756, 1757)
\item \lp{18. 6. 1757}{bitva u Kolína}  (Prusko $\times$ Rakousko (L. J. Daun))
	\begin{itemize}
	\item první porážka Pruské armády \ra stažení pruských vojsk z českých zemí
	\end{itemize}
\item carevna Alžběta zemřela \ra car Petr III. \ra Rusko neutrální
\item \lp{1763}{mír v Hubertsburgu} -- Rakousko s Pruskem
\item mír v Paříži -- Francie s Anglií
\item definitivní ztráta Slezska
\end{itemize}

\paragraph{Centralizační reformy}
\begin{itemize}
\item snaha o pevné \textbf{spojení Českých zemí} s Rakouskem -- nejdůležitější země monarchie
\item snaha o \textbf{germanizaci} celé monarchie
\item \textbf{správní}:
	\begin{itemize}
	\item Direktorium pro věci veřejné a komorní -- vláda
	\item Dvorská státní kancelář -- kancléřem -- hrábě Václav Antonín Kounic
	\end{itemize}
\item \textbf{právní}:
	\begin{itemize}
	\item nejvyšší soudní dvůr ve Vídni
	\item \lp{1769}{trestní zákoník}
	\item \lp{1776}{zákaz mučení}
	\item zákaz procesů o čarodějnictví
	\end{itemize}
\item \textbf{berní}:
	\begin{itemize}
	\item \lp{1748}{Tereziánský katastr}
	\item \lp{1754}{sčítání obyvatelstva}  -- od 1762 každý rok
	\end{itemize}
\item \textbf{ekonomické}:
	\begin{itemize}
	\item sjednocení míry, váhy, měny, podpora manufaktur, stavba silnic
	\end{itemize}
\item \textbf{školské}:
	\begin{itemize}
	\item povinná školní docházka (1774)
	\item školy triviální (v mateřském jazyce), hlavní (města), normální (větší města), přípravky pro učitele 
	\item industriální školy
	\item univerzita vyňata z dohledu církve
	\item hlavním vyučovacím jazykem je němčina
	\end{itemize}
\item \textbf{vojenské}:
	\begin{itemize}
	\item 7-letá vojenská povinnost pro všechny muže (17--40 let)
	\item vojenská akademie
	\end{itemize}
\item \textbf{sociálně ekonomické}:
	\begin{itemize}
	\item 1775 -- povstání sedláků na Náchodsku (poraženo u Chlumce)
	\item[\ra] \textbf{robotní patent podle majetku}
		\begin{itemize}
		\item čím více majetku, tím více roboty
		\item nejvíc -- tři dny týdně
		\item nejméně -- 13 dní ročně
		\end{itemize}
	\item 1777 -- F. A. Raab -- placená robota
	\end{itemize}
\end{itemize}

\subsection{Josef II. (1780--1790)}
\begin{itemize}
\item dynastie habsbursko-lotrinská
\item spoluvládce Marie Terezie od 1765
\item 1773 -- zrušen jezuitský řád \ra ruší "neprospěšné" řády \ra vznik kasáren, chudobinců, nemocnic
\item nová biskupství: Brno (1777), České Budějovice (1785)
\item prosazování germanizace
\item zlepšeno posílení Židů
\item \lp{13. 10. 1781}{\textbf{Toleranční patent}} --luterství, kalvinismus, ortodoxní víra
\item \lp{1. 11. 1718}{\textbf{Patent o zrušení nevolnictví}} (robota zrušena až o století a půl později)
	\begin{itemize}
	\item zisk osobní svobody
	\item možnost svatby, poslání dětí do školy, etc.
	\end{itemize}
\item vybudoval pevnosti Terezín a Josefov
\end{itemize}


\subsection{Leopold II. (1790--1792)}
\begin{itemize}
\item vévoda toskánský
\item pokračoval v politice Josefa II.
\item ústup osvícenství
\end{itemize}

\section{Baroko a rokoko}
\begin{itemize}
\item poslední dva ucelené slohy
\end{itemize}
\paragraph{Společnost}
\begin{itemize}
\item Evropa nábožensky rozdělena
	\begin{itemize}
	\item rekatolizace: české země, Falc, Polsko, Francie
	\item jansenismus -- C. Jansen; Ypres; klášter Port Royal; Blaise Pascal	
		\begin{itemize}
		\item v katolických zemích
		\item "Srdce má své důvody, o kterých rozum neví" -- Blaise Pascal
		\end{itemize}
	\item pietismus 
		\begin{itemize}
		\item v protestantských zemích
		\item důraz na zbožnost, charitu, školství
		\end{itemize}
	\end{itemize}
	\item francouzská literatura -- dominantní
		\begin{itemize}
		\item 
		\end{itemize}
\end{itemize}

\paragraph{Rozvoj vědy}
\begin{itemize}
\item empirický výzkum + teorie + experiment
\item astronomie: G. Galielei
\item fyzika: B. Pascal, Giovanni Torricelli, I. Newton
	\begin{itemize}
	\item Newton -- 1687 objev gravitačního zákona
	\item Pascal -- vymyslel první počítací stroj
	\end{itemize}
\item fyziologie: W. Harvey, A. van Leevenhoek (mikroskop)
\end{itemize}

\paragraph{Filosofie}
\begin{itemize}
\item empirismus
	\begin{itemize}
	\item F. Bacon -- "vědění je moc"
		\begin{itemize}
		\item indukce -- od jednotlivosti ke zobecnění
		\item Atlantida -- nedopsané filosofické dílo
		\end{itemize}
	\item T. Hobbes
		\begin{itemize}
		\item teoretik státního absolutismu
		\item sát je společenská smlouva mezi lidmi
		\item "homo homini lupus" = člověk člověku vlkem -- stav společnosti bez státu
		\end{itemize}
	\end{itemize}
\item osvícenství
	\begin{itemize}
	\item John Locke
		\begin{itemize}
		\item důležitá přirozená práva člověka
		\item oddělení moci výkonné a zákonodárné
		\item lidské vědomí je nepopsaný list papíru, zkušenostmi se popisuje
		\end{itemize}
	\end{itemize}
\item racionalismus
	\begin{itemize}
	\item R. Descartes
		\begin{itemize}
		\item matematik \ra dedukce -- hledal filosofické axiomy
		\end{itemize}
	\end{itemize}
	\item B. Spinoza
		\begin{itemize}
		\item pantheismus (příroda = bůh)
		\item matematická teorie etiky
		\end{itemize}
	\item G. W. Leibniz
		\begin{itemize}
		\item diferenciální a integrální počet
		\item doplnil Aristotelovy logické zákony
		\item "svět = harmonie monád"
		\end{itemize}

\end{itemize}

\paragraph{Výtvarné umění}
\begin{itemize}
\item baroko ("nepravidelná perla")
\item protestantské země -- klasicizující, umírněné baroko
\item katolické země -- ???
\item vznik na konci 16. stol. v Římě
\item hlavní znaky: disharmonie, dynamičnost, iracionalismus, nadpřirozeno, nadsázka, popírání hmoty (vznášející se postavy), ilusionismus (fake okna)
\item cílem je podnícení duchovního života
\end{itemize}

\paragraph{Architektura}
\begin{itemize}
\item ovál, stlačený oblouk, oválná nebo lichoběžníková okna, kupole s lucernou, iluzionistické fresky
\item Lorenzo Bernini -- kolonáda na nám. sv. Petra
\item Fontána di Trevi v Římě
\item Francesco Boromini 
\item Jacques Lemercier -- kostel v Sorboně
\item J. Hardouin-Mansart -- chrám v Paříži -- Napoleonovo mauzoleum
\item Jan B. Fischer z Erlachu
\item J. L. Hildebrandt
\end{itemize}

\paragraph{Čechy a Morava}
\begin{itemize}
\item Kryštof Dientzenhofer -- Chrám sv. Mikuláše na Malostranském náměstí
\item Kilián Dientzenhofer -- Chrám sv. Mikuláše na Staroměstském náměstí (koncertní sál)
\item Francesco Caratti + F. M. Kaňka -- Černínský palác (MZV)
\item Jan Blažej Santini-Aichel -- Kostel na Zelené hoře
\item Svatý kopeček u Olomouce
\item Sousoší svaté trojice -- Olomouc
\end{itemize}

\paragraph{Sochařství}
\begin{itemize}
\item dynamismus, nařasené šaty, dvojesovitě prohnuté postvy, tvarová a výrazová nadsázka
\item "memento mori" -- pamatuj na smrt
\item L. Bernini -- sousoší sv. Terezie
\item Jan Brokoff, Ferdinand Maxmilián Brokoff -- sochy v Praze na Karlově Mostě
\item Matyáš Bernard Braun -- hrad Kuks
\end{itemize}

\paragraph{Malířství}
\begin{itemize}
\item Pohyb, dramatické scény, napští
\item barevný ilusionismus, temnosvit, šerosvit (osvícení nejdůležitější postavy)
\item nábožensko mytologické a alegorické výjevy
\item fresky, závěsné obrazy
\item Caravaggio -- zakladatel
\item Francie: N. Poussin, C. Lorrain, L. Le Nain, Ch. Le Brun
\item Španělsko: D. Velázquez, B. E. Murillo
\item Nizozemí: P. P. Rubens, A. van Dyck, Rembrandt van Rijn, J. Jordanes, F. Smyders, J. Vermeer, F. Hals
\item Anglie: T. Geinsborough
\item Čechy: K. Škréta, P. Brandl, J. Kupecký, V. V. Reiner
\item V. Hollar -- český protestant, působí v Anglii (Dobrá kočka, která nemlsá)
\end{itemize}

\paragraph{Hudba}
\begin{itemize}
\item polyfonie
\item komorní hudba, opera -- nový žánr
\item Claudio Monteverdi -- zakladatel žánru opery (Orfeus, Euridika)
\item A. Corelli, A. Vivaldi
\item N: J. S. Bach (Fugy), F . Händel 
\item A: H. Purcell
\item Č: Fr. Míča, J. Mysliveček, F. Brixi, J. Stamic, J. D. Zelenka, J. a J. Bendové, \ldots
\end{itemize}

\subsection{Rokoko}
\begin{itemize}
\item 2. třetina 18. století -- sloh Ludvíka XV.
\item od francouzského "kamenný"
\item česko -- od 1730, po 1750 se prolíná s klasicismem
\item asymetrický ornament C, nebo S
\end{itemize}

\paragraph{Architektura}
\begin{itemize}
\item Malý Trianon
\item Sanssouci
\item Place de la Concorde
\end{itemize}

\paragraph{Malířství}
\begin{itemize}
\item Antoin Watteai -- Pierot, Dostaveníčko
\item Francois Boucher, Jean Honoré Fragonard  -- lechtivé náměty
\end{itemize}

\paragraph{Hudba}
\begin{itemize}
\item Alessandro Scarlatti, G. F. Teleman, Joseph Haydn, w. aA. Mozart 
\end{itemize}

\paragraph{Literatura}


\section{Osvícenství}
\begin{itemize}
\item reakce na temné baroko 
\item John Locke, Isaac Newton
\item od víry k rozumu
\item od církevních dogmat k vědeckému poznání
\item deismus
\item od fanatismu k toleranci
\item smyslem života = prosazování pokroku
\item návrat k antice
	\begin{itemize}
	\item vzorem řečtí a římští učenci a demokracie
	\end{itemize}
\end{itemize}

\subsection{Francie}
\begin{itemize}
\item ohlas v aristokratické společnosti (Voltaire, Montesquieu)
\item svět lze vysvětlit racionálně
\item přírodu lze ovládnout
\item odmítali feudální systém
\item deismus -- mechanické zákony, bůh je velký mechanik
\item mechaničtí materialisté
\item P. H. Holbach, J. O. de la Mettrie (Člověk stroj)
\item A. Helvetius
\end{itemize}

\paragraph{Charles Louis Montesquieu}
\begin{itemize}
\item O duchu zákonů
\item respektování přirozených práv člověka o(osobní svoboda, rovnost před zákony)
\item oddělení moci výkonné, zákonodárné a soudní
\end{itemize}

\paragraph{Jean Jacques Rousseau} (1712 -- 1778)
\begin{itemize}
\item člověk je od přírody dobrý, civilizace ho poškodila  (soukromé vlastnícitví, egoismus)
\item navrhoval návrat k původní rovnosti lidí, k přírodě
\item prosazoval přímou demokracii, rovnost, volnost
\item "rovnost, volnost, ..." -- autorem prvních dvou slov
\item Společenská smlouva
\item pedagog: Emil aneb o výchově, sám své děti poslal do nalezince
\item ekonom
	\begin{itemize}
	\item
	\end{itemize}
\end{itemize}


\section{Napoleon a Evropa}
\subsection{Životopis}
\begin{itemize}
\item Napoleon Bonaparte (*1769 Ajaccio -- 1821 sv. Helena)
\item chudá rodina + výhodné podmínky \ra studium na vojenské škole
\item "jsem člověk, jsem občan"
\item Desire Claryová -- snoubenka, později švédská královna (vdána za generála, který se stal králem)
\item Josefina Beauharnaisová -- nemohla mu dát syna
\item Marie Luisa -- syn "Orlík"
\end{itemize}

\subsection{Období konzulátu (1799--1804)}
\begin{itemize}
\item Napoleon konzulem
\item inteligentní \ra reformovaná státní správa
\item \lp{1799}{4. francouzská ústava}
	\begin{itemize}
	\item zachovává rovnost lidí před zákonem
	\item nezachovává volební rovnost
	\item svoboda podnikání, nedotknutelnost existujícího majetku
	\item zákaz návratu emigrantů, zkonfiskovaný majetek je nenavratitelný
	\end{itemize}
\item \lp{1801}{Konkordát s papežem} (dohoda mezi církevní a světskou institucí)
	\begin{itemize}
	\item církvi se nevrací žádný majetek ani privilegia
	\item kněží se mohou vrátit do far a sloužit mše
	\end{itemize}
\item \lp{1804}{Code Napoleon}
	\begin{itemize}
	\item opraven občanský zákoník Code Civil
	\end{itemize}
\end{itemize}

\paragraph{Válečné úspěchy}
\begin{itemize}
\item \lp{1800}{porážka Rakušanů u Marenga} -- absolutní porážka
\item \lp{1801}{mír Francie s Rakouskem v Lunévill}
	\begin{itemize}
	\item západní břeh Rýna je Francouzský
	\end{itemize}
\item \lp{1802}{mír Francie s Británií v Amiesnu}
	\begin{itemize}
	\item Británie hledá nového spojence proti Francii
	\item[\ra] vnější ohrožení Francie
	\end{itemize}
\end{itemize}

\paragraph{\lp{1804}{Napoleon císařem}}
\begin{itemize}
\item jednota před napadením
\item vládce, který je dobrý vojevůdce
\item kvůli svému původu podporoval chudé
\item korunovace schválená senátem
\item korunoval sám sebe a svou manželku
\end{itemize}

\subsection{Výbojné války}
\begin{itemize}
\item k dispozici pozemní i námořní armáda
\item \lp{20. 10. 1805}{Pruská armáda poražena u Ulmu}
\item \lp{21. 10. 1805}{Prohra u Trafalgaru}
	\begin{itemize}
	\item Britové (Nelson) zničili 3/4 francouzského loďstva
	\end{itemize}
\item Napoleon pokračuje do Rakouska
\item \lp{2. 12. 1805}{bitva tří císařů u Slavkova}
	\begin{itemize}
	\item Francie (císař Napoleon) vs. Rusko (car Alexandr) + Rakousko (císař František)
	\item Napoleon prozkoumal bojiště a vymyslel strategii, druhá strana ne
	\item na památku postavena Mohyla Míru -- Alois Sova
	\item do dnes se na polích dají najít pozůstatky vojáků
	\item mír podepsán u Slavkova
	\end{itemize}
\end{itemize}

\paragraph{1806}
\begin{itemize}
\item \lp{1806}{sjednocena Itálie} (kromě Sardinie a Vatikánu) -- do vlády dosazeni sourozenci
\item \lp{1806}{porážka Pruska} = konec Svaté říše římské
	\begin{itemize}
	\item císař František II. \ra I. rakouský (už od 1804)
	\end{itemize}
\item \lp{1806}{vytvořen Rýnský spolek}
\item \lp{1806}{kontinentální blokáda}
	\begin{itemize}
	\item cílem ekonomicky zničit Anglii 
	\item nahradit zboží Anglické francouzským
	\item Francouzské zboží bylo příliš luxusní a drahé a bylo ho málo
	\item[\ra] rozvoj domácí výroby (Brno významným městem textilního průmyslu), národních trhů, národní ekonomiky
	\end{itemize}
\end{itemize}

\paragraph{1807}
\begin{itemize}
\item Tylžský mír s Alexandrem I.
\item Velkovévodství varšavské (na úkor Pruska) -- závislé na Napoleonovi
\end{itemize}

\paragraph{Vnitřní politika}
\begin{itemize}
\item Joseph Fouché -- velitel tajné policie
	\begin{itemize}
	\item kontroloval kdo ohrožuje Napoleonovu vládu
	\item porušoval 4. ústavu
	\end{itemize}
\end{itemize}

\paragraph{1808}
\begin{itemize}
\item Napoleon napadl Španělsko
	\begin{itemize}
	\item Francouzi se chovali nadřazeně, drancovali vesnice
	\end{itemize}
\item bratr Joseph Bonapart jmenován Španělským králem (nikdy ho neovládl celé)
\item guerilla -- partyzánská válka (vzpoura kvůli Francouzskému chování)
\item Francisco Goya -- cyklus obrazů Hrůzy války
\end{itemize}

\paragraph{1809}
\begin{itemize}
\item Anglie na pomoc Španělsku
\item Francouzi poraženi ve Španělsku (Murat x Wllington)
\item Rakousko vyhlašuje novou válku
\item červenec -- bitva u Wagramu => ztráta Haliče a Terstu
\item kníže K. L. Metternich 
	\begin{itemize}
	\item od 1809 min zahraničí
	\item 1821 kancléřem
	\end{itemize}
\end{itemize}

\paragraph{1812}
\begin{itemize}
\item Rusko se zřeklo blokády
\item[\ra] tažení do Ruska -- Grande Armée (600 000)
	\begin{itemize}
	\item z Francie i žoldáci z Polska
	\end{itemize}
\end{itemize}

\paragraph{Průběh tažení}
\begin{itemize}
\item Ruský generál Michal Illarinovič Kutuzukov ustupoval až k Borodinu
\item popisováno ve Vojně a Míru
\item taktika spálené země 
\item 7. 9. 1812 bitva u Borodina
\item velké ztráty na obou stranách \ra další ústup
\item Francouzi nebyli vybaveni na Ruské podmínky
\item Napoleon byl tlačen spálenou zemí k ústupu, armáda rozdělena na oddíly
\item armáda byla pronásledována Ruskou armádou a partyzány
\item list. 1812 bitva na Berezině 
\item zbylo 40 000 mužů z 600 000
\end{itemize}

\paragraph{1813}
\begin{itemize}
\item Francouzi vyhnáni ze španělska
\item 19. 10. bitva národů u Lipska
\item vznik nové koalice Prusko, Rusko, Rakousko, Švédsko proti Francii
	\begin{itemize}
	\item švédský král (bývalý francouzký generál) znal Napoleonovu taktiku
	\end{itemize}
\end{itemize}

\paragraph{1814--1815}
\begin{itemize}
\item březen 1814 -- výtězné armády vstoupily do Paříže
\item Napoleon zajat a poslán na Elbu
\item stodenní císařství 20. 3. -- 18. 6. 1815
	\begin{itemize}
	\item Napoleon se vylodil na pobřeží Francie
	\item mladí francouzští vojáci mu byli věrní
	\item 18. 6. poražen u Waterloo  ( generál Blücher + Wellington)
	\item Napoleon deportován na sv. Helenu
	\item 1821 zemřel (možná otráven)
	\end{itemize}
\end{itemize}



\section{Vídeňský kongres}
\begin{itemize}
\item 1814 -- 1815
\item iniciován Metternichem
\item pro Vídeň ekonomicky prospěšné -- hromada šlechticů s celým dvorem
\item cíle
	\begin{itemize}
	\item Napoleon zničil řadu monarchií a rodů
	\item potřeba uznání legitimnosti rodů a nároku na jejich restauraci
	\end{itemize}
\item zásada rovnoprávnosti
\item nutnost potrestání Francie
\end{itemize}

\paragraph{Výsledky}
\begin{itemize}
\item uzavřen Pařížský mír s Francií
	\begin{itemize}
	\item hranice vráceny do roku 1792 -- před zahájením výbojných válek
	\item 3 roky okupace východofrancouzských pevností
	\item Ludvík XVIII. králem
	\item válečné reparace 700 milionů zlatých franků -- spláceno několik let
	\item ztráta kolonií
	\end{itemize}
\item Anglické územní zisky
	\begin{itemize}
	\item části francouzských a holandských kolonií
	\item Helgoland (Strategické kvůli přístavům Hamburg a Brehmy)
	\item Malta
	\item Kapsko
	\item Cejlon
	\item[\ra] převaha na moři
	\end{itemize}
\item Rusko
	\begin{itemize}
	\item zisk velkovévodství varšavského -- Kongresovka, Polské království (personální unie)
	\item Finsko
	\item Besarábie
	\end{itemize}
\item Prusko
	\begin{itemize}
	\item zisk 1/2 Saska
	\item Poznaňsko
	\item Porýní
	\item Vestfálsko
	\item Přední Pomořany
	\end{itemize}
\item Švédsko -- spojené království s Norskem
\item spolením Belgie a Nizozemska vytvořeno Nizozemské království
\item Švýcarsko zůstává neutrální
\item Itálie
	\begin{itemize}
	\item rozdělená
	\item Rakousko: Lombardie, Benátsko
	\item Bourboni: Neapolsko, Sicílie
	\item dyn. savojská: Království Sardinské
	\end{itemize}
\end{itemize}

\newpage
\timeline

\end{document}