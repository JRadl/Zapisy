\title{Ranný novověk}
\documentclass[10pt,a4paper]{article}
\usepackage[utf8]{inputenc}
\usepackage[czech]{babel}
\usepackage{amsmath}
\usepackage{amsfonts}
\usepackage{amssymb}
\usepackage{chemfig}
\usepackage{geometry}
\usepackage{wrapfig}
\usepackage{graphicx}
\usepackage{floatflt}
\usepackage{hyperref}
\usepackage{fancyhdr}
\usepackage{tabularx}
\usepackage{makecell}
\usepackage{csquotes}
\usepackage{footnote}

\MakeOuterQuote{"}

\renewcommand{\labelitemii}{$\circ$}
\renewcommand{\labelitemiii}{--}
\newcommand{\ra}{$\rightarrow$ }
\newcommand{\x}{$\times$ }
\newcommand{\lp}[2]{#1 -- #2}
\newcommand{\timeline}{\input{timeline}}


\geometry{lmargin = 0.8in, rmargin = 0.8in, tmargin = 0.8in, bmargin = 0.8in}
\date{\today}
\author{Jakub Rádl}

\makeatletter
\let\thetitle\@title
\let\theauthor\@author
\makeatother

\hypersetup{
colorlinks=true,
linkcolor=black,
urlcolor=cyan,
}



\begin{document}
\maketitle
\tableofcontents
\begin{figure}[b]
Toto dílo \textit{\thetitle} podléhá licenci Creative Commons \href{https://creativecommons.org/licenses/by-nc/4.0/}{CC BY-NC 4.0}.\\ (creativecommons.org/licenses/by-nc/4.0/)
\end{figure}
\newpage

\section{Zámořské cesty}
\paragraph{Politické důvody}
\begin{itemize}
\item \textbf{Arabové}
\item cesty do orientu obsazeny \textbf{Osmanskými turky} (po 1453)
	\begin{itemize}
	\item okrádali a zabíjeli obchodníky \ra nebezpečné těmito cestami cestovat
	\end{itemize}
\end{itemize}

\paragraph{Ekonomické důvody a předpoklady}
\begin{itemize}
\item potřeba silného financování -- financováno bohatými \textbf{kupci}
\item hledání \textbf{odbytišť} zboží vyráběného v Evropě
\item hledání nových \textbf{zdrojů surovin} (v Čechách útlum těžby), nejen Ag a Au, ale i drahokamů, vzácného dřeva, koření
\end{itemize}

\paragraph{Technické a vědecké předpoklady}
\begin{itemize}
\item začali se stavět lodě s hlubokým kýlem a kormidlem
	\begin{itemize}
	\item \textbf{velký ponor} \ra stabilita a větší nosnost
	\end{itemize}
\item zdokonalení stěžňového systému (4 stěžně, 6 plachet)
\item od 14. století používání \textbf{kompasů} (Flavio Giola)
\item převzato od Arabů
	\begin{itemize}
	\item astroláb -- slouží ke stanovení zeměpisné polohy lodě 
	\item astronomické tabulky -- pojmenování souhvězdí pro orientaci 
	\end{itemize}
\item geografické znalosti -- mapovali pobřeží
	\begin{itemize}
	\item první glóbus sestaven Martinem Behenim (zmapovány pouze známé kontinenty)
	\end{itemize}
\end{itemize}


\end{document}