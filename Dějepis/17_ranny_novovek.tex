\title{Ranný novověk}
\documentclass[10pt,a4paper]{article}
\usepackage[utf8]{inputenc}
\usepackage[czech]{babel}
\usepackage{amsmath}
\usepackage{amsfonts}
\usepackage{amssymb}
\usepackage{chemfig}
\usepackage{geometry}
\usepackage{wrapfig}
\usepackage{graphicx}
\usepackage{floatflt}
\usepackage{hyperref}
\usepackage{fancyhdr}
\usepackage{tabularx}
\usepackage{makecell}
\usepackage{csquotes}
\usepackage{footnote}

\MakeOuterQuote{"}

\renewcommand{\labelitemii}{$\circ$}
\renewcommand{\labelitemiii}{--}
\newcommand{\ra}{$\rightarrow$ }
\newcommand{\x}{$\times$ }
\newcommand{\lp}[2]{#1 -- #2}
\newcommand{\timeline}{\input{timeline}}


\geometry{lmargin = 0.8in, rmargin = 0.8in, tmargin = 0.8in, bmargin = 0.8in}
\date{\today}
\author{Jakub Rádl}

\makeatletter
\let\thetitle\@title
\let\theauthor\@author
\makeatother

\hypersetup{
colorlinks=true,
linkcolor=black,
urlcolor=cyan,
}



\begin{document}
\maketitle
\tableofcontents
\begin{figure}[b]
Toto dílo \textit{\thetitle} podléhá licenci Creative Commons \href{https://creativecommons.org/licenses/by-nc/4.0/}{CC BY-NC 4.0}.\\ (creativecommons.org/licenses/by-nc/4.0/)
\end{figure}
\newpage

\paragraph{}
\newpage
\section{Zámořské cesty}
\subsection{Důvody}
\paragraph{Politické důvody}
\begin{itemize}
\item Arabové obchodu nebránili
\item \lp{1453}{cesty do orientu Obsazeny \textbf{osmanskými turky}} (okrádali a zabíjeli obchodníky \ra nebezpečí)
\end{itemize}

\paragraph{Ekonomické důvody a předpoklady}
\begin{itemize}
\item potřeba silného financování -- financováno bohatými \textbf{kupci}
\item hledání \textbf{odbytišť} zboží vyráběného v Evropě
\item hledání nových \textbf{zdrojů surovin} (v Čechách útlum těžby), nejen Ag a Au, ale i drahokamů, vzácného dřeva, koření
\end{itemize}

\paragraph{Technické a vědecké předpoklady}
\begin{itemize}
\item lodě s hlubokým kýlem a kormidlem (velký ponor \ra stabilita a větší nosnost)
\item zdokonalení stěžňového systému (4 stěžně, 6 plachet)
\item od 14. století používání \textbf{kompasů} (Flavio Giola)
\item převzato od Arabů
	\begin{itemize}
	\item astroláb -- slouží ke stanovení zeměpisné polohy lodě 
	\item astronomické tabulky -- pojmenování souhvězdí pro orientaci 
	\end{itemize}
\item geografické znalosti -- mapovali pobřeží
	\begin{itemize}
	\item první glóbus sestaven Martinem Behenim (zmapovány pouze známé kontinenty)
	\end{itemize}
\end{itemize}

\subsection{Zprávy o orientálních zemích}
\begin{itemize}
\item námořníci si vymýšleli zvláštní bytosti a poklady (řeky, kde se zlato a drahokamy dají nabírat do dlaní)
\end{itemize}

\paragraph{Marco Polo}
\begin{itemize}
\item z Korčuly, uvězněn inkvizicí za vymyšlené zprávy o zámoří
\item cestoval z Benátek do Číny
	\begin{itemize}
	\item dostal se na dvůr Chána Kublaje (1271 -- 1295), zapisoval pro něj cesty
	\item zúčastnil se s ním výpravy na ostrov Zipangu (dnešní Japonsko)
	\end{itemize}
\end{itemize}

\paragraph{Správa zámořského území}
\begin{itemize}
\item Portugalsko, Španělsko -- státní monopol, následně kolonie rozděleny
\item Anglie, Nizozemí, Francie
	\begin{itemize}
	\item polostátní obchodní společnosti
	\item vykořisťovaly kolonie	
	\end{itemize}
\end{itemize}

\paragraph{Cesty do orientu}
\begin{itemize}
\item obeplutí Afriky
\item cesta na Západ
\end{itemize}

\subsection{Portugalsko}
\begin{itemize}
\item \lp{1415}{dobyta Ceuta} (poloostrov africké části Gibraltaru)
\item princ \textbf{Jindřich Mořeplavec} - král ho nechtěl pustit, aby neztratil dědice \ra finančně podporoval plavby
\end{itemize}


\paragraph{Cesty podél Afriky}
\begin{itemize}
\item hledali naleziště zlata
\item dopluli na Kapverdské ostrovy
\end{itemize}

\begin{itemize}
\item \lp{1444}{založení Společnosti pro obchod s koloniálním zbožím} (suroviny, otroci)
\item \lp{1487}{Bartolomeo Dias doplul na jih Afriky k mysu Dobré naděje}
\item \lp{1497}{Vasco da Gama}
	\begin{itemize}
	\item navazoval na cesty Bartolomea Diase
	\item cca rok obeplouval Afriku a doplul od Indie
	\item[\ra] dovoz drahokamů, hedvábí, koření, dřeva, \ldots
	\end{itemize}
\item \lp{1500}{Pedro Álvares Cabral -- Tra de Vera Cruz (Brazílie)}
\end{itemize}



\subsection{1. koloniální válka}
\begin{itemize}
\item \lp{1474--1479}
\item Portugalsko chtělo ovládnout Katílii
\item[\ra] první dělení světa
	\begin{itemize}
	\item spor musel řešit papež
	\end{itemize}
\item \lp{1494}{dohoda v Tordesillas}
	\begin{itemize}
	\item 2000km na západ od Kapverdských ostrovů je poledník
	\item západ je portugalský, východ je španělský
	\end{itemize}
\end{itemize}

\subsection{Španělsko}
\begin{itemize}
\item \lp{1479}{ovládli Kanárské ostrovy}
\item \lp{1479}{vznik Španělska}
\end{itemize}

\paragraph{Kryštof Kolumbus}
\begin{itemize}
\item portugalský král nechtěl cesty financovat \ra rozhodl se jít do Španělska (musel tajně uprchnout)
\item \textbf{\lp{3. 8. 1492}{První Kolumbova cesta}}
	\begin{itemize}
	\item \lp{11./12. 10. 1492}{dopluli na San Salvador}
	\item na pobřeží viděli fluoreskující řasy
	\item obyvatelé nazváni Indiány
	\item získali koření
	\end{itemize}
\item 2. Kolumbova cesta -- kolem ostrovů
\item 3. Kolumbova cesta -- k pobřeží Jižní Ameriky
\item 4. Kolumbova cesta -- pobřeží Panamské šíje
\item \lp{1506}{zemřel ve Valladolidu} (pravděpodobně pohřben na San Salvadoru)
\end{itemize}

\subsection{Ostatní země}
\begin{itemize}
\item \lp{1501}{\textbf{Amerigo Vespucci}} (jižní Amerika, popsal část pobřeží)
\item \lp{1513}{\textbf{Vasco Nunez De Balboa} překročil Panamskou šíji} (přenesli loď, objevili záliv Rica \ra potvrdili, že Amerika je kontinent)
\end{itemize}

\paragraph{Plavba kolem světa}
\begin{itemize}
\item \lp{1519}{Fernão Magalhães} (portugalský mořeplavec, financován Španěly)
\item vydali se na jihozápad s předpokladem, že obepluje Ameriku jako Afriku
\item ústí řeky Santa Cruz (Patagonie, část výpravy se vrátila zpět) \ra Magalhaesův průplav(mezi Amerikou a ohňovou zemí) \ra Filipíny
\item \lp{1521}{zabit domorodci při vyjednávání} (zbytek výpravy plul zpět)
\item \lp{1522}{cestu dokončil Sebastian del Cano}
\item[\ra] potvrzena kulatost Země
\end{itemize}



\section{Conquista}
\subsection{Hernán Cortés}
\begin{itemize}
\item \lp{1517}{cesta na Yucatan} (poloostrov v Mexiku)
\item zisk nového území od méně vyspělých civilizací
\end{itemize}
\paragraph{Mayové} (Mexiko)
	\begin{itemize}
	\item městské státy
	\item pyramidy, sochy, obrázkové písmo, astronomie, kalendáře, 
	\item dlážděné silnice \item poštovní styk, vodovody, mapy
	\item vyspělejší astronomie a kalendář, hrnčířství, řezbářství
	\item neznali železo, tažná zvířata
	\item časté přesidlování z důvodu vyčerpané půdy (neznali hnojení)
	\item považovaly Španěly za bohy (koně, palné zbraně, moře, \ldots)
	\end{itemize}
\paragraph{Aztécká říše} (Mexiko)
\begin{itemize}
\item strategické hlavní město na jezeře \textbf{Tenochtitlan}, král Montezuna
\item \textbf{nové plodiny} -- rýže, bavlna, pepř, vanilka, tabák, banány
\item \textbf{domácí zvířata} -- pes, krocan, husa, kachna
\item polyteistické náboženství (obětování stovek lidí Opeřenému hadovi)
\item \lp{1535}{Cortés objevil Kalifornii} (hledal El d'orado -- zemi zlata
\end{itemize}

\subsection{Říše Inků (Peru)}
\begin{itemize}
\item údolí Nasca, Mocha, centrum v údolí Cuzco, jezero Titicaca
\item \textbf{Machu Pichu} -- Španělé ho nikdy neobjevili
\item inka -- vládce (\ra chybně nazývaný kmen)
\item velké množství zlata -- "slzy boha slunce", nádobí, domy pobité plechy, 
\item bazény teplé vody okolo zámku, tepané květiny
\item \lp{1531-1535}{dobyli Franciscem Pizarro, Diego Almagro}
\end{itemize}

\paragraph{Vyspělá kultura inků}
\begin{itemize}
\item destilace medicíny z rostlin
\item vzdělání pro úředníky
\item sociální organizace 
	\begin{itemize}
	\item inka + dvůr, úředníci, řemeslníci
	\item při dosažení určitého věku povinný sňatek, případně přiřazen partner, půda (kdo má půdu, platí daně)
	\end{itemize}
\item terasovitá pole hnojená guanem (trus mořských ptáků)
\item lamy \ra vlna, mléko, maso
\end{itemize}



\section{Vznik nových koloniálních mocností}
\subsection{Nizozemsko}
\begin{itemize}
\item shromažďovali zboží přes pobřežní pevnosti
\item směřovali do tichomoří \ra Indonésie, Moluky
\item \lp{1602}{založena nizozemská Spojená východoindická společnost}
\end{itemize}

\subsection{Anglie}
\begin{itemize}
\item napadali španělské kolonie, snažili se je zabrat pro sebe, využívali piráty
\end{itemize}

\paragraph{Italští mořeplavci}
	\begin{itemize}
	\item \lp{1497}{\textbf{Giovanni Cabotto} (\textit{John Cabbot}) \ra Labrador, Newdoundland} (Kanada)
	\item Sebastian Cabotto \ra severoamerické pobřeží
	
	\end{itemize}


\paragraph{Angličané} 
\begin{itemize}
\item \textbf{Walter Raleigh} -- Virginie, 1627 kolonií
\item \textbf{Henry Hudson} + \textbf{William Baffin} -- hledali cestu kolem severní Ameriky
\item \lp{1596}{\textbf{W. Barents} objevil Špickberky} 
\item \textbf{John Hawkins} -- Guinea (podporoval obchod s otroky)
\end{itemize}
	
\subsection{Ostatní mocnosti}
\begin{itemize}
\item Francie -- část Kanady (1535)
\item Švédsko -- od ústí řeky Delawere vytlačeni Nizozemci
\item Dánsko a Německo -- pouze neúspěšné pokusy
\end{itemize}

\subsection{Výsledky a význam}
\begin{itemize}
\item objevení vyspělých civilizací i mimo Evropu \ra míšení kulturních vlivů
\item nové plodiny: brambory (plodina chudých), rajčata, cukrová třtina, tabák
\item potvrzena kulatost Země
\item základ světového obchodu
\item přesun center obchodu na pobřeží Atlantiku
\item \textbf{cenová revoluce \ra kapitalizace společnosti}
	\begin{itemize}
	\item do Evropy přiváženo mnoho zlata a stříbra \ra hodnota peněz jde dolů, cena zboží nahoru
	\item[\ra] rozdělení městské společnosti na \textbf{buržoazii} (bohatí) a \textbf{proletariát} (pracující)
	\item buržoazie platí daně \ra důležitá \ra chce se podílet na vládě \ra buržoazní revoluce 
	\end{itemize}
\end{itemize}

\section{Renesance}
\subsection{Vznik renezance}
\begin{itemize}
\item 14. století, Itálie
\item z italského \textit{re-nascire} \ldots nový život
\item emancipace a zvýšení významu bohatého měšťanstva
\item důležitý vliv antiky: každý je zodpovědný za sebe, důraz na vzdělání a společenskou aktivitu
\item typické rysy: individualismus, vzdělání, společenská aktivita
\end{itemize}

\subsection{Italské trecento (14. století)}
\begin{itemize}
\item koncept jednotné itálie
\end{itemize}

\subsubsection{Literatura}
\begin{itemize}
\item \textbf{Dante Alighieri} -- poprvé zavedl spisovnou latinu, \textit{Božská komedie}, \textit{De monarchia} [monarka]
\item \textbf{Francesco Petrarca} -- korespondence s Karlem IV., \textit{Canzoniere}, \textit{Sto sonetů lauře}
\item \textbf{Giovanni Boccaccio} -- \textit{Dekameron}
\item \textbf{Cola di Rienzo} -- 1347 ovládl Řím
\end{itemize}

\subsubsection{Malířství}
\begin{itemize}
\item \textbf{Giotto di Bondone} -- první plasticky vyvinuté postav
\item \textbf{Simone Martini} -- papežský palác v Avignonu
\end{itemize}


\subsection{Italské quattrocento}
\begin{itemize}
\item období rozkvětu Itálie pod vládou rodu Medici
\item centrem renesance je Florencie
	\begin{itemize}
	\item Cosimo I. Medici
	\item Lorenzo I. il Magnifico (vzdělaný, podporoval umělce)
	\end{itemize}
\item typickým znakem je hledáním harmonie a vyváženosti
\end{itemize}

\paragraph{Platónská akademie}
\begin{itemize}
\item 529 -- zrušena Justiniánem
\item za Lorenza obnovena
\end{itemize}

\subsubsection{Architektura}
\paragraph{Základní prvky}
\begin{itemize}
	\item horizontála, kopule, římsy, oblouky, sloupy, arkády (podloubí),  
	\item pilastry -- dekorativní polosloupy
	\item balustráda -- zábradlí tvořené kuželkami
	\item sgrafito dvoubarevná vyřezávaná omítka
	\item bosáž (rustika) -- imitace kamenných kvádrů
\end{itemize}
\paragraph{Autoři}
\begin{itemize}
\item Fillipo Brunelleschi -- nalezinec ve Florencii, S. Maria del Fiore, palác Pitti a jiné stavby
\item Leon Battista Alberti -- popsal prvky renezance v teoretických dálech
\end{itemize}

\subsubsection{Malířství}
\begin{itemize}
\item \textbf{Masaccio} -- perspektiva
\item \textbf{Piero della Francesca}
\item \textbf{Sandro Botticelli} -- rozměřil "krásné" proporce lidkého těla
\item \textbf{Lorenzo Ghiberti} -- dveře do Ráje
\end{itemize}


\subsection{Italské cinquecento (16. století)}
\begin{itemize}
\item vrcholné období renesance
\item přestavba chrámu sv. Petra
\end{itemize}

\paragraph{Autoři}
\begin{itemize}
\item \textbf{Giorgio Vasari} -- napsal teoretické dílo popisující renesanci a životy významných autorů
\item \textbf{Michelangelo Buonarroti} -- sochař, malíř (Sixtínská kaple -- \textit{Zrození Adama})
\item \textbf{Raffael Santi} -- \textit{Sixtinská madona}
\item \textbf{Leonardo da Vinci} -- malíř a všechno -- \textit{Mona Lisa}, \textit{Dáma s Hranostajem}, \textit{Svatá žena samotřetí}
\item \textbf{Mathias Grünewald}
\item \textbf{Hans Holbein} -- Nizozemec, púsobil v anglii, portrét Jindřicha, \textbf{Ježíš v Hrobě}
\item Hieronymus Bosh -- velmi detailní a rozsáhlé obrazy \textit{Zahrada pozemských rozkoší} -- Prado
\item \textbf{Jan van Eyck} -- výborná perspektiva, 	\textit{Svatba manželů Arnolfiniových}
\item Piter Breugel [brechl]
\item Donato Bramate
\end{itemize}

\subsection{Renesance u nás}
\begin{itemize}
\item zámek v Bučovicích (arkády ze tří stran nádvoří)
\item dům pánů z Kunštátu v Brně (nádvoří, arkády)
\item letohrádek královny Anny
\end{itemize}

\subsection{Manýrismus}
\begin{itemize}
\item protáhlé tvary
\item pitoreskní výjevy
\item Parmigiano, Tintoretto
\item \textbf{El Greco}
\end{itemize}

\paragraph{Mnýrismus na dvoře Rudolfa II.}
\begin{itemize}
\item Bartolomeus Spranger
\item Hans von Aachen
\item Arcimboldo
\item Ardien de Vries
\end{itemize}

\section{Humanismus}
\begin{itemize}
\item studia divina $\times$ studia humana
\item studium klasických jazyků -- aby si mohli přečíst antická díla
\item Erasmus Rotterdamský: Chvála bláznovství
\item \textbf{Johannes Gensfleisch von Gutenberg} -- vynález knihtisku
\end{itemize}

\paragraph{Politologie}
\begin{itemize}
\item Niccolo Machiavelli
\end{itemize}

\paragraph{Astronomie}
\begin{itemize}
\item Mikuláš Kusánský
\item Mikuláš Koperník
\item Johannes Kepler
\item Giordano Bruno
\item Galileo Galilei (sestroji dalekohled)
\end{itemize}

\paragraph{Význam}
\begin{itemize}
\item připravuje půdu pro další rozvoj novověké evropské kultury
\end{itemize}

\newpage
\section{Reformace}
\subsection{Příčiny reformace}
\begin{itemize}
\item reakce na společenskou krizi
\item krizový stav papežství 
\item všeobecná snaha reformovat církev
	\begin{itemize}
	\item \textbf{koncilialismus} -- říká, že koncil má větší autoritu, než papež
	\item \textbf{papalismus} -- autoritu má papež
	\end{itemize}
\end{itemize}

\paragraph{Cíle reformace}
\begin{itemize}
\item dodržování etických zásad křesťanství
\item odstranění světské části církve
\end{itemize}

\subsection{1. reformace (14.--15. stol.)}
\begin{itemize}
\item John Wycliffe + lolardi
\item M. Jan Hus, M. Jeroným Pražský (překládal Wycliffovy spisy)
\item husitství v Čechách
\item česká reformace \ra 1. země s oficiálně povolenými dvěma vyznáními (1436)
\end{itemize}

\subsection{Německá reformace}
\begin{itemize}
\item 16. stol. -- rozdrobenost, církev vlastní většinu pozemků, zvyšování církevních poplatků
\end{itemize}

\paragraph{Martin Luther}
\begin{itemize}
\item pocházel z Wittenberku
\item \lp{31. 10. 1517}{přibil \textbf{95 tezí} na dveře kostela} (v latině, rychle přeloženy)
\item usiluje o bezprostřední stav člověka a boha
\item papež jeho spisy nechal spálit
\item \lp{1521}{Luther pozván na říšský sněm do Wormsu}
\item hrozilo mu zatčení \ra cca rok se ukríval na Wartburgu, kde \textbf{přeložil Bibli}
\item 1522 -- zjistil, že má přívržence a začal znovu kázat \ra vznik nové církve
\item navazoval na Husa (přijímání podobojí, kritika odpustků)
\item důraz na vnitřní zbožnost
\item neuznává celibát
\item liturgický = národní jazyk
\item stoupenci zejm. šlechta, majetní měšťané
\item státní náboženství v Dánsku, Norsku, Švédsku, Islandu
\item stoupenci: České země, Uhersko, Sedmihradsko
\end{itemize}

\paragraph{\lp{1524--1526}{Německá selská válka}}
\begin{itemize}
\item nejradikálnější Durynsko -- \textbf{Thomas Müntzer} (Cvikov)
\item vycházel z Husa, ale radikálnější
\item \lp{1521}{pokus o podnícení nové revoluce v Praze}
\item rovnost všech lidí
\item \lp{1525}{poraženo, Müntzer popraven}
\item \lp{1526}{definitivní porážka i v Tyrolsku a Salzbursku}
\end{itemize}

\paragraph{Münsterská komuna}
\begin{itemize}
\item \textbf{novokřtěnci} -- křtít by se měli až dospělí lidé, kteří si jsou uvědomit váhu křtu
\item společný majetek, vojenská povinnost, protokomunistická diktatura 
\item \lp{1534}{obsazen Münster}
\item \lp{1535}{komuna vojensky poražena}
\item novokřtěnci na Moravě
\end{itemize}

\subsubsection{Katolíci $\times$ protestanti}
\begin{itemize}
\item říšský sněm ve Špýru -- protestanti tvrdí, že lidem nepřísluší hlasovat o správnosti náboženství
\item \lp{1531}{vznik šmalkadské jednoty} (vojensko-politicko-náboženský spolek luteránských knížat)
\item \lp{1546--1547}{šmalkaldská válka}
	\begin{itemize}
	\item ukončena porážkou lutheránů u Mühlberku 
	\end{itemize}
\item \lp{1555}{Augsburský mír}
	\begin{itemize}
	\item \textit{cuis regio, eius religio} -- čí vláda, toho víra
	\item všichni obyvatelé musí vyznávat náboženství panovníka \ra vznik \textbf{teritoriálních církví} \ra ještě rozdrobenější Německo
	\item[\ra] konec monopolu katolické církve i na území Německa
	\end{itemize}
\item Filip Melanchton dotvořil zásady luteránství
\end{itemize}

\subsection{Švýcarská reformace}
\paragraph{Huldrych Zwingli}
\begin{itemize}
\item navázal na Husa
\item lidé mají \textbf{právo vzdorovat} nespravedlivé vrchnosti
\item proti učení o transsubstanciaci -- přeměna podobojí je pouze symbolická (katolíci ji berou doslovně)
\end{itemize}

\paragraph{Jan Kalvín}
\begin{itemize}
\item od 1541 v Ženevě -- republika bez společenských privilegií (pronásledován z Francie)
\item konzistoř má být vedoucí orgán republiky
\item smyslem života je \textbf{práce, aktivní a mravný život}
\item učení o \textbf{predestinaci} -- člověk je svým chováním předurčen k spasení / zatracení
\item vrchnosti, která tak nežije je \textbf{nutné se vzepřít}, třeba i násilím
\item z Ženevy vzniká společenská republika -- všichni jsou si společensky rovni
\end{itemize}

\paragraph{Miguel Servetus}
\begin{itemize}
\item nizozemský biolog (objevil malý krevní objev)
\item fanatik
\item 1553 upálen Kalvínem v Ženevě
\end{itemize}

\subsection{Význam reformace}
\paragraph{Helvetská konfese}
\begin{itemize}
\item Skotsko (1560)
\item Uhersko, Polsko, Porýní
\item státní náboženství v Nizozemí
\item Francie -- hugenoti
\item Anglie -- puritáni
\item Amerika
\item reformace klade důraz na vnitřní zbožnost, prosazování národních jazyků a ospravedlňuje buržoazní revoluce
\end{itemize}

\section{Protireformace}
\begin{itemize}
\item protestanti -- protireformace
\item katolíci -- reforma katolické církve
\end{itemize}

\paragraph{Vztah katolíků k reformaci}
\begin{itemize}
\item papež Pavel III. (1534 -- 1549)
	\begin{itemize}
	\item Sanctum Officium (1542) -- "svatý úřad" -- obdoba inkvizice
	\end{itemize}
\item papež Pavel IV. (1549 - 55) -- bojovný
	\begin{itemize}
	\item převažuje vliv katolíků, kteří chtěli zlikvidovat protestantismus -- "bojovný katolicismus"
	\item\lp{1534}{založení řádu Societas Jesu Ignácem z Loyoly}
		\begin{itemize}
		\item 1540 uznán papežem
		\item misijní činnost i do protestantských zemí
		\item základní myšlenkou bylo, že řád je hlavní, jedinec neznamená nic, smyslem života je plnit řád
		\item pochopili důležitost působení v dětství \ra vznik propracovaného systému jezuitských škol
			\begin{itemize}
			\item poskytovaly výborné vzdělání za nízkou cenu
			\item učitelé byli vzdělaní vědci na špičce svého oboru
			\item jezuitský vědec Kamel \ra na Filipínách objevil rostlinu Kamélii
			\item na konci školního roku veřejné zkoušky \ra děti měly větší motivaci studovat
			\item do jezuitských škol posílali své děti i protestanti
			\end{itemize}
		\end{itemize}
	\end{itemize}
\end{itemize}

\paragraph{\lp{ 1545 -- 1563}{Tridentský koncil}}
\begin{itemize}
\item přelom v dějinách katolické církve
\item kněží hromadili stavy a funkce \ra definována nová pravidla
\item papež byl ustanoven větší autoritou než koncil (papalismus)
\item nesmí se hromadit církevní úřady
\item založeny biskupské semináře, nakonec nutné složit slib -- \textit{"credo ..."} (věřím)
\item některá usnesení jsou platná do 20. století
\item ekumenismus -- snaha o sjednocení křesťanských církví
\end{itemize}


\section{Šíření reformace}
\subsection{Francie od konce 15. století do počátku 17.}
\begin{itemize}
\item klasicky absolutistický stát (základy položeny Ludvíkem XI. po stoleté válce)
\item po r. 1453 stabilizace 
\item expanzivní politika
	\begin{itemize}
	\item do Itálie (Karel VIII.)
	\item boj s Habsburky o miláno 1521--1544
	\end{itemize}
\item od roku 1519 tlak Habsburků ze dvou stran
	\begin{itemize}
	\item Karel V. Habsburský se stává císařem
	\end{itemize}
\item František I. (1515--1547)
\item Jindřich II. (1547--1559)(syn Františka I.)
	\begin{itemize}
	\item nevěsta Kateřina Medicejská
	\item děti:
		\begin{itemize}
		\item František II. (1559--1560) (prvorozený syn a nástupce)
		\item Karel IX. (1560--1574) (ve vládě zastupován poručníky včetně Kateřiny Medicejské)
		\item Jindřich III. (1574--1589) 
		\item Markéta z Valois ("královna Margot")
		\end{itemize}
	\end{itemize}
\end{itemize}

\paragraph{2. pol. 16. století}
\begin{itemize}
\item roste vliv reformace -- kalvínistů -- hugenotů [igenotů]
\item oporou dynastie Bourbonů
\item od 1562 -- náboženské války 
	\begin{itemize}
	\item "krvavá lázeň ve Vassy" -- Jindřich nechal povraždit všechny hugenoty ve Vassy
	\end{itemize}
\item katolická liga (Jindřich de Guise) $\times$ protestantská Unie (Jindřich Navarrský de Bourbon)
\end{itemize}

\paragraph{Bartolomějská noc}
\begin{itemize}
\item 23. 8. 1572
\item sňatek Jindřicha Navarrského s Markétou z Valois s cílem sjednocení katolíků a protestantů
\item katoličtí kněží v Paříži svalovali na hugenoty všechna neštěstí \ra davová nenávist hugenotů
\item na hostině po sňatku zavražděn admirál Colginy v katedrále Nottre Dame
\item tímto odstartováno vraždění hugenotů -- 3000 v Paříži, celkem 10 000 ve Francii
\item Jindřich Navarrský se ukryl v komnatách Markéty z Valois a následně uprchl

\end{itemize}

\paragraph{Jindřich IV. de Bournon} (1589--1610)
\begin{itemize}
\item Edikt nantský (1598) -- zrovnoprávnění hugenotů s katolíky
\item rozvoj hospodářství
	\begin{itemize}
	\item monetaristická politika -- stát nezasahuje do ekonomiky (důležitý je trh)
	\item snížení daní
	\item reforma finanční správy (sjednocení měny)
	\item stavba silnic, mostů, vodních kanálů 
	\item kolonizace kanady
	\item podpora nizozemské revoluce
	\end{itemize}
\item 1610 zavražděn fanatickým mnichem
\item dvakrát ženatý
\end{itemize}

\paragraph{Ludvík XIII.} (1610--1643)
\begin{itemize}
\item poručnická vláda Marie Medicejské
\item vlády se ujímá kardinál Richelieu (dobrá vláda)
\item 1614--1789 se nescházejí generální stavy
\end{itemize}


\paragraph{70. léta 15. století}
\begin{itemize}
\item 1477 -- spor Fridricha II. Habs. s Ludvíkem XI. o Burgundsko
\item Fridrich získal ..........
\end{itemize}

\paragraph{Dvě větve habsburků}
\begin{itemize}
\item Državy Habsburků
\item děti .....
\item Karel V.
	\begin{itemize}
	\item Španělsko
	\item Nizozemí
	\item kolonie v Americe
	\end{itemize}
\item Ferdinand I.
	\begin{itemize}
	\item rakousko
	\item Čechy
	\item Uhersko
	\end{itemize}
\end{itemize}

\paragraph{Absolutismus ve Španělsku}
\begin{itemize}
\item Reconquista
\item 1469, 1479
\item Oporou Ferdinanda II. Aragonského města a hidalgové
\item Oporou Karla V. Habsburského -- grandi
\item 1520-1 povstání komunerů (Toledo)
\item Boh o Itálii -- sacco di Roma (1527)
\end{itemize}

\paragraph{•}
\begin{itemize}
\item 1556 nástupcem ve španělsku FIlip II
\item netolerantní fanatik
\item bez koncepce hospodářské politiky
\item zvyšoval daně kvůli válkam, vojensky potlačoval povstání
\item porážka Anglí v La Manche (1588)
\end{itemize}

\section{Buržoázní revoluce v Nizozemsku}
\paragraph{Vztah ke Španělsku}
\begin{itemize}
\item připojeno až počátkem 16. stol.
\item Filip II. 
\item 1557 -- státní úpadek -- vyšší daně -- omezování pravomocí
	\begin{itemize}
	\item Filip zadlužen \ra zavedeny dlužní úpisy
	\item zvýšení daní v Nizozemsku
	\end{itemize}
\item pronásledování protestantů (podpora od hugenotů z Francie)

\end{itemize}

\paragraph{Nizozemské provincie v 16. století}
\begin{itemize}
\item Nizozemí děleno na 17 provincií
	\begin{itemize}
	\item každá má svůj sněm, místodržitele, privilegia
	\end{itemize}
\item rozděleny na severní a jižní provincie
\item Jih
 	\begin{itemize}
	\item bohaté Flandry a Brabantsko (Antverpy)
	\item Valonie
	\item kalvínisti
 	\end{itemize}
\item Sever
	\begin{itemize}
	\item důležité provincie Holland, Zeeland, Utrecht
	\item Vlámové [flámové]
	\item lutheráni, novokřtěnci
	\end{itemize}
\item mezi provinciemi jsou hospodářské i společenské rozdíly
	\begin{itemize}
	\item na severu rybářství, provincie u moře bojují o půdu
	\item na jihu jsou nejdůležitější vrstvou měšťané a šlechta $\times$ na severu jen měšťané (obchodníci, řemeslníci)
	\end{itemize}
\end{itemize}

\paragraph{Generální sněm}
\begin{itemize}
\item schází se v něm 17 místodržitelů
\item předsedá mu \textbf{generální místodržící}, který je přímo zvolen králem jako jeho zástupce
\item Markéta Parmská (rádce kardinál de Granvelle)
	\begin{itemize}
	\item upírána privilegia
	\item zvyšovány daně
	\item velká vlna odporu z provincií \ra vznik opozice
	\end{itemize}
\end{itemize}

\paragraph{Opozice}
\begin{itemize}
\item umírněná: většinou šlechta
	\begin{itemize}
	\item Vilém Oranžský (p), hrabě Egmont (k), hrabě Hoorn (k)
	\item katolíci i protestanti
	\end{itemize}
\item radikálové:
	\begin{itemize}
	\item obrazoborecké hnutí (1566--1567)
	\end{itemize}
\item španělským králem vyslán vévoda z Alby \ra hrůzovláda 
\begin{itemize}
\item \ra rada pro nepokoje (1568)
\item popravování opozice (nejen protestantů) -- vévoda a vláda si nechávali majetek
\item daň alcabala [alcavala] -- zdvojnásobení daní (daň z prodeje i koupě)
\item \lp{1568}{útok Viléma Oranžského}
\end{itemize}

\paragraph{Gézové}
\begin{itemize}
\item
	\begin{itemize}
	\item poskytnut dočasný azyl v Anglii, hrozil ale útok od Španělů \ra vyhnáni
	\item[\ra] útok na Nizozemsko (nic jiného jim nezbývalo)
	\end{itemize}
\item \lp{31.3/1.4 1572}{ovládli přístav Brielle}
\item + nová armáda Viléma Oranžského
\item Alba odvolán
\item \lp{1576}{vydrancovány Antverpy}
\item nové povstání (Bruggy, Brusel, Gent)
\end{itemize}

\paragraph{Nizozemská kultura}
\begin{itemize}
\item pozemky oddělovány kanály
\item bohatí měšťané \ra vysoké domy 
\end{itemize}

\paragraph{Roztrhání Nizozemska}
\begin{itemize}
\item 1577 -- Věčný edikt
\item 1579 -- Arraská unie -- jižní provincie zůstávají se Španělskem
\item 1579 -- Utrechtská unie -- severní provincie nechtějí zůstat se španělskem
\item 1581 -- Haagská unie -- 7 severních provincií tvoří \textbf{Spojené Nizozemí}
	\begin{itemize}
	\item stavovská republika 
		\begin{itemize}
		\item generální stavy
		\item místodržící -- dědičný titul udělený Vilému Oranžskému
		\end{itemize}
	\item státním náboženstvím se stává kalvinismus, tolerují se ale všechna reformovaná náboženství
	\item velmi bohatý stát z obchodu
		\begin{itemize}
		\item 1602 -- založena Východoindická společnost
		\item 1621 -- založena Západoindická společnost
		\item 1626 -- založen nový Amsterodam (dnešní New York)
		\end{itemize}
	\end{itemize}
\end{itemize}
\end{itemize}


\timeline
\end{document}