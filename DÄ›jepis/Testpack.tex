\title{Testpack}
\documentclass[10pt,a4paper]{article}
\usepackage[utf8]{inputenc}
\usepackage[czech]{babel}
\usepackage{amsmath}
\usepackage{amsfonts}
\usepackage{amssymb}
\usepackage{chemfig}
\usepackage{geometry}
\usepackage{wrapfig}
\usepackage{graphicx}
\usepackage{floatflt}
\usepackage{hyperref}
\usepackage{fancyhdr}
\usepackage{tabularx}
\usepackage{makecell}
\usepackage{csquotes}
\usepackage{footnote}
\usepackage{movie15}
\MakeOuterQuote{"}

\renewcommand{\labelitemii}{$\circ$}
\renewcommand{\labelitemiii}{--}
\newcommand{\ra}{$\rightarrow$ }
\newcommand{\x}{$\times$ }
\newcommand{\lp}[2]{#1 -- #2}
\newcommand{\timeline}{\input{timeline}}


\geometry{lmargin = 0.8in, rmargin = 0.8in, tmargin = 0.8in, bmargin = 0.8in}
\date{\today}
\author{Jakub Rádl}

\makeatletter
\let\thetitle\@title
\let\theauthor\@author
\makeatother

\hypersetup{
colorlinks=true,
linkcolor=black,
urlcolor=cyan,
}



\begin{document}
\maketitle
\tableofcontents
\begin{figure}[b]
Toto dílo \textit{\thetitle} podléhá licenci Creative Commons \href{https://creativecommons.org/licenses/by-nc/4.0/}{CC BY-NC 4.0}.\\ (creativecommons.org/licenses/by-nc/4.0/)
\end{figure}
\newpage


\section{Absolutismus a parlamentarismus}
\begin{itemize}
\item od středověké stavovské monarchie k novým státním formám
\end{itemize}

\paragraph{Anglie po slavné revoluci}
\begin{itemize}
\item \textbf{Bill of Rights}: \textbf{rozhodující je parlament}
\item král bez parlamentu nemůže vyhlašovat a rušit zákony, vybírat daně, udržovat vojsko
\item dolní sněmovna -- volby
\item moc zákonodárná -- parlament
\item moc výkonná -- král a ministři
\item volební systém
	\begin{itemize}
	\item toryové -- konzervativci
	\item whigové -- liberálové
	\item vláda závisela na parlamentní většině
	\item menšinová strana -- právo a \textbf{povinnost kritizovat} vládu a hledat lepší řešení \ra parlamentní opozice
	\item částečně \textbf{zrušena cenzura} \ra v novinách se mohou psát i kritické články \ra vznik veřejného mínění  -- důležité pro korigování parlamentu
	\item volební právo -- jen \textbf{menšina majitelů půdy}
	\item tajné právo až v 19. stol.
	\item středověké rozdělení volebních obvodů
	\end{itemize}
\end{itemize}


\paragraph{Jiří I. Hanoverský}(1714--1727)
\begin{itemize}
\item vzdálený z rodu Stuartovců, Němec
\item \textbf{první ministr} = zástupce krále, komunikuje s vládou (král neuměl anglicky)
\item první první ministr (1721--1740) -- \textbf{Robert Walpole} (whig) 
\item \textbf{vznik občanské společnosti, demokracie}
\end{itemize}



\section{Francie v 17. až 18. století}
\begin{itemize}
\item \textbf{klasický absolutismus}
\end{itemize}
\subsection{Ludvík XIV.}(1643--1715)
\begin{itemize}
\item do 1661 -- poručník Mazarin (nástupce kardinála Richeliau)
\item posílena moc krále \ra "Stát jsme my" (majestátní plurál)
\item \textbf{fronda} -- povstání šlechticů
\item král rozhoduje o vnitřní i zahraniční politice, má výkonnou, zákonodárnou i soudní moc
\item \lp{1614}{přestávají se scházet generální stavy}(do 1789 -- VFR)
\end{itemize}

\paragraph{Armáda a výboje}
\begin{itemize}
\item \textbf{velké investice, malé zisky}
\item 119 000 mužů, doživotní vojenská služba
\item moderní zbraně (bajonety, píky, muškety)
\item \textbf{\lp{1667--1697}{výbojné "devoluční" války}}
	\begin{itemize}
	\item zabrat Nizozemsko, posunout hranice k Rýnu
	\item \textbf{devoluční právo} -- dcery z prvního manželství mají přednost v nástupnictví před syny z druhého \ra nároky na Nizozemí
	\end{itemize}
\item \textbf{{1701--1713 -- válka o španělské dědictví}}
	\begin{itemize}
	\item kandidáti
		\begin{itemize}
		\item Filip z Anjou (vnuk Ludvíka XIV.)
		\item Karel Habsburský
		\item Josef Ferdinand Bavorský
		\end{itemize}
	\item \lp{1701}{Velká aliance proti Francii} (aby se nemohla spojit se Španělskem)
		\begin{itemize}
		\item Anglie, Prusko, Nizozemsko, Portugalsko, SŘŘ, Savojsko, Hannovesko
		\end{itemize}
	\item \lp{1704}{Britové dobyli Gibraltar}
	\item \lp{1713}{mír v Utrechtu}
		\begin{itemize}
		\item Španělským králem se stal Filip z Anjou
		\item Habsburkové dostali Belgii
		\end{itemize}
	\item \lp{1714}{mír v Rastattu}  -- potvrzeno Britské vlastnictví Gibraltaru

	\end{itemize}
\end{itemize}



\paragraph{Nákladná politika}
\begin{itemize}
\item \textbf{výboje} -- neúspěšné \ra ztráta podpory buržoazie
\item nákladný dvůr -- král Slunce (zábavy, divadla, oděvy, \ldots)
\item vystavěn zámek ve Versailles
\item Palais Royal
\item oporou krále církev, úředníci (dobře placení), armáda
\item[\ra] \textbf{špatná ekonomická situace}
\end{itemize}

\paragraph{Merkantilismus}
\begin{itemize}
\item ministr financí \textbf{Jean-Baptist Colbert} [žán batist kolbér]
	\begin{itemize}
	\item z měšťanských řad, vynikající ekonom, zavedl nový systém
	\item stát je bohatý, když má aktivní obchodní bilance (vyváží více zboží než dováží)
	\item[\ra] \textbf{protekcionismus} -- vysoká cla na dovážené výrobky \ra větší odbyt domácího zboží
	\item podpora zakládání \textbf{manufaktur} -- výroba \textbf{drahých} parfémů, látek, gobelínů (nástěnné koberce) \ra nebyly příliš kupovány
	\item budování cest, průplavů (lodní doprava je nejlevnější) pro podporu obchodu
	\item koloniální \textbf{výboje} (Louisiana, Kanada, Madagaskar, Přední Indie, Indonésie; Východo a Západoindická společnost)
	\end{itemize}
\end{itemize}

\paragraph{Náboženská politika}
\begin{itemize}
\item snaha o sjednocení -- ("un roi, une foi, une loi" [in roa, in foa, in loa] = "jeden král, jedna víra, jeden zákon")
\item \textbf{\lp{1685}{zrušen Edikt nantský}} \ra emigrace protestantů \ra ztráta intelektuální a pracovní síly
\end{itemize}

\paragraph{Shrnutí}
\begin{itemize}
\item začátek Ludvíkovy vlády byl úspěšný
\item konec -- Francie před ekonomickým bankrotem
\item do testu: "\textbf{absolutismus, centralismus, byrokracie}"
\end{itemize}

\subsection{Ludvík XV. (1715--1774)}
\begin{itemize}
\item děti Ludvíka XIV. moc umíraly \ra pravnuk
\item mladý \ra regentská vláda Filipa Orleánského
\item neúspěšná snaha o hospodářské reformy \ra pokračování ekonomického úpadku
\end{itemize}




\section{Francouzská hegemonie}
\begin{itemize}
\item hegemonie = převaha
\end{itemize}
\subsection{Situace po 30-leté válce}
\begin{itemize}
\item Francie je nejsilnější mocnost \ra \textbf{expanzivní politika}
\item přirozené hranice -- Pyreneje, Alpy, Rýn
\item zisk strategicky významných míst -- jih. šp. Nizozemí, Alsasko, Štrasburk
\item spojencem Švédsko
\end{itemize}

\paragraph{Střední a východní Evropa}
\begin{itemize}
\item \lp{1683}{Turci oblehli Vídeň}
\item Jan III. Sobieski porazil Mustafa Pašu
\item pomoc prince Evžena Savojského
\end{itemize}

\paragraph{Rakouský protiútok}
\begin{itemize}
\item princ Evžen Savojský
	\begin{itemize}
	\item \lp{1686}{dobyl Budín}
	\item \lp{1697}{porazil u Zenty Mustafu II.}
	\item \lp{1699}{mír -- Turci se zřekli vlády v Uhrách}
	\end{itemize}
\end{itemize}

\paragraph{\lp{1701 -- 1713}{Válka o španělské dědictví}}
\begin{itemize}
\item Habsburkové vs. Bourboni
\item \lp{1713}{Utrechtský mír}
\item Šp. králem Filip z Anjou
\item Habsburkové -- J. Nizozemí, Neapolsko, Milánsko
\item \lp{1713}{Karel VI. \textbf{pragmatická sankce} }
	\begin{itemize}
	\item pokud vymře dynastie po meči, dědičná práva dostane ženská větev
	\end{itemize}
\item Anglie -- Gibraltar \ra obchod se šp. koloniemi
\item konec francouzské převahy
\end{itemize}


\section{Východní Evropa po 30leté válce}
\subsection{Rusko v 17. -- 18. století}
\paragraph{Expanze}
\begin{itemize}
\item zájem o \textbf{Pobaltí} (kvůli zamrzání Archangelsku)
\item boj s Polskem o \textbf{Ukrajinu}
	\begin{itemize}
	\item kozáci -- svobodní obyvatelé jihoruských stepí
	\item \lp{od 1648}{odboj kozáků proti Polákům} \ra spojení s Ruskem
	\item \lp{1667}{hranicí Dněpr}
	\end{itemize}
\item \lp{1679}{ovládnuta Kamčatka} \ra hranice s Čínou
\item \lp{1689}{smlouva v Něrčinsku} -- stanovena pevná hranice mezi Ruskem a Čínou
\end{itemize}

\subsubsection{Petr I. Veliký (1689--1725)}
\begin{itemize}
\item snaha překonat \textbf{zaostalost} Ruska
\item studijní cesta do západní Evropy (inkognito pracoval v manufaktuře)
\item usoudil \textbf{nutnost reforem} a přístupu k Baltu
\end{itemize}

\paragraph{Severní válka (1700 -- 1721)}
\begin{itemize}
\item protišvédská koalice -- Sasko, Polsko, Dánsko
\item \lp{1700}{porážka Rusů u Narvy} (v jejím důsledku reformy)
\item \lp{1703}{založen Petrohrad} v ústí řeky Něvy
	\begin{itemize}
	\item zamýšlena jako vojenská pevnost
	\item od roku 1712 hlavním městem
	\end{itemize}
\item \lp{1708-09}{Karel XII. táhl do Ruska}
\item \lp{1709}{bitva u Poltavy} -- obrat ve válce
\item \lp{1721}{\textbf{Nystädský mír}}
	\begin{itemize}
	\item Rusko získalo Rižský a Finský záliv (dnešní Estonsko a Lotyšsko)
	\end{itemize}
\end{itemize}

\paragraph{Reformy Petra I.}
\begin{itemize}
\item armáda, baltské loďstvo
\item hospodářské reformy
	\begin{itemize}
	\item vznik nevolnických manufaktur
	\item stavba lodí \ra zisk \ra prodej velmocem
	\item těžba železa v Urale
	\end{itemize}
\item politické reformy
	\begin{itemize}
	\item vznik senátu
	\item území rozděleno na gubernie
	\end{itemize}
\item kulturní reformy
	\begin{itemize}
	\item přivezena ze západu 1. tiskárna (zničena církví)
	\item tištěny první noviny
	\item přivážena a překládána technická literatura ze západu
	\item \lp{1725}{založena Akademie věd}
	\item \lp{1755}{Založena Univerzita v Moskvě}
	\end{itemize}
\end{itemize}

\subsubsection{Kateřina II. Veliká (1762--1796)}
\begin{itemize}
\item po smrti Petra I. zmatky ve vládě
\item nástup po \textbf{palácovém převratu} (vraždě Petra III.)
\item reorganizace říše, navázala na Petra I.
\item 1773--1775 -- poraženo povstání kozáků pod Jemeljana Pugačova (Ukrajina--Ural)
\item úspěšné boje s Turky \ra ovládnut Krym a černomořské pobřeží
	\begin{itemize}
	\item přístup jak k Baltu, tak k Černému moři
	\end{itemize}
\item \textbf{Grigorij Alexandrovič Potěmkin} -- rádce a milenec, významný vojevůdce
	\begin{itemize}
	\item nechal domy a cesty vylepšit podél inspekční cesty \ra "Potěmkinova vesnice"
	\end{itemize}
\item korespondence s francouzskými osvícenci
	\begin{itemize}
	\item v praxi nepoužila osvícenské reformy, protože se domnívala, že by je zaostalé Rusko nepochopilo
	\end{itemize}
\end{itemize}

\subsection{Vznik Pruska}
\paragraph{Celky, ze kterých Prusko vzniklo}
\begin{itemize}
\item \textbf{Braniborsko} -- 1415 koupili od Zikmunda Lucemburského Hohenzollerové
\item \textbf{Východní Prusko} -- řádové území, velmistr řádu Albrecht Hohenzollern r. 1525 přistoupil k luterství (1618)
\item \textbf{Západní Pomořany} -- připojeny r. 1648 k Braniborsku
\end{itemize}

\paragraph{Friedrich Vilém (1640--1688)}
\begin{itemize}
\item velký kurfiřt 
\item reformy armády
\item panovnický absolutismus
\item chytrá diplomacie za 30leté války
\item základy pro vznik prusko-braniborského státu
\item syn Fridrich I.
\end{itemize}

\paragraph{Fridrich I.}
\begin{itemize}
\item ve války o španělské dědictví se postavil na stranu Habsburků
\item za odměnu byl korunován králem od Leopolda I.
\item \lp{1701}{první pruský král}
\item podpora umění a věd
	\begin{itemize}
	\item \lp{1694}{založena univerzita v Halle}
	\item \lp{1696}{založena Akademie umění v Berlíně}
	\item \lp{1700}{založena Společnost věd}
	\end{itemize}
\item přestavba Berlína
\end{itemize}

\paragraph{Fridrich Vilém I.}
\begin{itemize}
\item ne příliš inteligentní
\item spořivý, nepodporoval vědu a kulturu
\item financování silné armády
	\begin{itemize}
	\item moderní výzbroj a výstroj
	\item branná povinnost -- možnost se vykoupit (šlechta)
	\item tvrdý výcvik -- tvrdé fyzické tresty (kůl, ulička, zasypání)
	\end{itemize}
\item junkeři -- šlechtic, který je feudálem a zároveň významným důstojníkem v armádě
\item zavedena povinná školní docházka (gramotnost vojáků)
\item armáda pouze na obrannou funkci
\end{itemize}

\paragraph{Fridrich II. Veliký}
\begin{itemize}
\item \lp{1740}{dohoda s Bavorskem a Francií}
\item válka o habsburské dědictví
	\begin{itemize}
	\item dvě války slezské (1740--42, 1744--45) \ra zisk Slezska
	\item sedmiletá válka (1756--1763)
		\begin{itemize}
		\item Prusko + Anglie $\times$ Rakousko + Francie + Rusko
		\item 1763 -- definitivní zisk Slezska
		\end{itemize}
	\end{itemize}
\item vnitřní politika: osvícenský panovnický absolutismus
	\begin{itemize}
	\item merkantilismus
	\item \textbf{Všeobecné pruské právo} -- nový právní systém
	\item náboženská tolerance -- příchod intelektuálů z Francie (zrušený edikt nantský)
	\item tvrdý postih neposlušnosti
	\item podpora věd a umění
	\item Ch. Wolf (vytvořil německé filozofické názvosloví), La Mettrie, Voltaire, I. Kant
	\end{itemize}
\item \textit{"Filosof ze Sanssouci"} (zámek podle vzoru Versailles)
\end{itemize}

\paragraph{Fridrich Vilém II.}(1786--1797)
\begin{itemize}
\item 2. a 3. dělení Polska
\end{itemize}

\paragraph{Fridrich Vilém III.}(1797--1840)

\subsection{Rozpad Polska}
\paragraph{1. dělení Polska (1772)}
\begin{itemize}
\item navrhl Filip II. Veliký (Prusko)
\item Prusko: vojvodství Pomořanské, Malborské, Chelmiňské a menší části Velkopolska
\item Rakousko: Halič a část vojvodství Sandoměřského, krakovského
\item Rusko: část Běloruska a Livonska
\item ztráta 30\% území
\item \lp{1791}{Sejm vypracoval ústavu} podle francouzského vzoru
\item \lp{1792}{útok Ruska}
\end{itemize}

\paragraph{2. dělení Polska (1793)}
\begin{itemize}
\item Rusko: oblast Vilna, Minsku, Ukrajina
\item Prusko: zbytek Velkopolska, část Kujavska a Mazovska, Gdaňsk
\item Krakov -- povstání šlechty 
	\begin{itemize}
	\item Tadeusz Kosciuszko ovládli Varšavu
	\item poraženi ruskou armádou
	\end{itemize}
\end{itemize}

\paragraph{3. dělení Polska (1795)}
\begin{itemize}
\item Rusko: Litva, Bělorusko, Ukrajina
\item Prusko: Velkopolsko s Varšavou až po Němen
\item Rakousko: Malopolsko s Krakovem, Halič, Bug
\item Polsko zaniká, obnoveno až po 1. světové válce 
\end{itemize}


\section{České země po 30leté válce (1648--1740)}
\begin{itemize}
\item Ferdinand III. (1637--1657)
\item Leopold I. (1657--1705)
\item Josef I. (1705--1711)
\item Karel VI. (1711--1740)
\end{itemize}

\paragraph{Vláda a správa}
\begin{itemize}
\item nejvyšší kancléř království českého (Vídeň)
\item zemská vláda
\item zemští úředníci jsou zodpovědní jen králi
\item snaha o centralizaci
\item ztráta Horní a Dolní Lužice
\end{itemize}

\paragraph{Náboženské poměry}
\begin{itemize}
\item netolerance nekatolických vyznání
\item \lp{1651}{soupis obyvatel podle vyznání}
\item rekatolizace
	\begin{itemize}
	\item nová biskupství -- Hradec Králové, Litoměřice
	\item \lp{1729}{svatořečení Jana Nepomuckého}
	\item působení jezuitů
	\end{itemize}
\end{itemize}

\paragraph{Sociální a ekonomické poměry}
\begin{itemize}
\item úbytek obyvatelstva o 1/3 
\item přísnější přehledy rustikálu (poddanské půdy)
\item právní normy \ra utužení nevolnictví
\item nevolnická povstání
\item \lp{1680}{Čechy, Plzeňsko, Ovčí vrch u Čeliva}
\item \lp{1680}{robotní patent} Leopolda I.
	\begin{itemize}
	\item robotník je povinen robotovat 3 dny v týdnu
	\end{itemize}
\item \lp{1692}{Chodové x Volf Maxmilián Lamingen z Albenreuthu}
	\begin{itemize}
	\item Chodové chápali patent jako úplné osvobození
	\item poraženo, vůdci zatčeni a potrestáni
	\item \lp{1695}{popraven Jan Sladký Kozina}
		\begin{itemize}
		\item "Lomikare, Lomikare, ..."
		\item Lamingen doopravdy po roce zemřel (pravděpodobně strachem že se prokletí vyplní)
		\end{itemize}
	\end{itemize}
\item počátky průmyslu
	\begin{itemize}
	\item podniká šlechta 
		\begin{itemize}
		\item města jsou omezena privilegii
		\item cechy bránily vývoji
		\end{itemize}
	\item sklářství, vlnařství, soukenictví
	\item manufaktury: Kounicové (Slavkov), Valdštejnové (Horní Litvínov)
	\item česká a moravská šlechta ve státních službách 
	\end{itemize}
\end{itemize}
\paragraph{Školství a výchova}
\begin{itemize}
\item jezuitské koleje
\item piaristická gymnázia
\item 1654 -- Karlo-Ferdinandova univerzita
	\begin{itemize}
	\item čeština se vůbec nevyučovala
	\end{itemize}
\item barokní kultura -- vrchol barokního umění na světě
\item zemský patriotismus -- vlastenectví šlechty (vlast je tam, kde mám majetek)
\item po Bílé hoře konec náboženské tolerance
\item úpadek českého jazyka (poněmčování slov, používání německých slov \ra hantec)
	\begin{itemize}
	\item udržován na venkově (Hájkova kronika česká)
	\end{itemize}
\end{itemize}


\section{Uhersko po 30leté válce}
\paragraph{Stavovská povstání}
\begin{itemize}
\item v Uhersku se nikdy nepovedlo Habsburkům prosadit absolutismus (odpor šlechty a poddaných)
\item náboženské otázky nedůležité -- tolerance (\ra imigrace protestantů)
\end{itemize}

\paragraph{Povstání Imricha Tökölyho}
\begin{itemize}
\item \lp{1678}{vzniklo u Mukačeva}
\item obsazeno území Horních Uher až po Váh
\item využívají Turci -- postup na Vídeň (1683)
\item \lp{1687}{prešovské jatky}
	\begin{itemize}
	\item veřejná poprava 24 osob jako trest za povstání (především popraveni měšťané za zločin šlechty)
	\end{itemize}
\end{itemize}

\paragraph{Povstání Františka Rákoczyho II.}
\begin{itemize}
\item syn Imricha Tökölyho
\item \lp{1703}{počátek povstání}
\item Habsburkové sesazeni z trůnu
	\begin{itemize}
	\item volba Maxmiliána Emanuela Bavorkského za nového krále
	\item spojenece Ludvíka XIV.
	\item očekávali pomoc Francie a Švédska
	\end{itemize}
\item \lp{1708}{bitva u Trenčína}
	\begin{itemize}
	\item Francie vyčerpána bitvou o Španělské dědictví, Švédové zastaveni u Poltavy \ra neměli podporu
	\item \lp{1711}{Szatmárský mír}
		\begin{itemize}
		\item zaručoval beztrestnost povstalcům
		\item potvrzeny náboženské svobody
		\item daně musí být schváleny uherským sněmem
		\item na oplátku trůn vrácen Habsburkům
		\item první kroky k osamostatnění
		\item nevolníci \ra zbojníci (Juraj Jánošík -- 1713 popraven)
		\end{itemize}
	\end{itemize}
\end{itemize}

\timeline

\end{document}