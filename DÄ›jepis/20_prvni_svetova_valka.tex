\title{První světová válka}
\documentclass[10pt,a4paper]{article}
\usepackage[utf8]{inputenc}
\usepackage[czech]{babel}
\usepackage{amsmath}
\usepackage{amsfonts}
\usepackage{amssymb}
\usepackage{chemfig}
\usepackage{geometry}
\usepackage{wrapfig}
\usepackage{graphicx}
\usepackage{floatflt}
\usepackage{hyperref}
\usepackage{fancyhdr}
\usepackage{tabularx}
\usepackage{makecell}
\usepackage{csquotes}
\usepackage{footnote}
\usepackage{movie15}
\MakeOuterQuote{"}

\renewcommand{\labelitemii}{$\circ$}
\renewcommand{\labelitemiii}{--}
\newcommand{\ra}{$\rightarrow$ }
\newcommand{\x}{$\times$ }
\newcommand{\lp}[2]{#1 -- #2}
\newcommand{\timeline}{\input{timeline}}


\geometry{lmargin = 0.8in, rmargin = 0.8in, tmargin = 0.8in, bmargin = 0.8in}
\date{\today}
\author{Jakub Rádl}

\makeatletter
\let\thetitle\@title
\let\theauthor\@author
\makeatother

\hypersetup{
colorlinks=true,
linkcolor=black,
urlcolor=cyan,
}



\begin{document}
\maketitle
\tableofcontents
\begin{figure}[b]
Toto dílo \textit{\thetitle} podléhá licenci Creative Commons \href{https://creativecommons.org/licenses/by-nc/4.0/}{CC BY-NC 4.0}.\\ (creativecommons.org/licenses/by-nc/4.0/)
\end{figure}
\newpage



\section{První světová válka}
\paragraph{Příčiny}
\begin{itemize}
\item expanzivní snahy velmocí
\item boj o kolonie
\item boj o sféru vlivu
\item extrémní nacionalismus
\item růst napětí mezi Trojspolkem a Dohodou
\end{itemize}

\paragraph{Vstup velmocí do války}
\begin{itemize}
\item Srajevský atentát (28. 6. 1914) na Františka Ferdinanda d'Este
\item Mladá Bosna -- organizace s cílem dosažení nezávislosti, Černá ruka -- teroristická frakce
	\begin{itemize}
	\item studenti, učitelé, řemeslníci
	\end{itemize}
\item Ultimátum R-U Srbsku \ra vyhlášení války 28. 7.
\item Mým národům
\item \lp{3. 8. 1914}{vstup V Británie, Francie do války}
\item Černá ruka \ra atentát na Františka Ferdinanda
	\begin{itemize}
	\item měli po něm hodit granát, ale netrefili se
	\item zastřelen po cestě (Gavrilo Princip)
	\end{itemize}
\item Rakousko-Uhersko chce vyšetřovat atentát, další podmínky
\item Srbsko podmínky odmítá
\item 28. 7. 1914 -- František Josef I.: Mým národům
	\begin{itemize}
	\item manifest, vyhlašuje válku Srbsku
	\end{itemize}
\end{itemize}

\paragraph{Nová technika}
\begin{itemize}
\item tanky, letadla, vzducholodě, rychlopalné zbraně, kulomety, děla, ponorky, otravné plyny (yperit), elektronkový vysílač
\item zákopová válka
\item totální válka \ra militarizace hospodářství
\end{itemize}


\subsection{1914}
\paragraph{Západní fronta}
\begin{itemize}
\item blesková válka (schleifen 1905)
\item Útok na Belgii a Lucembursko \ra zamýšlen útok na Francii
\item 20. 8. obsazen Brusel
\item 5. -- 15. 9. 1914 bitva na Marně -- blesková válka zastavena \ra zákopová válka
	\begin{itemize}
	\item Němci -- běh k moři
	\item 16. 10. dobyty Antverpy
	\end{itemize}
\end{itemize}

\paragraph{Jižní fronta}

\paragraph{Východní fronta}
\begin{itemize}
\item Ruskox R-U + N
\item útok na Východní Prusko
\item 
\item 
\item 
\end{itemize}

\paragraph{Další fronty}
\begin{itemize}
\item kavkazská -- poč. ledna 1915 Turecko poraženo Rusy
\item mezopotamská -- 6. 11. 1914 Angličané ovládli Basru, Kurny
\item syrsko-palestinská -- 3, 2, 1915 Turci .........
\end{itemize}


\subsection{1915}
\paragraph{Východní fronta}
\begin{itemize}
\item hlavní cíl -- porážka Ruska, potom západní fronta
\item ofenzica R-U v Karpatech
	\begin{itemize}
	\item únor -- dobyt Černovcy, Stanislav
	\end{itemize}
\item ruský útok \ra Přemyšl (22. 3.)
\item ofenzíva německa (průlom u Gorlice květen 1915)
\item ruský ústup
\item
\item 
\end{itemize}

\paragraph{Západní frona}
\begin{itemize}
\item březen Francie útoří v Champagni
\item duben německý útok na Ypry
	\begin{itemize}
	\item otráveno 15 000 mužů, 5000 zahynulo
	\item linie posunuta o 4km
	\end{itemize}
\end{itemize}

\paragraph{Jižní fronta}
\begin{itemize}
\item září 1915 -- útok Bulharska na Srbsko
\item listopad Srbsko poraženo
\item evakuace armády na Korfu
\item 23. 5. Itálie se připojila k Dohodě, 1916 vyhlásila váku německu
\item prosinece konference v Chantilly
\end{itemize}

































\subsection{Ruské revoluce 1917}
\begin{itemize}
\item 12. 3. -- demokratická revoluce (únorová 27. 2.)
	\begin{itemize}
	\item svržen car
	\item prozatimní vládna (kn. Georgij Lvov)
	\item sověty -- rady (esseři -- , menševici -- sociální demokraté)
	\end{itemize}
\item autonomie Finsku a Estonsku, nezávislost Polsku
\item pokračování ve válce
\item V. I. Lenin -- Dubnové teze
	\begin{itemize}
	\item návrat do Ruska v německém císařském vagónu
	\item hlásal proti imperialismu
	\item chtěl proletářskou revoluci \ra proletářská diktatura \ra mír s Německem
	\item \ra Němci mohou přesunout jednotky na západní frontu
	\end{itemize}
\item přibývají stoupenci bolševiků
\item protesty obyvatel (čeven, červenec)
	\begin{itemize}
	\item zvyšuje se pracovní doba, snižuje minimální mzda
	\item protest v Petrohradě \ra krvavě potlačeno armádou
	\end{itemize}
\item Alexandr Kerenský
	\begin{itemize}
	\item zakázal bolševiky, stoupenci ale stále přibývají
	\end{itemize}
\item září -- gen. Lavr Georg Kornilov -- pokus o vojenský převrat v Petrohradě
	\begin{itemize}
	\item obyvatelstvo Ruska není zvyklé a připraveno na demokracii, potřebuje pevný řád \ra potřebuje diktaturu
	\item domluva s dalšími členy vlády
	\item převrat se nepovedl, protože moc velká část vlády byla pod vlivem bolševiků
	\end{itemize}
\end{itemize}

\paragraph{Říjnová revoluce 6.--7. 11. 1917}
\begin{itemize}
\item 6. 11. -- povstání v Petrohradě 
\item Rudé gardy (ozbrojení civilisté) + baltské loďstvo
\item útok na Zimní palác
\item 7./8. 11. -- zatčeni ministři Prozatimní vlády
\end{itemize}

\paragraph{Všeruský sjezd sovětů}
\begin{itemize}
\item Rada lidovcýh komisařů -- Lenin
\item schváleny dekrety
	\begin{itemize}
	\item Dekret o míru (s Německem)
	\item Dekret o půdě -- konfiskuje půdu šlechty, dává ji rolníkům (Stalin ji zase vzal -- kolektivizace)
	\item Deklarace práv národů Ruska -- každý národ má právo na sebeurčení, má právo říct v jakém chce žít státě \ra rozpad císařství na republiky
	\end{itemize}
\end{itemize}

\paragraph{Význam revoluce}
\begin{itemize}
\item vzdělávání obyvatelstva
\item 
\end{itemize}

\paragraph{1917}
\begin{itemize}
\item revoluce v Rusku \ra vliv na konec války
\item Vilém II. souhlasí s vojenskou diktaturou a ponorkovou válkou
\item květen 1915 -- parník Lusitania (Cobh) \ra kongres vyhlašuje válku Německu
\item Němci asi 100 ponorek
\item 6. 4. -- USA vyhlašuje válku Německu (W. Wilson)
\item k vyhlášení války se přidávají další státy v J. Americe
\end{itemize}

\paragraph{Západní fronta}
\begin{itemize}
\item Němci v defenzívě
\item obrané postavení na Siegfriedově linii
\item doběn, květe -- Britové se pokouší o průlom Arrasu
\item demoralizace fr. armády předem ztracenými ofenzívami
\item Henri Philippe P=tain
\item Němci bolševickou revoluci vítali
\end{itemize}


\paragraph{Jižní fornta}
\begin{itemize}
\item úspěchy Němců a Rakušanů v Itálii
\item fronta na řece Piavě
\item Vittorio Orlando
\item pomoc fr. a britských jednotek
\end{itemize}


\section{1918}
\paragraph{Východní fronta}
\begin{itemize}
\item listopad 1917 -- jednání o separátním míru (pouze mezi dvěma stranami)
\item Lev Davidovič Trocký
	\begin{itemize}
	\item měl přijmout jakékoli podmínky
	\item Němci chtěli Lotyšsko, Litvu, Bělorusko, část Polska
	\item Trocký nechtěl podmínky přijmout ("ani mír ani válka")
	\item němci využili příležitosti k útoku
		\begin{itemize}
		\item zaútočili po celé délce fronty
		\item ovládli celou Ukrajinu, ...
		\end{itemize}
	\item[\ra] organizace Rudé armády
	\item únor 1918 bitva u Narvy a Pskova
	\item 3. 3. 1918 -- Brest Litevský mír -- Němci mají i nově nabité území
	\end{itemize}
\end{itemize}

\paragraph{Západní fronta}
\begin{itemize}
\item březen -- německá ofenziva
\item do června 5x
\item francouzský vrchní velitel maršál Ferdinand Foch
\item červenec-srpen 2. bitva na Marně 
	\begin{itemize}
	\item s Americkou pomocí Němci poraženi
	\end{itemize}
\item Německo je válkou vyčerpáno
\item \lp{8. 8. 1918}{průlom Britů u Amiensu}
\item počátekzáří -- Němci na Siegfriedově linii
\item \lp{26. 9.}{počátek rozhodující ofenzívy}
\item \lp{29. 9.}{gen. Ludendorf žádá o jednnání o příměří}
\item \lp{4. 10.}{žádost odeslána W. Wilsonovi}
\end{itemize}

\paragraph{Jižní Fronta}
\begin{itemize}
\item \lp{15. 6.}{pokus R-U o průlom na Piavě}
	\begin{itemize}
	\item umírá většina rakouských vojáků
	\end{itemize}
\item 29. 9. -- Doss Alto -- čs. legie v Itálii
\item říjen -- italská protiofenzíva
\item 25. 10. -- zhroucení pozice R-U
\item 18. 10. nóta prezidenta WIlsona
\item 27. 10. odpověď hr. Gyuly Andrassyho 
\item zánik R-U
\end{itemize}

\paragraph{Balkán}
\begin{itemize}
\item září 1918 -- zhroucení bulharské armády
\item 29. 9. Soluň -- podpis dohody o příměří
\item 1. 11. osvobozen Bělehrad
\item Rumunsko osvobozeno od německé nadvlády
\item 31. 10. kapitulace Turecka
\end{itemize}

\paragraph{Iniciativa prezidenta W. Wilsona}
\begin{itemize}
\item 8. 1. 1918 -- 14 bodů (program na ukončení války)
\item 8. 11. 1918 -- Nejvyšší spojenecká rada oznámila podmínky příměří
\item 11. 11. 1918 11:00 -- konec války
\end{itemize}


\paragraph{Hospodářská a politický situace u nás na počátku}
\begin{itemize}
\item mimořádná opatření vlády
\item válečná konjunktura (hutní, strojírenský, textilní, potravinářský průmysl)
\item suroviny potřeba na frontách \ra přídělové hospodářství
\item černý trh
\item hladové bouře (4. 17 Prostějov, 7. 17 Ostravsko)
\item růst nacionalismu
\item ženy musely pracovat protože muži bojovali \ra volební právo
\end{itemize}

\paragraph{•}

\paragraph{Česká mafie}
\begin{itemize}
\item Masaryk od podzimu 1914 v zahraničí
\item březen 1915 -- česká Mafie
\item E. Beneš -> Př. šámal (benešovi hrozí zatčení tak emigruje)
\item Kramář, Rašín
\item spolupráce s Dohodou
\item spolupráce se Slováky (Škrobár, Hodža)
\item 1915 -- Clevelandská dohoda -- s krajany v USA
	\begin{itemize}
	\item Masaryk + zástupci českých a slovenských krajanů
	\item poprvé mluví o společném státu 
	\end{itemize}
\end{itemize}


\end{document}