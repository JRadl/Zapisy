\title{Moderní dějiny}
\documentclass[10pt,a4paper]{article}
\usepackage[utf8]{inputenc}
\usepackage[czech]{babel}
\usepackage{amsmath}
\usepackage{amsfonts}
\usepackage{amssymb}
\usepackage{chemfig}
\usepackage{geometry}
\usepackage{wrapfig}
\usepackage{graphicx}
\usepackage{floatflt}
\usepackage{hyperref}
\usepackage{fancyhdr}
\usepackage{tabularx}
\usepackage{makecell}
\usepackage{csquotes}
\usepackage{footnote}
\usepackage{movie15}
\MakeOuterQuote{"}

\renewcommand{\labelitemii}{$\circ$}
\renewcommand{\labelitemiii}{--}
\newcommand{\ra}{$\rightarrow$ }
\newcommand{\x}{$\times$ }
\newcommand{\lp}[2]{#1 -- #2}
\newcommand{\timeline}{\input{timeline}}


\geometry{lmargin = 0.8in, rmargin = 0.8in, tmargin = 0.8in, bmargin = 0.8in}
\date{\today}
\author{Jakub Rádl}

\makeatletter
\let\thetitle\@title
\let\theauthor\@author
\makeatother

\hypersetup{
colorlinks=true,
linkcolor=black,
urlcolor=cyan,
}



\begin{document}
\maketitle
\tableofcontents
\begin{figure}[b]
Toto dílo \textit{\thetitle} podléhá licenci Creative Commons \href{https://creativecommons.org/licenses/by-nc/4.0/}{CC BY-NC 4.0}.\\ (creativecommons.org/licenses/by-nc/4.0/)
\end{figure}
\newpage

\section{Vznik USA}
\begin{itemize}
\item \lp{1624--1752}{\textbf{založeno 13 osad}} (Atlantik \ra Alleghany, Denver)
	\begin{itemize}
	\item zabírání půdy, pěstování kukuřice
	\end{itemize}
\item \lp{1636}{\textbf{založena Harvardská univerzita}}
\item \lp{1664}{po 2. angloholandské válce ovládla Anglie New York}
\item \lp{1667}{\textbf{Bredský mír}} -- Nizozemci dostali některé kolonie
\item správa kolonií
	\begin{itemize}
	\item kolonisté volí zástupce anglického krále \ra "guvernér"
	\item zákaz obchodu mezi koloniemi
	\end{itemize}
\item na severu farmáři, menší části půdy, otroci
\item na jihu bohatší, bavlna, otroci
\end{itemize}

\paragraph{Po sedmileté válce}
\begin{itemize}
\item Anglie vyčerpaná \ra větší daně, cla (kolkovní daň na listiny) \ra \textbf{bouře kolonistů}
\item \lp{16.12.1773}{\textbf{Bostonské pití čaje}} -- osadníci napadli lodě Východoindické společnosti a vyházeli čaj do moře
\item \lp{1. 6. 1774}{uzavřen bostonský přístav} -- odezva na Boston
\item \lp{1774}{vnucen \textbf{Quebecký pakt}}
	\begin{itemize}
	\item kolonie nesmí obsazovat území dobytá Francií (jsou Kanady)
	\item mají povinnost ubytovávat anglické vojáky
	\end{itemize}
\item \lp{1774}{\textbf{První kontinentální kongres ve Filadelfii}}
	\begin{itemize}
	\item zástupci 13 osad, snaha o usmíření s Anglií, Anglie ale nechce kompromisy
	\end{itemize}
\item \lp{1775}{sestaveno kolonistické vojsko}
	\begin{itemize}
	\item v čele\textbf{ George Washington}
	\item Angličané je neúspěšně napadají 
	\end{itemize}
\item Thomas Paine: Zdravý rozum -- brožura předkládající argumenty pro americkou samostatnost	
\end{itemize}

\paragraph{\lp{1775--1788}{Druhý kontinentální kongres}}
\begin{itemize}
\item \lp{4. 7. 1776}{prohlášení nezávislosti}
\item \textbf{Thomas Jefferson} -- právník, vypracoval ústavu
\item neúspěch povstalců \ra hledání pomoci v Evropě -- vyslán \textbf{Benjamin Franklin}
\item pomoc od Francie (nepřátelství s Anglií) -- vojsko, loďstvo, půjčka ; Ruska, Španělska
\item \lp{září 1777}{vítězství\textbf{ u Saratogy}}
\item \lp{1781}{kapitulace Angličanů \textbf{u Yorktownu}}
\item \lp{1782--1783}{mírová jednání v Paříži} \ra uznání samostatnosti
\item zadluženost farmářů \ra povstání
\end{itemize}

\paragraph{Vývoj americké ústavy}
\begin{itemize}
\item \lp{1777}{první ústava}
\item \lp{po 1781}{diskuze o nutnosti nové ústavy}
\item \textbf{Thomas Jefferson} -- konfederace, decentralizace, zemědělská produkce, podpora Francie
\item \textbf{A. Hamilton} -- federace, silná ústřední vláda, průmyslový rozvoj
\item \lp{1787}{2. ústava USA} -- (\lp{1788}{ratifikována})
\item \textbf{rozdělení moci}
	\begin{itemize}
	\item USA -- federativní stát
	\item výkonná moc -- prezident (G. Washington) + vláda
	\item zákonodárná moc -- kongres = sněmovna reprezentantů + senát
	\item soudní moc -- nejvyšší soud USA
	\item 1. nezávislý stát na kontinentu
	\end{itemize}
\end{itemize}




\section{Velká francouzská revoluce (1789--1799)}
\paragraph{Před revolucí}
\begin{itemize}
\item vyspělá kultura -- osvícenství, literatura, filosofie, móda
\item nárůst obyvatelstva o 7 milionů během 18. století \ra 18-20 milionů (většina 3. stav)
\item hospodářství -- buržoazie, sedláci platí daně, ale nemají politickou moc
\item politická moc -- král (šlechta, duchovenstvo) -- absolutismus
\item 80. léta -- zhoršování hospodářské situace, neúroda, hlad, pomoc USA
\ra nevyrovnaný rozpočet, zadlužení
\item bankéř Jean Necker (ministr financí) -- musí být zdaněny šlechta a duchovenstvo \ra silný odpor stavů
\item \lp{1789}{generální stavy se sešly ve Versailles}

\end{itemize}

\paragraph{Ludvík XVI. (1774--1793)}
\begin{itemize}
\item \lp{5. 5. 1789}{svolal generální stavy ve Versailles}
\item \lp{17. 6. 1789}{vznik \textbf{Národního shromáždení}} -- chtělo ústavu, zrušit krále
\item \lp{9. 7. 1789}{\textbf{Ústavodárné národní shromáždění}} -- připojení šlechty a duchovenstva
	\begin{itemize}
	\item nikdo už nechce absolutismus
	\end{itemize}
\item král chtěl vojensky zasáhnout \ra povstání
\end{itemize}

\subsection{První období revoluce (1789-1792)}
\begin{itemize}
\item \lp{14. 7. 1789}{\textbf{dobyta Bastila}}
\item Národní garda -- dobrovolnické oddíly nahrazující policii 
\item \lp{4. 8. 1789}{\textbf{dekrety rušící privilegia aristokracie}} \ra rovnost lidu (daně platí všichni)
\item \lp{26. 8. 1789}{\textbf{Deklarace práv člověka a občana}}
\item zrušeny závazky mezi sedláky a šlechtici
	\begin{itemize}
	\item rovnost všech stavů před zákonem, presumpce neviny, dnes součástí listiny práv
	\end{itemize}
\item konfiskován majetek církve
\item trikolora -- bílá = král, modrá, červená = Paříž
\end{itemize}

\paragraph{Diskuze o ústavě}
\begin{itemize}
\item klub \textbf{jakobínů} -- M. Roberspiere, L. Saint-Just -- podle kostela sv. Jakuba
	\begin{itemize}
	\item Francie by měla být republikou, podpora inteligence, buržoazie	 
	\end{itemize}			
\item klub \textbf{Cordeliérů} -- G. J. Dauton, J. P. Marat, J. Hébert -- podle opasku, který nosí františkáni
	\begin{itemize}
	\item též chtějí republiku, radikálnější, podpora nemajetných 
	\end{itemize}
\item \lp{1791}{\textbf{Le Chapelierův zákon}} (jakobín)
	\begin{itemize}
	\item zákaz vytváření spolků zaměstnanců a výrobců, snaha o udržení cen (do pol. 19. stol)
	\end{itemize}

\item \lp{21. 6. 1791}{\textbf{král se pokusil o útěk}} \ra domácí vězení
\end{itemize}

\paragraph{\lp{3. 9. 1791}{První francouzská ústava}}
\begin{itemize}
\item \textbf{konstituční monarchie}
\item \textit{zákonodárná moc} -- zákonodárné národní shromáždění
\item \textit{výkonná moc} -- král (právo veto) + ministři (vláda)
\item \textit{soudní moc} -- nejvyšší soud
\item volební census -- omezení politických práv, volit mohli ti, co platili nejvyšší daně \ra \textbf{sansculloti} -- chudí
\item občanská práva pro všechny
\item decentralizace -- rozdělení na departmenty (kraje)
\end{itemize}

\paragraph{Zákonodárné shromáždění}
\begin{itemize}
\item \textbf{pravice} -- stoupenci současné ústavy, chtějí konstituční monarchii (do 1792), nemají vyhraněný názor
\item \textbf{levice}
	\begin{itemize}
	\item \textbf{girondini} [žerondeni] -- voleni za kraj (Giron), bohatá buržoasie, J. P. Brissot \ra nespokojeni s monarchií, chtějí republiku
	\item \textbf{jakobíni}
	\item \textbf{cordelieři}
	\end{itemize}
\end{itemize}

\paragraph{První protifrancouzská koalice}
\begin{itemize}
\item Rakousko + Prusko (František II.) -- shromáždil armádu
\item girondini chtějí preventivně vyhásit válku, jakobíni nesouhlasí (Francie má slabou armádu)
\item \lp{1792}{vyhlášena válka Rakousku} -- 80 000 vojáků ve špatném
 stavu \ra neúspěch
\item Marseillaisa -- píseň složená 1. oddílem armády \ra hymna
\item \textbf{manifest vévody z Brunšviku z Koblence} -- pokud někdo ublíží králi, vypálí Paříž \ra povstání
\item \lp{10.8.1792}{povstalci útočí na Tuillerie} -- královo vězení v Paříži (král byl tajně přemístěn)
\item pařížská samospráva -- Komuna
\end{itemize}

\subsection{Druhé období revoluce (10. 8. 1792--27. 7. 1794)}
\begin{itemize}
\item všeobecné volební právo
\item \lp{20. 9. 1792}{\textbf{bitva u Valmy}} -- výhra Francie
\item \lp{21. 9. 1792}{\textbf{konvent vyhlásil republiku}}
\item \textbf{Konvent} -- nástupce ZNS
	\begin{itemize}
	\item pravice -- girondini
	\item levice 
		\begin{itemize}
		\item \textbf{montagnardi} -- hora (seděli nahoře), radikálnější (reformy pro
		\item jakobíni, cordeliéři
		\end{itemize}
	\end{itemize}
\end{itemize}

\paragraph{Diskuze o vině krále}
\begin{itemize}
\item jakobíni chtějí popravit krále (chtějí moc), girandini mírnější (bojí se evropských panovníků)
\item \lp{21. 1. 1793}{\textbf{Ludvík XVI. popraven}}
\item \lp{16. 10. 1793}{\textbf{popravena Marie Antoinette}}
\item[\ra] zhoršení mezinárodní situace
	\begin{itemize}
	\item ke koalici Rakouska a Pruska se přidala V. Británie, Španělsko, Nizozemí
	\end{itemize}
\item \lp{1793}{protirevoluční povstání} na venkově (Vendée)
\item vojenské neúspěchy \ra napětí mezi montagnardy a girondini
\item \lp{duben 1793}{\textbf{Výbor pro obecné blaho}} (součást Konventu)
\item \lp{květen 1793}{\textbf{Malé maximum}}
	\begin{itemize}
	\item tresty za černý obchod, omezení cen potravin
	\end{itemize}
\item novináři Mara, Hébert \ra kritizují vládu \ra obvyvatelstvo se bouří proti vládě
\item \lp{31. 5.--2. 6. 1793}{převrat v Konventu} \ra girondini vyhnáni
\end{itemize}

\paragraph{Jakobínská diktatura (1793--1794)}
\begin{itemize}
\item radikální
\item \lp{3. 6. 1793}{zákon o prodeji půdy emigrantů}
\item \lp{24. 6. 1793}{\textbf{nová republikánská ústava}}
	\begin{itemize}
	\item všichni mají právo na práci, ti co nepracují na sociální podporu, bezplatné vzdělání
	\item nevešla v platnost
	\end{itemize}
\item \lp{17. 7. 1793}{zákon o zrušení feudálních povinností}
\item \lp{léto 1793}{\textbf{protijakobínská povstání}}
	\begin{itemize}
	\item povstání girondinů -- Normandie, Provance, Bordeaux, Marseille
	\item royalistické (pro královské) -- Vendée
	\item \lp{13. 7. 1793}{zavražděn J. P. Marat} (cordelier) \ra povstání brutálně potlačeno
	\end{itemize}
\item \lp{srpen 1793}{Angličané obsadili Toulon a jiné přístavy} (zpět dobyto Napoleonem)
\item \lp{červenec 1793}{z Výboru pro všeobecné blaho vyloučeni cordeliéři} (v čele s Dantonem)
\item \lp{5. 9. 1793}{\textbf{Velká maxima}}
	\begin{itemize}
	\item nastavení min. a max. mzdy, max ceny \ra zabránění zvýšení inflace
	\end{itemize}
\item od počátku září politika teroru
\item zatčeni Hebértisté -- "zběsilí" \ra popraveni v březnu 1794
\end{itemize}

\paragraph{Teror proti církvi}
\begin{itemize}
\item konfiskace církevních pokladů
zaveden kult Nejvyšší bytosti (deismus)
\item duchovní vyháněni z farností
\item zaveden jakobínský kalendář (od 21. 9. 1792, dékady = měsíce)
\end{itemize}

\paragraph{1794}
\begin{itemize}
\item od března velké procesy -- Hébert a Danton popraveni
\item udavačství, hledání vnitřního nepřítele (září 1793 -- březen 1794 přes 30 000 lidí)
\item \textbf{Revoluční tribunály} -- souzení lidí proti revoluci
\item od června velký terror, atmosféra strachu \ra spikuntí mezi girondini a pár jakobíny
\end{itemize}
\paragraph{\lp{26. 7. 1794}{9. termidor}}
\begin{itemize}
\item Robespierre svržen, 27. zatčen, 28. popraven
\item společně popraven de Saint-Just a další jakobíni
\end{itemize}

\subsection{Třetí období revoluce (28. 7. 1794--9. 11. 1799)}
\begin{itemize}
\item období \textbf{thermidorské revoluce} -- girondéni u moci
\item uvolněn centralismus
\item uzavřen jakobínský klub
\item hospodářské reformy \ra svoboda podnikání a obchodu -- \textbf{zrušena maxima} \ra \textbf{hladomor}
\item \lp{jaro 1795}{jakobíni uskutečnili \textbf{neúspěšný pokus o povstání}}
\end{itemize}

\paragraph{\lp{22. 8. 1795}{Nová ústava}}
\begin{itemize}
\item republika, o\textbf{bčanská práva pro všechny}
\item \textbf{politická práva omezen}a -- vysoký census, dvoustupňový volební systém (jen bohatí volí prezidenta)
\item \textit{výkonná moc} -- \textbf{Direktorium} -- pěti členů
\item \textit{zákonodárná moc} -- \textbf{Rada starších}, \textbf{Rada pěti set} (inspirováno antikou)
\end{itemize}

\paragraph{Politická situace}
\begin{itemize}
\item moc Direktoria není pevná
\item \lp{1795--96}{spiknutí rovných} -- F. G. Babeuf \ra popraveni
\item \lp{1794}{poraženi Prusové v bitvě u Fleurus}
\item \lp{1795}{uzavřen separátní mír s Pruskem, Španělskem, Toskánskem, Holadskem}
\item Belgie připojena k Francii
\item ve válce pokračuje Británie a Rakousko
\item \lp{2. 3. 1796}{(generál) \textbf{Napoleon vpadl do Itálie}}
\item nastoleny demokratické svobody
\item \lp{17. 10. 1797}{\textbf{mír s Rakouskem v Campo Formio}} -- Rakousko se vzdalo Belgie a Lombardie za Benátsko Dalmácii a Istrii
\item \textbf{francouzská intervence} \ra Helvétská republika (Švýcarsko), Římská republika, Parthenopská republika
\end{itemize}

\paragraph{2. protifrancouzská koalice}
\begin{itemize}
\item Británie + Rusko, Rakousko, Bourboni (chtěli Neapolsko), Portugalsko, Turecko
\item Napoleon vyráží s expedičním sborem do Indie
	\begin{itemize}
	\item \lp{1798}{z Toulonu do Egypta} (Francie doufala, že se přidají, nestalo se tak)
	\item nalezena a rozluštěna \textbf{Rossetská deska}
	\end{itemize}
\item \lp{1. 8. 1798}{\textbf{Napoleon poražen Nelsonem u Abukiru a Akry}}
\item \textbf{ruská armáda (A. V. Suvorov) v Itálii} -- neshody s Rakouskem
	\begin{itemize}
	\item Svatogotthardským průsmykem do Švýcarska
	\end{itemize}
\end{itemize}

\paragraph{\lp{9.11.1799}{18. brumaire}}
\begin{itemize}
\item složitá vnitropolictická situace, destabilizace vlády (aristokraté a šlechta chtěli monarchii)
\item vláda uzavřela tajnou dohodu s Napoleonem \ra \textbf{Napoleonův převrat} \ra \textbf{vojenská diktatura}
\item Rada starších předala moc 3 konzulům
\item Rada pěti set rozehnána
\end{itemize}

\subsection{Důsledky a význam Velké Francouzské Revoluce}
\begin{itemize}
\item šíření liberálních myšlenek -- lidé jsou si rovni, svobodní, svobodné podnikání
\item zavádění občanských zákoníků -- Code Civil \ra moderní vztahy mezi lidmi
\item vznik států s centralizovanou správou
\item centrálně řízené školství
\item francouzská nadvláda \ra nacionalismus
	\begin{itemize}
	\item lidé odporovali Francii
	\end{itemize}
\item nedostatek koloniálního zboží \ra náhražkové produkty
	\begin{itemize}
    \item pěstování cukrové řepy
	\end{itemize}
\end{itemize}


\section{Napoleon a Evropa}
\paragraph{Životopis}
\begin{itemize}
\item Napoleon Bonaparte (*1769 Ajaccio, Korsica -- 1821 sv. Helena)
\item chudá rodina + výhodné podmínky \ra studium na vojenské škole
\item "jsem člověk, jsem občan" -- má vlastní mozek, učitelé ho neměli rádi
\item \textbf{Desire Claryová} -- snoubenka, rozešel se, později švédská královna (vdána za generála, který se stal králem)
\item \textbf{Josefina Beauharnaisová} -- vdova, nemohla mu dát syna
\item \textbf{Marie Luisa} -- syn "Orlík" (mladý umřel)
\end{itemize}

\subsection{Období konzulátu (1799--1804)}
\begin{itemize}
\item Napoleon konzulem
\item inteligentní \ra reformovaná státní správa
\item \lp{1799}{\textbf{4. francouzská ústava}}
	\begin{itemize}
	\item zachovává rovnost lidí před zákonem, volební nerovnost
	\item svoboda podnikání, nedotknutelnost existujícího majetku
	\item zákaz návratu emigrantů, zkonfiskovaný majetek je nenavratitelný
	\end{itemize}
\item \lp{1801}{\textbf{Konkordát s papežem}} (dohoda mezi církevní a světskou institucí)
	\begin{itemize}
	\item církvi se nevrací žádný majetek ani privilegia
	\item kněží se mohou vrátit do far a sloužit mše
	\end{itemize}
\item \lp{1804}{\textbf{Code Napoleon}} -- vylepšen občanský zákoník Code Civil
\end{itemize}

\paragraph{Válečné úspěchy}
\begin{itemize}
\item \lp{1800}{porážka Rakušanů u Marenga} -- absolutní porážka
\item \lp{1801}{\textbf{mír s Rakouskem v Lunévill}}
	\begin{itemize}
	\item západní břeh Rýna je Francouzský
	\item Rakušané se zřekli Belgie a Itálie
	\end{itemize}
\item \lp{1802}{\textbf{mír Francie s Británií v Amiesnu}}
	\begin{itemize}
	\item Británie má opustit Maltu, Francie Egypt
	\item \textbf{prodloužení času na hledání spojenců}
	\item[\ra] vnější ohrožení Francie
	\end{itemize}
\end{itemize}

\paragraph{\lp{1804}{Napoleon císařem}}
\begin{itemize}
\item jednota před napadením \ra vládce, který je dobrý vojevůdce
\item kvůli svému původu podporoval chudé
\item korunovace schválená senátem
\item korunoval sám sebe a svou manželku
\end{itemize}

\subsection{Výbojné války}
\begin{itemize}
\item k dispozici pozemní i námořní armáda
\item \lp{20. 10. 1805}{pruská armáda poražena \textbf{u Ulmu}}
\item \lp{21. 10. 1805}{prohra \textbf{u Trafalgaru}} (jih Španělska)
	\begin{itemize}
	\item Britové (Nelson) zničili 3/4 francouzského loďstva
	\end{itemize}
\item Napoleon pokračuje do Rakouska
\item \lp{2. 12. 1805}{bitva tří císařů \textbf{u Slavkova}}
	\begin{itemize}
	\item Francie (císař Napoleon) vs. Rusko (car Alexandr) + Rakousko (císař František)
	\item Napoleon prozkoumal bojiště a vymyslel strategii, druhá strana ne
	\item na památku postavena Mohyla Míru -- Alois Sova
	\item do dnes se na polích dají najít pozůstatky vojáků
	\item mír podepsán u Slavkova
	\end{itemize}
\end{itemize}

\paragraph{1806}
\begin{itemize}
\item \lp{1806}{sjednocena Itálie} (kromě Sardinie a Vatikánu) -- do vlády dosazeni sourozenci
\item \lp{1806}{porážka Pruska} = konec Svaté říše římské
	\begin{itemize}
	\item císař František II. \ra I. rakouský (už od 1804)
	\end{itemize}
\item \lp{1806}{vytvořen Rýnský spolek}
\item \lp{1806}{kontinentální blokáda}
	\begin{itemize}
	\item cílem ekonomicky zničit Anglii 
	\item nahradit zboží Anglické francouzským
	\item Francouzské zboží bylo příliš luxusní a drahé a bylo ho málo
	\item[\ra] rozvoj domácí výroby (Brno významným městem textilního průmyslu), národních trhů, národní ekonomiky
	\end{itemize}
\end{itemize}

\paragraph{1807}
\begin{itemize}
\item Tylžský mír s Alexandrem I.
\item Velkovévodství varšavské (na úkor Pruska) -- závislé na Napoleonovi
\item vnitřní politika
	\begin{itemize}
	\item Joseph Fouché -- velitel tajné policie
	\item kontroloval kdo ohrožuje Napoleonovu vládu \ra porušoval 4. ústavu
	\end{itemize}
\end{itemize}

\paragraph{1808}
\begin{itemize}
\item Napoleon napadl Španělsko
	\begin{itemize}
	\item Francouzi se chovali nadřazeně, drancovali vesnice \ra Francisco Goya: cyklus obrazů \textbf{Hrůzy války}
	\end{itemize}
\item bratr Joseph Bonapart jmenován Španělským králem (nikdy ho neovládl celé)
\item \textbf{guerilla} -- partyzánská válka (vzpoura kvůli Francouzskému chování)
\end{itemize}

\paragraph{1809}
\begin{itemize}
\item Francouzi poraženi ve Španělsku (pomoc od Britů)  (Murat x Wllington)
\item Rakousko vyhlašuje novou válku
\item červenec -- bitva u Wagramu => ztráta Haliče a Terstu
\item \lp{1809}{kníže K. L. Metternich ministrem zahraničí}
\item \lp{1821}{Metternich kancléřem} 
\end{itemize}

\paragraph{1812}
\begin{itemize}
\item Rusko se zřeklo blokády
\item[\ra] tažení do Ruska -- \textbf{Grande Armée} (600 000 z Francie i žoldáci z Polska)
\item Ruský generál Michal Illarinovič \textbf{Kutuzukov} stále ustupoval až k Borodinu
\item \lp{7. 9. 1812}{bitva u Borodina}
\item velké ztráty na obou stranách \ra další ústup
\item Francouzi \textbf{nebyli vybaveni} na Ruské podmínky
\item Napoleon byl tlačen \textbf{spálenou zemí} k ústupu, armáda rozdělena na oddíly
\item armáda byla pronásledována Ruskou armádou a partyzány
\item \lp{listopad 1812}{\textbf{bitva na Berezině}}
\item zbylo 40 000 mužů z 600 000
\end{itemize}

\paragraph{1813}
\begin{itemize}
\item Francouzi vyhnáni ze Španělska
\item \lp{19. 10. 1813}{bitva národů u Lipska}
\item vznik nové koalice Prusko, Rusko, Rakousko, Švédsko proti Francii
	\begin{itemize}
	\item švédský král (bývalý francouzský generál) znal Napoleonovu taktiku
	\end{itemize}
\end{itemize}

\paragraph{1814, 1815}
\begin{itemize}
\item \lp{březen 1814}{vítězné armády vstoupily do Paříže}
\item Napoleon zajat a poslán na Elbu
\item \lp{20. 3. -- 18. 6. 1815}{\textbf{sto denní císařství}}
	\begin{itemize}
	\item Napoleon se vylodil na pobřeží Francie
	\item mladí francouzští vojáci mu byli věrní
	\item \lp{18. 6. 1815}{Napoleon poražen u Waterloo}  (generál Blücher + Wellington)
	\item Napoleon deportován na sv. Helenu
	\item \lp{1821}{Napoleon zemřel} (možná otráven)
	\end{itemize}
\end{itemize}



\section{Vídeňský kongres (1814--1815)}
\begin{itemize}
\item iniciován Metternichem
\item pro Vídeň ekonomicky prospěšné -- hromada šlechticů s celým dvorem
\item cílem -- vyšetření legitimnosti rodů a nároku na jejich restauraci do Napoleonem napadených států
\item nutnost potrestání Francie
\end{itemize}

\paragraph{Pařížský mír s Francií}
	\begin{itemize}
	\item hranice vráceny do roku 1792 -- před zahájením výbojných válek
	\item 3 roky okupace východofrancouzských pevností
	\item \textbf{Ludvík XVIII.} králem
	\item válečné reparace \textbf{700 milionů zlatých franků} -- spláceno několik let
	\item ztráta kolonií
	\end{itemize}
	
\paragraph{Anglie}
	\begin{itemize}
	\item zisk  části francouzských a holandských kolonií
	\item Helgoland (Strategické kvůli přístavům Hamburg a Brehmy)
	\item Malta, Kapsko, Cejlon
	\item[\ra] převaha na moři
	\end{itemize}
	
\paragraph{Rusko}
	\begin{itemize}
	\item zisk velkovévodství varšavského -- Kongresovka, Polské království (personální unie)
	\item Finsko, Besarábie
	\end{itemize}
	
\paragraph{Prusko}
	\begin{itemize}
	\item zisk 1/2 Saska, Poznaňsko, Porýní, Vestfálsko, Přední Pomořany
	\end{itemize}
	
	
\paragraph{Švédsko} -- spojené království s Norskem

\paragraph{Nizozemské království} -- spojením Belgie a Nizozemska 

\paragraph{Švýcarsko} -- zůstává neutrální

\paragraph{Itálie} -- rozdělená
\begin{itemize}
\item \textbf{Rakousko} -- Lombardie, Benátsko
\item \textbf{Bourboni} -- Neapolsko, Sicílie
\item \textbf{dynastie savojská} -- Království Sardinské (Piemont, Sardinie)
\item papežský stát
\end{itemize}

\paragraph{Německo} -- rozdrobené 
\begin{itemize}
\item \lp{1814}{Německý spolek} -- neměl vládu, armádu, \ldots
\end{itemize}

\subsection{Výsledek kongresu}
\begin{itemize}
\item restaurace ve Francii, Španělsku, Portugalsku, Neapolsku
\item kongres = zklamání pro většinu Evropy
	\begin{itemize}
	\item v 19. století \textbf{začíná vlastenectví}
	\end{itemize}
\item prospěšný k vyrovnání sil v Evropě \ra půl století míru
	\begin{itemize}
	\item pět hlavních mocností -- \textbf{pentarchie}
	\end{itemize}
\item \textbf{kvietismus} -- nastolení policejních režimů, tajné policie, ukončení vzpoury v zárodku
\item \lp{září 1815}{\textbf{vznik Svaté aliance}}
	\begin{itemize}
	\item iniciováno Alexandrem I. -- strach, že zaostalé Rusko propadne revoluci
	\item \textbf{zásada intervence} -- cizí armáda pomůže zastavit revoluci 
	\end{itemize}
\end{itemize}






\section{Klasicismus}
\begin{itemize}
\item reakce na baroko a rokoko, které preferovaly city, vášně náboženské zaujetí
\item vyrovnanost, souměrnost
\item uctívání antiky a renesance
\item rozumem lze změnit společnost
\item odklon od víry
\item láska ke svobodě (opěvování antického boje za svobodu)
\item ideál občanských ctností 
\item akademismus -- kopírování, dílo postrádá hlubší smysl
\end{itemize}

\subsection{Malířství}
\begin{itemize}
\item \textbf{Jacques Louis David} -- dvorní malíř Napoleona (přechod přes Alpy)
\item \textbf{Antoin-Jean Gros} 
\item \textbf{Jean Dominiqu Ingrese} [engr] 
\item \textbf{Antonín Mánes} -- Letohrádek královny Anny
\item \textbf{František Tkadlík} -- portréty známých osobností (Josef Dobrovský)
\end{itemize}

\subsection{Sochařství}
\begin{itemize}
\item vyumělkované, nerozeznatelné od antiky
\item \textbf{V. Prachner} -- pramen Vltavy
\item \textbf{Bertel Thorvaldsen} 
\end{itemize}

\subsection{Architektura}
\begin{itemize}
\item Stavovské divadlo (Don Giovanni)
\item Capitol ve Washingtonu
\end{itemize}

\subsection{Hudba}
\paragraph{Vídeňský klasicismus}
\begin{itemize}
\item Joseph Haydn
\item W. A. Mozart (Don Giovanni, Kouzelná flétna, Figarova svatba, Requiem, Malá noční hudba, Turecký pochod)
	\begin{itemize}
	\item socha Mozarta před Redutou
	\end{itemize}
\item Ludwig van Bethoven (opera Fidelio, simfonie Eroica, pátá, devátá)
\end{itemize}

\subsection{Literatura}
\begin{itemize}
\item J. W. Goethe
\end{itemize}


\subsection{Empír}
\begin{itemize}
\item sloh Napoleonova císařství
\item kolonády
\item 
\end{itemize}

\subsection{Romantismus}
\begin{itemize}
\item různá datace (po r. 1790; 30. léta 19. stol.)
\item nový doživotní pocit -- proti rozumu lidská duše
\item dobrodružnství, nepředvídatelné situace, příroda, lidová kultura
\item vlastenectví, národní hnutí, národní tradice (proti osvícenskému světoobčanství)
\item mystické představy -- "duch doby", "duch národa"
\item uctíván středověk (trubadůři, úcta k ženám)
\item přeceňována úloha jedince -- hrdiny
\item typické životní osudy -- Byron, Puškin, Lermontov, Mácha
\end{itemize}

\paragraph{Malířství}
\begin{itemize}
\item F \textbf{Théodore Géricault} [žeryko] 
\item F \textbf{Eugéne Delacroix}
\item Š \textbf{Francisco Goya} -- Nahá Maja [macha] (pohoršení Španěleské společnosti)
\item A \textbf{John Constable} 
\item A \textbf{William Turner} -- Déšť pára a rychlost
\item N \textbf{Caspar Davide Friedrich} -- Kříž v horách
\item Č \textbf{Antonín Mánes}, \textbf{Josef Mánes} -- Dostaveníčko
\item Č \textbf{Adolf Kosárek} 
\item Č \textbf{Josef Matěj Navrátil}
\end{itemize}

\paragraph{Architektura}
\begin{itemize}
\item napodobování středověkých hradů a slohů
\item anglický park se zříceninou (francouzský park má geometrické tvary, anglický nepravidelné)
\item altány
\item lidové stavby 
\item minarety
\item \textbf{Lednicko valtický areál}
	\begin{itemize}
	\item zámek Lednice, Janův hrad, minaret, vodárna, antický chrám
	\end{itemize}
\end{itemize}

\paragraph{Hudba}
\begin{itemize}
\item Beethoven -- pozdní díla
\item Robert Schuman
\item Franz Schubert
\item Felix Mendelson-Bartholdy
\item novoromantici
	\begin{itemize}
	\item Franz Liszt -- maďarsko
	\item Hektor Berlioz
	\item Richard Wagner -- tvůrce německé opery (nacionalista, antisemita)
	\item Petr Iljič Čajkovský -- Labutí Jezero, Louskáček
	\item Bedřich Smetana -- opery, Vltava
	\item Antonín Dvořák -- Rusalka, Novosvětská
	\end{itemize}
\end{itemize}

Romantismus ovlivnil symbolismus, secesi, impresionismus a silou výrazu expresionismus



\newpage
\section{Období restaurace a rovnováhy (1815--1847)}
\begin{itemize}
\item na trůny se vracejí původní dynastie
\item pentarchie zajišťuje, že mezi sebou neválčí
\item konají se kongresy, kde se řeší významné problémy
\end{itemize}

\paragraph{Hlavní rysy}
\begin{itemize}
\item konsolidace tradičních monarchií (policie, ekonomická stabilizace)
\item neúspěšná povstání a spiknutí
\item doba prvních národních bojů a osvobozeneckých hnutí
\item národy v područí jiných se snaží o dosažení nezávislosti
\item rozvoj ekonomiky, kapitalistická industrializace
\item moderní státní správa -- obec, okres, kraj, stát
\item konfliktní otázky se řeší na pravidelných kongresech
\item \lp{1818}{kongres v Cáchách}
	\begin{itemize}
	\item Francie (Ludvík 18.) se stává rovnoprávným státem k evropským velmocím
	\item okupační vojska se stahují z Francie
	\end{itemize}
\end{itemize}

\subsection{Národní hnutí a revoluce ve 20. letech 19. století}
\paragraph{Německo}
\begin{itemize}
\item \lp{1815}{vznik vlasteneckých spolků} -- \textbf{buršenšafty}
	\begin{itemize}
	\item cílem sjednocení Německa
	\item považovány za neškodné
	\item členy často mladí studenti a učitelé
	\end{itemize}
\item \lp{1817}{setkání vůdců spolků na Wartburgu}
	\begin{itemize}
	\item v roce 1517 vystoupil Martin Luther, na Wartburgu se skrýval před Papežem \ra výročí
	\item[\ra] Metternich pochopil, že spolky mohou být nebezpečné
	\end{itemize}
\item \lp{1819}{karlovarské usnesení}
	\begin{itemize}
	\item spolky byly zakázány, studenti vyloučeni z univerzit, cenzurované přednášky (porušení svobody projevu i na univerzitách)
	\item záminkou byly politické atentáty
	\end{itemize}
\item[\ra] národní hnutí na deset let umlčeno
\end{itemize}

\paragraph{Itálie}
\begin{itemize}
\item intelektuálové se schází tajně v horách u milířů (kryti kouřem) \ra \textbf{karbonáři}
\item \lp{1820}{povstání v Neapolsku} \ra Ferdinand I. Bourbonský se dostává k moci
	\begin{itemize}
	\item nutno uznat ústavu \ra mnoho dalších převratů
	\item převrat v sardinském království
	\end{itemize}
\item \lp{1821}{rakouská intervence} -- pozatýkání karbonářů
\item \textbf{Silvio Pellico} -- známý karbonář, zajat policií, vězněn na Špilberku
\end{itemize}

\paragraph{Španělsko}
\begin{itemize}
\item \lp{leden 1820}{revoluce}
\item inspirace v americkém boji za nezávislost
\item \lp{1823}{francouzská intervence}
\item potlačení povstání ve Španělsku bylo obtížné \ra nedostatek sil na udržení kolonií v Americe (Mexiko, \ldots)
\item známí revolucionáři \textbf{Simon Bolívar}, \textbf{José San Martín}
\end{itemize}

\paragraph{Rusko}
\begin{itemize}
\item 97\% obyvatelstva negramotných
\item mladí lidé odjíždí studovat do Francie, Belgie, Anglie -- velke společenské rozdíly
\item[\ra] vznik tajných spolků
	\begin{itemize}
	\item vysoká tajnost \ra data jsou orientační
	\item \lp{1821}{vznik Jižního spolku}
		\begin{itemize}
		\item inspirováni Francií chtěli republiku, volební právo pro všechny, zrušení nevolnictví
		\end{itemize}
	\item \lp{1822}{vznik Severního spolku}
		\begin{itemize}
		\item chtěli alespoň konstituční monarchii, volební právo pro majetné, zrušení nevolnictví
		\end{itemize}
	\item 18?? -- společnost sjednocených Slovanů
	\end{itemize}
\item \lp{1825}{zemřel Alexandr I.} -- náhlé \ra několik měsíců bezvládí
\item \lp{26. 12. 1825}{povstání děkabristů} (14. děkabr podle pravoslavného kalendáře)
	\begin{itemize}
	\item shromážděno 3000 důstojníků, pokládali požadavky novému carovi
	\item povstání tvrdě potlačeno věrnými oddíly
	\item 5 vůdců popraveno, 120 posláno do pracovních táborů (A. S. Puškin)
	\end{itemize}
\end{itemize}

\subsection{Národně osvobozenecké hnutí na Balkáně}
\begin{itemize}
\item začátek 19. století -- pod nadvládou Turků
\end{itemize}

\paragraph{Srbsko}
\begin{itemize}
\item \lp{1804--13}{první buržoázní revoluce proti Turkům}
\item nová šlechta majetných sedláků (stará vyvražděna turky po bitvě u Kosova)
\item Dorde Petrovic Kradorde [Karodžordže]
\item obklíčen bělehrad
\item na straně Srbů stojí Rusko -- "hájí pravoslavné vyznání", chctějí ovládnout černé moře, Bospor, Dardanelli
\item \lp{1812}{rusko ovládlo Besarábii, Moldávii} -- mír s Tureckem
\item \lp{1813}{Turci táhnou na bělehrad} -- potlačí povstání
\item \textbf{Miloš Obrenovič}
	\begin{itemize}
	\item chtěl vyjednávat s Turky o Srbské autonomii
	\item Turci vyjednávat nechtěli
	\end{itemize}
\item \lp{1815--1817}{nové povstání}
	\begin{itemize}
	\item vrátil se Kradorde
	\item \lp{1817}{potlačeno}
	\item zahrnuti do podmínek drinopolského míru (1829)
		\begin{itemize}
		\item Francie a Británie pomohli zastavit Turecko
		\end{itemize}
	\end{itemize}
\item Srbsko získává autonomii
\item Obrenovič dědičným srbským knížetem
\item svoboda vyznání, srbské tiskárny, pošta, školy
\item půda je soukromým majetkem (ne majetkem sultána)
\end{itemize}

\paragraph{Řecko}
\begin{itemize}
\item Řekové žili v Turecku a Osmanské říši
\item zachovali si pravoslavnou církev i přes velký tlak islamizace \ra spojenec Rusko
\item kontakty s Evropou
\item kníže \textbf{Alexandr Ypsilanti} -- přítel Ruského cara \ra příslib vojenské pomoci
\item \lp{1821}{povstání v Moldavsku} (s pomocí Ruska)
\item \lp{1822}{protiřecký teror na ostrově Chios}
	\begin{itemize}
	\item vyvražďování obyvatelstva, vypalování vesnic
	\end{itemize}
\item filhelénské hnutí (prořecké) 
	\begin{itemize}
	\item vybírány suroviny, financí
	\item dobrovolníci -- G. Byron
	\end{itemize}
\item Rusko nabírá síly na osvobození Balkánu \ra nelíbí se to Turecku
\item \lp{1827--29}{ruskoturecká válka}
\item \lp{1827}{útok loďstva A, F, R}
\item \lp{1829}{drinopolský mír}
\begin{itemize}
\item \lp{1830}{Londýn -- dokončování hranic} \ra dědičná monarchie
\item \lp{1832}{králem Otto Wittelsbach} -- Bavorsko (protlačen Angličany)
\end{itemize}
\end{itemize}

\paragraph{Rumunsko}
\begin{itemize}
\item knížectví \textbf{Valachie, Moldávie} (zatím to není Rumunsko)
\item \lp{1831--32}{autonomie}
\item \lp{do 1851}{obsazena ruským a tureckým vojskem}
\item \lp{1861}{sjednocena} \ra Rumunsko (Alexander Ion Cuza) -- autonomie v rámci Osmanské Říše
\end{itemize}

\begin{itemize}
\item národně osvobozenecké hnutí až ve 30. a 40. letech
\end{itemize}



\subsection{Revoluce ve 30. letech 19. století}
\begin{itemize}
\item růst vlivu liberálů
\end{itemize}

\paragraph{Liberalismus}
\begin{itemize}
\item svobodný občan má mít svobodu slova, podnikání, vyznání
\item stát občana neomezuje, pouze chrání jeho majetek
\item rovná občanská práva pro všechny
\item politická práva pouze pro majetné (platící daně)
\item svoboda každého občana končí tam, kde začíná svoboda druhého
\end{itemize}

\paragraph{Francie}

\begin{itemize}
\item 20. léta
	\begin{itemize}
	\item Karel X. (1824--1830)
	\item pokus o obnovu absolutismu
	\item svoboda tisku
	\item nelegální opozice, demokraté, republikáni, bonapartisté
	\end{itemize}



\item \lp{1827}{ve volbách výtěží liberálové \ra král nechal rozpustit sněmovnu}
	\begin{itemize}
	\item červenec -- ordonance = omezená politická práva
	\end{itemize}
\item červencová revoluce
	\begin{itemize}
	\item \lp{26. 7. 1830}{barikády v ulicích}
	\item vojáci poslaní na potlačení se přidali k povstání
	\item[\ra] Karel X. uprchl
	\end{itemize}
\item prozatímní vláda \ra zvolen za krále Ludvík Filip Orleánský
\item červencová monarchie (1830--1848)
	\begin{itemize}
	\item nový politický systém
	\item rozšířeny občanské svobody
	\item snížen census -- většina Francouzů ale stále nemůže volit
	\item prostor pro hospodářský vývoj
	\end{itemize}
\item 30. léta -- hospodářská prosperita
\item po r. 1839 -- hospodářská krize -- korupční aféry
\end{itemize}

\paragraph{Belgie}
\begin{itemize}
\item součástí Nizozemského království
	\begin{itemize}
	\item Vilém I. Oranžský
	\item centralismus, holandština, (B) katolicismus x (H) kalvinismus
	\item persekuce belgických politiků snažících se o rovnoprávnost
	\item státní dluh (Belgie měla splácet Holandský dluh)
	\item vliv Francie
	\end{itemize}


\item[\ra] revoluce
\begin{itemize}
\item \lp{25. 8. 1830}{povstání v Bruselu} (24. uvedena opera s revolučním obsahem)
\item poslána armáda na potlačení \ra v září armáda poražena, donucena k ústupu
\item \lp{24. 9. 1830}{prozatímní vláda}
\item samostatnost Belgie = průlom do Vídeňského kongresu
\item \lp{1831}{ústava}
	\begin{itemize}
	\item nejliberálnější 
	\item plná občanská práva, svoboda slova, tisku, náboženství
	\end{itemize}
\item král Leopold I. Sasko-Koburský
\item vyhlášena věčná neutralita, porušena až Němci za 1. sv. války
\end{itemize}
\end{itemize}

\paragraph{Německo}
\begin{itemize}
\item liberální hnutí Mladé Německo (Mladá Evropa)
	\begin{itemize}
	\item snaha o sjednocení německa
	\end{itemize}
\item[\ra] Říšská rada -- persekuce (1834) \ra emigrace
	\begin{itemize}
	\item mezi emigranty také Karel Marx
	\end{itemize}
\end{itemize}


\paragraph{Polsko}
\begin{itemize}
\item Království polské
	\begin{itemize}
	\item ohlas revoluce ve Francii a Belgii
	\item tajné organizace proti Mikulášovi I.
	\end{itemize}
\item konzervativci -- Adam Czartoryský
\item radikálové -- Jaroslav Lelewel
\item \lp{30. 11. 1830}{povstání ve Varšavě}
	\begin{itemize}
	\item polské vojsko obsadilo celou kongresovku, sesadilo Ruské úředníky
	\item svolán sejm: Mikuláš sesazen
	\end{itemize}
\item Poláci špatně spoléhali na francouzskou a belgickou pomoc
\item \lp{květen 1831}{útok ruského vojska}
\item povstalci poraženi \ra persekuce, emigrace (A. Mickiewicz, F. Chopin, J. Lewelel)
\item výsledky
	\begin{itemize}
	\item zrušení autonomie Kongresovky \ra přímá součást Ruska
	\item zrušeno polské vojsko
	\item zrušena univerzita + stř. školy
	\item vyučovací a úřední jazyk je ruština
	\item ochromen kulturní život
	\item rusifikace
	\item vliv na české národní obrození
		\begin{itemize}
		\item pomáhalo tříbit názory (starší -- proč slované bojují proti slovanům, mladší generace -- politické myšlení)
		\item čeští vlastenci pomáhali Polákům v prchání do Belgie/Francie
		\end{itemize}
	\end{itemize}
\end{itemize}

\paragraph{Velká Británie}
\begin{itemize}
\item 30. a 40. léta -- politické a sociální otřesy
	\begin{itemize}
	\item hnutí za volební reformu
	\item \textbf{Charta lidu} (1838)
		\begin{itemize}
		\item všeobecné volební právo
		\item zrušení majetkového censu pro poslance
		\item úprava velikosti volebních obvodů
		\end{itemize}
	\item 40. léta -- nová petice s 3 mil. podpisy
	\item chartismus postupně upadal -- chudí lidé se soustředili spíše na sociální než na politické otázky
	\end{itemize}
\item \textbf{Trade unions} -- sdružovaly pracující v jednotlivé oblasti -- první odborová unie na světě
\end{itemize}

\paragraph{Irsko}
\begin{itemize}
\item konec 20. let -- zákon o zrovnoprávnění katolíků s protestanty
\item boj za autonomii a zrovnoprávnění (oranžisté -- protestanti)
\item hlavní představitel hnutí Daniel O`Conell 
\item hnutí je součástí hnutí Mladá Evropa
\item 40. léta -- hladomor (v důsledku nákazy brambor) \ra utlumení snah, smrt 1/3 obyvatelstva, emigrace do USA
\item \lp{1848}{pokus o povstání} (Mladé Irsko)
\end{itemize}

\paragraph{Východní otázka}
\begin{itemize}
\item Rusko se snaží kontrolovat plavbu na Černém moři, přes Bospor a Dardaneli, do Středozemního moře
\item propagováno jako ochrana pravoslavných států
\item Balkán je strategicky důležitý \ra Francie, Velká Británe nechtějí, aby ho Rusko ovládalo
\item Rakousko má zájem o Podunají -- dopravní tepna, ústí do Černého moře
\item[\ra] rozpory mezi velmocemi \ra rozpad Svaté aliance, pentarchie, rovnováhy sil
\end{itemize}

\section{Nové politické směry 19. století (Vznik kapitalistické společnosti)}
\paragraph{Liberalismus}
\begin{itemize}
\item rovná občasnká práva
\item svobodný občan, svoboda slova, svoboda podnikání, svoboda vyznání
\item ekonomika nemá být omezována státem
\item stát chrání občana, majetek
\end{itemize}

\paragraph{Ekonomický liberalismus}
\begin{itemize}
\item \textit{"laissez-faire, laissez-passer"} -- nechte to běžet
\item zásada volné soutěže, svobodného trhu -- stát nezasahuje do ekonomiky
\item  stát nezasahuje ani do vztahu mezi zaměstnavatelem a zaměstnancem (neexistují pravidla, pojištění, \ldots \ra špatné zacházení)
\end{itemize}

\paragraph{Demokratické hnutí}
\begin{itemize}
\item oddělilo se od liberalismu
\item vychází z \textbf{Jeana Jacquese Rousseaua}
\item každý občan bez rozdílu na majetku má mít podíl na politickém rozhodování \ra všeobecné volební právo
\item sociální reformy
\item demokracie má vyřešit sociální problémy -- člověk neschopný pracovat je společnosti na obtíž
\item stát se má více starat o lidi \ra samospráva na řešení problémů, podpora řemeslníků, dělníku
\end{itemize}

\paragraph{Utopický socialismus}
\begin{itemize}
\item \textit{utopos} -- nikde
\item \textbf{Claude Henri de Saint-Simon}
	\begin{itemize}
	\item záchranou společnosti je kolektivní vlastnictví
	\item všichni lidé mají pracovat
	\item založeno na křesťanské lásce k bližnímu \ra nikdo nebude nikoho vykořisťovat
	\end{itemize}
\item \textbf{Charles Fourier}
	\begin{itemize}
	\item obchodník \ra navrhoval sdružování do družstev "falang"
	\item rozdělování rovným dílem zisku z prodeje družstva
	\item každý člověk by měl pracovat, nepracující člověk je na obtíž společnosti
	\end{itemize}
\item \textbf{Robert Owe}n 
	\begin{itemize}
	\item dělníci strádají, podnikatelé s nimi pohrdají
	\item člověk = výtvorem společenského prostředí
	\item podnikatel -- vlastnil mnoho továren
	\item nutná industrializace
	\item sociální zabezpečení 
		\begin{itemize}
		\item budování škol pro dělníky
		\item důchod
		\item odstranil práci dětí
		\item otevírány čítárny pro dělníky
		\end{itemize}
	\item zkrachoval -- ostatní podnikatelé od něj nenakupovali, protože nechtěli dělat stejná opatření pro své zaměstnance
	\end{itemize}
\item \textbf{František Matouš Klácel}
	\begin{itemize}
	\item mnich v augustiniánském klášteře v Brně
	\item přítel mnoha významných osobností (B. Němcová)
	\item je možné žít v křesťanském společenství, kde bude majetek spravedlivě rozdělen
	\item pokusil se o založení této společnosti v Americe
	\end{itemize}
\item význam
	\begin{itemize}
	\item analýza příčin sociální nerovnosti -- kde se bere vykořisťování
	\item liberalismus sám o sobě nemůže vytvořit spravedlivou společnost
	\item navázaly další programy
	\end{itemize}
\end{itemize}

\paragraph{Francouzští socialisté}
\begin{itemize}
\item \textbf{Piere Joseph Proudhon} \ra proudhonisté
\item \textbf{August Blanqui}
	\begin{itemize}
	\item výborný řečník, nelegální názory \ra většina života ve vězení
	\end{itemize}
\item program
	\begin{itemize}
	\item vycházel u utopických socialistů
	\item velké vlastnictví = krádež
	\item výrobci se sdružují \ra stát je společný
	\item industrializace je nutností
	\item společnost má kontrolovat, regulovat a rozdělovat výrobu 
	\item revoluce musí proběhnout v podobě násilného převratu v čele s intelektuály
	\end{itemize}
\end{itemize}

\paragraph{Vědecký socialismus (Marxismus)}
\begin{itemize}
\item \textbf{Karl Marx}
	\begin{itemize}
	\item syn advokáta \ra univerzitní vzdělání, měl převzat praxi
	\item na univerzitě vyučován Hegelem, převzal jeho filosofii
	\end{itemize}
\item \textbf{Fridrich Engles}
	\begin{itemize}
	\item syn obchodníka, jako mladý poslán na zkušenou do Anglie
	\item poznal chudé dělníky \ra zájem o sociální problémy
	\item odešel do Francie, potkal Marxe \ra přátelé
	\item zdědil otcův majetek \ra možnost propagace
	\end{itemize}
\item inspirace v utopickém socialismu, anglické politické ekonomii, G. W. Hegel
	\begin{itemize}
	\item Hegel -- společnost se pořád dokola vrací do stejného stavu
	\end{itemize}
\item \lp{1848}{Manifest komunistické strany}
	\begin{itemize}
	\item dějiny se vyvíjejí tak, že proti sobě bojují vždy dvě sociální třídy (buržoazie x proletariát) \ra třídní boj posouvá společnost
	\item revoluce nesmí být spiknutí, musí ji podporovat všichni \ra seznamují proletariát s právy, aby věděli co mohou udělat
	\item společenské vlastnictví výrobních prostředků
	\end{itemize}
\item po neúspěšné revoluci 1848/49 umírněnější názory
\item vliv na mezinárodní organizace dělníků \ra zakládání sociálně demokratických organizací
	\begin{itemize}
	\item potřeba spojit sílu dělníků
	\item \lp{1864}{I. internacionála v Londýně}
	\item \lp{1889}{II. internacionála v Paříži}
	\item \lp{1919}{III. internacionála v Moskvě (Kominterna)}
		\begin{itemize}
		\item sdružovala bolševické strany, ne sociální demokraty
		\item rozpad 1939 -- podepsání smlouvy s Německem
		\end{itemize}
	\item socialistická internacionála
		\begin{itemize}
		\item sdružuje socialistické strany (Labour Party)
		\end{itemize}
	\end{itemize}
\end{itemize}

\paragraph{Anarchisté}
\begin{itemize}
\item An arche -- nevláda
\item \textbf{Michail Bakunin}
\item stát je zlo, jedině jeho odstraněním budou lidé osvobozeni
\item snaha o dosažení sociální spravedlnosti násilím na jednotlivcích
\item sdružování do malých obcí dobrovolné federace -- komuny
\item krajní křídlo -- politické atentáty
\end{itemize}

\paragraph{Nový názor na národ}
\begin{itemize}
\item pospolitost rovných občanství
\item společná minulost
\item národní zájmy
\item každý vlastenec slouží národu
\item \textbf{Johan Gottfried Herder}
	\begin{itemize}
	\item pro národ je důležitý jazyk, nejen historie a lidová kultura
	\end{itemize}
\item vlastenectví se místy vyvíjí v nacionalismus a šovinismus
\item \textbf{Národní hnutí} -- Čechy, Morava, Slovensko, Uhersko, Norsko, Chorvatsko
\end{itemize}

\section{Průmyslová revoluce}
\paragraph{}
\begin{itemize}
\item James Watt -- parní stroj, do té doby stroje poháněny vodními koly
\item vyšší nároky na suroviny -- nutnost kvalitního železa \ra nutnost kvalitního uhlí \ra ... 
	\begin{itemize}
	\item ovlivňování mezi průmyslovými oblastmi, počátek v textilním průmyslu
	\end{itemize}
\item rozvoj kolejové dopravy
\item Anglie 1825 -- \textbf{George Stephenson}
\item vynález telegrafu
\end{itemize}

\paragraph{Kolejová doprava u nás}
\begin{itemize}
\item koňská dráha
\item \textbf{František Gerstner}
\item \lp{1832}{České Budějovice -- Linec} (kvůli přepravě soli)
\item \lp{1833}{Praha -- Lány -- Plzeň}
\item koňská síla je nedostatečná \ra parní lokomotivy
\item Nutnost propojení Rakouska s Haličem
\item \lp{1839}{Vídeň \ra Brno} (později  \ra Olomouc \ra Opava)
\item \lp{1845}{Olomouc \ra Praha}
	\begin{itemize}
	\item postavil český inženýr Jan Perner \ra první česká průmyslová škola v Praze
	\end{itemize}
\item \lp{1851}{Praha \ra Drážďany} (první mezinárodní trať)
\end{itemize}

\paragraph{Význam zavádění strojů}
\begin{itemize}
\item anglické nebo jejich ukradené modely 
\item Belgie, Francie, Porýní, Sasko, české země
\item pol. 19. stol. \ra růst koupěschopnosti obyvatelstva
\item růst spotřeby textilu (Brno, Liberec)
	\begin{itemize}
	\item v Brně první parní stroj
	\end{itemize}
\item dělníci měli pocit, že jim stroje berou práci \ra hnutí dělníků, rozbíjení strojů
\item do r. 1846 vznik 62 hutí
\item podnikatel -- dělník; svobodný trh pracovní sály
\item zakladatelské období
	\begin{itemize}
	\item neexistují pravidla pro pracovníky -- dlouhá pracovní doba atd.
	\item neexistují pravidla konkurence
	\end{itemize}
\end{itemize}

\paragraph{Kritika sociálních křivd}
\begin{itemize}
\item stát chce zákonodárství omezující vykořisťování
\item křesťanská sdružení
\item dělnická sdružení
\item konzervativci proti liberálům
	\begin{itemize}
	\item chtěli zpět k moci, upozorňovaly na díry v liberálním programu
	\item chrání zájmy tradiční malovýroby, která je zastíněna továrnami
	\end{itemize}
\item radikální demokraté, socialisté
\end{itemize}

\paragraph{Protirobotní povstání}
\begin{itemize}
\item i přes existenci továren ve městech stále robota
\item \lp{1821}{jižní Morava}	
	\begin{itemize}
	\item potlačeno mírně, bez fyzických trestů
	\end{itemize}
\item \lp{1831}{východní slovensko}
\item Hospodářské změny \ra politické změny po roce 1848/9
\end{itemize}

\section{České národní obrození (60./70. léta 18. stol -- 1. pol. 19. stol.)}
\begin{itemize}
\item proces přeměny středověké společnosti na moderní
\item později probíhá též moravské obrození
\item utlačování habsburskou monarchií \ra snaha o obnovu českého jazyka 
\item \textbf{národ} -- skupina lidí spojená tradicí, historií a jazykem (Johan Gottfried Herder)
\item Rakouské císařství = mnohonárodní stát
	\begin{itemize}
	\item 37 mil. obyvatel \ra Slované 17, Němci 8, Maďaři 5, Rumuni+Italové 5
	\item Slované se v této době považují za jeden národ
	\end{itemize}

\end{itemize}

\paragraph{Struktura české společnosti}
\begin{itemize}
\item oslabeno státoprávní postavení Království českého
	\begin{itemize}
	\item snaha Habsburků minimalizovat autonomní české státní orgány
	\end{itemize}
\item chybí měšťanstvo, inteligence
	\begin{itemize}
	\item odchod kvůli bitvě na Bílé hoře
	\end{itemize}
\item šlechta -- Němci \ra národně odcizená

\item germanizace nižších vrstev
\item výhody
	\begin{itemize}
	\item na venkově se Čeština udržela
	\item na triviálních školách vyučováno v češtině
	\item vyspělá česká literatura z doby předbělohorské 
		\begin{itemize}
		\item některé spisy nebyly zničeny, protože neobsahovaly proticírkevní myšlenky
		\end{itemize}
	\item u nás národní obrození splývá s obnovou jazyka
	\end{itemize}
\end{itemize}

\subsection{První fáze}
\begin{itemize}
\item osvícenství
	\begin{itemize}
	\item vědecký zájem o studium jazyka, zvyků, kultury, dějin
	\end{itemize}
\item panslavismus -- myšlenka společného slovanského národa
\item zemský patriotismus -- vlastenectví je vázáno k místu, kde mám majetek
\item nová česká inteligence vzniká z lidových vrstev -- kněží, učitelé
\item prestižní kněžské povolání, po vystudování přidělena fara \ra existenční jistota
\item kněží jsou významní \ra ovlivňují lidi kázáními 
\item \lp{1775}{\textbf{Bohuslav Balbín: Rozprava na obranu jazyka slovanského, zvláště pak českého}}
\item \lp{1783}{\textbf{K. Ignác Thám: Obrana jazyka českého}}
\item \textbf{Gelasius Dobner}
	\begin{itemize}
	\item zakladatel historiografie
	\item kritizoval nepřesnost Hájkovy kroniky
	\item neviděl význam předbělohorské češtiny
	\end{itemize}
\item \textbf{František Martin Pelcl}
	\begin{itemize}
	\item \lp{1793}{katedra českého jazyka a literatury na Karlo-Ferdinandově univerzitě}
	\item[\ra] učena čeština a učeno v češtině
	\item prvním profesorem
	\end{itemize}
\item Morava: \textbf{Jan Petr Cerroni} -- prozkoumal Moravsko slezské historické prameny a uspořádal je
\end{itemize}

\paragraph{Josef Dobrovský} (1753--1829) 
\begin{itemize}
\item osvícenský kriticismus
	\begin{itemize}
	\item legenda o smrti Jana Nepomuckého (debata s Dobnerem o jeho smrti) 
	\item evangelium sv. Marka v korunovačních klenotech -- zjistil, že není napsáno sv. Markem, ale pochází z 6 století
	\end{itemize}
\item \lp{1791}{veřejně formuloval program českého národního obrození}
	\begin{itemize}
	\item požadovali nepotlačování češtiny jako úředního jazyka, vyučování česky ve školách
	\end{itemize}
\end{itemize}

\paragraph{Žurnalistika}
\begin{itemize}
\item Václav Matěj Kramerius
	\begin{itemize}
	\item Nakladatelství Česká expedice (1790)
	\item Krameriovy c.k. noviny
	\end{itemize}
\end{itemize}

\paragraph{České divadlo}
\begin{itemize}
\item Nosticovo \ra Stavovské divadlo
	\begin{itemize}
	\item Otevřeno operou Don Giovanni
	\end{itemize}
\item Václav Thám
	\begin{itemize}
	\item Vlastenecké divadlo \ra "Bouda"
	\item U Hybernů (řád irských mnichů)
	\item Břetislav a Jitka
	\end{itemize}
\end{itemize}

\subsection{Druhá fáze (poč. 19. stol.--30. let. 19. stol.)}
\begin{itemize}
\item silně ovlivněna napoleonskými válkami \ra panslavismus
\item romantismus
\item snaha o odražení germanizace 
\item 1. novodobý český národně kulturní program 
\end{itemize}

\paragraph{Josef Jungman}
\begin{itemize}
\item jazyk = nejdůležitější znak národa
\item snaha dokázat, že čeština je rovnocenná s ostatními jazyky
\item Slovník česko-německý, Historie literatury české, Slovesnost (mluvnice)
\end{itemize}

\paragraph{Přírodní a humanitní vědy}
\begin{itemize}
\item A. Marek -- filosofická terminologie
\item Jan Sv. Presl -- chemická terminologie
\item Karel Bořivoj Presl -- botanická terminologie
\item Josef Vojtěch Sedláček  -- matfyz terminologie
\item Jan Evangelista Purkyně -- zkoumal buňku
\end{itemize}

\paragraph{RKZ}
\begin{itemize}
\item \lp{1817}{Rukopis královédvorský}
\item \lp{1818}{Rukopis zelenodvorský}
\item Josef Linda, Václav Hanka
\item nepravost prokázána až v 20. století
\end{itemize}

\paragraph{Vlastenecké museum v Čechách (1818)}
\begin{itemize}
\item hr. Fr. Ant. Kolovrat
\item F. Palacký -- ČČM
	\begin{itemize}
	\item Staří letopisové čeští
	\item Idea státu rakouského
	\end{itemize}
\item P. J. Šafařík
\item hr. Kašpar Šternberk
\end{itemize}

\paragraph{Národní museum v Brně (1818)}
\begin{itemize}
\item hrabě Antonín Mitrovský
\item A. Boček (další falzifikáty)
\end{itemize}


\subsection{Třetí fáze}
\begin{itemize}
\item vliv polského povstání \ra politické myšlení
\item nová generace
	\begin{itemize}
	\item František Palacký, Karel Havlíček Borovský, Karel Hynek Mácha, Karel Sabina, Fratišek Rieger
	\end{itemize}
\item na Moravě
	\begin{itemize}
	\item F. M. Klácel
	\item F. C. Kampelík -- zakládal spořitelny
	\item J. Helcelet -- filosof, Olomouc
	\item J. Ohéral -- 
	\end{itemize}
\end{itemize}

\paragraph{František Palacký} (1798--1876)
\begin{itemize}
\item Dějiny národu českého v Čechách a v Moravě
	\begin{itemize}
	\item přehodnotil husitství z kacířského hnutí \ra první velká reformace, vyzdvihuje hodnoty
	\end{itemize}
\item příprava 1. českého politického programu -- austroslavismus
	\begin{itemize}
	\item Karel Havlíček Borovský
	\item podstatou je sjednocení slovanů pod Rakouskem 
	\item[\ra] federace, slované chtějí rovná práva (úměrná podle obyvatelstva)
	\end{itemize}
\end{itemize}

\paragraph{Národní instituce}
\begin{itemize}
\item \lp{1831}{Matice česká}

\item \lp{1833}{Jednota pro povzbuzení průmyslu v Čechách} -- cílem otevřít českou průmyslovou školu 
	\begin{itemize}
	\item Rieger, Trojan, Perner
	\end{itemize}
\item \lp{1845}{Měšťanská beseda}
	\begin{itemize}
	\item Rieger, Trojan
	\end{itemize}
\item \lp{1853}{Matice moravská}
\end{itemize}

\paragraph{Divadlo}
\begin{itemize}
\item neexistují stálá divadla \ra kočovné společnosti a ochotnické divadlo
\item \textbf{J. K. Tyl} (1808--1856) -- Fidlovačka (Hudba Fr. Škroup)
\item \textbf{Václav Thám}
\item cílem probudit české vlastenectví, zachovat češtinu
\end{itemize}

\paragraph{Žurnalistika}
\begin{itemize}
\item Pražské noviny -- Kramerius \ra Čelakovský \ra Borovský
	\begin{itemize}
	\item Česká včela
	\end{itemize}
\item Čechoslav (Langer) -- austroslavismus
\item Jindy a nyní \ra Květy české (Tyl)
\end{itemize}

\paragraph{Radikálové}
\begin{itemize}
\item nesouhlasili s nadvládou Habsburské monarchie, chtěli vlastní vládu
\item spolek \textbf{Repeal} -- inspirováno Iry
\item E. Arnold
\item K. Sabina
\item F. C. Kampelík
\end{itemize}

\section{Slovenské národní obrození (1780--2. pol. 19. stol.)}
\subsection{První období (1780--1820)}
\paragraph{Katolíci}
\begin{itemize}
\item vytvořit spisovný jazyk
\item Generání seminář -- soustředění v Prešpurku (Bratislava)
\item Michal Kratochvíl, Anton Bernolák \ra bernoláčtina -- společný jazyk, příliš rozdílný na východě a západě
\item bernoláčtina -- jazyk se neujal, ale bylo v něm napsáno několik her
\item Ján Hollý, Ján Chalupka
\end{itemize}

\paragraph{Evangelíci}
\begin{itemize}
\item chtějí používat předbělohorskou češtinu \ra bibličtina (Bible kralická)
\item centrem lyceum v Prešpurku
\item představitelé Juraj Ribay, Juraj Palkovič
\item Ján Kollár (1793-1852) -- Slávy dcera
\item Pavel Josef Šafařík  (1795--1861) -- Slovanské starožitnosti (popisuje význam Slovanů při vytváření kultury)
\end{itemize}

\subsection{Druhé období (1820--pol. 19. stol.)}
\begin{itemize}
\item nová generace na Bratislavském lyceu
\item L. Štúr, M. M. Hodža, J. M. Hurban
\item studentské spolky (Kežmarok, Levoča, Prešov, Nitra)
\item \lp{1843}{Štúr} -- spisovný jazyk
\item \lp{1844}{spolek Tatrín} v Liptovském Mikuláši
\item od 1847 Štúr poslancem (za Zvolen) uherského zemského sněmu
\end{itemize}

\paragraph{Radikální demokraté}
\begin{itemize}
\item \textbf{Ján Francisci}
\item \textbf{Janko Král} -- slovenský básník
\item program -- zrušení poddanství
\item umění -- Malíři: Peter Bohúň, Jozef Božetěch Klemens
\end{itemize}

\newpage
\section{Revoluce (1848--1849)}
\begin{itemize}
\item série revolucí, které zasáhly celou Evropu
\item vznik občanské společnosti, programy národní emancipace (zrovnoprávnění)
\end{itemize}

\subsection{Itálie}
\begin{itemize}
\item do teď rozdrobená, z ekonomických a národních důvodů by bylo lepší se sjednotit \ra "risorgimento"
\item \lp{12. 1. 1848}{bouře na Sicílii a v Neapoli}
	\begin{itemize}
	\item cílem získat nezávislost od Bourbonů
	\item Ferdinand II. Bourbonský -- ústava (vychází z francouzské 1830)
	\end{itemize}
\item[\ra] vzrůst napětí v Lombardii a Benátsku
\item \lp{17. 3. 1848}{povstání v Miláně a v Benátkách}
\item do čela sjednocení se postavil sardinský král \textbf{Karel Albert} (savojská dynastie -- jediná domácí)
\item \lp{24. 3. 1848}{vyhlašuje válku Rakousku}
\item \lp{25. 7. 1848}{porážka u Custozzy}
\item \lp{léto 1848}{demokraté se dostávají do čela \ra republika}
\item \lp{leden 1849}{Římská republika}
\item \lp{1849}{nová válka sardinského krále} proti Rakousku
	\begin{itemize}
	\item \lp{březen 1849}{poražen u Novarry} \ra 
	\end{itemize}
\item rakouská intervence v Toskánsku
	\begin{itemize}
	\item neapolská vojska \ra Sicílie
	\item francouzský expediční sbor \ra Římská republika
	\end{itemize}
\item \lp{srpen 1849}{poražena republika sv. Marka}
\item revoluce, ač neúspěšná, byla významná tím, že rozšířila mezi obyvatelstvo myšlenku jednoty a občanských svobod
\end{itemize}


\subsection{Francie}
\begin{itemize}
\item 40. léta opozice proti \textbf{Ludvíku Filipovi} a vládě (od 39 špatná ekonomická situace)
	\begin{itemize}
	\item předseda vlády \textbf{F. P. Guizota}
	\end{itemize}
\item liberálové, demokraté, socialisté
	\begin{itemize}
	\item liberálové jediní legální
	\item pořádali veřejná shromáždění -- bankety
	\end{itemize}
\end{itemize}

\paragraph{Únorová revoluce}
\begin{itemize}
\item bankety bouřlivé \ra zakázány
\item \lp{22. 2. 1848}{bouře}
\item \lp{24. 2.}{barikády v Paříži}
	\begin{itemize}
	\item[\ra]Ludvík Filip abdikoval
	\end{itemize}
\item vytvořena prozatímní vláda
	\begin{itemize}
	\item \lp{25. 2. 1848}{vyhlášena 2. republika}
	\item nová ústava, než byla hotova, vydávány dekrety
		\begin{itemize}
		\item všeobecné volební právo
		\item sdružovací právo (negace Le Chapelierova zákona)
		\item právo na práci \ra národní dílny 
			\begin{itemize}
			\item v národních dílnách vyráběny jednoduché nekvalitní výrobky nekvalifikovanými lidmi
			\item nízký výdělek \ra zvýšeny daně rolníkům
			\end{itemize}
		\end{itemize}
	\end{itemize}
\item \lp{duben 1848}{volby}
	\begin{itemize}
	\item vítězství liberálních republikánů
	\item květen -- socialisté pokus o převrat
	\item \lp{23. 6. 1848}{zrušeny sociální dílny}
	\item \lp{do 26. 6.}{živelné povstání}
		\begin{itemize}
		\item tvrdě potlačeno po třech dnech
		\item ovlivnilo práci na ústavě \ra 
		\end{itemize}
	\end{itemize}
\end{itemize}

\paragraph{Cesta k 2. císařství}
\begin{itemize}
\item \lp{4. 11. 1848}{nová ústava} -- vytvořen úřad prezidenta
\item \lp{10. 12. 1848}{prezidentem Ludvík Napoleon Bonaparte}
\item \lp{jaro 1849} parlamentní volby
	\begin{itemize}
	\item vítězí strana pořádku (liberálové)
	\item omezeno hlasovací právo
	\item omezeny výsledky únorové revoluce 
	\end{itemize}
\item \lp{2. 12. 1851}{státní převrat}
\item \lp{2. 12. 1852}{Napoleon se nechal prohlásit císařem (\ra  III.)}
\end{itemize}

\subsection{Německo}
\begin{itemize}
\item Německo je rozdrobené na malé státečky a města, napětí od r. 1847
\item \lp{březen 1848}{bouře v Porýní}
\item \lp{18.3.}{povstání v Berlíně}
	\begin{itemize}
	\item němečtí liberálové, chtěli do vlády
	\item proti absolutismu
	\item pro sjednocení Německa, zrušení roboty, \ldots
	\end{itemize}
\item \lp{19.3.}{Fridrich Vilém IV. slíbil ústavu}
	\begin{itemize}
	\item revoluce zuří po celé Evropě \ra nemá kde vzít podporu na potlačení
	\item ustoupil požadavkům
	\end{itemize}
\item \lp{březen 1848}{předfrankfurtský sněm}
	\begin{itemize}
	\item organizace velkého pangermánského sněmu
	\end{itemize}
\item \lp{18. 5. 1848}{Pangermánský sněm ve Frankfurtu}
	\begin{itemize}
	\item velkoněmecká koncepce -- sjednotit všechny německy mluvící země (nelíbí se Habsburkům)
	\item maloněmecká koncepce -- sjednocení Německa v přibližně dnešních hranicích
	\item liberálové x demokraté
	\end{itemize}
\item \lp{březen 1849}{nová ústava}
	\begin{itemize}
	\item titul císaře nabídnut Fridrichu Vilémovi IV.
	\item zrušeno poddanství
	\end{itemize}
\item výsledky a význam
	\begin{itemize}
	\item \lp{jaro 1849}{poslední pokusy o povstání v Porýní, Drážďanech}
	\item \lp{18. 6. 1849}{pruské vojsko rozehnalo frankfurtský sněm}
	\item Prusko -- \textbf{oktrojovaná ústava}
		\begin{itemize}
		\item "vnucená" -- vypracována na základě panovníka
		\item \textbf{zrušení roboty} za výkup
		\item někde zrušena cenzura
		\item zostření vztahů s Rakouskem
		\end{itemize}
	\end{itemize}
\end{itemize}

\subsection{Habsburská monarchie}
\begin{itemize}
\item vliv Itálie a Francie
\item důvody
	\begin{itemize}
	\item \lp{11. 3. 1848}{veřejné lidové shromáždění ve Svatováclavských lázních}
	\item F. A. Brauner -- významný český politik, liberál
	\end{itemize}
\item petice císaři
	\begin{itemize}
	\item Země Koruny České -- samostatný celek, sněm v Praze
	\item rovnoprávnost češtiny
	\item občanské svobody
	\item zrušení roboty
	\item zřízení národních gard
	\item \textbf{Svatováclavský výbor}
	\end{itemize}
\item Kabinetní list (8. 4. 1848)
	\begin{itemize}
	\item odpověď až na třetí petici
	\item slibuje splnit hlavní body
	\end{itemize}
\item \lp{13. 3. 1848}{povstání ve Vídni}
\item Metternich odstoupil
\item sliby císaře Ferdinanda I.
\item rakouští liberálové -- přípravy na frankfurtský sněm
\item \lp{10. 4. 1848}{vznikl Národní výbor -- liberálové + konzervativci}
\item \lp{11. 4. 1848}{Palackého psaní do Frankfurtu} -- argumentoval, proč by Česko nemělo být součástí Německa
\item \lp{květen 1848}{kníže Alfred Windischgrätz} -- velitel rakouské armády -- potlačoval povstání, provokoval
\end{itemize}

\paragraph{Slovanský sjezd}
\begin{itemize}
\item \lp{2. 6. 1848}{začal v Praze}
\item slovanská vzájemnost
\item společný přístup
\item Šafařík, Bakunin, Zach, Palacký, K. Havlíček, \ldots...
\item diferenciace národů
\item \textbf{Manifest k národům evropským}
\item austroslavismus -- Slované mají dostat práva podle svého počtu, ale mají zůstat pod ochranou 
\end{itemize}

\paragraph{Červnové povstání, Svatodušní bouře}
\begin{itemize}
\item \lp{12. 6. 1848}{mše zbratření -- Koňský trh}
\item lidé po cestě z mše ucpali ulice \ra označeno Windischgrätzem za nedovolené shromáždění \ra vojensky potlačeno
\item srážky na Celetné
\item barikády
	\begin{itemize}
	\item nedostatek zbraní, střeliva, materiálu, potravin
	\end{itemize}
\item \lp{16. 6. 1848}{dělostřelecký útok}
\item \lp{17. 6. 1848}{kapitulace}
\item Praha -- stav obležení
\item \lp{10. 7. 1848}{svolán ústavodárný říšský sněm ve Vídni}
\item \lp{7. 9. 1848}{zrušení roboty za náhradu}
\item \lp{říjen 1848}{nové bouře v Uhrách a Vídni}
\item \lp{1. 11. 1848}{konec revoluce v Předlitavsku}
\item \lp{22. 11. 1848}{pokračuje sněm v Kroměříži}
\end{itemize}

\paragraph{František Josef I.}
\begin{itemize}
\item korunován v osmnácti letech, inteligentní, pracovitý, dal na rady zkušenějších
\item \lp{4. 3. 1849}{Stadionova oktrojovaná ústava} -- hrabě Stadion
	\begin{itemize}
	\item ministerstva, vláda
	\item 
	\end{itemize}
\item \lp{7. 3. 1849}{rozehnán kroměřížský sněm} -- nepotřebný, když už ústava existuje
\item \lp{1849}{májové spiknutí}
\item radikálové K. Sabina, E. Arnold
\end{itemize}

\subsection{Uhersko}
\begin{itemize}
\item nelíbí se jim centralizace -- řízení z Vídně
\item od 1713 -- odpor
\item demokraté + liberálové
\item revolucionář Lajos Kossuth
\item \lp{3. 3. 1848}{uherský sněm v Prešpurku}
\item \lp{17. 3. 1848}{císař uznal Uherskou vládu}
\item \lp{18. 3. 1848}{zrušeny feudální výsady, poddanství, obč. svobody, centralizace}
\item Maďaři se stavěli proti centralizaci, ale neuznávali Slováky, Chorvaty, Rumuny, Srby
\item \lp{11. 4. 1848}{uherská ústava}
	\begin{itemize}
	\item na území Uherska žije jen Maďarský jazyk, ostatní jsou jen jazykové skupiny \ra povstání
	\end{itemize}
\item \lp{duben 1848}{Nitranské žiadosti}
	\begin{itemize}
	\item žádost o uznání Slováků jako vlastní národ (zatím umírněné)
	\item Josef Miloslav Hurban
	\end{itemize}
\item \lp{10. 5. 1848}{Žiadosti slovenského národa}
	\begin{itemize}
	\item požadovali jazykovou autonomii a tvorbu vlastní samosprávy pro Horní uhry (Slovensko ještě neexistuje)
	\item reakce Uherska -- vydání zatykače na Hurbana a Štura
	\end{itemize}
\item národnostních rozporů využívá vídeňská vláda
	\begin{itemize}
	\item chorvatská armáda
	\item dobrovolnická armáda na Slovensku
	\item[\ra] oslabení Maďarů
	\item když byly armády moc silné \ra vydán příkaz k navrácení dobrovolníků
	\end{itemize}
\end{itemize}

\paragraph{Konec revoluce}
\begin{itemize}
\item \lp{14. 4. 1849}{uherský sněm v Debrecíně}
	\begin{itemize}
	\item sesazeni Habsburkové 
	\item vyhlášena samostatnost
	\end{itemize}
\item \lp{13. 8. 1849}{porážka u Világose}
	\begin{itemize}
	\item porážka revoluce s pomocí ruského vojska (Svatá aliance)
	\end{itemize}
\item Sandor Petöfi -- zakladatel Maďarské poezie, zajat, vězněn na Sibiři
\end{itemize}

\paragraph{Výsledky a význam}
\begin{itemize}
\item odstraněny feudální výsady
\item zrušeno poddanství
\item částečná občanská práva
\item politické programy
\item získání zkušenosti 
\item změna struktury společnosti hospodářská, sociální
	\begin{itemize}
	\item uvolněno podnikání
	\item nejdůležitější společenskou vrstvou je buržoazie
	\end{itemize}
\end{itemize}

\section{Velká Británie -- vedoucí světová mocnost}
\paragraph{Soupeření toryů a whigů}
\begin{itemize}
\item konzervativci (vznik z toryů)
	\begin{itemize}
	\item zemědělci -- ochrana před dovozem levného obilí
	\end{itemize}
\item liberálové (z whigů)
	\begin{itemize}
	\item obchodníci -- chtějí dovážet obilí 
	\end{itemize}
\item \lp{1846}{předseda Robert Peel} -- konzervativec, ale hlasoval s liberály, protože dovoz obilí je ekonomicky prospěšný \ra převaha
\end{itemize}

\paragraph{Vláda Whigů}
\begin{itemize}
\item premiér Henry John Palmerston: aktivní zahraniční politika
\item koloniální výboje do Indie
	\begin{itemize}
	\item východoindická společnost
	\item zotročování obyvatelstva, etc.
	\item povstání Sikhů, Sipahiů \ra východoindická společnost zrušena
	\end{itemize}
\item zrušen úřad místokrále
\item \lp{1876}{Indie císařstvím} pod vládou Anglické královny
\item \lp{později kolonizována i Barma}
\end{itemize}

\paragraph{Expanze do Číny}
\begin{itemize}
\item \lp{1840--42}{1. opiová válka}
\begin{itemize}
	\item Čína musela otevřít 5 přístavů 
	\item Hongkong pod Britskou kontrolou (\ra později navrácen \ra momentálně problémy, protože lidé jsou zvyklí na britskou demokracii)
\end{itemize}
\item \lp{1865--1858}{2. opiová válka}
	\begin{itemize}
	\item účast Francie \ra více otevřených přístavů
	\end{itemize}
\item \lp{1859--1860}{3. opiová válka} 
	\begin{itemize}
	\item Britští vojáci v Pekingu \ra kapitulace Číny \ra politika otevřených dveří
	\end{itemize}
\end{itemize}

\paragraph{60. léta 19. století}
\begin{itemize}
\item \textbf{Benjamin Disraeli} -- konzervativec
\item \textbf{William Gladstone} -- liberál
\item \lp{1865}{střet -- boj o volební reformu}
	\begin{itemize}
	\item volební právo rozšířeno na majitele nemovitostí
	\item změna volebních okrsků
	\end{itemize}
\end{itemize}

\paragraph{Vláda liberálů}
\begin{itemize}
\item předsedou vlády \textbf{W. Gladstone}
\item reformy: 
	\begin{itemize}
	\item omezení privilegií anglikánské církve
	\item rozšířeno vzdělání
	\item odbory se staly politickým partnerem
	\item zákaz stávek
	\item zrušena dovozní cla
	\end{itemize}
\item program volného trhu (i s lidskou pracovní silou \ra odmítání práv pro zaměstnance)
\end{itemize}

\paragraph{Vláda konzervativců}
\begin{itemize}
\item \lp{1874--1880}{první období}
\item \lp{1886--1892}{druhé období}
\item předseda -- \textbf{Benjamin Disraeli}
\item sociální zákonodárství \ra vyhráli volby 74
	\begin{itemize}
	\item max. prac. doba 56h/týden
	qitem zdravotní péče
	\end{itemize}
\item výbojná politika
	\begin{itemize}
	\item zisk Kypru
	\item Egypr, Barma, Malajso -- politicky závislé
	\item Afrika 1899--1902
		\begin{itemize}
		\item búrské války
		\end{itemize}
	\end{itemize}
\end{itemize}

\paragraph{Dominia}
\begin{itemize}
\item kolonie nechtěly být koloniemi \ra dominia
\item Kanada, Jihoafrická unie (těžba zlata a diamantů), Austrálie, Nový Zéland
\end{itemize}

\paragraph{Polarizace politického života}
\begin{itemize}
\item \lp{1893}{Nezávislá dělnická strana}
\item \lp{1900}{Výbor pro dělnické zastoupení}
\item \lp{1906}{Labour Party}
\item \lp{1906}{vláda liberálů}
	\begin{itemize}
	\item premiér -- David Lloyd George
	\end{itemize}
\end{itemize}

\paragraph{Fabiáni}
\begin{itemize}
\item vzdělanci
\item evoluce \ra socialismus
\item G. B. Shaw (Pygmalion), H. G. Wells (Válka světů) -- významní spisovatelé
\end{itemize}

\paragraph{Feniáni}
\begin{itemize}
\item podle Irské teroristické organizace Sinn Féin
\item Ulster
\item od 60. let násilí k prosazení
\end{itemize}

\paragraph{Viktoriánská Anglie}
\begin{itemize}
\item královna Viktorie (*1819, 1837--1901)
\item 1876 -- císařovna indická
\item manžel Albert I. Sachsen Coburg-Gotha
\item 9 dětí
\item za druhé světové války přejmenována dynastie na Windsorskou
\item sídlem Buckinghamský palác
\item kulturní centrum Albert Hall
\end{itemize}

\section{Mezinárodní vztahy v 2. pol. 19. století}
\subsection{Itálie}
\paragraph{Sjednocení Itálie (1870)}
\begin{itemize}
\item \lp{po roce 1849}{8 větších států}
\item Sardinské království (Sardinie + Piemont)
\item \textbf{Viktor Emanuel II.} -- Savojská dynastie
	\begin{itemize}
	\item chtěl sjednotit Itálii
	\end{itemize}
\item \lp{1852}{předseda vlády -- hrabě \textbf{Cavour}}
	\begin{itemize}
	\item snaha o diplomatické sjednocení
	\item[\ra] spojenectví s Francií za Savojsko a Nizzu
	\end{itemize}
\end{itemize}

\paragraph{1859}
\begin{itemize}
\item červen -- bitva u Magenty a Solferina
\item srpen -- připojeno Toskánsko, Modena, Parma, Romagna
\item Giuseppe Garibaldi
\item listopad -- Napoleon III. -- mír s Rakouskem 
	\begin{itemize}
	\item Sardinie se stává soupeřem \ra připojena pouze Lombardie
	\end{itemize}
\end{itemize}


\paragraph{1860}
\begin{itemize}
\item jaro -- povstání na Sicílii proti Bourbonům
\item květen -- Garibaldi -- výprava tisíce 
	\begin{itemize}
	\item porážka krále 
	\item[\ra] plebiscit \ra připojení k Itálii
	\end{itemize}
\end{itemize}

\paragraph{Italské království}
\begin{itemize}
\item \lp{březen 1861}{svolán všeitalský parlament v Turíně}
\item králem Viktor Emanuel \ra I.
\item hlavní město: Florencie
\end{itemize}

\paragraph{\lp{1866}{Pruskorakouská válka}}
\begin{itemize}
\item Itálie spojencem Pruska
\item Custozza -- porážka Itálie
\item \lp{1866}{pražský mír} -- připojeno Benátsko
\item \lp{1870}{Francie poražena Pruskem}
	\begin{itemize}
	\item Francouzská armáda nezvládá bránit Papežský stát
	\end{itemize}
\item \lp{září 1870}{Papežský stát připojen k Itálie}
	\begin{itemize}
	\item papeži zůstává území Vatikánu
	\end{itemize}
\item \lp{1871}{hlavním městem Řím}
\end{itemize}

\subsection{Německo}
\paragraph{Sjednocení Německa 1871}
\begin{itemize}
\item prudký rozvoj průmyslu
\item rozdrobenost je překážkou podnikání
\item Prusko -- nejsilnější stát
	\begin{itemize}
	\item Junkeři -- nejvýznamnější vrstva
	\end{itemize}
\item \textbf{Vilém I. Hohenzollern}
	\begin{itemize}
	\item 
	\end{itemize}
\item \lp{1862}{\textbf{Otto von Bismarck} předsedou vlády}
\end{itemize}

\paragraph{Sjednocení "krví a železem"}
\begin{itemize}
\item \lp{1863--64}{válka s Dánskem}
	\begin{itemize}
	\item Šlesvicko Prusku
	\item Holštýnsko Rakousku
	\end{itemize}
\item \lp{1866}{Bismarck prohlásil Holštýnsko za pruské}
\item \lp{1866}{prusko-rakouská válka}
	\begin{itemize}
	\item spojenectví s Itálií
	\item \lp{3. 7. 1866}{bitva u Hradce králové (Sadové)}
	\item Pražský mír: Rakouska ztratilo vliv v Německu
	\end{itemize}
\end{itemize}

\paragraph{\lp{1866}{Severoněmecký spolek}}
\begin{itemize}
\item Prusko + 21 států
\item \lp{duben 1867}{vypracována ústava Říšským sněmem} -- všeobecné volební právo, císařství
\item Francie proti sjednocení Německa
\item záminka pro válku -- španělský trůn
\end{itemize}

\paragraph{\lp{1870--1871}{prusko-francouzská válka}}
\begin{itemize}
\item \lp{19. 7. 1870}{vyhlásil Napoleon III. válku}
\item \lp{2. 9. 1870}{Francouzi poraženi u Sedanu}
\item \lp{27. 10. 1870}{druhá polovina Francouzů poražena u Metz}
\item \lp{18. 1. 1871}{vyhlášeno Německé císařství ve Versailles}
\end{itemize}

\paragraph{Německé císařství}
\begin{itemize}
\item spolkový stát -- 22 monarchií + 3 svobodná města + Alsasko Lotrinsko
\item císařem \textbf{Vilém I.}
\item kancléřem \textbf{Bismarck}
\item \lp{10. 5. 1871}{mír ve Frankfurtu}
	\begin{itemize}
	\item Francie musí platit velké reparace
	\end{itemize}
\end{itemize}

\paragraph{jaro 1871}
\begin{itemize}
\item volby do ŘS
\item nová ústava
	\begin{itemize}
	\item ŘS jen moc zákonodárná
	\item kancléř -- pravomoci
	\item císař -- silná moc
	\item spolková rada - omezená moc (zástupci států)
	\end{itemize}
\item[\ra] Německo se stává velmocí -- rozvoj průmyslu, budování železnic, rozvoj těžby
\end{itemize}

\paragraph{60. a 70. léta}
\begin{itemize}
\item konjunktura -- období maximální výroby, kdy je na trhu hodně výrobků, hodně lidí pracuje, vysoká koupěschopnost
\item konjunktura se střídá s krizí z nadvýroby \ra deprese krize \ra obnovení výroby \ra konjunktura
\item průmysl
\item Porúří
\item růst počtu dělníků
\item vliv Marxismu
\item \lp{1869}{Sociálně demokratická dělnická strana Německa}
	\begin{itemize}
	\item Eisenašský (Eisenach) program
	\item hájí zájmy pracujících, sociální zabezpečení, \ldots
	\end{itemize}	 
\end{itemize}

\paragraph{Bismarckova politika}
\begin{itemize}
\item proti katolické církvi
	\begin{itemize}
	\item Kulturkampf 
	\item \lp{1871}{zákaz zasahování do politiky}
	\end{itemize}
\item proti sociálně demokratickému hnutí
	\begin{itemize}
	\item \lp{1878}{zákon o zakázání socialistické strany}
	\item politika cukru a biče
	\item zavedl první sociální zákony
	\end{itemize}
\item sociální zákony
	\begin{itemize}
	\item zbezpečení ve stáří
	\item při úrazu
	\item zlepšení pracovních podmínek
	\item zkrácení pracovní doby (12h denně)
	\end{itemize}
\item vliv sociálních demokratů narůstal
\item \lp{1889}{stávka v Porúří}
\item \lp{1890}{\textbf{Bismarck} odvolán \textbf{Vilémem II.}}
\end{itemize}

\paragraph{Sociální demokracie}
\begin{itemize}
\item povolena činnost
\item \lp{1891}{volby} -- 20%
\item růst vlivu
\item diferenciace	
	\begin{itemize}
	\item revizionisté -- chtěli opravit marxistické učení podle vývoje světa; \textbf{E. Bernstein}
	\item radikálové (Spartakovci) -- nechtěli nic měnit, chtěli revoluci proletariátu; \textbf{Rosa Luxemburková}
	\end{itemize}
\item \lp{1905}{vlna stávek}
\end{itemize}

\paragraph{Bismarckova zahraniční politika}
\begin{itemize}
\item tradiční spojení Rakousko-Uhersko, Rusko
	\begin{itemize}
	\item v 70. letech balkánská krize -- střed zájmů těchto mocností
	\end{itemize}
\item Afrika -- první kolonie
	\begin{itemize}
	\item dovoz surovin, odbyt zboží
	\end{itemize}
\item 80. léta -- buduje se námořnictvo
\item růst napětí s Anglií a Francií
\item přední východ -- Rusko
\end{itemize}

\subsection{Francie}
\paragraph{2. císařství}
\begin{itemize}
\item \lp{1851}{převrat \textbf{Ludvíka Bonaparta}}
\item \lp{1852}{císařství}
\item nová ústava
\item 60. léta -- liberalizace režimu \ra opozice: republikáni, monarchisté
\item hospodářský růst
\item \lp{1860}{změna celních předpisů} \ra potíže
\item \textbf{Napoleon III.} povolil zakládat nepolitické dělnické spolky
\item přestavba Paříže -- aby se snížila nezaměstnanost
	\begin{itemize}
	\item vznik opery, typických Pařížských širokých ulic
	\end{itemize}
\item \lp{60. léta}{agitace socialistů a anarchistů}
\end{itemize}

\paragraph{Koloniální výboje}
\begin{itemize}
\item \lp{1853}{Krymská válka}
\item Suezský průplav (Ferdinand Lesseps; 1869)
\item 1859 -- podpora Itálie
\item snaha proniknout do Alžíru, Indočíny
\item Mexiko: 1861 císařství (1861)
	\begin{itemize}
	\item arcivévoda Maxmilián Habsburský
	\item \lp{1867}{popraven}
	\end{itemize}
\end{itemize}

\paragraph{Vztah k Prusku}
\begin{itemize}
\item Bismarck vyprovokoval Napoleona III.
\item \lp{19. 7. 1870}{zahájení války}
\item \lp{2. 9. 1870}{Sedan} -- Napoleon zajat
\item \lp{4. 9. 1870}{revoluce \ra 3. republika}
\item prozatímní vláda -- neuzavřela příměří s Německem
\item národní gardy, partyzáni
\item \lp{27. 10. 1870}{porážka u Metz}
\item pruská vojska obklíčila Paříž
\item 18. 1. 1871 -- vstup Pruských vojsk do Paříže
	\begin{itemize}
	\item Prusové vyhlásili císařství ve Versailles
	\end{itemize}
\item nová vláda -- Thiers
\item \lp{28. 1. 1871}{příměří}
\item Bismarck vyprovokoval vznik Pařížské komuny
\end{itemize}

\paragraph{Pařížská komuna}
\begin{itemize}
\item vláda chce odzbrojit nárdní gardy
\item v noci \lp{17./18. 3. 1871}{nařízeno vládním jednotkám odzbrojení gard}
	\begin{itemize}
	\item vládní jednotky poraženy, někteří vůdcové popraveni
	\item vznik komunny -- stát ve státu
	\end{itemize}
\item \lp{26. 3. 1781}{volby do Rady komuny}
	\begin{itemize}
	\item orgán vyjadřující demokratické názory občanů
	\item vydává dekrety
		\begin{itemize}
		\item národní garda nahradí policii a  armádu
		\item odluka státu od církve
		\item zavedení volených a sesaditelných úředníků
		\item zavedení sociálních opatření ve prospěch pracujících
		\item řešení bytové politiky -- byty uprchlých bohatých předány chudým
		\item spoluúčast dělníků při vedení podniku
		\item bezplatné vyučování
		\item vyhlášení rovnoprávnosti žen
		\end{itemize}
	\end{itemize}
\end{itemize}

\paragraph{Obrana Paříže}
\begin{itemize}
\item na komunu útočí pruské vojsko a vládní vojsko
\item \lp{10. 5. 1871}{mír ve Frankfurtu}
	\begin{itemize}
	\item odstoupení Alsaska-Lotrinska
	\item reparace 5. mld. zl. franků
	\item pruská armáda opouští Francii \ra vládní armáda může zaútočit na Paříž
		\begin{itemize}
		\item májový týden 21.--28. 5. 1871
		\item represálie -- 30 000 mrtvých
		\item vojenské soudy
		\item ohlas v Evropě -- propagace demokratické myšlenky, 
		\end{itemize}
	\end{itemize}
\end{itemize}

\paragraph{Stabilizace republiky}
\begin{itemize}
\item snahy monarchistů -- prezident Mac-Mahon
\item \lp{1875}{nová ústava třetí republiky}
	\begin{itemize}
	\item republika 
	\item všeobecné hlasovací právo pro muže
	\item bezplatné vzdělávání
	\item silná prezidentská moc
	\item obnovení občanských svobody
	\end{itemize}
\end{itemize}

\paragraph{Politické stany}
\begin{itemize}
\item radikální strana
	\begin{itemize}
	\item odkaz VFR
	\item odluka státu od církve
	\item omezení pravomocí prezidenta
	\item \textbf{Georges Clémenceau} [klemanso]
	\end{itemize}
\item socialisté
	\begin{itemize}
	\item blanquisté, marxisté, etc.
	\item \lp{1905}{socialisté sjednoceni}
	\item \textbf{Jean Jaurés} [žoré]
		\begin{itemize}
		\item není důležitá národnost a ostatní věci, je důležité získat práva \ra za tuto myšlenku zabit
		\end{itemize}
	\end{itemize}
\end{itemize}

\paragraph{Politické a korupční aféry}
\begin{itemize}
\item \lp{90. léta 19. století}{Dreyfusova aféra}
	\begin{itemize}
	\item Dreyfus obviněn ze špionáže \ra vyhnán 
	\item zapojen Emile Zola \ra odjel do Anglie
	\item později očištěn
	\end{itemize}
\item aféra s akciemi Panamského průplavu
	\begin{itemize}
	\item společnost na stavbu průplavu zkrachovala \ra lidé přišli o investované peníze
	\item stavbu převzala Anglie (založila stát Panama)
	\end{itemize}
\end{itemize}

\paragraph{Volby 1902}
\begin{itemize}
\item koalice Levý blok (umírnění socialisté + Radikálové)
\item Radikální strana je nejvlivnější do 1914
\item antiklerikalismus
\item protiněmecká orientace
\item sblížení s V. B. a Itálií
\end{itemize}






\section{Naše země v habsburské monarchii (1849--1867)}
\subsection{Období státního absolutismu (1852--1861)}
\begin{itemize}
\item v čele státu stojí panovník, který vládne prostřednictvím vlády a ministerstev
\item \textbf{František Josef I.} (1848--1916)
\item \lp{1849}{oktrojovaná ústava}
\item \lp{31.12.1851}{silvestrovské patenty}
	\begin{itemize}
	\item rušily oktrojovanou ústavu
	\end{itemize}
\item nová ústava vydána až 1861 \ra do té doby státní absolutismus
\item ministerský předseda \textbf{Felix Schwarzenberg}
\item cílem udržet celistvost státu a centralizovat moc v rukou císaře
\item ministr vnitra \textbf{Alexandr Bach} -- období nesprávně nazýváno Bachovský absolutismus
\end{itemize}

\paragraph{Reformy státní správy, soudnictví, berní}
\begin{itemize}
\item správa -- město (obec) \ra okres (hejtman) \ra kraj \ra místodržitelství (Praha, Brno, Opava) \ra ministerstvo vnitra
\item soudnictví -- okresní soud \ra krajský \ra zemský \ra vrchní zemské \ra nejvyšší soud \ra ministerstvo spravedlnosti
\item berní -- obdobně, ministerstvo financí
\end{itemize}

\paragraph{Rozvoj podnikání}
\begin{itemize}
\item \lp{7. 9. 1848}{zrušení roboty a poddanství}
\item \lp{1859}{Živnostenský řád} -- zrušeny cech, kodifikace svobodného podnikání	
\end{itemize}

\paragraph{Průmyslový rozvoj}
\begin{itemize}
\item nejrychlejší rozvoj v českých zemích
\item textilní průmysl -- Brno
\item potravinářství -- lihovarnictví, cukrovarnictví, pivovarnictví
\item strojírenství, železářství
	\begin{itemize}
	\item Vítkovická huť -- 1826 (do 2. sv. v. největší ve stř. Evropě)
	\item Ringhofferova strojírna 
	\item Daňkova strojírna 
	\item Pražská železářská společnost
	\end{itemize}
\item těžba uhlí
\end{itemize}

\paragraph{Národní a sociální problémy}
\begin{itemize}
\item policie četnictvo
\item pronásledováni Palacký, K. Havlíček Borovský -- vyhnán do Brixenu
\item mezinárodní problémy \ra změny
	\begin{itemize}
	\item červen 1859 -- porážka Rakouska u Magenty a Solferina
	\item \lp{listopad 1859}{ztráta Lombardie}
	\end{itemize}
\item za problémy viněn Bach \ra odvolán
\end{itemize}

\paragraph{Konstituce}
\begin{itemize}
\item \lp{20. 10. 1860}{Říjnový diplom}
\item \lp{26. 2. 1861}{únorová "Schmerlingova" ústava}
	\begin{itemize}
	\item výkonná -- panovník a vláda
	\item zákonodárná -- Říšská rada
		\begin{itemize}
		\item horní sněmovna -- šlechta a čestní členové (Palacký)
		\item dolní sněmovna -- 
		\end{itemize}
	\item volební cenzus -- 10zl.
	\item 4 volební kurie
	\end{itemize}
\item \lp{1863--1879}{pasivní opozice}
\end{itemize}

\paragraph{Kulturní a spolkový život po roce 1860}
\begin{itemize}
\item časopis Čas  -- první časopis s politickým programem liberálů
\item Národní listy -- politické články
\item Riegerův naučný slovník 
\item Sokol (1862, Fünger, Tyrš)
\item Prozatímní divadlo
\item umělecká beseda: B. Smetana, J. Mánes, K. Purkyně, J. Čermák, V. Hálek, J. Hlávka aj.
\end{itemize}

\paragraph{Politická diferenciace v 60. letech 19. stol.}
\begin{itemize}
\item Česká národní strana
	\begin{itemize}
	\item staročeši -- Palacký, Brauner, Rieger
	\item mladočeši -- radikální, moderní názory -- K. Sladkovský, E. Vávra, J. Grégr, J. Arbes, J. V. Sládek
	\end{itemize}
\end{itemize}

\paragraph{Rakousko-uherské království}
\begin{itemize}
\item prusho-rakouská válka (1866)
	\begin{itemize}
	\item 3. 7. bitva u Sadové
	\item ztráta Benátska
	\item jednání vlády s Uhry
	\end{itemize}
	
\item \lp{7. 2. 1867}{Rakousko-Uhersko}
	\begin{itemize}
	\item rovnocenné státy spojené personální unií
	\item společné: panovník, armáda, finance, zahraniční politika
	\item Uhersko -- ústava z r. 1848
		\begin{itemize}
		\item jeden uherský národ, ostatní jsou jen jazykové skupiny
		\end{itemize}
	\item Předlitavsko -- \lp{21. 12. 1867}{prosincová ústava}
		\begin{itemize}
		\item 
		\end{itemize}		 
	\end{itemize}
\end{itemize}

\paragraph{Pokus o české vyrovnání}
\begin{itemize}
\item \lp{1867}{zklamání}
\item \lp{1868--71}{tábory lidu}
\item \lp{1870--71}{prusko-francouzská válka}
	\begin{itemize}
	\item Rakousko na francouzské straně
	\item jednání vlády s českou opozicí (Palacký, Rieger, Pražák)
	\end{itemize}
\item \lp{10.10.1871}{český zemský sněm přijal 18 Fundamentálních článků}
	\begin{itemize}
	\item schváleno i MZS
	\item nevyšlo v platnost kvůli Rakouské vládě
	\item vystupovali proti nim rakouští Němci, Maďaři (aby nechtěli i Slováci a Chorvati)
	\end{itemize}
\item \lp{1874}{Národní strana svobodomyslná}
	\begin{itemize}
	\item Julius Grégr
	\item mladočeši
	\end{itemize}
\end{itemize}

\paragraph{Zahraniční politika}
\begin{itemize}
\item \lp{1878}{berlínský kongres}
	\begin{itemize}
	\item RU dostalo do správy Bosnu a Hercegovinu
	\end{itemize}
\end{itemize}

\paragraph{Drobečková politika}
\begin{itemize}
\item premiér - hr. Eduard Taaffe (1879--1893)
\item reformy:
	\begin{itemize}
	\item snížen census na 5 zl.
	\item omezení prac. doby
	\end{itemize}
\item drobečkové reformy
	\begin{itemize}
	\item úřad musí odpovídat i česky
	\item Alois Pražák členem vlády -- ministr bez úřadu
	\item \lp{1882}{rozdělení univerzity Karlo-Ferdinandovy} na českou a německou část
	\end{itemize}
\end{itemize}

\paragraph{Politické snahy T. G. Masaryka}
\begin{itemize}
\item 3 politický subjekt mezi staročechy a mladočechy \ra realisté
\end{itemize}

\paragraph{Punktace}
\begin{itemize}
\item staročeši utrpěli smrtí Palackého
\item \lp{1890}{jednání staročechů s německými liberály}
\item němečtí liberálové chtějí německé a německočeské okresy
\item proti mladočeši, veřejnost
\item 1891 -- volby \ra vítězí mladočeši
\item konec staročeské strany
\end{itemize}

\paragraph{Ekonomický vývoj a sociální otázka}
\begin{itemize}
\item po 1848 a vydání živnostenského řádu -- průmyslový rozvoj, růst počtu obyvatel \ra emigrace
\item odchod do Německa a USA -- pracovní příležitosti
\item rolníci odchod do Ruska a Ukrajiny
\item \lp{1873--1879}{1. krize z nadvýroby}
\item Brno
	\begin{itemize}
	\item 140000 obyvatel
	\item	320 podniků
\item moravský Manchester
	\end{itemize}
\end{itemize}

\paragraph{Modernizace v technologii}
\begin{itemize}
\item ing. V. Kaplan 
\item E. Kolben -- elektromotory Kolben Daněk
\item F. Křižík -- oblouková lampa, první elektrárna, elektrifikace Národního divadla
\item ing. J. Kašpar -- letectví
\item 1897 Kopřivnice; President
\item \lp{1906}{Mladá Boleslav} -- Laurin a Klement
\item \lp{1890}{Plzeň -- Emil Škoda}
\item \lp{1868}{Živnostenská banka} (nejstarší česká) 
\end{itemize}

\paragraph{Dělnické organizace}
\begin{itemize}
\item \lp{60. léta}{odborové svazy}
\item svépomocný spolek Oul; dr. F. Chleborad
\item \lp{1874}{Neudörfl -- vznik Sociálně demokratické strany Rakouska} -- Pecka, Zápotocký
\item \lp{1878}{Českoslovanská sociálně demokratická strana dělnická (Břevnov)}
	\begin{itemize}
	\item represe po pařížské komuně \ra rozpad
	\item předseda Viktor Adler -- revoluce je transformace společnosti do vyšších vrstev
	\end{itemize}
\item Josef Hybeš
\item \lp{1888/89}{Heifneld}{českoslovanský slučovací sjezd}
	\begin{itemize}
	\item spojit se pro zlepšení sociální situace
	\end{itemize}
\end{itemize}

\paragraph{Boj o politické reformy}
\begin{itemize}
\item cíl = všeobecné volební právo
\item \lp{1893}{protirakouské demostrace}
\item pokrokové hnutí Omladina -- A. Rašín, K. Neuman
\item \lp{1894}{neprávem odsouzeni k vězení}

\paragraph{Boj za všeobecné volební právo}
\begin{itemize}
\item \lp{1893}{demonstrace}
\item \lp{1896}{Kazimír Badeni}
\end{itemize}

\paragraph{Nové politické směry}
\begin{itemize}
\item \lp{1891}{Rerum novarum} (papež Lev XIII.)
	\begin{itemize}
	\item výzvy papeže, aby se vyjadřovala církev k novým věcem a zakládala křesťanské strany 
	\end{itemize}
\item \lp{1894}{Křesťansko sociální strana} (Jan Šrámek) (->KDU-ČSL)
\item \lp{1896}{Česká sociálně demokratická strana dělnická}
\item \lp{1898}{Národně sociální strana} (Václav Klofáč)
	\begin{itemize}
\item	narozdíl od socialistické uznává rozdíli mezi národy
	\end{itemize}
\item \lp{1899}{Agrární strana} (Alfons Šťastný, Antonín Švehla)
	\begin{itemize}
	\item majetní obyvatelé venkova
	\end{itemize}
\item \lp{1900}{Česká strana lidová} (realisté) (T. G. M., F. Drtina, Jos. Gruber)
\end{itemize}
\paragraph{Morava}
\begin{itemize}
\item \textbf{Lidová strana} -- zemská org. mladočeské strany
	\begin{itemize}
	\item Adolf Stránský -- Lidové noviny
	\end{itemize}
\item Agrární strana
\item katolické strany
\item jednání mezi Čechy a Němci
\item \lp{1905}{moravský pakt} (moravské vyrovnání)
\end{itemize}

\paragraph{Boj za rovné volební právo}
\begin{itemize}
\item vrchol od října do listopadu 1905
\item demonstrace, manifestace
\item 28. 11. 1905 -- generální stávka
\item vláda -- příslib reforem
\item 26. 1. 1907 -- volební reforma Maxe von Becka
\item všeobecné, rovné, tajné, přímé
\item jen muži,ne vojáci, ne bez trvalého bydliště delšího než 1 rok
\end{itemize}

\paragraph{Politické strany na počátku 20. století}
\begin{itemize}
\item politická orientace podle:
	\begin{itemize}
	\item souhlas s balkánskou politikou
	\item spojenectví s Německem
	\item mladočeši
	\item K. Kramář, A. Rašín, J. Preis
	\item neopanslavismus
	\item česká a moravská agrární strana
	\end{itemize}
\item 1906 Česká strana lidová \ra Česká strana pokroková (T. G. Masaryk)
\item 1908 Česká strana státoprávně pokroková (K. S. Sokol) -- jako první chce osamostatnit české země 
\item klerikální strany -- pod státní kontrolou
\item 
\end{itemize}

\paragraph{Slovensko}
\begin{itemize}
\item 1874 -- zrušena slovenská gymnázia, Matice slovenská
\item maďarizace
\item 1898 Slovenská nírodní strana
	\begin{itemize}
	\item měsíčník Hlas -- dr. Vavro Šrobár -- Slováci mají používat češtinu 
	\end{itemize}
\item  1903 Slovenská ľudová strana
	\begin{itemize}
	\item Andrej Hlinka, Ferdiš Juriga
	\end{itemize}
\item Černová (27. 10. 1907)
	\begin{itemize}
	\item národní nepokoje
	\item svěcení kostela, nechtěli pustit vojáky
	\end{itemize}
\end{itemize}
\end{itemize}




\section{Rusko v 2. polovině 19. století}
\paragraph{Vláda Mikuláše I. (1825--1855)}
\begin{itemize}
\item snaha o izolaci Ruska
\item omezený přístup ke vzdělání (gymnázium jen pro děti úředníků)
\item tajná policie
\item opozice:
	\begin{itemize}
	\item slavjanofilové -- chtěli vrádu cara s radou
	\item západníci -- problémem nevolnictví, absolutismus
	\end{itemize}
\end{itemize}

\paragraph{Krymská válka (1853--1856)}
\begin{itemize}
\item Rusko chce kontrolovat Černé moře
\item \lp{1853}{ruské vojsko obsadilo Moldavsko a Valašsko, zničena Turecká flotila}
\item \lp{1854}{anglicko-francouzská flotila na Krymu}
	\begin{itemize}
	\item A+F hájí Turecko
	\item 13 měsíců obléhání Sevastopolu
	\end{itemize}
\item porážka Ruska
\end{itemize}

\paragraph{Pařížský mír}
\begin{itemize}
\item březen 1856
\item ztráty Ruska
	\begin{itemize}
	\item 
	\end{itemize}
\end{itemize}

\paragraph{Sociální a ekonomické problémy}
\begin{itemize}
\item slabá železniční síť, špatná komunikace (není telefonní síť)
\item ruští vojáci byli nevolníci \ra vzpoury, negramotnost
	\begin{itemize}
	\item útěky na Krym
	\end{itemize}
\item[\ra] nutnost Reforem
\end{itemize}

\paragraph{Alexandr II. (1855--1881)}
\begin{itemize}
\item \lp{1861}{manifest o zrušení nevolnictví}
\item opozice
	\begin{itemize}
	\item narodnici
		\begin{itemize}
		\item umírnění -- potřeba vzdělávat obyvatelstvo
		\item radikálové -- 
		\end{itemize}
		
	\end{itemize}
\item z radikálních narodniků se se oddělila teroristická frakce s cílem odstranit cara \ra \lp{1881}{zavražděn}
\end{itemize}

\paragraph{Diferenciace politických názorů}
\begin{itemize}
\item ruský liberalismus -- z umírněných narodniků
\item marxistický socialismus
\item \lp{1903}{rozdělení}
	\begin{itemize}
	\item menševici -- soc. demokracie
	\item bolševici
	\item 1912 -- definitivní rozdělení (v Praze)
	\end{itemize}
\end{itemize}

\paragraph{Industrializace}
\begin{itemize}
\item zahraniční kapitál hlavně A, F (malá konkurence, levná pracovní síla)
	\begin{itemize}
	\item železnice
	\item železářství
	\item textilní průmysl
	\end{itemize}
\item růst počtu dělníků
\item projevuje se ve městech -- Petrohrad, Moskva, Riga, Varšava, Lodž
\end{itemize}

\paragraph{Expanze}
\begin{itemize}
\item Kavkaz
\item stř. Asie
\item Afghánistán -- střet s Brity
\item obavy z pronikání Britů do Číny
\end{itemize}

\paragraph{Rusko-Japonská válka (1904--1905)}
\begin{itemize}
\item Japonci chtějí ovládnout Mandžusko
\item únor 1904 Japonci napadají Port Artur
\item Rusko má špatnou armádu
\item březen 1905 Japonci vyhrávají u Mukudenu, květen u Cušimi
\end{itemize}

\paragraph{Revoluce (1905--1907)}
\begin{itemize}
\item požadavek občanských práv
\item konec samoděržaví
\item 22. 1. 1905 -- krvavá neděle v Petrohradu
	\begin{itemize}
	\item vojsko během dvou hodin zabilo tisíc lidí
	\end{itemize}
\item 27. 6. 1905 -- vzpoura na křižníku Potěmkin (dělníci dostávali špatné jídlo)
\item říjen -- celoruská válka
	\begin{itemize}
	\item 8. hodinová pracovní doba
	\item demokratické svobody
	\item svoláno ústavodárné shromáždění
	\item požadavkům bylo ustoupeno
	\end{itemize}
\end{itemize}

\paragraph{Pokus bolševiků}
\begin{itemize}
\item V. I. Uljanov Lenin 
\item prosinec 1905 -- Moskva
\item 3. 6. 1907 -- diskuze o ústavě trvala příliš dlouha \ra rozpuštěna Duma
	\begin{itemize}
	\item některé svobody ponechány
	\item hospodářské reformy
	\item
	\item 
	\end{itemize}
\end{itemize}



\newpage
\timeline

\end{document}