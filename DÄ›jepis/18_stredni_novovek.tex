\title{Střední novověk}
\documentclass[10pt,a4paper]{article}
\usepackage[utf8]{inputenc}
\usepackage[czech]{babel}
\usepackage{amsmath}
\usepackage{amsfonts}
\usepackage{amssymb}
\usepackage{chemfig}
\usepackage{geometry}
\usepackage{wrapfig}
\usepackage{graphicx}
\usepackage{floatflt}
\usepackage{hyperref}
\usepackage{fancyhdr}
\usepackage{tabularx}
\usepackage{makecell}
\usepackage{csquotes}
\usepackage{footnote}
\usepackage{movie15}
\MakeOuterQuote{"}

\renewcommand{\labelitemii}{$\circ$}
\renewcommand{\labelitemiii}{--}
\newcommand{\ra}{$\rightarrow$ }
\newcommand{\x}{$\times$ }
\newcommand{\lp}[2]{#1 -- #2}
\newcommand{\timeline}{\input{timeline}}


\geometry{lmargin = 0.8in, rmargin = 0.8in, tmargin = 0.8in, bmargin = 0.8in}
\date{\today}
\author{Jakub Rádl}

\makeatletter
\let\thetitle\@title
\let\theauthor\@author
\makeatother

\hypersetup{
colorlinks=true,
linkcolor=black,
urlcolor=cyan,
}



\begin{document}
\maketitle
\tableofcontents
\begin{figure}[b]
Toto dílo \textit{\thetitle} podléhá licenci Creative Commons \href{https://creativecommons.org/licenses/by-nc/4.0/}{CC BY-NC 4.0}.\\ (creativecommons.org/licenses/by-nc/4.0/)
\end{figure}
\newpage

\section{Napoleon a Evropa}
\subsection{Životopis}
\begin{itemize}
\item Napoleon Bonaparte (*1769 Ajaccio -- 1821 sv. Helena)
\item chudá rodina + výhodné podmínky \ra studium na vojenské škole
\item "jsem člověk, jsem občan"
\item Desire Claryová -- snoubenka, později švédská královna (vdána za generála, který se stal králem)
\item Josefina Beauharnaisová -- nemohla mu dát syna
\item Marie Luisa -- syn "Orlík"
\end{itemize}

\subsection{Období konzulátu (1799--1804)}
\begin{itemize}
\item Napoleon konzulem
\item inteligentní \ra reformovaná státní správa
\item \lp{1799}{4. francouzská ústava}
	\begin{itemize}
	\item zachovává rovnost lidí před zákonem
	\item nezachovává volební rovnost
	\item svoboda podnikání, nedotknutelnost existujícího majetku
	\item zákaz návratu emigrantů, zkonfiskovaný majetek je nenavratitelný
	\end{itemize}
\item \lp{1801}{Konkordát s papežem} (dohoda mezi církevní a světskou institucí)
	\begin{itemize}
	\item církvi se nevrací žádný majetek ani privilegia
	\item kněží se mohou vrátit do far a sloužit mše
	\end{itemize}
\item \lp{1804}{Code Napoleon}
	\begin{itemize}
	\item opraven občanský zákoník Code Civil
	\end{itemize}
\end{itemize}

\paragraph{Válečné úspěchy}
\begin{itemize}
\item \lp{1800}{porážka Rakušanů u Marenga} -- absolutní porážka
\item \lp{1801}{mír Francie s Rakouskem v Lunévill}
	\begin{itemize}
	\item západní břeh Rýna je Francouzský
	\end{itemize}
\item \lp{1802}{mír Francie s Británií v Amiesnu}
	\begin{itemize}
	\item Británie hledá nového spojence proti Francii
	\item[\ra] vnější ohrožení Francie
	\end{itemize}
\end{itemize}

\paragraph{\lp{1804}{Napoleon císařem}}
\begin{itemize}
\item jednota před napadením
\item vládce, který je dobrý vojevůdce
\item kvůli svému původu podporoval chudé
\item korunovace schválená senátem
\item korunoval sám sebe a svou manželku
\end{itemize}

\subsection{Výbojné války}
\begin{itemize}
\item k dispozici pozemní i námořní armáda
\item \lp{20. 10. 1805}{Pruská armáda poražena u Ulmu}
\item \lp{21. 10. 1805}{Prohra u Trafalgaru}
	\begin{itemize}
	\item Britové (Nelson) zničili 3/4 francouzského loďstva
	\end{itemize}
\item Napoleon pokračuje do Rakouska
\item \lp{2. 12. 1805}{bitva tří císařů u Slavkova}
	\begin{itemize}
	\item Francie (císař Napoleon) vs. Rusko (car Alexandr) + Rakousko (císař František)
	\item Napoleon prozkoumal bojiště a vymyslel strategii, druhá strana ne
	\item na památku postavena Mohyla Míru -- Alois Sova
	\item do dnes se na polích dají najít pozůstatky vojáků
	\item mír podepsán u Slavkova
	\end{itemize}
\end{itemize}

\paragraph{1806}
\begin{itemize}
\item \lp{1806}{sjednocena Itálie} (kromě Sardinie a Vatikánu) -- do vlády dosazeni sourozenci
\item \lp{1806}{porážka Pruska} = konec Svaté říše římské
	\begin{itemize}
	\item císař František II. \ra I. rakouský (už od 1804)
	\end{itemize}
\item \lp{1806}{vytvořen Rýnský spolek}
\item \lp{1806}{kontinentální blokáda}
	\begin{itemize}
	\item cílem ekonomicky zničit Anglii 
	\item nahradit zboží Anglické francouzským
	\item Francouzské zboží bylo příliš luxusní a drahé a bylo ho málo
	\item[\ra] rozvoj domácí výroby (Brno významným městem textilního průmyslu), národních trhů, národní ekonomiky
	\end{itemize}
\end{itemize}

\paragraph{1807}
\begin{itemize}
\item Tylžský mír s Alexandrem I.
\item Velkovévodství varšavské (na úkor Pruska) -- závislé na Napoleonovi
\end{itemize}

\paragraph{Vnitřní politika}
\begin{itemize}
\item Joseph Fouché -- velitel tajné policie
	\begin{itemize}
	\item kontroloval kdo ohrožuje Napoleonovu vládu
	\item porušoval 4. ústavu
	\end{itemize}
\end{itemize}

\paragraph{1808}
\begin{itemize}
\item Napoleon napadl Španělsko
	\begin{itemize}
	\item Francouzi se chovali nadřazeně, drancovali vesnice
	\end{itemize}
\item bratr Joseph Bonapart jmenován Španělským králem (nikdy ho neovládl celé)
\item guerilla -- partyzánská válka (vzpoura kvůli Francouzskému chování)
\item Francisco Goya -- cyklus obrazů Hrůzy války
\end{itemize}

\paragraph{1809}
\begin{itemize}
\item Anglie na pomoc Španělsku
\item Francouzi poraženi ve Španělsku (Murat x Wllington)
\item Rakousko vyhlašuje novou válku
\item červenec -- bitva u Wagramu => ztráta Haliče a Terstu
\item kníže K. L. Metternich 
	\begin{itemize}
	\item od 1809 min zahraničí
	\item 1821 kancléřem
	\end{itemize}
\end{itemize}

\paragraph{1812}
\begin{itemize}
\item Rusko se zřeklo blokády
\item[\ra] tažení do Ruska -- Grande Armée (600 000)
	\begin{itemize}
	\item z Francie i žoldáci z Polska
	\end{itemize}
\end{itemize}

\paragraph{Průběh tažení}
\begin{itemize}
\item Ruský generál Michal Illarinovič Kutuzukov ustupoval až k Borodinu
\item popisováno ve Vojně a Míru
\item taktika spálené země 
\item 7. 9. 1812 bitva u Borodina
\item velké ztráty na obou stranách \ra další ústup
\item Francouzi nebyli vybaveni na Ruské podmínky
\item Napoleon byl tlačen spálenou zemí k ústupu, armáda rozdělena na oddíly
\item armáda byla pronásledována Ruskou armádou a partyzány
\item list. 1812 bitva na Berezině 
\item zbylo 40 000 mužů z 600 000
\end{itemize}

\paragraph{1813}
\begin{itemize}
\item Francouzi vyhnáni ze španělska
\item 19. 10. bitva národů u Lipska
\item vznik nové koalice Prusko, Rusko, Rakousko, Švédsko proti Francii
	\begin{itemize}
	\item švédský král (bývalý francouzký generál) znal Napoleonovu taktiku
	\end{itemize}
\end{itemize}

\paragraph{1814--1815}
\begin{itemize}
\item březen 1814 -- výtězné armády vstoupily do Paříže
\item Napoleon zajat a poslán na Elbu
\item stodenní císařství 20. 3. -- 18. 6. 1815
	\begin{itemize}
	\item Napoleon se vylodil na pobřeží Francie
	\item mladí francouzští vojáci mu byli věrní
	\item 18. 6. poražen u Waterloo  ( generál Blücher + Wellington)
	\item Napoleon deportován na sv. Helenu
	\item 1821 zemřel (možná otráven)
	\end{itemize}
\end{itemize}



\section{Vídeňský kongres}
\begin{itemize}
\item 1814 -- 1815
\item iniciován Metternichem
\item pro Vídeň ekonomicky prospěšné -- hromada šlechticů s celým dvorem
\item cíle
	\begin{itemize}
	\item Napoleon zničil řadu monarchií a rodů
	\item potřeba uznání legitimnosti rodů a nároku na jejich restauraci
	\end{itemize}
\item zásada rovnoprávnosti
\item nutnost potrestání Francie
\end{itemize}

\paragraph{Výsledky}
\begin{itemize}
\item uzavřen Pařížský mír s Francií
	\begin{itemize}
	\item hranice vráceny do roku 1792 -- před zahájením výbojných válek
	\item 3 roky okupace východofrancouzských pevností
	\item Ludvík XVIII. králem
	\item válečné reparace 700 milionů zlatých franků -- spláceno několik let
	\item ztráta kolonií
	\end{itemize}
\item Anglické územní zisky
	\begin{itemize}
	\item části francouzských a holandských kolonií
	\item Helgoland (Strategické kvůli přístavům Hamburg a Brehmy)
	\item Malta
	\item Kapsko
	\item Cejlon
	\item[\ra] převaha na moři
	\end{itemize}
\item Rusko
	\begin{itemize}
	\item zisk velkovévodství varšavského -- Kongresovka, Polské království (personální unie)
	\item Finsko
	\item Besarábie
	\end{itemize}
\item Prusko
	\begin{itemize}
	\item zisk 1/2 Saska
	\item Poznaňsko
	\item Porýní
	\item Vestfálsko
	\item Přední Pomořany
	\end{itemize}
\item Švédsko -- spojené království s Norskem
\item spolením Belgie a Nizozemska vytvořeno Nizozemské království
\item Švýcarsko zůstává neutrální
\item Itálie
	\begin{itemize}
	\item rozdělená
	\item Rakousko: Lombardie, Benátsko
	\item Bourboni: Neapolsko, Sicílie
	\iten dyn. savojská: Království Sardinské
	\end{itemize}
\end{itemize}

\newpage
\timeline

\end{document}