\title{Moderní dějiny 2}
\documentclass[10pt,a4paper]{article}
\usepackage[utf8]{inputenc}
\usepackage[czech]{babel}
\usepackage{amsmath}
\usepackage{amsfonts}
\usepackage{amssymb}
\usepackage{chemfig}
\usepackage{geometry}
\usepackage{wrapfig}
\usepackage{graphicx}
\usepackage{floatflt}
\usepackage{hyperref}
\usepackage{fancyhdr}
\usepackage{tabularx}
\usepackage{makecell}
\usepackage{csquotes}
\usepackage{footnote}
\usepackage{movie15}
\MakeOuterQuote{"}

\renewcommand{\labelitemii}{$\circ$}
\renewcommand{\labelitemiii}{--}
\newcommand{\ra}{$\rightarrow$ }
\newcommand{\x}{$\times$ }
\newcommand{\lp}[2]{#1 -- #2}
\newcommand{\timeline}{\input{timeline}}


\geometry{lmargin = 0.8in, rmargin = 0.8in, tmargin = 0.8in, bmargin = 0.8in}
\date{\today}
\author{Jakub Rádl}

\makeatletter
\let\thetitle\@title
\let\theauthor\@author
\makeatother

\hypersetup{
colorlinks=true,
linkcolor=black,
urlcolor=cyan,
}



\begin{document}
\maketitle
\tableofcontents
\begin{figure}[b]
Toto dílo \textit{\thetitle} podléhá licenci Creative Commons \href{https://creativecommons.org/licenses/by-nc/4.0/}{CC BY-NC 4.0}.\\ (creativecommons.org/licenses/by-nc/4.0/)
\end{figure}
\newpage
\newpage

\section{Občanská válka v USA}

\paragraph{Rozsah USA od poč. 19. stol.}
\begin{itemize}
\item Louisiana (od Francie)
\item Florida (od Španělska)
\item Kalifornie, Texas, Nové Mexiko (zisk z války s Mexikem)
\item 1800 \ra 1830 -- rozšíření obyvatelstva  z 5 na 13 milionů (zlatá horečka, levná půda)
\end{itemize}

\paragraph{Hospodářství}
\begin{itemize}
\item jih -- plantáže
\item sever -- průmysl, železnice
\item zlatá horečka
\item od 1815 -- průmyslová revoluce
\item 1820 -- zákaz šíření otrokářství do nově ovládnutých států
	\begin{itemize}
	\item převládalo především na jihu
	\item na severu obchodníci, otrok nemá peníze \ra obchodník nemá zisk
	\end{itemize}
\item Abolicisté (Louis Garrison)
	\begin{itemize}
	\item 2000 spolků na pomoc otroků
	\item Underground Raleway (ilegální, ne podzemní)
	\item \lp{1859}{Johrn Brown}
		\begin{itemize}
		\item chtěl vyvolat hromadné povstání otroků, přepadl zbrojnici, pověšen
		\end{itemize}
	\end{itemize}
\end{itemize}

\paragraph{Politické strany}
\begin{itemize}
\item 1828 -- Demokratická strana
\item 1854 -- Republikánská strana
	\begin{itemize}
	\item hospodáská jednota
	\item zrušení otroctví
	\item bezplatné příděly půdy na Západě
	\end{itemize}
\item 1860 -- zvolen prezidentem Abraham Lincoln
	\begin{itemize}
	\item umírněné názory na otroctví, nesmí se zrušit hromadně, Jih by se naštval
	\item Jih se stejně naštval 
	\item[\ra] 20. 12. 1860 -- zástupci J. Karolíny odvoláni z Washingtonu
	\item[\ra] 4. 2. 1861 -- 11 států vystoupilo z unie
		\begin{itemize}
		\item Konfederované státy americké (hl. m. Richmond)
		\item příprava na vojenský střet
		\end{itemize}
	\end{itemize}
\end{itemize}

\paragraph{Občanská válka severu proti jihu}
\subsection{1. fáze (1861--1865)}
\begin{itemize}
\item \lp{12. 4. 1861}{útok na Fort Sumter} (úspěch jižanů -- R. E. Lee)
\item \lp{15. 4. 18}{počátek války}
\item
\item
\item
\item
\item 1862 -- zákon o bezplatých přídělech půdy (za malý registrašní poplatek, po 5 letech přejde půda do vlastnictví)
\item 1863 -- zákon o zrušení otroctví
\item 
\end{itemize}

\end{document}