\title{Ranný novověk}
\documentclass[10pt,a4paper]{article}
\usepackage[utf8]{inputenc}
\usepackage[czech]{babel}
\usepackage{amsmath}
\usepackage{amsfonts}
\usepackage{amssymb}
\usepackage{chemfig}
\usepackage{geometry}
\usepackage{wrapfig}
\usepackage{graphicx}
\usepackage{floatflt}
\usepackage{hyperref}
\usepackage{fancyhdr}
\usepackage{tabularx}
\usepackage{makecell}
\usepackage{csquotes}
\usepackage{footnote}
\usepackage{movie15}
\MakeOuterQuote{"}

\renewcommand{\labelitemii}{$\circ$}
\renewcommand{\labelitemiii}{--}
\newcommand{\ra}{$\rightarrow$ }
\newcommand{\x}{$\times$ }
\newcommand{\lp}[2]{#1 -- #2}
\newcommand{\timeline}{\input{timeline}}


\geometry{lmargin = 0.8in, rmargin = 0.8in, tmargin = 0.8in, bmargin = 0.8in}
\date{\today}
\author{Jakub Rádl}

\makeatletter
\let\thetitle\@title
\let\theauthor\@author
\makeatother

\hypersetup{
colorlinks=true,
linkcolor=black,
urlcolor=cyan,
}



\begin{document}
\maketitle
\tableofcontents
\begin{figure}[b]
Toto dílo \textit{\thetitle} podléhá licenci Creative Commons \href{https://creativecommons.org/licenses/by-nc/4.0/}{CC BY-NC 4.0}.\\ (creativecommons.org/licenses/by-nc/4.0/)
\end{figure}
\newpage

\section{Zámořské cesty}
\subsection{Důvody}
\paragraph{Politické důvody}
\begin{itemize}
\item \textbf{Arabové}
\item cesty do orientu obsazeny \textbf{Osmanskými turky} (po 1453)
	\begin{itemize}
	\item okrádali a zabíjeli obchodníky \ra nebezpečné těmito cestami cestovat
	\end{itemize}
\end{itemize}

\paragraph{Ekonomické důvody a předpoklady}
\begin{itemize}
\item potřeba silného financování -- financováno bohatými \textbf{kupci}
\item hledání \textbf{odbytišť} zboží vyráběného v Evropě
\item hledání nových \textbf{zdrojů surovin} (v Čechách útlum těžby), nejen Ag a Au, ale i drahokamů, vzácného dřeva, koření
\end{itemize}

\paragraph{Technické a vědecké předpoklady}
\begin{itemize}
\item začali se stavět lodě s hlubokým kýlem a kormidlem
	\begin{itemize}
	\item \textbf{velký ponor} \ra stabilita a větší nosnost
	\end{itemize}
\item zdokonalení stěžňového systému (4 stěžně, 6 plachet)
\item od 14. století používání \textbf{kompasů} (Flavio Giola)
\item převzato od Arabů
	\begin{itemize}
	\item astroláb -- slouží ke stanovení zeměpisné polohy lodě 
	\item astronomické tabulky -- pojmenování souhvězdí pro orientaci 
	\end{itemize}
\item geografické znalosti -- mapovali pobřeží
	\begin{itemize}
	\item první glóbus sestaven Martinem Behenim (zmapovány pouze známé kontinenty)
	\end{itemize}
\end{itemize}

\subsection{Zprávy o orientálních zemích}
\begin{itemize}
\item námořníci si vymýšleli zvláštní bytosti a poklady (řeky, kde se zlato a drahokamy dají nabírat do dlaní
\end{itemize}

\paragraph{Marco Polo}
\begin{itemize}
\item z Korčuly, uvězněn inkvizicí za vymyšlené zprávy o zámoří
\item cestoval z Benátek do Číny
	\begin{itemize}
	\item dostal se na dvůr Chána Kublaje (1271 -- 1295), zapisoval pro něj cesty
	\item zúčastnil se s ním výpravy na ostrov Zipangu (dnešní Japonsko)
	\end{itemize}
\end{itemize}

\subsection{Správa zámořského území}
\paragraph{Portugalsko, Španělsko}
\begin{itemize}
\item státní monopol, následně kolonie rozděleny
\end{itemize}

\paragraph{Anglie, Nizozemí, Francie}
\begin{itemize}
\item polostátní obchodní společnosti
\item vykořisťovaly kolonie
\end{itemize}

\paragraph{Cesty do orientu}
\begin{itemize}
\item obeplutí Afriky
\item cesta na Západ
\end{itemize}

\subsection{Cesty Portugalců}
\begin{itemize}
\item 1415 dobyta Ceuta
\end{itemize}

\paragraph{Princ Jindřich Mořeplavec}
\begin{itemize}
\item třetí syn portugalského krále \ra král ho nechtěl pustit, aby měl dědice
\item podporoval platby svými prostředky
\end{itemize}

\paragraph{Cesty podél Afriky}
\begin{itemize}
\item hledali naleziště zlata
\item dopluli na Kapverdské ostrovy
\end{itemize}

\paragraph{\lp{1444}{založení Společnosti pro obchod s koloniálním zbožím}}
\begin{itemize}
\item suroviny, otroci
\end{itemize}

\paragraph{\lp{1487}{cesta Bartolomea Diase}}
\begin{itemize}
\item doplul na jih Afriky k mysu Dobré naděje
\end{itemize}

\paragraph{\lp{1497}{Vasco da Gama}}
\begin{itemize}
\item navazoval na cesty Bartolomea Diase
\item cca rok obeplouval Afriku a doplul od Indie
\item[\ra] dovoz drahokamů, hedvábí, koření, dřeva, \ldots
\end{itemize}

\paragraph{\lp{1500}{}}


\subsection{\lp{1474}{1. koloniální válka}}
\begin{itemize}
\item Portugalsko chtělo ovládnout Katílii
\item[\ra] první dělení světa
	\begin{itemize}
	\item spor musel řešit papež
	\end{itemize}
\end{itemize}

\paragraph{\lp{1494}{dohoda v Tordesillas}}
\begin{itemize}
\item 200km na západ od Kapverdských ostrovů je poledník
\item západ je portugalský, východ je španělský
\end{itemize}

\paragraph{1500}

\subsection{Cesty Španělů}
\begin{itemize}
\item \lp{1479}{ovládli Kanárské ostrovy}
\item \lp{1479}{vznik Španělska}
\end{itemize}

\paragraph{Kryštof Kolumbus}
\begin{itemize}
\item portugalský král nechtěl cesty financovat
\item rozhodl se jít do Španělska (musel tajně uprchnout)
\item \lp{3. 8. 1492}{Kolumbus vyplul}
\end{itemize}

\paragraph{\lp{11./12. 10. 1492}{Kolumbus doplul na San Salvador}}
\begin{itemize}
\item na pobřeží viděli fluoreskující řasy
\item obyvatelé nazváni Indiány
\item získali koření
\end{itemize}

\paragraph{2. Kolumbova cesta} -- kolem ostrovů
\paragraph{3. Kolumbova cesta} -- k pobřeží Jižní Ameriky
\paragraph{4. Kolumbova cesta} -- pobřeží Panamské šíje

\paragraph{\lp{1506}{zemřel ve Valladolidu}}
\begin{itemize}
\item je pravděpodobně pohřben na San Salvadoru
\end{itemize}

\paragraph{1501}{Amerigo Vespucci}
\begin{itemize}
\item cestoval do jižní Ameriky
\item popsal část pobřeží
\end{itemize}

\paragraph{\lp{1513}{Vasco Nunez De Balboa překročil Panamskou šíji}}
\begin{itemize}
\item přenesli loď
\item objevili záliv Rica \ra potvrdili, že Amerika je kontinent
\end{itemize}

\paragraph{\lp{1519}{Fernao Magalhaes -- plavba kolem světa}}
\begin{itemize}
\item portugalský mořeplavec
\item financován Španěly
\item vydali se na jihozápad s předpokladem, že obepluje Ameriku jako Afriku
\item u ústí řeky ... se část výpravy vrátila zpět
\item propluli Magalhaesovým průplavem (mezi Amerikou a ohňovou zemí)
\item doplul do Filipín
\item \lp{1521}{zabit domorodci při vyjednávání}, zbytek výpravy plul zpět
\item \lp{1522}{cesta kolem světa dokončena}
\item[\ra] potvrzena kulatost Země
\end{itemize}
\end{document}