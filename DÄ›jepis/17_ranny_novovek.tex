\title{Raný novověk}
\documentclass[10pt,a4paper]{article}
\usepackage[utf8]{inputenc}
\usepackage[czech]{babel}
\usepackage{amsmath}
\usepackage{amsfonts}
\usepackage{amssymb}
\usepackage{chemfig}
\usepackage{geometry}
\usepackage{wrapfig}
\usepackage{graphicx}
\usepackage{floatflt}
\usepackage{hyperref}
\usepackage{fancyhdr}
\usepackage{tabularx}
\usepackage{makecell}
\usepackage{csquotes}
\usepackage{footnote}
\usepackage{movie15}
\MakeOuterQuote{"}

\renewcommand{\labelitemii}{$\circ$}
\renewcommand{\labelitemiii}{--}
\newcommand{\ra}{$\rightarrow$ }
\newcommand{\x}{$\times$ }
\newcommand{\lp}[2]{#1 -- #2}
\newcommand{\timeline}{\input{timeline}}


\geometry{lmargin = 0.8in, rmargin = 0.8in, tmargin = 0.8in, bmargin = 0.8in}
\date{\today}
\author{Jakub Rádl}

\makeatletter
\let\thetitle\@title
\let\theauthor\@author
\makeatother

\hypersetup{
colorlinks=true,
linkcolor=black,
urlcolor=cyan,
}



\begin{document}
\maketitle
\tableofcontents
\begin{figure}[b]
Toto dílo \textit{\thetitle} podléhá licenci Creative Commons \href{https://creativecommons.org/licenses/by-nc/4.0/}{CC BY-NC 4.0}.\\ (creativecommons.org/licenses/by-nc/4.0/)
\end{figure}
\newpage

\section{Zámořské cesty}
\subsection{Důvody}
\paragraph{Politické důvody}
\begin{itemize}
\item Arabové obchodu nebránili
\item \lp{1453}{cesty do orientu Obsazeny \textbf{osmanskými turky}} (okrádali a zabíjeli obchodníky \ra nebezpečí)
\end{itemize}

\paragraph{Ekonomické důvody a předpoklady}
\begin{itemize}
\item potřeba silného financování -- financováno bohatými \textbf{kupci}
\item hledání \textbf{odbytišť} zboží vyráběného v Evropě
\item hledání nových \textbf{zdrojů surovin} (v Čechách útlum těžby), nejen Ag a Au, ale i drahokamů, vzácného dřeva, koření
\end{itemize}

\paragraph{Technické a vědecké předpoklady}
\begin{itemize}
\item lodě s hlubokým kýlem a kormidlem (velký ponor \ra stabilita a větší nosnost)
\item zdokonalení stěžňového systému (4 stěžně, 6 plachet)
\item od 14. století používání \textbf{kompasů} (Flavio Giola)
\item převzato od Arabů
	\begin{itemize}
	\item astroláb -- slouží ke stanovení zeměpisné polohy lodě 
	\item astronomické tabulky -- pojmenování souhvězdí pro orientaci 
	\end{itemize}
\item geografické znalosti -- mapovali pobřeží
	\begin{itemize}
	\item první glóbus sestaven Martinem Behenim (zmapovány pouze známé kontinenty)
	\end{itemize}
\end{itemize}

\subsection{Zprávy o orientálních zemích}
\begin{itemize}
\item námořníci si vymýšleli zvláštní bytosti a poklady (řeky, kde se zlato a drahokamy dají nabírat do dlaní)
\end{itemize}

\paragraph{Marco Polo}
\begin{itemize}
\item z Korčuly, uvězněn inkvizicí za vymyšlené zprávy o zámoří
\item cestoval z Benátek do Číny
	\begin{itemize}
	\item dostal se na dvůr Chána Kublaje (1271 -- 1295), zapisoval pro něj cesty
	\item zúčastnil se s ním výpravy na ostrov Zipangu (dnešní Japonsko)
	\end{itemize}
\end{itemize}

\paragraph{Správa zámořského území}
\begin{itemize}
\item Portugalsko, Španělsko -- státní monopol, následně kolonie rozděleny
\item Anglie, Nizozemí, Francie
	\begin{itemize}
	\item polostátní obchodní společnosti
	\item vykořisťovaly kolonie	
	\end{itemize}
\end{itemize}

\paragraph{Cesty do orientu}
\begin{itemize}
\item obeplutí Afriky
\item cesta na Západ
\end{itemize}

\subsection{Portugalsko}
\begin{itemize}
\item \lp{1415}{dobyta Ceuta} (poloostrov africké části Gibraltaru)
\item princ \textbf{Jindřich Mořeplavec} - král ho nechtěl pustit, aby neztratil dědice \ra finančně podporoval plavby
\end{itemize}


\paragraph{Cesty podél Afriky}
\begin{itemize}
\item hledali naleziště zlata
\item dopluli na Kapverdské ostrovy
\end{itemize}

\begin{itemize}
\item \lp{1444}{založení Společnosti pro obchod s koloniálním zbožím} (suroviny, otroci)
\item \lp{1487}{Bartolomeo Dias doplul na jih Afriky k mysu Dobré naděje}
\item \lp{1497}{Vasco da Gama}
	\begin{itemize}
	\item navazoval na cesty Bartolomea Diase
	\item cca rok obeplouval Afriku a doplul od Indie
	\item[\ra] dovoz drahokamů, hedvábí, koření, dřeva, \ldots
	\end{itemize}
\item \lp{1500}{Pedro Álvares Cabral -- Tra de Vera Cruz (Brazílie)}
\end{itemize}



\subsection{1. koloniální válka}
\begin{itemize}
\item \lp{1474--1479}
\item Portugalsko chtělo ovládnout Katílii
\item[\ra] první dělení světa
	\begin{itemize}
	\item spor musel řešit papež
	\end{itemize}
\item \lp{1494}{dohoda v Tordesillas}
	\begin{itemize}
	\item 2000km na západ od Kapverdských ostrovů je poledník
	\item západ je portugalský, východ je španělský
	\end{itemize}
\end{itemize}

\subsection{Španělsko}
\begin{itemize}
\item \lp{1479}{ovládli Kanárské ostrovy}
\item \lp{1479}{vznik Španělska}
\end{itemize}

\paragraph{Kryštof Kolumbus}
\begin{itemize}
\item portugalský král nechtěl cesty financovat \ra rozhodl se jít do Španělska (musel tajně uprchnout)
\item \textbf{\lp{3. 8. 1492}{První Kolumbova cesta}}
	\begin{itemize}
	\item \lp{11./12. 10. 1492}{dopluli na San Salvador}
	\item na pobřeží viděli fluoreskující řasy
	\item obyvatelé nazváni Indiány
	\item získali koření
	\end{itemize}
\item 2. Kolumbova cesta -- kolem ostrovů
\item 3. Kolumbova cesta -- k pobřeží Jižní Ameriky
\item 4. Kolumbova cesta -- pobřeží Panamské šíje
\item \lp{1506}{zemřel ve Valladolidu} (pravděpodobně pohřben na San Salvadoru)
\end{itemize}

\subsection{Ostatní země}
\begin{itemize}
\item \lp{1501}{\textbf{Amerigo Vespucci}} (jižní Amerika, popsal část pobřeží)
\item \lp{1513}{\textbf{Vasco Nunez De Balboa} překročil Panamskou šíji} (přenesli loď, objevili záliv Rica \ra potvrdili, že Amerika je kontinent)
\end{itemize}

\paragraph{Plavba kolem světa}
\begin{itemize}
\item \lp{1519}{Fernão Magalhães} (portugalský mořeplavec, financován Španěly)
\item vydali se na jihozápad s předpokladem, že obepluje Ameriku jako Afriku
\item ústí řeky Santa Cruz (Patagonie, část výpravy se vrátila zpět) \ra Magalhaesův průplav(mezi Amerikou a ohňovou zemí) \ra Filipíny
\item \lp{1521}{zabit domorodci při vyjednávání} (zbytek výpravy plul zpět)
\item \lp{1522}{cestu dokončil Sebastian del Cano}
\item[\ra] potvrzena kulatost Země
\end{itemize}



\section{Conquista}
\subsection{Hernán Cortés}
\begin{itemize}
\item \lp{1517}{cesta na Yucatan} (poloostrov v Mexiku)
\item zisk nového území od méně vyspělých civilizací
\end{itemize}
\paragraph{Mayové} (Mexiko)
	\begin{itemize}
	\item městské státy
	\item pyramidy, sochy, obrázkové písmo, astronomie, kalendáře, 
	\item dlážděné silnice \item poštovní styk, vodovody, mapy
	\item vyspělejší astronomie a kalendář, hrnčířství, řezbářství
	\item neznali železo, tažná zvířata
	\item časté přesidlování z důvodu vyčerpané půdy (neznali hnojení)
	\item považovaly Španěly za bohy (koně, palné zbraně, moře, \ldots)
	\end{itemize}
\paragraph{Aztécká říše} (Mexiko)
\begin{itemize}
\item strategické hlavní město na jezeře \textbf{Tenochtitlan}, král Montezuna
\item \textbf{nové plodiny} -- rýže, bavlna, pepř, vanilka, tabák, banány
\item \textbf{domácí zvířata} -- pes, krocan, husa, kachna
\item polyteistické náboženství (obětování stovek lidí Opeřenému hadovi)
\item \lp{1535}{Cortés objevil Kalifornii} (hledal El d'orado -- zemi zlata
\end{itemize}

\subsection{Říše Inků (Peru)}
\begin{itemize}
\item údolí Nasca, Mocha, centrum v údolí Cuzco, jezero Titicaca
\item \textbf{Machu Pichu} -- Španělé ho nikdy neobjevili
\item inka -- vládce (\ra chybně nazývaný kmen)
\item velké množství zlata -- "slzy boha slunce", nádobí, domy pobité plechy, 
\item bazény teplé vody okolo zámku, tepané květiny
\item \lp{1531-1535}{dobyli Franciscem Pizarro, Diego Almagro}
\end{itemize}

\paragraph{Vyspělá kultura inků}
\begin{itemize}
\item destilace medicíny z rostlin
\item vzdělání pro úředníky
\item sociální organizace 
	\begin{itemize}
	\item inka + dvůr, úředníci, řemeslníci
	\item při dosažení určitého věku povinný sňatek, případně přiřazen partner, půda (kdo má půdu, platí daně)
	\end{itemize}
\item terasovitá pole hnojená guanem (trus mořských ptáků)
\item lamy \ra vlna, mléko, maso
\end{itemize}



\section{Vznik nových koloniálních mocností}
\subsection{Nizozemsko}
\begin{itemize}
\item shromažďovali zboží přes pobřežní pevnosti
\item směřovali do tichomoří \ra Indonésie, Moluky
\item \lp{1602}{založena nizozemská Spojená východoindická společnost}
\end{itemize}

\subsection{Anglie}
\begin{itemize}
\item napadali španělské kolonie, snažili se je zabrat pro sebe, využívali piráty
\end{itemize}

\paragraph{Italští mořeplavci}
	\begin{itemize}
	\item \lp{1497}{\textbf{Giovanni Cabotto} (\textit{John Cabbot}) \ra Labrador, Newdoundland} (Kanada)
	\item Sebastian Cabotto \ra severoamerické pobřeží
	
	\end{itemize}


\paragraph{Angličané} 
\begin{itemize}
\item \textbf{Walter Raleigh} -- Virginie, 1627 kolonií
\item \textbf{Henry Hudson} + \textbf{William Baffin} -- hledali cestu kolem severní Ameriky
\item \lp{1596}{\textbf{W. Barents} objevil Špickberky} 
\item \textbf{John Hawkins} -- Guinea (podporoval obchod s otroky)
\end{itemize}
	
\subsection{Ostatní mocnosti}
\begin{itemize}
\item Francie -- část Kanady (1535)
\item Švédsko -- od ústí řeky Delawere vytlačeni Nizozemci
\item Dánsko a Německo -- pouze neúspěšné pokusy
\end{itemize}

\subsection{Výsledky a význam}
\begin{itemize}
\item objevení vyspělých civilizací i mimo Evropu \ra míšení kulturních vlivů
\item nové plodiny: brambory (plodina chudých), rajčata, cukrová třtina, tabák
\item potvrzena kulatost Země
\item základ světového obchodu
\item přesun center obchodu na pobřeží Atlantiku
\item \textbf{cenová revoluce \ra kapitalizace společnosti}
	\begin{itemize}
	\item do Evropy přiváženo mnoho zlata a stříbra \ra hodnota peněz jde dolů, cena zboží nahoru
	\item[\ra] rozdělení městské společnosti na \textbf{buržoazii} (bohatí) a \textbf{proletariát} (pracující)
	\item buržoazie platí daně \ra důležitá \ra chce se podílet na vládě \ra buržoazní revoluce 
	\end{itemize}
\end{itemize}

\section{Renesance}
\subsection{Vznik renezance}
\begin{itemize}
\item 14. století, Itálie
\item z italského \textit{re-nascire} \ldots nový život
\item emancipace a zvýšení významu bohatého měšťanstva
\item důležitý vliv antiky: každý je zodpovědný za sebe, důraz na vzdělání a společenskou aktivitu
\item typické rysy: individualismus, vzdělání, společenská aktivita
\end{itemize}

\subsection{Italské trecento (14. století)}
\begin{itemize}
\item koncept jednotné itálie
\end{itemize}

\subsubsection{Literatura}
\begin{itemize}
\item \textbf{Dante Alighieri} -- poprvé zavedl spisovnou latinu, \textit{Božská komedie}, \textit{De monarchia} [monarka]
\item \textbf{Francesco Petrarca} -- korespondence s Karlem IV., \textit{Canzoniere}, \textit{Sto sonetů lauře}
\item \textbf{Giovanni Boccaccio} -- \textit{Dekameron}
\item \textbf{Cola di Rienzo} -- 1347 ovládl Řím
\end{itemize}

\subsubsection{Malířství}
\begin{itemize}
\item \textbf{Giotto di Bondone} -- první plasticky vyvinuté postav
\item \textbf{Simone Martini} -- papežský palác v Avignonu
\end{itemize}


\subsection{Italské quattrocento}
\begin{itemize}
\item období rozkvětu Itálie pod vládou rodu Medici
\item centrem renesance je Florencie
	\begin{itemize}
	\item Cosimo I. Medici
	\item Lorenzo I. il Magnifico (vzdělaný, podporoval umělce)
	\end{itemize}
\item typickým znakem je hledáním harmonie a vyváženosti
\end{itemize}

\paragraph{Platónská akademie}
\begin{itemize}
\item 529 -- zrušena Justiniánem
\item za Lorenza obnovena
\end{itemize}

\subsubsection{Architektura}
\paragraph{Základní prvky}
\begin{itemize}
	\item horizontála, kopule, římsy, oblouky, sloupy, arkády (podloubí),  
	\item pilastry -- dekorativní polosloupy
	\item balustráda -- zábradlí tvořené kuželkami
	\item sgrafito dvoubarevná vyřezávaná omítka
	\item bosáž (rustika) -- imitace kamenných kvádrů
\end{itemize}
\paragraph{Autoři}
\begin{itemize}
\item Fillipo Brunelleschi -- nalezinec ve Florencii, S. Maria del Fiore, palác Pitti a jiné stavby
\item Leon Battista Alberti -- popsal prvky renezance v teoretických dálech
\end{itemize}

\subsubsection{Malířství}
\begin{itemize}
\item \textbf{Masaccio} -- perspektiva
\item \textbf{Piero della Francesca}
\item \textbf{Sandro Botticelli} -- rozměřil "krásné" proporce lidkého těla
\item \textbf{Lorenzo Ghiberti} -- dveře do Ráje
\end{itemize}


\subsection{Italské cinquecento (16. století)}
\begin{itemize}
\item vrcholné období renesance
\item přestavba chrámu sv. Petra
\end{itemize}

\paragraph{Autoři}
\begin{itemize}
\item \textbf{Giorgio Vasari} -- napsal teoretické dílo popisující renesanci a životy významných autorů
\item \textbf{Michelangelo Buonarroti} -- sochař, malíř (Sixtínská kaple -- \textit{Zrození Adama})
\item \textbf{Raffael Santi} -- \textit{Sixtinská madona}
\item \textbf{Leonardo da Vinci} -- malíř a všechno -- \textit{Mona Lisa}, \textit{Dáma s Hranostajem}, \textit{Svatá žena samotřetí}
\item \textbf{Mathias Grünewald}
\item \textbf{Hans Holbein} -- Nizozemec, púsobil v anglii, portrét Jindřicha, \textbf{Ježíš v Hrobě}
\item Hieronymus Bosh -- velmi detailní a rozsáhlé obrazy \textit{Zahrada pozemských rozkoší} -- Prado
\item \textbf{Jan van Eyck} -- výborná perspektiva, 	\textit{Svatba manželů Arnolfiniových}
\item Piter Breugel [brechl]
\item Donato Bramate
\end{itemize}

\subsection{Renesance u nás}
\begin{itemize}
\item zámek v Bučovicích (arkády ze tří stran nádvoří)
\item dům pánů z Kunštátu v Brně (nádvoří, arkády)
\item letohrádek královny Anny
\end{itemize}

\subsection{Manýrismus}
\begin{itemize}
\item protáhlé tvary
\item pitoreskní výjevy
\item Parmigiano, Tintoretto
\item \textbf{El Greco}
\end{itemize}

\paragraph{Mnýrismus na dvoře Rudolfa II.}
\begin{itemize}
\item Bartolomeus Spranger
\item Hans von Aachen
\item Arcimboldo
\item Ardien de Vries
\end{itemize}

\section{Humanismus}
\begin{itemize}
\item studia divina $\times$ studia humana
\item studium klasických jazyků -- aby si mohli přečíst antická díla
\item Erasmus Rotterdamský: Chvála bláznovství
\item \textbf{Johannes Gensfleisch von Gutenberg} -- vynález knihtisku
\end{itemize}

\paragraph{Politologie}
\begin{itemize}
\item Niccolo Machiavelli
\end{itemize}

\paragraph{Astronomie}
\begin{itemize}
\item Mikuláš Kusánský
\item Mikuláš Koperník
\item Johannes Kepler
\item Giordano Bruno
\item Galileo Galilei (sestroji dalekohled)
\end{itemize}

\paragraph{Význam}
\begin{itemize}
\item připravuje půdu pro další rozvoj novověké evropské kultury
\end{itemize}

\newpage
\section{Reformace}
\subsection{Příčiny reformace}
\begin{itemize}
\item reakce na společenskou krizi
\item krizový stav papežství 
\item všeobecná snaha reformovat církev
	\begin{itemize}
	\item \textbf{koncilialismus} -- říká, že koncil má větší autoritu, než papež
	\item \textbf{papalismus} -- autoritu má papež
	\end{itemize}
\end{itemize}

\paragraph{Cíle reformace}
\begin{itemize}
\item dodržování etických zásad křesťanství
\item odstranění světské části církve
\end{itemize}

\subsection{1. reformace (14.--15. stol.)}
\begin{itemize}
\item John Wycliffe + lolardi
\item M. Jan Hus, M. Jeroným Pražský (překládal Wycliffovy spisy)
\item husitství v Čechách
\item česká reformace \ra 1. země s oficiálně povolenými dvěma vyznáními (1436)
\end{itemize}

\subsection{Německá reformace}
\begin{itemize}
\item 16. stol. -- rozdrobenost, církev vlastní většinu pozemků, zvyšování církevních poplatků
\end{itemize}

\paragraph{Martin Luther}
\begin{itemize}
\item pocházel z Wittenberku
\item \lp{31. 10. 1517}{přibil \textbf{95 tezí} na dveře kostela} (v latině, rychle přeloženy)
\item usiluje o bezprostřední stav člověka a boha
\item papež jeho spisy nechal spálit
\item \lp{1521}{Luther pozván na říšský sněm do Wormsu}
\item hrozilo mu zatčení \ra cca rok se ukríval na Wartburgu, kde \textbf{přeložil Bibli}
\item 1522 -- zjistil, že má přívržence a začal znovu kázat \ra vznik nové církve
\item navazoval na Husa (přijímání podobojí, kritika odpustků)
\item důraz na vnitřní zbožnost
\item neuznává celibát
\item liturgický = národní jazyk
\item stoupenci zejm. šlechta, majetní měšťané
\item státní náboženství v Dánsku, Norsku, Švédsku, Islandu
\item stoupenci: České země, Uhersko, Sedmihradsko
\end{itemize}

\paragraph{\lp{1524--1526}{Německá selská válka}}
\begin{itemize}
\item nejradikálnější Durynsko -- \textbf{Thomas Müntzer} (Cvikov)
\item vycházel z Husa, ale radikálnější
\item \lp{1521}{pokus o podnícení nové revoluce v Praze}
\item rovnost všech lidí
\item \lp{1525}{poraženo, Müntzer popraven}
\item \lp{1526}{definitivní porážka i v Tyrolsku a Salzbursku}
\end{itemize}

\paragraph{Münsterská komuna}
\begin{itemize}
\item \textbf{novokřtěnci} -- křtít by se měli až dospělí lidé, kteří si jsou uvědomit váhu křtu
\item společný majetek, vojenská povinnost, protokomunistická diktatura 
\item \lp{1534}{obsazen Münster}
\item \lp{1535}{komuna vojensky poražena}
\item novokřtěnci na Moravě
\end{itemize}

\subsubsection{Katolíci $\times$ protestanti}
\begin{itemize}
\item říšský sněm ve Špýru -- protestanti tvrdí, že lidem nepřísluší hlasovat o správnosti náboženství
\item \lp{1531}{vznik šmalkadské jednoty} (vojensko-politicko-náboženský spolek luteránských knížat)
\item \lp{1546--1547}{šmalkaldská válka}
	\begin{itemize}
	\item ukončena porážkou lutheránů u Mühlberku 
	\end{itemize}
\item \lp{1555}{Augsburský mír}
	\begin{itemize}
	\item \textit{cuis regio, eius religio} -- čí vláda, toho víra
	\item všichni obyvatelé musí vyznávat náboženství panovníka \ra vznik \textbf{teritoriálních církví} \ra ještě rozdrobenější Německo
	\item[\ra] konec monopolu katolické církve i na území Německa
	\end{itemize}
\item Filip Melanchton dotvořil zásady luteránství
\end{itemize}

\subsection{Švýcarská reformace}
\paragraph{Huldrych Zwingli}
\begin{itemize}
\item navázal na Husa
\item lidé mají \textbf{právo vzdorovat} nespravedlivé vrchnosti
\item proti učení o transsubstanciaci -- přeměna podobojí je pouze symbolická (katolíci ji berou doslovně)
\end{itemize}

\paragraph{Jan Kalvín}
\begin{itemize}
\item od 1541 v Ženevě -- republika bez společenských privilegií (pronásledován z Francie)
\item konzistoř má být vedoucí orgán republiky
\item smyslem života je \textbf{práce, aktivní a mravný život}
\item učení o \textbf{predestinaci} -- člověk je svým chováním předurčen k spasení / zatracení
\item vrchnosti, která tak nežije je \textbf{nutné se vzepřít}, třeba i násilím
\item z Ženevy vzniká společenská republika -- všichni jsou si společensky rovni
\end{itemize}

\paragraph{Miguel Servetus}
\begin{itemize}
\item nizozemský biolog (objevil malý krevní objev)
\item fanatik
\item 1553 upálen Kalvínem v Ženevě
\end{itemize}

\subsection{Význam reformace}
\paragraph{Helvetská konfese}
\begin{itemize}
\item Skotsko (1560)
\item Uhersko, Polsko, Porýní
\item státní náboženství v Nizozemí
\item Francie -- hugenoti
\item Anglie -- puritáni
\item Amerika
\item reformace klade důraz na vnitřní zbožnost, prosazování národních jazyků a ospravedlňuje buržoazní revoluce
\end{itemize}

\section{Protireformace}
\begin{itemize}
\item protestanti -- protireformace
\item katolíci -- reforma katolické církve
\end{itemize}

\paragraph{Vztah katolíků k reformaci}
\begin{itemize}
\item papež Pavel III. (1534 -- 1549)
	\begin{itemize}
	\item Sanctum Officium (1542) -- "svatý úřad" -- obdoba inkvizice
	\end{itemize}
\item papež Pavel IV. (1549 - 55) -- bojovný
	\begin{itemize}
	\item převažuje vliv katolíků, kteří chtěli zlikvidovat protestantismus -- "bojovný katolicismus"
	\item\lp{1534}{založení řádu Societas Jesu Ignácem z Loyoly}
		\begin{itemize}
		\item 1540 uznán papežem
		\item misijní činnost i do protestantských zemí
		\item základní myšlenkou bylo, že řád je hlavní, jedinec neznamená nic, smyslem života je plnit řád
		\item pochopili důležitost působení v dětství \ra vznik propracovaného systému jezuitských škol
			\begin{itemize}
			\item poskytovaly výborné vzdělání za nízkou cenu
			\item učitelé byli vzdělaní vědci na špičce svého oboru
			\item jezuitský vědec Kamel \ra na Filipínách objevil rostlinu Kamélii
			\item na konci školního roku veřejné zkoušky \ra děti měly větší motivaci studovat
			\item do jezuitských škol posílali své děti i protestanti
			\end{itemize}
		\end{itemize}
	\end{itemize}
\end{itemize}

\paragraph{\lp{ 1545 -- 1563}{Tridentský koncil}}
\begin{itemize}
\item přelom v dějinách katolické církve
\item kněží hromadili stavy a funkce \ra definována nová pravidla
\item papež byl ustanoven větší autoritou než koncil (papalismus)
\item nesmí se hromadit církevní úřady
\item založeny biskupské semináře, nakonec nutné složit slib -- \textit{"credo ..."} (věřím)
\item některá usnesení jsou platná do 20. století
\item ekumenismus -- snaha o sjednocení křesťanských církví
\end{itemize}


\section{Šíření reformace}
\subsection{Francie od konce 15. století do počátku 17.}
\begin{itemize}
\item klasicky absolutistický stát (základy položeny Ludvíkem XI. po stoleté válce)
\item po r. 1453 stabilizace 
\item expanzivní politika
	\begin{itemize}
	\item do Itálie (Karel VIII.)
	\item boj s Habsburky o miláno 1521--1544
	\end{itemize}
\item od roku 1519 tlak Habsburků ze dvou stran
	\begin{itemize}
	\item Karel V. Habsburský se stává císařem
	\end{itemize}
\item František I. (1515--1547)
\item Jindřich II. (1547--1559)(syn Františka I.)
	\begin{itemize}
	\item nevěsta Kateřina Medicejská
	\item děti:
		\begin{itemize}
		\item František II. (1559--1560) (prvorozený syn a nástupce)
		\item Karel IX. (1560--1574) (ve vládě zastupován poručníky včetně Kateřiny Medicejské)
		\item Jindřich III. (1574--1589) 
		\item Markéta z Valois ("královna Margot")
		\end{itemize}
	\end{itemize}
\end{itemize}

\paragraph{2. pol. 16. století}
\begin{itemize}
\item roste vliv reformace -- kalvínistů -- hugenotů [igenotů]
\item oporou dynastie Bourbonů
\item \lp{1562}{\textbf{krvavá lázeň ve Vassy}}  -- \textbf{František de Guise} nechal podpálit stodolu sloužící jako hugenotský kostel (63 mrtvých, $>$100 zraněných)
\item katolická liga (\textbf{Jindřich de Guise}) $\times$ protestantská Unie (\textbf{Jindřich Navarrský de Bourbon})
\end{itemize}

\paragraph{Bartolomějská noc}
\begin{itemize}
\item 23. 8. 1572
\item \textbf{sňatek Jindřicha Navarrského s Markétou z Valois} s cílem sjednocení katolíků a protestantů
\item katoličtí kněží v Paříži svalovali na hugenoty všechna neštěstí \ra davová nenávist hugenotů
\item na hostině po sňatku zavražděn \textbf{admirál Colginy} v katedrále Nottre Dame
\item tímto odstartováno vraždění hugenotů -- 3000 v Paříži, celkem 10 000 ve Francii
\item Jindřich Navarrský se ukryl v komnatách Markéty z Valois a následně uprchl

\end{itemize}

\paragraph{Jindřich IV. Navarrský de Bournon} (1589--1610)
\begin{itemize}
\item \lp{1598}{\textbf{edikt nantský}} -- zrovnoprávnění hugenotů s katolíky
\item rozvoj hospodářství
	\begin{itemize}
	\item monetaristická politika -- stát nezasahuje do ekonomiky (důležitý je trh)
	\item snížení daní, reforma finanční správy (sjednocení měny)
	\item stavba silnic, mostů, vodních kanálů 
	\item kolonizace Kanady
	\item podpora nizozemské revoluce
	\end{itemize}
\item \lp{1610}{Jindřich IV. zavražděn} fanatickým mnichem
\item druhé manželství s \textbf{Marií Medicejskou} \ra Ludvík XIII.
\end{itemize}

\subsection{Vzestup Habsburků}
\paragraph{Ludvík XIII.} (1610--1643)
\begin{itemize}
\item poručnická vláda Marie Medicejské
\item vlády se ujímá \textbf{kardinál Richelieu} (dobrá vláda)
\item \lp{1614--1789}{nescházejí se generální stavy} -- absolutismus
\end{itemize}

\paragraph{70. léta 15. století}
\begin{itemize}
\item \lp{1477}{spor o Burgundsko} Fridricha III. Habsburský s Ludvíkem XI. Francouzským
\item Fridrich získal část Burgundska (Nizozemí) sňatkem s Marií Burgundskou
\end{itemize}

\paragraph{Dvě větve Habsburků}
\begin{itemize}
\item Državy Habsburků
\item děti Filipa a Johany Kastilských 
\item \textbf{Karel V.} -- \textbf{Španělsko}, Nizozemí, Sardinie, Sicilie, kolonie v J. Americe, kolonie v Jižní Americe
\item \textbf{Ferdinand I.} -- \textbf{Rakousko}, Čechy, Uhersko
\end{itemize}

\paragraph{Absolutismus ve Španělsku}
\begin{itemize}
\item Reconquista
\item \lp{1479}{vznik Španělska}
\item oporou Ferdinanda II. Aragonského města a hidalgové
\item oporou Karla V. Habsburského -- grandi
\item \lp{1520-1521}{povstání komunérů (Toledo)}
\item \lp{1527}{sacco di Roma} -- boj o itálii
\item hospodářský úpadek
\item neúspěšný boj s protestanty
\item 1555 -- Augsburský mír (\textit{"Cuius regio, eius religio"})
\item Karel V. se vzdal moci \ra Ferdinand I. císařem
\end{itemize}

\paragraph{Filip II} (1556-1598)
\begin{itemize}
\item \lp{1556}{Karel V. předává vládu ve Španělsku synu Filipu II.}
\item netolerantní katolický fanatik
\item bez koncepce hospodářské politiky
\item zvyšoval daně kvůli válkám, vojensky potlačoval povstání
\item \lp{1588}{Filip II. poražen Anglií v La Manche}
\end{itemize}

\section{Buržoazní revoluce v Nizozemsku}
\paragraph{Vztah ke Španělsku}
\begin{itemize}
\item připojeno na počátku 16. stol. za Filipa II.
\item \lp{1557}{státní úpadek}
	\begin{itemize}
	\item \textbf{Filip II. zadlužen} šlechtě \ra zavedeny dlužní úpisy
	\item zvýšení daní v Nizozemsku, omezování pravomocí provincií, pronásledování protestantů
	\end{itemize}
\item pronásledovaní protestanti podporováni hugenoty z Francie
\end{itemize}

\paragraph{Nizozemské provincie v 16. století}
\begin{itemize}
\item Nizozemí děleno na \textbf{17 provincií} -- každá má svůj sněm, místodržitele, privilegia
\item Jih -- Valoni
 	\begin{itemize}
	\item bohaté Flandry a Brabantsko (Antverpy)
	\item kalvínisti, katolíci
 	\end{itemize}
\item Sever -- Vlámové [flámové]
	\begin{itemize}
	\item důležité provincie Holland, Zeeland, Utrecht
	\item lutheráni, kalvínisti, novokřtěnci
	\end{itemize}
\item mezi provinciemi jsou hospodářské i společenské rozdíly
	\begin{itemize}
	\item sever chudší než jih
	\item na severu rybářství, provincie u moře bojují o půdu $\times$ na jihu rozvoj řemesla
	\item na jihu jsou nejdůležitější vrstvou měšťané a šlechta $\times$ na severu jen měšťané (obchodníci, řemeslníci)
	\item na severu kalvínisti, lutheráni $\times$ na jihu umírněnější i katolíci
	\end{itemize}
\end{itemize}

\paragraph{Generální sněm}
\begin{itemize}
\item schází se v něm 17 místodržitelů
\item předsedá mu \textbf{generální místodržící}, který je přímo zvolen králem jako jeho zástupce
\item \textbf{Markéta Parmská} (1559--1567)
	\begin{itemize}
	\item rádce kardinál \textbf{de Granvelle}
	\item tyran, upírána privilegia, zvyšovány daně
	\item velká vlna odporu z provincií \ra vznik opozice
	\item \lp{1564}{de Granvelle odvolán}
	\end{itemize}
\end{itemize}

\paragraph{Opozice}
\begin{itemize}
\item umírněná: většinou šlechta
	\begin{itemize}
	\item Vilém Oranžský (p), hrabě Egmont (k), hrabě Hoorn (k)
	\end{itemize}
\item radikálové:
	\begin{itemize}
	\item \lp{1566--1567}{obrazoborecké hnutí}
	\end{itemize}
\item španělským králem vyslán \textbf{vévoda z Alby} \ra hrůzovláda 
\begin{itemize}
\item[\ra] rada pro nepokoje (1568)
\item mučení, vyslýchání, popravování opozice (nejen protestantů) -- vévoda a vláda si nechávali majetek
\item daň \textbf{alcabala} [alcavala] -- dvojatá daň (z prodeje i koupě)
\item \lp{1568}{neúspěšný útok Viléma Oranžského}
\end{itemize}
\end{itemize}

\paragraph{Gézové}
\begin{itemize}
\item chudší část obyvatelstva, bojovala proti Španělům
	\begin{itemize}
	\item poskytnut dočasný azyl v Anglii, hrozil ale útok od Španělů \ra vyhnáni
	\item[\ra] útok na Nizozemsko (nic jiného jim nezbývalo)
	\end{itemize}
\item \lp{31.3/1.4 1572}{ovládli přístav Brielle} s pomocí nové armády Viléma Oranžského
\item[\ra] Alba odvolán
\end{itemize}

\paragraph{Roztrhání Nizozemska}
\begin{itemize}
\item \lp{1576}{vydrancovány Antverpy Španěly}
\item nová povstání proti Španělům (Bruggy, Brusel, Gent)
\item \lp{1577}{věčný edikt} -- Španělé se stáhnou \ra vyjednávání o odstoupení
\item \lp{1579}{\textbf{Arraská unie}} -- jižní provincie zůstávají se Španělskem
\item \lp{1579}{\textbf{Utrechtská unie}} -- severní provincie nechtějí zůstat se Španělskem
\item \lp{1581}{\textbf{Haagská unie}} -- 7 severních provincií tvoří \textbf{Spojené Nizozemí}
	\begin{itemize}
	\item stavovská republika (generální stavy, místodržící -- dědičně titul Vilém Oranžský)
	\item státním náboženství -- kalvinismus, tolerují se ale všechna reformovaná náboženství
	\item velmi bohatý stát z obchodu
		\begin{itemize}
		\item \lp{1602}{založena Východoindická společnost}
		\item \lp{1621}{založena Západoindická společnost}
		\item \lp{1626}{založen nový Amsterodam}(dnešní New York)
		\end{itemize}
	\end{itemize}
\item \lp{1609}{Nizozemsko uznáno jako samostatný stát}
\item \lp{1648}{samostatnost potvrzena \textbf{vestfálským mírem}}
\end{itemize}

\section{Anglie}
\paragraph{Jindřich VII.} (1509--1547)
\begin{itemize}
\item stabilizace
\item rozdělení společnosti
	\begin{itemize}
	\item \textbf{yeomani} -- zemědělci
	\item \textbf{gentry} -- venkovská šlechta, feudálové
	\item \textbf{nová šlechta} (noví baroni) -- kupují si titul od krále
	\end{itemize}
\item hvězdná komora -- soudní instituce zabývající se politickými záležitostmi
\item správní systém farnost $<$ hrabství $<$ parlament -- dolní sněmovna (voleni poslanci z hrabství)
\end{itemize}

\paragraph{Způsoby podnikání}
\begin{itemize}
\item chov ovcí -- ohrazování, prodej vlny Nizozemcům
\item[\ra] odliv pracovní síly z venkova do měst
\item obchod, stavba lodí
\end{itemize}

\paragraph{Jindřich VIII.} (1509--1547)
\begin{itemize}
\item zvláštní rysy reformace
\item královská reformace
\item oženil se s vdovou po svém bratru -- Kateřinou Aragonskou (jediná dcera přežila porod -- Marie)
\item[\ra] nutnost dědice \ra pokus o rozvod -- papež nepovolil kvůli vlivu španělské královské rodiny 
\item osobnosti
	\begin{itemize}
	\item \textbf{Thomas Wolsey} -- kardinál, arcibiskup canterburský, hříšník, kritizován
	\item \textbf{Thomas More} -- spisovatel (\textit{Utopie}), učenec 
	\item \textbf{Thomas Cranmer} -- reformátor, první anglikánský arcibiskup, Cambridge, popraven
	\item \textbf{Thomas Cromwel} -- reformátor, rádce krále, popraven
	\end{itemize}
\item manželky
	\begin{itemize}
	\item Kateřina Argaonská -- neměla syna
	\item Anna Boleynová -- nevěrná, míchala se do politiky, popravena za vlastizradu
	\item Jana Seymourová -- syn Eduard VI.
	\item Anna Clévská -- dobrovolně se rozvedly
	\item Kateřina Howardová -- popravena za nevěru
	\item Kateřina Paarová 
	\end{itemize}
\end{itemize}

\paragraph{Královská reformace}
\begin{itemize}
\item \lp{1533}{oddělení anglikánské církve}
\item \lp{1534}{Act of Supremacy} -- revize církevních zákonů
\end{itemize}

\paragraph{Anglikánství}
\begin{itemize}
\item král spravuje církevní majetek
\item král ruší církevní úřady
\item král jmenuje kardinály, biskupy, \ldots
\end{itemize}

\paragraph{Eduard VI.} (1547--1553)
\begin{itemize}
\item regentská vláda \textbf{lorda Somerseta}
\item zemřel jako chlapec
\end{itemize}

\paragraph{Marie Tudorovna} (1553--1558)
\begin{itemize}
\item perzekuce nekatolíků -- "Bloody Mary"
\item manželka Filipa II. Španělského
\end{itemize}

\paragraph{Alžběta I. Anglická} (1558--1603)
\begin{itemize}
\item krásná, ale nikdy se neprovdala
\item cílem udržení stability a míru
\item nové podnikání
	\begin{itemize}
	\item chov ovcí -- vlnařství
	\item námořní obchod
	\end{itemize}
\item podpora hugenotů a gézů
\item soupeření se Španělskem -- Filip II.
\item Marie Stuartovna -- Skotsko
	\begin{itemize}
	\item uprchla do Anglie kvůli podezření z vraždy dvou svých manželů
	\item \lp{1578}{Marie Stuartovna popravena za spiknutí proti Anglii s Filipem II.}
	\end{itemize}
\item \lp{1588}{Neporazitelná Armáda Filipa II. poražena v La Manchském průlivu}
\end{itemize}	

\subsection{Alžbětínská Anglie}
\begin{itemize}
\item \textbf{Erasmus Rotterdamský}
\item sir \textbf{Walter Raleigh} [roli]
	\begin{itemize}
	\item založil Virginii
	\item 1603 -- uvězněn v Tauru
	\end{itemize}
\item \textbf{William Shakespeare} -- existenciální otázky \ra hrají se do dnes
\item vliv nizozemského malířství a francouzské architektury
\end{itemize}

\section{Rusko}
\subsection{Opakování}
\begin{itemize}
\item 13. století -- Rozpad Ruska 
\item Novgorodské knížectví
\item Vladimirsko suzdalské knížectví
\item Ivan I. Kalita
\item Dimitrij Donský
\item 14. stol. -- úpadek Zlaté hordy
\item 1478 -- připojil Ivan III. Novgorod a jiná knížectví k Moskvě
\item 1480 -- Ivan III. přestal platit daně Tatarům
\item nový stát Moskevská Rus
\end{itemize}

\subsection{Stav v 2. pol. 15. století}
\paragraph{Sousedé Rusi}
\begin{itemize}
\item na západě Litva, Livonsko
\item na jihu Tataři
\item na východě Sibiř -- možnost výbojů
\end{itemize}

\paragraph{Hospodářství}
\begin{itemize}
\item zemědělství
	\begin{itemize}
	\item \textbf{votčina} -- dědičná půda
	\item \textbf{poměstí} (nedědičná, udělena panovníkem) \ra poměšči = služebná šlechta
	\end{itemize}
\item obchod s Evropou vázl -- strach z Tatarů
	\begin{itemize}
	\item trasa Moskva -- Lublin -- Krakov -- Praha
	\item trasa Moskva -- Lublin -- Poznaň -- Hanza
	\end{itemize}
\item města mají zemědělský charakter (dřevěné domy, \ldots)
\item největšími městy jsou Novgorod, Pskov, Nižní Novgorod, Vladimir, Suzdal, Moskva
\end{itemize}

\paragraph{Církev}
\begin{itemize}
\item pokusy o reformaci neúspěšné
\item církevní dogmata označena za nedotknutelná
\item církev má právo na světskou moc
\item až do 20. stol. populární idea třetího Říma 
	\begin{itemize}
	\item jediná správná církev je ruská pravoslavná
	\end{itemize}
\item \lp{1589}{moskevský metropolita se stává patriarchou}
\end{itemize}

\subsection{Ivan IV. Hrozný} (1533--1584)
\begin{itemize}
\item otec Vasilij III.
\item na trůn nastupuje ve 3 letech \ra ovládán bojary
\item \textbf{reformy} -- správní, právní, vojenské, hospodářské
\item \lp{1547}{korunován prvním ruským carem}
\item \lp{1547--1550}{ruské stavovské shromáždění} \textbf{Duma}
	\begin{itemize}
	\item připravovala zákonní \textbf{Stoglav}
	\end{itemize}
\item \lp{1552}{dobyl Kazaň}
	\begin{itemize}
	\item vyvražďování a přesidlování obyvatel \ra porušťování
	\item otevření cesty ke Kaspickému moři
	\end{itemize}
\end{itemize}

\paragraph{Výboje do Evropy}
\begin{itemize}
\item snaha oslabit Livonsko a získat přístup k Baltu
\item Livonské války (1558--1582)
	\begin{itemize}
	\item Moskevská Rus poražena Lublinskou unií
	\item 1582 -- mír pro Rusko ztrátou, vinu dával bojarům
	\end{itemize}
\end{itemize}

\paragraph{Samoděržaví}
\begin{itemize}
\item odebral půdu bojarům -- snaha o zlomení jejich moci
\item předána nižší šlechtě výměnou za vojenskou povinnost \ra \textbf{opričina} (opriční vojsko)
\item bojaři vysídleni do okrajových oblastí na \textbf{zemštinu} -- neúrodná
\item reforma prosazována terorem \ra katastrofické důsledky
	\begin{itemize}
	\item půda nemohla být obdělávána
	\item hladomor
	\item povaražděno obrovské množství mužů, žen i dětí
	\end{itemize}
\item psychicky narušený -- zabil svého syna
\item reformy silně narušili Rusko, ale vedli k ruskému absolutismu -- samoděržaví
\end{itemize}

\subsection{Car Fjodor} (1584--1598)
\begin{itemize}
\item syn Ivana IV.
\item opožděný \ra regent \textbf{Boris Godunov}
\item likvidace bojarské opozice (Sujští, Bělští, Romanovci, Glinští)
\item \lp{1584}{založen přístav Archangelsk}
\item \lp{1587}{příměří na 15 let s Polskem}
\item \lp{1590--93}{úspěšná válka se Švédskem}
\item \lp{1595}{uzavřen Věčný mír}\ra přístup k Finskému moři
\item \lp{1598}{zemřel Fjord \ra Godunov korunován carem} 
\end{itemize}

\paragraph{Car Boris Godunov}(1598--1605)
\begin{itemize}
\item hromadné útěky poddaných \ra nevolnictví
\item opozice do vyhnanství
\item \lp{1600--1603}{hladomor}
\item bojaři se spojili s Poláky
\item \lp{1605}{Godunov zemřel}
\end{itemize}

\subsection{Období zmatků na přelomu 16. a 17. století}
\begin{itemize}
\item Ivan IV. byl mnohokrát ženatý \ra spory o trůn
\begin{itemize}
\item \textbf{Lžidimitrij I.}, \textbf{Lžidimitrij II.} -- vydávali se za Ivanovy syny
\end{itemize}
\item \lp{1613}{zvolen carem Michal Fjordovič Romanov} (mladý \ra lehko ovladatelný)
\end{itemize}

\subsection{Kultura}
\begin{itemize}
\item architektura
	\begin{itemize}
	\item chrámy -- barevná fasáda
	\item chrám Vasila Blaženého v Moskvě
	\end{itemize}
\item malířství 
\begin{itemize}
	\item ovlivněno Itálií -- biblické výjevy, hierarchická perspektiva
\end{itemize}
\item literatura
	\begin{itemize}
	\item 16. stol. -- polemické spisy
	\item svod ruských letopisů -- vznik jednotné kroniky
	\item dopisy a poselství Ivana IV.
	\item \textbf{Azbukovnik} -- encyklopedie
	\item \textbf{Domostroj} -- o morálce a náboženství
	\end{itemize}
\end{itemize}

\newpage
\section{Vznik habsburské monarchie}
\subsection{První Habsburkové}
\paragraph{Ferdinand I. Habsburský} (1526--1564)
\begin{itemize}
\item na základě nástupnických smluv si dělal nároky na český trůn, podle Zlaté buly sicilské mají ale slovo stavové
\item český trůn byl silně zadlužený \ra zvolen Ferdinand
	\begin{itemize}
	\item zvolen za podmínky že nebude omezovat stavy, \ldots
	\end{itemize}
\item pokus o centralizaci \ra odboj
	\begin{itemize}
	\item ignoroval zemské úřady
	\item \lp{1546}{svolání zemské hotovosti} -- šlechta je povinna poskytnout vojsko, poskytnuto velmi málo, většinu musel zaplatit sám
	\item šmalkaldská válka
	\item \lp{březen 1547}{protestantská šlechta a města chtějí omezit moc panovníka}
	\end{itemize}
\item \lp{srpen 1547}{Bartolomějský sněm v Praze}
	\begin{itemize}
	\item před sněmem k zastrašení stavů popraveni dva páni a dva měšťané
	\item král svolává sněmy a obsazuje zemské úřady
	\item královská města ztratila hlas na zemském sněmu \ra přestala být samostatným politickým subjektem
	\item dědicové mohou být přijati za krále za vlády předchůdce \ra možnost ovládání volby svých nástupců
	\item městům omezena samospráva, nesmí mít armádu
	\end{itemize}
\item perzekuce jednoty bratrské
\item rekatolizace -- jezuité (katolíků bylo jen 10\%)
\end{itemize}

\paragraph{Nástup Habsburků v Uhersku}
\begin{itemize}
\item střední a drobná šlechta pro \textbf{Jana Zápolského} -- vedl úspěšné boje s Turky
\item \textbf{magnáti} = vyšší šlechta soustředěna kolem Marie Uherské \ra pro jejího bratra \textbf{Ferdinanda Habsburského}
\item oba korunováni uherským králem \ra vnitřní válka
\item \lp{1540}{Jan Záposlký umírá}
\item Sülejman I. "hájil zájmy" syna J. Z.
\item Ferdinand podporován císařem Karlem V.
\item \lp{1541}{poraženi Habsburkové u Budína}
	\begin{itemize}
	\item Turci obsadili většinu Uher (do 1547)
	\item Habsburkům zůstalo Slovensko
	\end{itemize}
\item \lp{1547}{Uhersko rozděleno}
	\begin{itemize}
	\item Budínský pašalík
	\item Sedmihradské knížectví
	\item Uherské království
	\end{itemize}
\end{itemize}

\paragraph{Maxmilián II.}(1564--1576)
\begin{itemize}
\item ovlivněn lutheránstvím \ra nedůvěra katolíků a papeže 
\item neúspěšný pokus o zisk polského trůnu od vymřelých Jagelonců
\item \lp{1575}{česká konfese}
	\begin{itemize}
	\item spojení novoutrakvistů, českých bratrů a lutheránů
	\item stvořili nové vyznání a vyžadovali jeho uznání
	\item pán nesmí nutit své poddané ke své víře (x Ausburský mír)
	\end{itemize}
\item \lp{1575}{účast na českém zemském sněmu}
	\begin{itemize}
	\item chtěl uznat syna Rudolfa za nástupce, schválit mimořádné daně na posílení koruny
	\item konfesi potvrdil ústně
	\item ve Vídni konfesi odvolal
	\end{itemize}
\end{itemize}

\subsection{Rudolf II.}(1576--1611)
\begin{itemize}
\item \textbf{španělská strana}
	\begin{itemize}
	\item katolická
	\item \textbf{Jaroslav Bořita z Martinic}, \textbf{Vilém Slavnata z Chlumu}, Karel z Lichtenštejna, Albrecht z Valdštejna 
	\end{itemize}
\item \textbf{stavovská opozice}
	\begin{itemize}
	\item 90\%
	\item Karel st. ze Žerotína, Václav Budovec z Budova, Petr Vok z Rožmberka
	\end{itemize}
\item \lp{1583}{Rudolf přesídlil do Prahy} \ra posílení českých katolíků 
	\begin{itemize}
	\item aby se zbavil své matky
	\item stavební a umělecký rozvoj Prahy
	\item přesídlování učenců a diplomatů do Prahy
	\item 70 alchymistů (John Dee, Edward Kelly, Tadeáš Hájek z Hájku (botanik, astronom), Johannes Kepler (matematik), Tycho de Brahe (astronom))
	\item Rudolfovy sbírky -- část ve Vídni, velká část ukradena Švédy 
	\item malíři: Bartolomeus Spranger, Hans von Aachen, Arcimboldo
	\item sochař: Adrien de Vries
	\end{itemize}
\end{itemize}


\paragraph{Uhersko}
\begin{itemize}
\item \lp{1593--1606}{Uhersko pleněno Turky}
\item Rudolf II. se pokouší o protireformaci
\item \lp{1604}{1. protihabsburký odboj} -- podporován Turky
\item pověřil bratra Matyáše aby vyjednal mír s Uherskem
\item \lp{1606}{Matyáš vyjednal mír pro Rudolfa nevýhodný}
\item \lp{1607--1608}{odboj rakouských a moravských stavů} -- chtějí raději vládu Matyáše
\item \lp{25.6. 1608}{Libeňský mír}{zůstaly mu Čechy}
\item \lp{9. 7. 1609}{\textbf{Majestas Rudolfina}}
	\begin{itemize}
	\item potvrzuje českou konfesi
	\item 30 defenzorů -- dodržují na dodržování konfese (ze všech stavů)
	\end{itemize}
\end{itemize}


\paragraph{Konec vlády Rudolfa II.}
\begin{itemize}
\item \lp{1611}{vpád pasovských do Čech}
	\begin{itemize}
	\item požádal o pomoc bratrance \ra chtěl vojensky dosáhnout odvolání Majestas Rudolfina
	\end{itemize}
\item \lp{1611}{Rudolf II. abdikoval}
\item \lp{1612}{v Praze zemřel} -- pohřben v chrámu sv. Víta
\end{itemize}

\section{Třicetiletá válka (1618--1648)}
\subsection{Válka česká -- České stavovské povstání (1618--1620)}
\paragraph{Matyáš}(1611--1619)
\begin{itemize}
\item \lp{1612}{Matyáš korunován císařem římským}
\item nevládnul v praze \ra vláda místodržících
\item nedodržoval konfesi -- protireformační politika
\item \lp{1617}{Oňatova smlouva}
	\begin{itemize}
	\item Španělští Habsburkové se vzdaly nároku na středoevropské državy \ra kandidátem Ferdinand Štýrský (Habsburk)
	\end{itemize}
\item \lp{1617}{zemský sněm -- Ferdinand Štýrský přijat za budoucího krále} -- domnívali se, že ho ještě mohou odvolat
\item porušování Majestátu
	\begin{itemize}
	\item \lp{březen 1618}{svoláni defenzoři do prahy}
	\item sněm zakázán Matyášem \ra \lp{21.5.}{druhý sjezd v Karolínu} (radikálové)
	\item[\ra] za porušování majestátu mohou místodržící
	\item \lp{23.5.1618}{\textbf{3. česká defenestrace}}
		\begin{itemize}
		\item J. Bořita, V. Slavata, Fabricius (písař) vyhozeni
		\item spadli na svah \ra nezemřeli
		\end{itemize}
	\end{itemize}
\end{itemize}

\paragraph{Průběh povstání}
\begin{itemize}
\item cíl odboje -- přerozdělení moci ve prospěch nekatolíků
\item \lp{24. 5. 1618}{svržen král}
	\begin{itemize}
	\item zemská vláda 30 direktorů
	\item Václav Vilém z Roupova, Jindřich Matyáš Thurn
	\end{itemize}
\item připojily se Lužice a Slezsko
\item Morava neutrální (Karel st. ze Žerotína)
\item \lp{květen 1619}{brněnský převrat} -- Ladislav Velen ze Žerotína
\item chyby -- spoléhají na zahraniční pomoc (Unie, Jakub I., Nizozemí, Uhersko)
\item \lp{20. 3. 1619}{zemřel Matyáš}
\item tažení na Vídeň
\end{itemize}

\paragraph{\lp{31. 7. 1619}{Generální sněm zemí Koruny české}}
\begin{itemize}
\item konfederace Čech, Moravy, Slezska, Lužicí (všechna jsou rovnoprávná)
\item nová ústava 
\begin{itemize}
	\item pokroková \ra chtějí ji i Rakouští stavové
	\item sesazen Ferdinand II.
	\item[\ra] \lp{26. 8. 1619}{zvolen kurfiřt Fridrich Falcký}
	\item \lp{28. 8. 1619}{Ferdinand II. zvolen císařem římským} (1619--1637)
\end{itemize}
\item \lp{leden 1620}{připojení uherské opozice}(Gábor Bethlen)
\end{itemize}

\paragraph{Obrat v povstání}
\begin{itemize}
\item \lp{1620}{Maxmilián Bavorský nabídl pomoc císaři} (chtěl se stát kurfiřtem místo F. Falckého)
\item \lp{8. 11. 1620}{bitva na Bílé hoře}
	\begin{itemize}
	\item císařské vs. české vojsko
	\item obojí žoldnéřské, podobně velké, ale české špatně placené
	\item ústup směrem k Praze na Bílou horu \ra císařští obsadili Prahu
	\item stavové bitvu prohráli
	\item zatčeni členové opozice
	\item konfiskace majetku nekatolické odbojné šlechty -- $\frac{3}{4}$ pozemků \ra skupováno německou a italskou šlechtou
	
	\end{itemize}
\item \lp{21. 6. 1621}{staroměstská exekuce} -- 3 páni, 7 rytířů, 17 měšťanů
\item násilná rekatolizace (univerzita pod, dohledem jezuitů, Majestát zrušen)

\item \lp{1624}{vypovězeni nekatoličtí duchovní}
\item \lp{1627 Č, 1628 M}{vypovězeni nekatoličtí stavové} (150 000 emigrantů)
\item \lp{1627 Č, 1628 M}{Obnovené zřízení zemské}
	\begin{itemize}
	\item český trůn dědičně Habsburkové
	\item panovník je nejvyšší hlavou moci zákonodárné a soudní
	\item úředníci jsou odpovědni jen králi
	\item katolické = jediné movolené náboženství
	\item němčina zrovnoprávněna s češtinou (postupně vytlačuje)
	\item vláda má sídlo ve Vídni
	\item položeny základy absolutismu
	\end{itemize}
\end{itemize}

\paragraph{Kultura doby předbělohorské}
\begin{itemize}
\item jazyk a literatura -- Jan Blahoslav, Jiří melantrich z Aventina, Daniel Adam z Veleslavína, Pavel Skála ze Zhoře, Pavel Stránský, Václav Vratislav z Mitrovic, literátská bratrstva
\item věda -- Tadeáš Hájek z Hájku, Václav Hájek z Libočan
\item hudba -- Kryštof Harant z Polžic a Bezdružic
\end{itemize}

\subsection{Evropský kontext války}
\paragraph{Příčiny}
\begin{itemize}
\item náboženské
\item hospodářské -- nárok Habsburků na středoevropské državy
\item společenské -- absolutismus
\end{itemize}

\paragraph{2 tábory}
\begin{itemize}
\item katolický -- španělští a rakouští Habsburkové + papež, podpora Polska
\item protestantský -- Nizozemí, Anglie, Francie, Dánsko, Švédsko
\end{itemize}

\paragraph{3 tábory v SŘŘ}
\begin{itemize}
\item kalvinistická unie -- kuřfiřt Fridrich Falcký (1608)
\item katolická liga -- Maxmilián Bavorský (1609)
\item luteráni -- saský kuřfiřt Jan Jiří -- přiklánění k vedoucí straně
\end{itemize}

\subsection{II. válka falcká}(1621--1623)
\begin{itemize}
\item císařská vojska (Johann Tilly) obsadila se španělskou pomocí Falc
\item Morava -- Jan Jiří Krnovský + Valaši
\item uherský odboj -- Gabriel Bethlen
\item \lp{1622}{Bethlen uzavřel mír s císařem}
\item \lp{1623}{Bavorsko získalo kuřfiřstkou hodnost}
\end{itemize}

\subsection{III. válka dánská}(1625--1629)
\begin{itemize}
\item nekatoličtí stavové: Nizozemsko + Anglie, Dánsko, severní německo, F. Falcký, G. Bethlen
\item \lp{1625}{haagská koalice} -- vedení Kristián IV.
\item Albrecht z Valdštejna $\times$ Kristián IV.
	\begin{itemize}
	\item Albrecht sestavil na vlastní náklady armádu pro císaře
	\item duben 1626 -- Albrecht poražen u Dessau
	\end{itemize}
\item srpen 1626 J. T. Tilly porazil Kristiána IV. u lutteru
\item \lp{1627}{Valdštejn zlikvidoval zbytky protihabsburského vojska} na Moravě a Slezsku
\ra vrchním velitelem císařské armády (Meklenbursko)
\item \lp{1629}{resttituční} edikt -- navrácení majetku katolické církvi
\item \lp{22. 5. 1629}{mír v Lübecku}
\item s\lp{rpen 1629}{Valdštejn propuštěn z císařských služeb}
\end{itemize}

\subsection{IV. válka švédská (1630--1635)}
\begin{itemize}
\item \lp{1630}{Gustav II. Adolf se vylodil v Pomořanech}
\item \lp{1631}{spojenci Švédska: Sasko, Braniborsko}
\item \lp{září 1631}{bitva u Breitenfeltu} (poražen Tilly)
\item Švédové vpadli do Bavorska
\item \lp{listopad 1631}{Sasové obsadili severní Čechy a Prahu} \ra dočasný návrat exilantů do Čech
\item \lp{1632}{Tilly padl u Rainu}
	\begin{itemize}
	\item \lp{1632}{povolán Valdštejn} (pravomoci, nová armáda)
	\item Sasové vytlačeni z Čech
	\end{itemize}
\item \lp{16.11.1632}{bitva u Lützenu} -- padl Gustav II. Adolf
	\begin{itemize}
	\item královna Kristina -- vzdělaná 
	\item rádce generál Axel Oxenstierna [Oxnšterna]
	\end{itemize}
\item \lp{25. 2. 1634}{Valdštejn zavražděn v Chebu} -- nedůvěryhodný
\item \lp{září 1634}{Švédové poraženi u nördlingenu}
\item Sasové začali vyjednávat o míru
\item \lp{30.5.1635}{pražský separátní mír}
	\begin{itemize}
	\item separátní -- pouze mezi císařem a Sasy, neúčastnily se všechny dotčené strany
	\item nelegální podle zlaté buly Karal IV.
	\end{itemize}
\item Sasko získalo Lužice
\item \lp{1635}{mír Braniborsko}
\end{itemize}

\subsection{V. válka švédsko-francouzská} (1635--1648)
\begin{itemize}
\item \lp{květen 1635}{Richelieu vypověděl válku Španělsku}
\item spojenci císaře -- Sasko, Braniborsko, Dánsko
\item \lp{leden 1639}{Švédové vpadli do severního Německa}
	\begin{itemize}
	\item \lp{1642}{Leonard Torstenson vpadl do Slezska a na Moravu}
	\item \lp{1643}{1. obléhání Brna}
	\end{itemize}
\item \lp{1645}{Torstenson zvítězil u Jankova}
\item \lp{4.5 -- 23. 8. 1645}{2. obléhání Brna}
	\begin{itemize}
	\item Švédové drancovali okolní vesnice
	\item obránce Brna -- Radouit de Souches [radui de suš]
		\begin{itemize}
		\item Francouz, kalvínista
		\end{itemize}
	\item páter Martin Středa (jezuita, mobilizoval koleje)
	\item generál Ogilvy (skot)
	\item několikrát výpravy pro zásoby potravin
	\item zvonění v 11. hodin
	\end{itemize}
\item Ferdinanad III. (1637--1657)
\item od 1644 mírová jednávní ve Vestfálsku -- Můnster, Osnabrück
\item dopis J. A. Komenského švédskému kancléři A. Oxenstiernovi
	\begin{itemize}
	\item srpen 1648 vpád Švédů (gen. Königsmark) do Prahy
	\end{itemize}

\paragraph{\lp{24. 10. 1648}{Vestfálský mír}}
\begin{itemize}
\item situace vrácena do roku 1624
\item císař oslaben v ŘŘ -- zakázán absolutismus
\item uznání území Sasku (Lužice) a Braniborsku
\item Francie -- zisk Alsaska (až k Rýnu)
\item Švédsko -- ústí řeky Odry, Labe, Vesery
\item odvolán restituční edikt
\item Nizozemí a Švýcarsko samostatné
\item Německo a Itálie nadále rozdrobené
\end{itemize}

\paragraph{Důsledky války}
\begin{itemize}
\item úbytek cca 1/3 obyvatel (zůst. Čechy -- 1mil., Morava, Slezsko -- 1/2 mil.)
\item šlechta získala zabavené pozemky \ra potřeba obdělávat \ra němečtí imigranti
\item málo lidí \ra zvyšování daní \ra utužení nevolnictví ("druhé nevolnictví")
\item zvýšení rozdílů ve vývoji západní $>$ střední $>$ východní Evropa
\end{itemize}

\end{itemize}



\section{Buržoazní revoluce v Anglii (1640--1660)}
\begin{itemize}
\item Stuartovci
\end{itemize}

\paragraph{Jakub I. Stuart}(1603--1625)
\begin{itemize}
\item král skotský a anglický
\item konkurence Nizozemců -- dobří v obchodu
\item státní monopoly (sůl, víno, mýdlo)
\end{itemize}

\paragraph{Karel I.}(1625--1649)
\begin{itemize}
\item psychicky a fyzicky slabý
\item spor o moc s parlamentem
\item \lp{1629}{rozpuštěn parlament} \ra absolutismus
\item oporou vysoká šlechta, biskupové, některé obchodní společnosti
\item opozicí (od 30. let) puritáni (angličtí kalvínisté)
\item \lp{1639}{povstání ve Skotsku} (chtěl zavést anglikánskou církev)
\item \lp{1640}{Karel nucen svolat parlament} aby mohl vypsat mimořádné daně na válku se Skotskem
\end{itemize}

\subsection{1. období (1640--1642)}
\begin{itemize}
\item král chtěl využít rozporů mezi poslanci
\item \lp{4. 1. 1642}{Karel I. vtrhl s vojskem do parlamentu} aby zatkl 5 hlavních představitelů
	\begin{itemize}
	\item varováni, parlament se obrátil kvůli králi
	\item[\ra] kvůli bezpečnosti Karel I. přesídlil do Oxfordu
	\end{itemize}
\end{itemize}

\paragraph{Diferenciace puritánů}
\begin{itemize}
\item \textbf{presbitariáni} -- umírnění, pro dohodu s králem, bohatá šlechta a buržoazie
\item \textbf{independenti} -- král se má podrobit parlamentu, střední šlechta, obchodníci a majitelé manufaktur
	\begin{itemize}
	\item v druhé fázi dále rozděleni
	\item \textbf{levelleři} -- politická práva pro "závislé" muže, střední městské a venkovské vrstvy
	\item \textbf{diggeři} -- nejradikálnější, chudina, chtějí politickou a majetkovou rovnoprávnost pro všechny obyvatele
	\end{itemize}
\end{itemize}


\subsection{2. období (1642--1649)}
\begin{itemize}
\item \lp{22. 8. 1642}{Karel I. vyhlásil parlamentu válku}
\item z počátku parlament nevítězil výrazně
\item vedení parlamentních vojsk převzal \textbf{Oliver Cromwell}, zavedl reformy \ra "New Model Army"
	\begin{itemize}
	\item do armády přijímal i nemajetné a zařídil jim výbavu
	\item vojáci byli povyšováni v hodnostech za zásluhy \ra i neurození vojáci měli šanci dosáhnout vysoké pozice -- motivace
	\item všichni vojáci procházeli výcvikem
	\end{itemize}
\item \lp{1644}{bitva u Marston Moor}
\item \lp{1645}{bitva u Naseby}
\item \lp{1647 }{Karel I. uprchl do Skotska}
\item \ra 2. občasnká válka
\item independenti vyloučili presbytariány z parlamentu -- Kusý parlament (do 1653)
\item Soud s králem
\item \lp{30. 1. 1649}{Karel I. popraven}
\end{itemize}

\subsection{3. období (1649--1660)}
\begin{itemize}
\item \lp{19.5.1649}{parlament vyhlásil Commonwealth}
\item potlačno povstání diggerů
\item \lp{1649}{1652 připojeno Irsko}
\item \lp{1651}{navigační akta}
	\begin{itemize}
	\item popisuje, kdo může do anglie dovážet zboží
	\item anglické lodě / lodě ze zemí, kde se zboží vyrobilo
	\item proti Nizozemcům, kteří se živili překupnictvím
	\item podpora buržoazie 
	\end{itemize}
\item \lp{1653}{Cromwell rozehnal parlament}, kvůli zneužívání svého postavení
\item nový parlament \ra Cromwell jmenován lordem protektorem \ra vojenská diktatura
\item \lp{29. 5. 1660}{Karel II. v Anglii} -- konec revoluce
\end{itemize}

\subsection{Restaurace Stuartovců}
\paragraph{Karel II.}(1660--1685)
\begin{itemize}
\item syn Karla I.
\item potrestat zodpovědné za smrt Karla I. (nechal oběsit mrtvého Cromwella)
\item pokus o návrat k absolutismu
\item \lp{1679}{Habeas Corpus Akt}
	\begin{itemize}
	\item presumpce neviny
	\end{itemize}
\item \lp{1666}{požár Londýna}
	\begin{itemize}
	\item umožnění vzniku nových budov
	\end{itemize}
\item ženatý s katolickou portugalskou princeznou
\end{itemize}

\paragraph{Jakub II. (1685--1688)}
\begin{itemize}
\item bratr Karla II.
\item otevřeně chtěl obnovit absolutismus a katolicismus
\item[\ra] všeobecný odpor
\item utekl
\end{itemize}

\subsection{Slavná revoluce (1688--1689)}
\begin{itemize}
\item \lp{30. 6. 1688} -- Vilém III. Oranžský a Marie II. Stuartovna (dcera Jakuba II.) podepsali \textbf{Listinu práv}
\item \lp{13. 2. 1689} -- Vilém III. korunován králem \ra Anglie = konstituční monarchií
\item toryové -- konzervativci
\item wigové -- liberálové
\end{itemize}

\paragraph{Anna Anglická Stuartovna(1702--1714)}
\begin{itemize}
\item dcera Jakuba II.
\item \lp{1707}{sloučen anglický a skotský parlament} \ra \textbf{Království Velké Británie}
\end{itemize}

\subsection{Výsledky a význam}
\begin{itemize}
\item buržoazie se může podílet na vládě
\item ekonomický význam -- předstihnutí Nizozemska, 
\end{itemize}



\section{Absolutismus a parlamentarismus}
\begin{itemize}
\item od středověké stavovské monarchie k novým státním formám
\end{itemize}

\paragraph{Anglie po slavné revoluci}
\begin{itemize}
\item Bill of Rights: rozhodující je Parlament
\item král bez parlamentu nemůže vyhlašovat a rušit zákony, vybírat daně, udržovat vojsko
\item dolní sněmovna -- volby
\item moc zákonodárná -- parlament
\item moc výkonná -- král a ministři
\item volební systém
	\begin{itemize}
	\item toryové -- konzervativci
	\item whigové -- liberálové
	\item vláda závisela na parlamentní většině
	\item menšinová strana -- právo a povinnost kritizovat vládu a hledat lepší řešení \ra parlamentní opozice
	\item částečně zrušena cenzura \ra v novinách se mohou psát i kritické články \ra vznik veřejného mínění  -- důležité pro korigování parlamentu
	\item volební právo -- jen menšina majitelů půdy
	\item tajné právo až v 19. stol.
	\item středověké rozdělení volebních obvodů
	\end{itemize}
\end{itemize}


\paragraph{Jiří I. Hanoverský}(1714--1727)
\begin{itemize}
\item vzdálený z rodu Stuartovců
\item němec 
\item první ministr = zástupce krále, komunikuje s vládou (král neuměl anglicky)
\item první první ministr (1721--1740) -- \textbf{Robert Walpole} (whig) 
\item vznik občanské společnosti, demokracie
\end{itemize}



\section{Francie v 17. až 18. století}
\begin{itemize}
\item klasický absolutismus
\end{itemize}
\subsection{Ludvík XIV.}(1643--1715)
\begin{itemize}
\item do 1661 -- poručník Mazarin (nástupce kardinála Richeliau)
\item posílena moc krále 
\item fronda
\item král rozhoduje o vnitřní i zahraniční politice, má výkonnou, zákonodárnou i soudní moc
\item od 1614 se neschází generální stavy (do 1789 -- VFR)
\item "Stát jsme my" (majestátní plurál)
\item armáda a výboje
	\begin{itemize}
	\item 119 000 mužů, doživtní vojenská služba
	\item moderní zbraně (bajonety, píky, muškety)
	\item 1667--1697 -- výbojné války
		\begin{itemize}
		\item zabrat Nizozemsko, posunout hranice k rýnu
		\item devoluční právo -- dcery z prvního manželství mají přednost v nástupnictví před syny z druhého \ra nároky na Nizozemí -- "devoluční válka"
		\item obecně do válek hodně investoval a moc nezískal
		\item ze začátku podporován buržoazií
		\end{itemize}
	\end{itemize}
\end{itemize}

\paragraph{1701--1713 -- válka o španělské dědictví}
\begin{itemize}
\item kandidáti
	\begin{itemize}
	\item Filip z Anjou (vnuk Ludvíka XIV)
	\item Karel Habsburský
	\item Josef Ferdinand Bavorský
	\end{itemize}
\item 1701 -- velká aliance proti Francii (aby se nemohla spojit se Španělskem)
	\begin{itemize}
	\item Anglie, Prusko, Nizozemsko, Portugalsko, SŘŘ, Savojsko, Hannovesko
	\end{itemize}
\item 1704 -- Britové dobyli Gibraltar
\item 1713 -- mír v Utrechtu -- potvrzeno Britské vlastnictví Gibraltaru
\item 1714 -- mír v Rastattu
	\begin{itemize}
	\item Španělským králem se stal Filip z Anjou
	\item Habsburkové dostali Belgii
	\end{itemize}
\end{itemize}



\paragraph{Nákladná politika}
\begin{itemize}
\item výboje -- neúspěšné \ra ztráta podpory buržoazie
\item nákladný dvůr -- král Slunce (zábavy, divadla, oděvy, \ldots)
\item vystavěn zámek ve Versailles
\item Palais Royal
\item oporou krále církev, úředníci (dobře placení), armáda
\item[\ra] \textbf{špatná ekonomická situace}
\end{itemize}

\paragraph{Merkantilismus}
\begin{itemize}
\item ministr financí \textbf{Jean-Baptist Colbert} [žán batist kolbér]
	\begin{itemize}
	\item z měšťanských řad, vynikající ekonom, zavedl nový systém
	\item stát je bohatý, když má aktivní obchodní bilance (vyváží více zboží než dováží)
	\item[\ra] \textbf{protekcionismus} -- vysoká cla na dovážené výrobky \ra větší odbyt domácího zboží
	\item podpora zakládání \textbf{manufaktur} -- výroba \textbf{drahých} parfémů, látek, gobelínů (nástěnné koberce) \ra nebyly příliš kupovány
	\item budování cest, průplavů (lodní doprava je nejlevnější) pro podporu obchodu
	\item koloniální \textbf{výboje} (Louisiana, Kanada, Madagaskar, Přední Indie, Indonésie; Východo a Západoindická společnost)
	\end{itemize}
\end{itemize}

\paragraph{Náboženská politika}
\begin{itemize}
\item snaha o sjednocení -- ("un roi, une foi, une loi" [in roa, in foa, in loa] = "jeden král, jedna víra, jeden zákon")
\item \lp{1685}{zrušen Edikt nantský} \ra emigrace protestantů \ra ztráta intelektuální a pracovní síly
\end{itemize}

\paragraph{Shrnutí}
\begin{itemize}
\item začátek Ludvíkovy vlády byl úspěšný
\item konec -- Francie před ekonomickým bankrotem
\item do testu: "absolutismus, centralismus, byrokracie"
\end{itemize}

\subsection{Ludvík XV. (1715--1774)}
\begin{itemize}
\item děti Ludvíka XIV. moc umíraly \ra pravnuk
\item mladý \ra regentská vláda Filipa Orleánského
\item neúspěšná snaha o hospodářské reformy \ra pokračování ekonomického úpadku
\end{itemize}




\section{Francouzská hegemonie}
\begin{itemize}
\item hegemonie = převaha
\end{itemize}
\subsection{Situace po 30-leté válce}
\begin{itemize}
\item Francie je nejsilnější mocnost \ra \textbf{expanzivní politika}
\item přirozené hranice -- Pyreneje, Alpy, Rýn
\item zisk strategicky významných míst -- jih. šp. Nizozemí, Alsasko, Štrasburk
\item spojencem Švédsko
\end{itemize}

\paragraph{Střední a východní Evropa}
\begin{itemize}
\item \lp{1683}{Turci oblehli Vídeň}
\item Jan III. Sobieski porazil Mustafa Pašu
\item pomoc prince Evžena Savojského
\end{itemize}

\paragraph{Rakouský protiútok}
\begin{itemize}
\item princ Evžen Savojský
	\begin{itemize}
	\item \lp{1686}{dobyl Budín}
	\item \lp{1697}{porazil u Zenty Mustafu II.}
	\item \lp{1699}{mír -- Turci se zřekli vlády v Uhrách}
	\end{itemize}
\end{itemize}

\paragraph{Válka o španělské dědictví}
\begin{itemize}
\item 1701 -- 1703
\item Habsburkové vs. Bourboni
\item 1713 -- Utrechtský mír
\item Šp. králem Filip z Anjou
\item Habsburkové -- J. Nizozemí, Neapolsko, Milánsko
\item \textbf{pragmatická sankce} (1713)
	\begin{itemize}
	\item Karel VI.
	\item pokud vymře dynastie po meči, dědičná práva dostane ženská větev
	\end{itemize}
\item Anglie -- Gibraltar, obchod se šp. koloniemi
\item konec francouzské převahy
\end{itemize}



\section{Východní Evropa po 30leté válce}
\subsection{Rusko v 17. -- 18. století}
\paragraph{Expanze}
\begin{itemize}
\item zájem o pobaltí (kvůli zamrzání Archangelsku)
\item boj o Ukrajinu s Polskem 
	\begin{titlepage}
	\item kozáci -- svobodní obyvatelé jihoruských stepí
	\item \lp{od 1648}{odboj kozáků proti Polákům}, spojení s Ruskem
	\item \lp{1667}{hranicí Dněpr}
	\end{titlepage}
\item \lp{1679}{ovládnuta Kamčatka \ra čínské hranice}
\item \lp{1689}{smlouva v Něrčinsku} -- stanovena pevná hranice mezi Ruskem a Čínou
\end{itemize}

\subsubsection{Petr I. Veliký}
\begin{itemize}
\item snaha překonat zaostalost Ruska
\item studijní cesta do západní Evropy (inkognito pracoval v manufaktuře)
\item usoudil nutnost reforem a přístupu k Baltu
\end{itemize}

\paragraph{Severní válka (1700 -- 1721)}
\begin{itemize}
\item protišvédská koalice -- Sasko, Polsko, Dánsko
\item \lp{1700}{porážka Rusů u Narvy} (v jejím důsledku reformy)
\item \lp{1703}{založen Petrohrad} v ústí řeky Něvy
	\begin{itemize}
	\item zamýšlena jako vojenská pevnost
	\item od roku 1712 hlavním městem
	\end{itemize}
\item \lp{1708-09}{Karel XII. táhl do Ruska}
\item \lp{1709}{bitva u Poltavy} -- obrat ve válce
\item \lp{1721}{\textbf{Nystädský mír}}
	\begin{itemize}
	\item Rusko získalo Rižského a Finského zálivu (dnešní Estonsko a Lotyšsko)
	\end{itemize}
\end{itemize}

\paragraph{Reformy Petra I.}
\begin{itemize}
\item armáda, baltské loďstvo
\item hospodářské reformy
	\begin{itemize}
	\item vznik nevolnických manufaktur
	\item stavba lodí \ra zisk \ra prodej velmocem
	\item těžba železa v Urale
	\end{itemize}
\item politické reformy
	\begin{itemize}
	\item vznik senátu
	\item území rozděleno na gubernie
	\end{itemize}
\item kulturní reformy
	\begin{itemize}
	\item přivezena ze západu 1. tiskárna (zničena církví)
	\item tištěny první noviny
	\item přivážena a překládána technická literatura ze západu
	\item \lp{1725}{založena Akademie věd}
	\item \lp{1755}{Založena Univerzita v Moskvě}
	\end{itemize}
\end{itemize}

\subsubsection{Kateřina II. Veliká (1762--1796)}
\begin{itemize}
\item po smrti Petra I. zmatky ve vládě
\item nástup po \textbf{palácovém převratu} (vraždě Petra III.)
\item reorganizace říše, navázala na Petra I.
\item 1773--1775 -- poraženo povstání kozáků pod Jemeljana Pugačova (Ukrajina--Ural)
\item úspěšné boje s Turky \ra ovládnut Krym a černomořské pobřeží
	\begin{itemize}
	\item přístup jak k Baltu, tak k Černému moři
	\end{itemize}
\item \textbf{Grigorij Alexandrovič Potěmkin} -- rádce a milenec, významný vojevůdce
	\begin{itemize}
	\item nechal domy a cesty vylepšit podél inspekční cesty \ra "Potěmkinova vesnice"
	\end{itemize}
\item korespondence s francouzskými osvícenci
	\begin{itemize}
	\item v praxi nepoužila osvícenské reformy, protože se domnívala, že by je zaostalé Rusko nepochopilo
	\end{itemize}
\end{itemize}

\subsection{Vznik Pruska}
\paragraph{Celky, ze kterých Prusko vzniklo}
\begin{itemize}
\item \textbf{Braniborsko} -- 1415 koupili od Zikmunda Lucemburského Hohenzollerové
\item \textbf{Východní Prusko} -- řádové území, velmistr řádu Albrecht Hohenzollern r. 1525 přistoupil k luterství (1618)
\item \textbf{Západní Pomořany} -- připojeny r. 1648 k Braniborsku
\end{itemize}

\paragraph{Velký kurfiřt}
\begin{itemize}
\item \textbf{Friedrich Vilém} (1640--1688)
\item reformy armády
\item panovnický absolutismus
\item chytrá diplomacie za 30leté války
\item základy pro vznik prusko-braniborského státu
\item syn Fridrich I.
\end{itemize}

\paragraph{Fridrich I.}
\begin{itemize}
\item ve válce o španělské dědictví se postavil na stranu Habsburků
\item za odměnu byl korunován králem od Leopolda I.
\item \lp{1701}{první pruský král}
\item podpora umění a věd
	\begin{itemize}
	\item \lp{1694}{založena univerzita v Halle}
	\item \lp{1696}{založena Akademie umění v Berlíně}
	\item \lp{1700}{založena Společnost věd}
	\end{itemize}
\item přestavba Berlína
\end{itemize}

\paragraph{Fridrich Vilém I.}
\begin{itemize}
\item ne příliš inteligentní
\item spořivý, nepodporoval vědu a kulturu
\item financování silné armády
	\begin{itemize}
	\item moderní výzbroj a výstroj
	\item branná povinnost -- možnost se vykoupit (šlechta)
	\item tvrdý výcvik -- tvrdé fyzické tresty (kůl, ulička, zasypání)
	\end{itemize}
\item junkeři -- šlechtic, který je feudálem a zároveň významným důstojníkem v armádě
\item zavedena povinná školní docházka (gramotnost vojáků)
\item armáda pouze na obrannou funkci
\end{itemize}

\paragraph{Fridrich II. Veliký}
\begin{itemize}
\item \lp{1740}{dohoda s Bavorskem a Francií}
\item válka o habsburské dědictví
	\begin{itemize}
	\item dvě války slezské (1740--42, 1744--45) \ra zisk Slezska
	\item sedmiletá válka (1756--1763)
		\begin{itemize}
		\item Prusko + Anglie $\times$ Rakousko + Francie + Rusko
		\item 1763 -- definitivní zisk Slezska
		\end{itemize}
	\end{itemize}
\item vnitřní politika: osvícenský panovnický absolutismus
	\begin{itemize}
	\item merkantilismus
	\item \textbf{Všeobecné pruské právo} -- nový právní systém
	\item náboženská tolerance -- příchod intelektuálů z Francie (zrušený edikt nantský)
	\item tvrdý postih neposlušnosti
	\item podpora věd a umění
	\item Ch. Wolf (vytvořil německé filozofické názvosloví), La Mettrie, Voltaire, I. Kant
	\end{itemize}
\item \textit{"Filosof ze Sanssouci"} (zámek podle vzoru Versailles)
\end{itemize}

\paragraph{Fridrich Vilém II.}(1786--1797)
\begin{itemize}
\item 2. a 3. dělení Polska
\end{itemize}

\paragraph{Fridrich Vilém III.}(1797--1840)

\subsection{Rozpad Polska}
\paragraph{1. dělení Polska}
\begin{itemize}
\item navrhl Filip II. Veliký (Prusko)
\item Prusko: vojvodství Pomořanské, Malborské, Chelmiňské a menší části Velkopolska
\item Rakousko: Halič a část vojvodství Sandoměřského, krakovského
\item Rusko: část Běloruska a Livonska
\item ztráta 30\% území
\item \lp{1791}{Sejm vypracoval ústavu} podle francouzského vzoru
\item \lp{1792}{útok Ruska}
\end{itemize}

\paragraph{2. dělení Polska (1793)}
\begin{itemize}
\item Rusko: oblast Vilna, Minsku, Ukrajina
\item Prusko: zbytek Velkopolska, část Kujavska a Mazovska, Gdaňsk
\item Krakov -- povstání šlechty 
	\begin{itemize}
	\item Tadeusz Kosciuszko ovládli Varšavu
	\item poraženi ruskou armádou
	\end{itemize}
\end{itemize}

\paragraph{3. dělení Polska (1795)}
\begin{itemize}
\item Rusko: Litva, Bělorusko, Ukrajina
\item Prusko: Velkopolsko s Varšavou až po Němen
\item Rakousko: Malopolsko s Krakovem, Halič, Bug
\item Polsko zaniká, obnoveno až po 1. světové válce 
\end{itemize}


\section{České země po 30leté válce (1648--1740)}
\begin{itemize}
\item Ferdinand III. (1637--1657)
\item Leopold I. (1657--1705)
\item Josef I. (1705--1711)
\item Karel VI. (1711--1740)
\end{itemize}

\paragraph{Vláda a správa}
\begin{itemize}
\item nejvyšší kancléř království českého (Vídeň)
\item zemská vláda
\item zemští úředníci jsou zodpovědní jen králi
\item snaha o centralizaci
\item ztráta Horní a Dolní Lužice
\end{itemize}

\paragraph{Náboženské poměry}
\begin{itemize}
\item netolerance nekatolických vyznání
\item \lp{1651}{soupis obyvatel podle vyznání}
\item rekatolizace
	\begin{itemize}
	\item nová biskupství -- Hradec Králové, Litoměřice
	\item 1729 -- svatořečení Jana Nepomuckého
	\item působení jezuitů
	\end{itemize}
\end{itemize}

\paragraph{Sociální a ekonomické poměry}
\begin{itemize}
\item úbytek obyvatelstva o 1/3 
\item přísnější přehledy rustikálu (poddanské půdy)
\item právní normy \ra utužení nevolnictví
\item nevolnictví povstání
\item \lp{1680}{Čechy, Plzeňsko, Ovčí vrch u Čeliva}
\item \lp{1680}{robotní patent} Leopolda I.
	\begin{itemize}
	\item robotník je povinen robotovat 3 dny v týdnu
	\end{itemize}
\item \lp{1692}{Chodové x Volf Maxmilián Lamingen z Albenreuthu}
	\begin{itemize}
	\item Chodové chápali patent jako úplné osvobození
	\item poraženo
	\item vůdci zatčeni a potrestáni
	\item \lp{1695}{popraven Jan Sladký Kozina}
		\begin{itemize}
		\item "Lomikare, Lomikare, ..."
		\item Lamingen doopravdy po roce zemřel (pravděpodobně strachem že se prokletí vyplní)
		\end{itemize}
	\end{itemize}
\item počátky průmyslu
	\begin{itemize}
	\item podniká šlechta 
		\begin{itemize}
		\item města jsou omezena privilegii
		\item cechy bránily vývoji
		\end{itemize}
	\item sklářství, vlnařství, soukenictví
	\item manufaktury: Kounicové (Slavkov), Valdštejnové (Horní Litvínov)
	\item česká a moravská šlechta ve státních službách 
	\end{itemize}
\end{itemize}

\paragraph{Školství a výchova}
\begin{itemize}
\item jezuitské koleje
\item piaristická gimnázia
\item 1654 -- Karlo-Ferdinandova univerzita
	\begin{itemize}
	\item čeština se vůbec nevyučovala
	\end{itemize}
\item barokní kultura -- vrchol barokního umění na světě
\item zemský patriotismus -- vlastenectví šlechty (vlast je tam, kde mám majetek)
\item po Bílé hoře konec náboženské tolerance
\item úpadek českého jazyka (poněmčování slov, používání německých slov \ra hantec)
	\begin{itemize}
	\item udržován na venkově (Hájkova kronika česká)
	\end{itemize}
\end{itemize}

\section{Uhersko po 30leté válce}
\paragraph{Stavovská povstání}
\begin{itemize}
\item v Uhersku se nikdy nepovedlo Habsburkům prosadit absolutismus (odpor šlechty a poddaných)
\item náboženské otázky nedůležité -- tolerance (\ra imigrace protestantů)
\end{itemize}

\paragraph{Povstání Imricha Tökölyho}
\begin{itemize}
\item \lp{1678}{vzniklo u Mukačeva}
\item obsazeno území Horních Uher až po Váh
\item využívají Turci -- postup na Vídeň (1683)
\item 1687 prešovské jatky
	\begin{itemize}
	\item veřejná poprava 24 osob jako trest za povstání (především popraveni měšťané za zločin šlechty)
	\end{itemize}
\end{itemize}

\paragraph{Povstání Františka Rákoczyho II.}
\begin{itemize}
\item syn Imricha Tökölyho
\item \lp{1703}{počátek povstání}
\item Habsburkové sesazeni z trůnu
	\begin{itemize}
	\item volba Maxmiliána Emanuela Bavorkského za nového krále
	\item spojenece Ludvíka XIV.
	\item očekávali pomoc Francie a Švédska
	\end{itemize}
\item \lp{1708}{bitva u Trenčína}
	\begin{itemize}
	\item Francie vyčerpána bitvou o Španělské dědictví, Švédové zastaveni u Poltavy \ra neměli podporu
	\item \lp{1711}{Szatmárský mír 1711}
		\begin{itemize}
		\item zaručoval beztrestnost povstalcům
		\item potvrzeny náboženské svobody
		\item daně musí být schváleny uherským sněmem
		\item na oplátku trůn vrácen Habsburkům
		\item první kroky k osamostatnění
		\item nevolníci \ra zbojníci (Juraj Jánošík -- 1713 popraven)
		\end{itemize}
	\end{itemize}
\end{itemize}

\section{Osvícenský absolutismus v habsburské monarchii}
\subsection{Války o španělské dědictví}
\begin{itemize}
\item ztráta nároku středoevropských Habsburků na Španělsko
\item zůstala jim Belgie a území v Itálii
\item Karel VI. -- pragmatická sankce (1713) -- trůn se dědí i přes ženskou linii
\item synové Karla VI. zemřeli jako děti \ra dynastie vymírá po meči \ra Marie Terezie
\end{itemize}

\subsection{Marie Terezie (1740--1780)}
\begin{itemize}
\item provdána za Františka Štěpána Lotrinského (nevýhodné, Lotrinsko vzdáno Francii \ra velkovévoda Toskánský)
\item Fridrich II. Veliký
\end{itemize}

\paragraph{Slezské války}
\begin{itemize}
\item vs. Prusko + Bavorsko, Sasko, Francie
\item 1. 1740--1742
\item 2. 1744--1745
\item ztráta Slezska 
\item korunována královnou českou
\item František Štěpán císařem římským (Marie Terezie nebyla nikdy korunována císařovnou)
\end{itemize}

\paragraph{Sedmiletá válka}
\begin{itemize}
\item spojenci Rakouska: Francie, Rusko, Švédsko, Sasko
\item spojenci Pruska (Fridrich II. Veliký): Anglie
\item vpád Prusů do Saska, Čech (1756, 1757)
\item bitva u Kolína 18. 6. 1757 (L. J. Daun)
\item carevna Alžběta zemřela \ra car Petr III.
\item mír v Hubertsburgu (1763) -- Rak. s Pruskem
\item mír v Paříži -- Francie s Anglií
\item definitivní ztráta Slezska
\end{itemize}

\paragraph{Centralizační reformy}
\begin{itemize}
\item snaha o pevné spojení Českých zemí s Rakouskem -- nejdůležitější země monarchie
\item snaha o germanizaci celé monarchie
\item správní:, 
	\begin{itemize}
	\item  Direktorium pro věci veřejné a komorní -- vláda
	\item Dvorská státní kancelář -- kancléřem -- hrábě Václav Antonín Kounic
	\end{itemize}
\item právní:
	\begin{itemize}
	\item nejvyšší soudní dvůr ve Vídni
	\item trestní zákoník (1769)
	\item zákaz mučení (1776)
	\item zákaz procesů o čarodějnictví
	\end{itemize}
\item berní:
	\begin{itemize}
	\item Tereziánský katastr (1748)
	\item sčítání obyvatelstva (1754) -- od 1762 každý rok
	\end{itemize}
\item ekonomické
	\begin{itemize}
	\item sjednocení míry, váhy, měny, podpora manufaktur, stavba silnic
	\end{itemize}
\item školství
	\begin{itemize}
	\item povinná školní docházka (1774)
	\item školy triviální (v mateřském jazyce), hlavní (města), normální (větší města), přípravky pro učitele 
	\item industriální školy
	\item univerzita vyňata z dohledu církve
	\item hlavním vyučovacím jazykem je němčina
	\end{itemize}
\item vojenské:
	\begin{itemize}
	\item 7-letá vojenská povinnost pro všechny muže (17--40 let)
	\item vojenská akademie
	\end{itemize}
\item sociálně ekonomické
	\begin{itemize}
	\item 1775 -- povstání sedláků na Náchodsku (poraženo u Chlumce)
	\item[\ra] \textbf{robotní patent podle majetku}
		\begin{itemize}
		\item čím více majetku, tím více roboty
		\item nejvíc -- tři dny týdně
		\item nejméně -- 13 dní ročně
		\end{itemize}
	\item 1777 -- F. A. Raab -- placená robota
	\end{itemize}
\end{itemize}

\subsection{Josef II. (1780--1790)}
\begin{itemize}
\item dynastie habsbursko-lotrinská
\item spoluvládce Marie Terezie od 1765
\item 1773 -- zrušen jezuitský řád \ra ruší "neprospěšné" řády \ra vznik kasáren, chudobinců, nemocnic
\item nová biskupství: Brno (1777), České Budějovice (1785)
\item prosazování germanizace
\item zlepšeno posílení Židů
\item \lp{13. 10. 1781}{\textbf{Toleranční patent}} --luterství, kalvinismus, ortodoxní víra
\item \lp{1. 11. 1718}{\textbf{Patent o zrušení nevolnictví}} (robota zrušena až o století a půl později)
	\begin{itemize}
	\item zisk osobní svobody
	\item možnost svatby, poslání dětí do školy, etc.
	\end{itemize}
\item vybudoval pevnosti Terezín a Josefov
\end{itemize}


\subsection{Leopold II. (1790--1792)}
\begin{itemize}
\item vévoda toskánský
\item pokračoval v politice Josefa II.
\item ústup osvícenství
\end{itemize}

\section{Baroko a rokoko}
\begin{itemize}
\item poslední dva ucelené slohy
\end{itemize}
\paragraph{Společnost}
\begin{itemize}
\item Evropa nábožensky rozdělena
	\begin{itemize}
	\item rekatolizace: české země, Falc, Polsko, Francie
	\item jansenismus -- C. Jansen; Ypres; klášter Port Royal; Blaise Pascal	
		\begin{itemize}
		\item v katolických zemích
		\item "Srdce má své důvody, o kterých rozum neví" -- Blaise Pascal
		\end{itemize}
	\item pietismus 
		\begin{itemize}
		\item v protestantských zemích
		\item důraz na zbožnost, charitu, školství
		\end{itemize}
	\end{itemize}
	\item francouzská literatura -- dominantní
		\begin{itemize}
		\item 
		\end{itemize}
\end{itemize}

\paragraph{Rozvoj vědy}
\begin{itemize}
\item empirický výzkum + teorie + experiment
\item astronomie: G. Galielei
\item fyzika: B. Pascal, Giovanni Torricelli, I. Newton
	\begin{itemize}
	\item Newton -- 1687 objev gravitačního zákona
	\item Pascal -- vymyslel první počítací stroj
	\end{itemize}
\item fyziologie: W. Harvey, A. van Leevenhoek (mikroskop)
\end{itemize}

\paragraph{Filosofie}
\begin{itemize}
\item empirismus
	\begin{itemize}
	\item F. Bacon -- "vědění je moc"
		\begin{itemize}
		\item indukce -- od jednotlivosti ke zobecnění
		\item Atlantida -- nedopsané filosofické dílo
		\end{itemize}
	\item T. Hobbes
		\begin{itemize}
		\item teoretik státního absolutismu
		\item sát je společenská smlouva mezi lidmi
		\item "homo homini lupus" = člověk člověku vlkem -- stav společnosti bez státu
		\end{itemize}
	\end{itemize}
\item osvícenství
	\begin{itemize}
	\item John Locke
		\begin{itemize}
		\item důležitá přirozená práva člověka
		\item oddělení moci výkonné a zákonodárné
		\item lidské vědomí je nepopsaný list papíru, zkušenostmi se popisuje
		\end{itemize}
	\end{itemize}
\item racionalismus
	\begin{itemize}
	\item R. Descartes
		\begin{itemize}
		\item matematik \ra dedukce -- hledal filosofické axiomy
		\end{itemize}
	\end{itemize}
	\item B. Spinoza
		\begin{itemize}
		\item pantheismus (příroda = bůh)
		\item matematická teorie etiky
		\end{itemize}
	\item G. W. Leibniz
		\begin{itemize}
		\item diferenciální a integrální počet
		\item doplnil Aristotelovy logické zákony
		\item "svět = harmonie monád"
		\end{itemize}

\end{itemize}

\paragraph{Výtvarné umění}
\begin{itemize}
\item baroko ("nepravidelná perla")
\item protestantské země -- klasicizující, umírněné baroko
\item katolické země -- ???
\item vznik na konci 16. stol. v Římě
\item hlavní znaky: disharmonie, dynamičnost, iracionalismus, nadpřirozeno, nadsázka, popírání hmoty (vznášející se postavy), ilusionismus (fake okna)
\item cílem je podnícení duchovního života
\end{itemize}

\paragraph{Architektura}
\begin{itemize}
\item ovál, stlačený oblouk, oválná nebo lichoběžníková okna, kupole s lucernou, iluzionistické fresky
\item Lorenzo Bernini -- kolonáda na nám. sv. Petra
\item Fontána di Trevi v Římě
\item Francesco Boromini 
\item Jacques Lemercier -- kostel v Sorboně
\item J. Hardouin-Mansart -- chrám v Paříži -- Napoleonovo mauzoleum
\item Jan B. Fischer z Erlachu
\item J. L. Hildebrandt
\end{itemize}



\timeline
\end{document}