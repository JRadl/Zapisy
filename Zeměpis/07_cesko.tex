\title{Česká republika}
\documentclass[10pt,a4paper]{article}
\usepackage[utf8]{inputenc}
\usepackage[czech]{babel}
\usepackage{amsmath}
\usepackage{amsfonts}
\usepackage{amssymb}
\usepackage{chemfig}
\usepackage{geometry}
\usepackage{wrapfig}
\usepackage{graphicx}
\usepackage{floatflt}
\usepackage{hyperref}
\usepackage{fancyhdr}
\usepackage{tabularx}
\usepackage{makecell}
\usepackage{csquotes}
\usepackage{footnote}
\usepackage{movie15}
\MakeOuterQuote{"}

\renewcommand{\labelitemii}{$\circ$}
\renewcommand{\labelitemiii}{--}
\newcommand{\ra}{$\rightarrow$ }
\newcommand{\x}{$\times$ }
\newcommand{\lp}[2]{#1 -- #2}
\newcommand{\timeline}{\input{timeline}}


\geometry{lmargin = 0.8in, rmargin = 0.8in, tmargin = 0.8in, bmargin = 0.8in}
\date{\today}
\author{Jakub Rádl}

\makeatletter
\let\thetitle\@title
\let\theauthor\@author
\makeatother

\hypersetup{
colorlinks=true,
linkcolor=black,
urlcolor=cyan,
}



\begin{document}
\maketitle
\tableofcontents
\begin{figure}[b]
Toto dílo \textit{\thetitle} podléhá licenci Creative Commons \href{https://creativecommons.org/licenses/by-nc/4.0/}{CC BY-NC 4.0}.\\ (creativecommons.org/licenses/by-nc/4.0/)
\end{figure}
\newpage

\section{Základní informace}
\paragraph{Vyhledej v atlase:}
\begin{enumerate}
\item rozlohu a počet obyvatel ČR [78 866km$^2$, 10.5 mil]
\item okrajové obce státu a jejich souřadnice [S:51,03 Lobendava; Z:12,05 Krásná; J:48,33 Vyšší brod; V: 18,51 Bukovec]
\item vzdálenost S--J, Z--V [278km, 493km]
\item nejkratší vzdálenost od moře [Šluknov -- Štětínský záliv v Baltském moři -- 326km]
\item délku státní hranice se sousedními zeměmi [N:810km, P:762km, R:466km, S:252km]
\item geografický střed: u obce Čihošť (49,44 s.š. 15,20 v.d.) (u Ledeče nad Sázavou)
\end{enumerate}

\paragraph{Státní znak}
\begin{itemize}
\item husitská pavéza
\item dvakrát český lev, moravská orlice, slezská orlice
\end{itemize}

\subsection{Poloha}
\paragraph{Matematickogeografická}
\begin{itemize}
\item ČR leží na 50. rovnoběžce a na 15. poledníku (středočeské časové pásmu)
\end{itemize}

\paragraph{Fyzickogeografická poloha}
\begin{itemize}
\item vnitrozemský stát ležící na rozhraní mezi oceánským a kontinentálním klimatem
\item střední nadmořská výška -- 450m (Evropa 350m)
\item ČR leží na hlavním evropském rozvodí -- nedostatek velkých toků
\item poloha na styku Českého masívu a Karpat -- hranice vede mezi Znojmem, Brnem, Olomoucí a Ostravou
\item nejvyšší bod -- Sněžka (1603)
\item nejnižší bod (115m)
\end{itemize}

\paragraph{Geopolitická}
\begin{itemize}
\item poloha na střetu mocností a mocenských zájmů
\item součást Rakousko-Uherska, zabrání Sudet, vytvoření protektorátu, sovětský blok, návrat do Evropy po r. 1989
\end{itemize}

\section{Historie}
\subsection{Hranice státu}
\begin{itemize}
\item přirozená nebo umělá
\item patří mezi nejstarší (cca 1000 let) a nejstabilnější v Evropě
\item tvořena pohraničním pásmem hor
\item celková délka --  2290km 
\item dlouhá vzhledem k rozloze státu
\item členitá -- výběžky: ašský, šluknovský, frýdlandský, broumovský, rychlebský, osoblažský
\item současná hranice vymezena mírovými smlouvami 
	\begin{itemize}
	\item Versailleská -- Německo (1919)
	\item st. germainská -- Rakousko (1919)
		\begin{itemize}
		\item zisk kusu dolních Rakous 
		\end{itemize}
	\item mezinárodní arbitráž -- Polsko (1920)
		\begin{itemize}
		\item spory o Těšín \ra město rozděleno na Polský a Český Těšín
		\end{itemize}
	\item mezinárodní smlouva -- Slovensko (1997)
	\end{itemize}
\end{itemize}

\subsection{Územní vývoj státu}
\begin{itemize}
\item 7. stol. -- Sámova říše
\item 9. stol. -- Velká Morava (Mikulčice, Uherské Hradiště, \ldots)
\item 10. stol. -- základy přemyslovského státu v Čechách, Morava připojena v polovině 11. století
\item 1212 -- Zlatá bula sicilská
\item 13. stol. -- rozšiřování území za vlády Přemysla Otakara II. a Václava II.
\item 14. stol. -- po vymření přemyslovců Lucemburkové -- vznik zemí Koruny české
\item 1526 -- bitva u Moháče, česko se stává součástí Habsburské říše
\item 28. 10. 1918 -- vznik samostatného Československa
\item 1938 -- mnichovská dohoda (konec tzv. první republiky)
\item 1939 -- Slovenský štát, Protektorát Čechy a Morava
\item 1945 -- ztráta Podkarpatské rusi
\item 1993 -- rozdělení státu na Českou a Slovenskou republiku
\end{itemize}

\subsection{Mapování našeho státu}
\begin{itemize}
\item 1518 -- nejstarší tištěná mapa Čech od Mikuláše Klaudyána
	\begin{itemize}
	\item mnoho erbů a textu, málo mapy
	\end{itemize}
\item 1569 -- mapa Moravy od Pavla Fabricia
\item 1627 -- mapa Moravy od J. A. Komenského
\item v letech 1760--1780 probíhá úřední mapování vojenské a civilní
\item 1935 -- Atlas Republiky Československé
\item 1966 -- Atlas Československé socialistické republiky
\end{itemize}


\section{Geologie}
\subsection{Geomorfologické členění}
\begin{itemize}
\item věda o tvaru zemského povrchu
\end{itemize}

\begin{tabular}{|c|c|}
	\hline
	\textbf{Provincie} & \textbf{Subprovincie}\\
	\hline
	Česká vysočina & Šumavská\\
	& Krušnohorská\\
	& Krkonošsko-jesenická\\
	& Poberounská\\
	& Česká tabule\\
	\hline
	Středoevropská nížina & Středopolské nížiny\\
	\hline
	Západní karpaty & Vněkarpatské sníženiny \\
	& Vnější západní karpaty\\
	\hline
	Panonská pánev & Vídeňská pánev\\
	\hline

\end{tabular}


\subsection{Česká vysočina}
\begin{itemize}
\item V České vysočině převládají předhercynské útvary -- hlavně krystalické břidlice (ČMV, Šumava)
	\begin{itemize}
	\item starší břidlice -- metamorfované horniny
	\end{itemize}
\item Mezi nejstarší nepřeměněné útvary náleží oblast Barrandienu (Praha -- Plzeň)
	\begin{itemize}
	\item Joachym Barand -- francouzský geolog, objevil spoustu zkamenělin, bylo zde moře
	\end{itemize}
	\item největší mocnost zemské kůry je v okolí Sedlčan -- 42km
	\item tektonicky aktivní oblasti jsou na Chebsku, Náchodsku, Opavsku
	\item v horninovém složení převažují žuly, pískovce, vápence a vulkanické horniny
\end{itemize}

\paragraph{Geologický vývoj}
\begin{itemize}
\item \textbf{prahory} (archaikum) -- vznik ČMV, Šumavy, jižních Čech
\item \textbf{starohory} (proterozoikum) -- nejstarší mořské usazeniny v Barandienu
\item \textbf{prvohory} (paleozoikum) --  kaledonské vrásnění, část pzemí zalita mořem, vznik devonských vápenců Moravského krasu, hercynské vrásnění -- vysoká pohoří, ložiska černého uhlí, moře pouze na okrajích masívu
\item \textbf{druhohory} (mezozoikum)  
	\begin{itemize}
	\item snižování horstev
	\item mělké křídové moře, po jeho ústupu vznik České křídové tabule
	\item labské pískovce v Českém Švýcarsku, Adržpach, Český ráj
	\end{itemize}
\item \textbf{třetihory} (kenozoikum)
	\begin{itemize}
	\item vliv alpinského vrásnění -- neproběhlo přímo Českou vysočinou, ale Západními Karpaty \ra zdvih okrajových pohoří, zlomy, poklesy
	\item vulkanismus v Českém středohoří, Doupovských horách a Nízkém Jeseníku (Venušina sopka, Velký a Malý roudný)
	\item vývoj říční sítě, hnědouhelné pánve -- Podkrušnohorský zlom
	\end{itemize}
\item \textbf{čtvrtohory} (kvartér)
	\begin{itemize}
	\item doznívající vulkanismus
	\item pleistocenní zalednění -- lokální horské ledovce v Krkonoších a na Šumavě
	\item váté písky a spraše, dotváření říční sítě
	\item vliv člověka
	\end{itemize}
\end{itemize}

\subsection{Šumavská subprovincie}
\begin{itemize}
\item Šumava -- Velký Javor(1456, nejvyšší na Šumavě), Plechý(1378, nejvyšší na české části Šumavy)
\item Šumavské podhůří
\item Český les -- Čerchov(1042)
\item Novohradské hory
\end{itemize}

\paragraph{Šumava}
\begin{itemize}
\item hraniční pohoří s Německem a z části s Rakouskem
\item jedna z geologicky zaoblených oblastí -- prahory, starohory
\item vliv alpinského vrásnění
\item lokální zalednění v pleistocénu
	\begin{itemize}
	\item ledovcová jezera -- Černé, Plešné, Čertovo, Prášilské, Laka
\item kary, morény
	\end{itemize}
\item národní park Šumava
	\begin{itemize}
	\item (rašeliniště) a pláně(náhorní plošiny)
	\item největší český národní park
	\item NPR Boubín
	\item dlouhodobé problémy s kůrovci
	\end{itemize}
\item společně s Bavorským lesem se jedná o nejzalesněnější část střední Evropy
\item prameny Vltavy (pod Černá horou) a Otavy (Vydra), Lipenská nádrž (horní tok Vltavy)
\item hřebeny Šumavy prochází hlavní evropské rozvodí
\end{itemize}

\subsection{Krkonošsko-jesenická subprovincie}
\begin{itemize}
\item na svazích Krkonoš se vytvořili čtyři vegetační výškové stupně
	\begin{itemize}
	\item listnaté a smíšené lesy s kulturními loukami 400--800 m
	\item horské smrčiny 800--1200 m
	\item lišejníková tundra, kamenité sutě 1450--1602 m
	\end{itemize}
\item Krkonoše -- poškozená příroda, od 89 pokusy o nápravu
\item pohoří má pestrou skladbu rostlin a živočichů s řadou endemitů a glaciálních reliktů
\item příroda hřebenů Krkonoš s drsným klimatem se podobá přírodě severní Evropy -- severská tundra
\item 
\end{itemize}

\paragraph{Broumovská vysočina}
\begin{itemize}
\item součástí je CHKO Broumovsko s Adršpašsko teplickými skálami a Broumovskými stěnami
\item lokalita patří k největším skalním městům střední Evropy
\item členitý reliéf způsobuje klimatickou inverzi
\item regionem protéká řeka Metuje
\end{itemize}

\paragraph{Kralický Sněžník}
\begin{itemize}
\item rulová klenba, ze které vybíhají směrem k jihu dva hřbety
\item pramenná oblast řeky Moravy
\item centrální body hlavního evropského rozvodí  .......
\end{itemize}

\paragraph{CHKO Hrubý Jeseník}
\begin{itemize}
\item nejvyšší pohoří Moravy se zaoblenými hřbety a hlubokými údolími
\item hřebeny Jeseníků patří k nejchladnějším oblastem v republice (Praděd 0.9)
\item vyvinuté výškové vegetační stupně
\item Ramzovské a Červenohorské sedlo
\item NPR Praděd -- vrchol Pradědu, Petrovy kameny, Velká a Malá kotlina, Bílá Opava
\item NPR Šerák-Keprník -- zbytky pralesních porostů horských smrčin
\item lázně Jeseník, Ramzová, Červenohorské sedlo, Malá Morávka, Karlov, Karlova Studánka
\end{itemize}

\subsubsection{Českomoravská subprovincie}
\begin{itemize}
\item Brněnská vrchovina, Českomoravská vrchovina (Javořice, 837), Středočeská pahorkatina, Jihočeské pánve
\end{itemize}

\paragraph{Brněnská vrchovina}
\begin{itemize}
\item Drahanská, Bobravská vrchovina, Boskovická brázda
\item součástí Drahanské vrchoviny je \textbf{Moravský kras}
	\begin{itemize}
	\item největší krasové území ČR, cca 100 km$^2$
	\item tvořen devonskými vápenci
	\item CHKO se sídlem v Blansku
	\item povrchové a podpovrchové krasové jevy (kaňony, údolí, závrty, propasti, jeskyně, ponory, 
vývěry, \ldots)
	\item nejnavštěvovanější místa v \textit{severní části}
		\begin{itemize}
		\item Sloupsko šošůvské jeskyně (Kůlna -- nálezy neandrtálců 120 000 let)
		\item Punkevní jeskyně (prof. Absolon, Masarykův dóm)
		\item Macocha -- 138 m
		\item Amatérská jeskyně (41 km, Punkva)
		\item Kateřinská jeskyně
		\item Balcarka
		\end{itemize}
	\item \textit{střední část} -- Rudické propadání, Býčí skála
	\item \textit{jižní část} -- Ochozská jeskyně, Pekárna -- rytiny koní a bizonů na kostech zvířat (13 000 let)
	\end{itemize}
\end{itemize}

\subsection{Západní karpaty}
\subsubsection{Vnější západní karpaty}
\begin{itemize}
\item pavlovské vrchy (Děvín 550), Žďánický les, Chřiby, Hostýnsko-vsetínská vrchovina, Kyjovská pahorkatina, Vizovická vrchovina, Bílé Karpaty (Velá Javořina, 970), Javorníky (Velký Javorník, 1071), Moravsko-slezské Beskydy (Lysá hora, 1323)
\end{itemize}

\paragraph{Pavlovské vrchy}
\begin{itemize}
\item CHKO, biosferická rezervace
\item slabě zkrasovělé jurské vápence táhnoucí se až do Rakouska
	\begin{itemize}
	\item přineseny Alpinským vrásněním
	\end{itemize}
\item jedna z nejteplejších a nejsušších oblastí v ČR
\item stepní a lesostepní společenstva v NPR Děvín
\item lovci mamutů -- Věstonická Venuše
\item pěstování vinné révy
\item nádrže Nové Mlýny na řece Dyji
\end{itemize}

\subsubsection{Vněkarpatské sníženiny}
\begin{itemize}
\item Dyjsko-svratecký úval, Vyškovská brána, Hornomoravský úval, Moravská brána, Ostravská pánev
\end{itemize}



\newpage


\section{Klima ČR}
\begin{itemize}
\item ČR leží v \textbf{přechodu oceánského a kontinentálního klimatu}
\item na klima mají vliv
	\begin{itemize}
	\item většinu roku ovlivňována \textbf{západním prouděním} od Atlantiku
	\item ovlivněno \textbf{reliéfem} (nadmořská výška, srážkový stín, expozice svahu)
	\item \textbf{lesní plochy} -- zachytávají vodu 
	\item \textbf{vodní plochy} -- zvlhčují místní mikroklima
	\item \textbf{zastavené plochy} -- produkce tepla
	\item rozložení \textbf{vzduchových hmot} (arktická, polární, tropická)
	\end{itemize}
\item počasí ovlivňuje rozložení cyklon a anticyklon nad Azorskými ostrovy, Islandem, Britskými ostrovy, Skandinávií a Ruskem
\item aprílové počasí -- časté změny v počasí počátkem dubna
\item ledoví muži -- jarní mrazy na svátek Pankráce, Serváce a Bonifáce (12. -- 14. 5.)
\item Medardov kápě -- 40 dní dešťů po svátku sv. Medarda (8. 6.)
\item babí léto -- doznění letního počasí počátkem podzimu
\end{itemize}

\paragraph{Synoptická mapa}
\begin{itemize}
\item zobrazuje předpověď počasí
\item tlakové výše, níže, \ldots
\end{itemize}

\paragraph{Rozložení teplot}
\begin{itemize}
\item vliv nadmořské výšky (0.6$^\circ$C/100m -- termický stupeň)
\item z hlediska vymezujeme klimatické oblasti
	\begin{itemize}
	\item \textbf{teplé} -- nížiny
	\item \textbf{mírně teplé} -- vrchoviny $<$ 700 m
	\item \textbf{chladné} -- vrchoviny, pohoří $>$ 700 m
	\end{itemize}
\end{itemize}

\paragraph{Rozložení srážek}
\begin{itemize}
\item ovlivněno \textbf{nadmořskou výškou}
\item nejsušší oblasti leží ve \textbf{srážkovém stín}u Krušných hor a ČMV, nejdeštivější jsou naše okrajová pohoří
\item \textbf{klimadiagram} -- určuje roční chod srážek na určitém území
\end{itemize}

\section{Hydrologické poměry v ČR}
\paragraph{Otázečky}
\begin{enumerate}
\item Lokalizuj naše řeky v rozsahu mapy v atlase na str. 11
\item Do kterých moří odtéká voda z našeho území a které řeky ji tam odvádějí?
	\begin{itemize}
	\item Morava -- Černé, Labe -- Severní, Odra -- Baltské
	\end{itemize}
\item Co bylo příčinou velkých povodní v posledních letech?
	\begin{itemize}
	\item 
	\end{itemize}
\item Z učebnice zpracuj téma Zásahy člověka do hydrosféry
	\begin{itemize}
	\item staví rybníky, přehrady, špiní vodu
	\end{itemize}
\end{enumerate}

\paragraph{Řeky}
\begin{itemize}
\item modelují reliéf, významné pro člověka (domácí spotřeba, energie, zalévání, \ldots)
\item ČR leží na \textbf{hlavním evropském rozvodí} (Klepý, 1143m) \ra sucho, nemožnost vodní dopravy
\item \textbf{sněhovo-dešťový} (středoevropský) režim odtoku \ra hlavním zdrojem vody jsou srážky, nevyrovnané (na jaře taje sníh)
\item voda z našeho území odtéká do Severního, Černého a Baltského moře
\end{itemize}

\paragraph{Jezera}
\begin{itemize}
\item ledovcová (Černé 18ha, 40m)
\item krasová -- malá (Hranická propast)
\item rašelinová -- slatě (Chalupská slať)
\item říční	
\end{itemize}

\paragraph{Rybníky}
\begin{itemize}
\item jižní Čechy (Rožmberk 489ha, Třeboňská, Českobudějovická pánev, osou řeka Lužnice) -- kulturní krajina
\item Českomoravská vrchovina (Velké Dářko)
\item jižní Morava (Pohořelicko, Lednické rybníky -- Nesyt)
	\begin{itemize}
	\item ryby umírají na nedostatek kyslíku
	\end{itemize}
\end{itemize}

\paragraph{Přehradní nádrže}
\begin{itemize}
\item význam energetický, zásoba pitné a užitkové vody, rybolov, zavlažování, regulace průtoku
\item největší množství vody je zadržováno na vltavské kaskádě (Lipno 4870 ha, Orlík 720 mil. m3, Kamýk, Slapy, Štěchovice)
\item Vranov, Dalešice, Nové Mlýny, Slezská Harta, Šance, Brněnská přehrada
\end{itemize}

\paragraph{Prosté podzemní vody}
\begin{itemize}
\item zásoby tam kde jsou propustné horniny, pískovcová skalní města (Polabí, České Švýcarsko, Adršpach)
\item Brno bere vodu z vírské přehrady a Březové
\end{itemize}

\paragraph{Minerální vody}
\begin{itemize}
\item obohacené vody využívané k léčebným účelům
\item zásoby v severních Čechách \ra lázně Karlovy vary, Františkovy, Mariánské, na Moravě Luhačovice
\end{itemize}

\paragraph{Vliv člověka na hydrosféru}
\begin{itemize}
\item stavba přehrad, rybníků
\item využívání \ra znečišťování vody
\item globální oteplování \ra tání ledovců
\end{itemize}

\section{Půdy ČR}
\begin{itemize}
\item půdní typ vzniká působením \textbf{půdotvorných činitelů}
	\begin{itemize}
	\item \textbf{geologický podklad} (vliv na chemismus, minerály), \textbf{reliéf} (expozice, eroze), \textbf{klima} (srážky, výpary, teploty), \textbf{vegetace} (zdroj humusu), \textbf{edafon} (řasy, bakterie, plísně, houby), \textbf{člověk} (odlesňování-eroze, zástavba, kyselé deště, umělá hnojiva)
	\end{itemize}
\item průřez půdou se skládá z několika horizontů
\item půdní druh vymezujeme na základě zrnitosti \ra jílovité, hlinité, písčité
\item vypsat z atlasu str. 12
\end{itemize}

\section{Živá příroda}
\begin{itemize}
\item vývoj rostlinstva a živočišstva na území ČR byl ovlivněn \textbf{polohou na rozhraní} několika \textbf{biogeografických zón}
	\begin{itemize}
	\item boreální, středoevropské lesní, alpské a karpatské
	\end{itemize}
\item také polohou na rozhraní \textbf{zaledněné a nezaledněné} Evropy
\item vegetační stupně údolních niv, dubový a bukový byly výrazně \textbf{přeměněny člověkem} -- zemědělské plochy, regulace řek, likvidace lužních lesů
\item význam lesních ekosystémů
	\begin{itemize}
	\item z rozlohy státu zabírají asi 33\%
	\item v druhové skladbě dominuje smrk 55\%
	\item význam lesa pro ekologickou stabilitu krajiny
	\item význam lesa pro ekologickou vodohospodářství
	\item význam lesa pro rekreaci
	\item zabraňování erozi
	\item lesy ohrožené kůrovcem, především Šumava	
	\end{itemize}
\end{itemize}


\subsection{Ochrana přírody}
\begin{enumerate}
\item Pomocí učebice a atlasu vyhledej naše NP, CHKO a biosferické rezervace
\item Která z těchto chráněných území leží na Moravě? [NP Podyjí, CHKO: Pálava, Moravský kras, Karpaty, Beskydy] 
\end{enumerate}


\begin{itemize}
\item 1838 -- Žofínský prales, prales Hojná Voda (Novohradské hory) -- první chráněná území
\item 1858 -- Boubínský prales
\item po 1989 -- zřízeno MŽP \ra vydávány nové právní normy pro ochranu přírody
\item 1991 -- státní program péče o životní prostředí
\item postupně dochází ke zkvalitňování životního prostředí, ale značná zátěž z minulosti a rostoucí problém s nárůstem automobilové dopravy, prašností atd.
\end{itemize}

\subsection{Chráněná území}
\paragraph{Velkoplošná CHÚ}
	\begin{itemize}
	\item \textbf{národní parky}: 
		\begin{itemize}
		\item vyhlašuje parlament české republiky
		\item 1963 -- Krkonošský NP (nejstarší)
		\item 1991 -- Šumava (největší)
		\item 1991 -- NP Podyjí
			\begin{itemize}
			\item mezi Znojmem a Vranovem
			\item kaňon řeky Dyje
			\item nejzachovanější kaňon v Evropě (za komunismu uzavřeno)
			\item signálky -- malé, asfaltové silnice pro pohraniční stráž, nyní pro turisty
			\end{itemize}
		\item 2000 -- NP České Švýcarsko
			\begin{itemize}
			\item největší pískovcové skalní město ve stř. Evropě
			\item Pravčická brána
			\item pozůstatek mělkého křídového moře
			\end{itemize}
		\end{itemize}
	\item \textbf{chráněné krajinné oblasti}:
		\begin{itemize}
		\item vyhlašuje vláda ČR
		\end{itemize}
	\item NP a CHKO jsou řízeny vlastní správou
	\item návštěvníci jsou povinni řídit se řádem těchto CHÚ
	\item v NP jsou vytvořeny ochranné zóny
	\end{itemize}
\paragraph{Maloplošná CHÚ}
\begin{itemize}
\item \textbf{národní přírodní rezervace} (NPR)
\item \textbf{přírodní rezervace} (PR)
\item \textbf{národní přírodní památka} (NPP)
\item \textbf{přírodní památka} (PP)
\end{itemize}

\paragraph{Biosferické rezervace UNESCO}
\begin{itemize}
\item lokality světového významu
\item Dolní Morava
	\begin{itemize}
	\item CHKO Pálava -- Pavlovské vrchy
	\item Lednicko valtický areál
	\item Soutok -- Dyje do Moravy
	\end{itemize}
\item Třeboňsko, Šumava, Křivoklátsko, Krkonoše, Bílé Karpaty, 
\end{itemize}

\paragraph{Mokřadní ekosystémy}
\begin{itemize}
\item bažiny, rašeliniště, vodní plochy
\item chráněné na základě Ramsarské úmluvy
\item Třeboňsko, Šumavské slatě, Lednické rybníky (Nesyt)
\end{itemize}

\paragraph{Natura 2000}
\begin{itemize}
\item projekt EU, zaměřený na mimořádně cenná přírodní území
\item ochranu kontrolují místní orgány
\end{itemize}

\newpage

\section{Regiony české republiky}
\begin{enumerate}
\item Vyhledej v atlase kraje ČR a  jejich hlavní města
\item z učebnice zjisti rozlohu JM kraje a porovnej s ostatními
\item Kdo je současným hejtmanem JM kraje a kde sídlí
\end{enumerate}

\paragraph{Kraje}
\begin{itemize}
\item v r. 2000 zavedeny \textbf{vyšší územní samosprávné celky} -- \textbf{kraje}
\item celkem 14 krajů
\item v čele hejtman -- současně
\item Jihomoravský -- 
\item Středočeský -- největší
\item Moravskoslezský -- nejlidnatější
\end{itemize}

\paragraph{Okresy}
\begin{itemize}
\item 76
\item 2003 -- zankají, dnes fungují jen jako statistické jednotky
\item nahrazeny \textbf{obcemi s rozšířenou působností} -- 205
\end{itemize}

\paragraph{Obec} -- nejmenší článek

\paragraph{Kraje}
\begin{itemize}
\item Jihomoravský -- Brno
	\begin{itemize}
	\item Blansko
	\item Brno-město
	\item Brno-venkov
	\item Břeclav
	\item Hodonín
	\item Vyškov
	\item Znojmo
	\end{itemize}
\item Vysočina -- Jihlava
\item Jihočeský -- Č. Budějovice
\item Plzeňský -- Plzeň
\item Karlovarský -- Karlovy vary
\item Ústecký -- Ústí nad Labem
\item Liberecký -- Liberec
\item Královéhradecký -- Hradec Králové
\item Pardubický -- Pardubice
\item Olomoucký -- Olomouc
\item Moravskoslezský -- Ostrava
\item Zlínský -- Zlín
\item Středočeský -- Praha 
\item Praha -- Praha
\end{itemize}

\paragraph{Historie}
\begin{itemize}
\item ČR se skládá ze tří historických území: Čechy, Morava, Slezsko (pro současné uspořádání toto nehraje roli)
\end{itemize}

\paragraph{NUTS}
\begin{itemize}
\item v rámci EU byla v ČR vytvořena \textbf{soustava územních statistických jednotek} -- \textbf{NUTS II}
\item jedná se o 8 oblastí, které slouží eurostatu pro porovnávání ekonomických ukazatelů členských zemí
\item jedná se o oblasti, které počtem obyvatel překračují 1 milion (proto některé kraje spojeny)
\item na základě členění jsou přijímány finanční prostředky ze strukturálních fondů EU, které jsou určeny na rozvoj zaostávajících regionů 
\item HDP $<$ 75\% EU -- chudý region
\item HDP $>$ 120\% EU -- bohatý region
\item NUTS 1 -- ČR
\item NUTS 2 -- 8 regionů vymezených EU
\item NUTS 3 -- 14 krajů
\item NUTS 4 -- 76 okrajů
\item NUTS 5 -- 6249 obcí
\end{itemize}

\section{Vývoj osídlení a zalidnění}
\begin{itemize}
\item původně usazení v nížinách 
\item nejpozději osídlovány vysoké části vrchovin a pohoří
\item potřeba rozšíření -- pastevectví, těžba rud, těžba dřeva
	\begin{itemize}
	\item motivace neplacením daní / půdou zdarma
	\item zváni cizinci ke kultivaci pohraničí -- "sudet"
	\end{itemize}
\item v současnosti velké množství malých obcí s malým počtem obyvatel -- spravují ale přes polovinu území, protože jsou na rozdíl od bodových velkoměst rovnoměrně rozloženy
\end{itemize}

\paragraph{Městská sídla}
\begin{itemize}
\item vymezení pojmu město:
	\begin{itemize}
	\item historicko-právní hledisko -- titul získán v minulosti
	\item kvantitativní hledisko -- od 10 000 
	\item kvalitativní hledisko -- služby, vícepodlažní budovy, 
	\end{itemize}
\item velkoměsto $>$ 100 000, světové velkoměsto $>$ 1000 000
\item statutární město -- má právo se členit na městské části
	\begin{itemize}
	\item primátorka: Markéta Vaňková (sídlí na magistrátu:  )
	\item městské části: 
		\begin{itemize}
		\item je jich 29
		\item v čele starosta
		\item Jaroška v Černých polích
		\end{itemize}
	\end{itemize}
\item funkce měst
\item problémy našich měst
\item kulturní památky UNESCO
	\begin{itemize}
	\item Český Krumlov (historické centrum)
	\item Telč (historické centrum)
	\item Praha (historické centrum)
	\item Kutná Hora (historické centrum, chrám sv. Barbory, ..)
	\item Zelená Hora
	\item Brno (vila Tugendhat)
	\item Kroměříž (zahrady, zámek, \ldots)
	\item Žďár nad Sázavou
	\item Lednicko-Valtický areál ()
	\item Holašovice
	\item Olomouc
	\item Třebíč
	\item Litomyšl
	\end{itemize}
\item vývoj urbanistických zón	
	\begin{itemize}
	\item jednotlivé části města vznikající v určitých obdobích
	\item historické jádro
	\item 19. stol -- rozvoj průmyslu \ra nová obytná zóna, továrny
	\item po 2. sv. v. -- rozšíření panelových sídlišť
		\begin{itemize}
		\item revitalizace po 1989 -- zateplování, nástavby, výměna jader
		\end{itemize}
	\item novodobě
		\begin{itemize}
		\item vznik nových průmyslových zón na okrajích měst, satelitních měst
		\item revitalizace volnočasových zón, městských parků
		\item vznik nákupních center
		\end{itemize}
	\end{itemize}
	\item funkce venkovských sídel:
		\begin{itemize}
		\item zemědělská
		\item obytná
		\item udržování lidových tradic
		\item rekreační (druhé bydlení)
		\item údržba krajiny
		\item revitalizace
			\begin{itemize}
			\item cílem zabránit vylidňování venkova
			\item malý potenciál pro mladé lidi 
			\item[\ra] zlepšování služeb (obchod, školství, doprava, kultura, \ldots)
			\item[\ra] vznik pracovních příležitostí
			\end{itemize}
		\item urbanizace -- růst měst rozšiřování městského způsobu života, stěhování obyvatel do měst –- rozdílný vývoj ve vyspělých zemích a  rozvojových zemích
		\item suburbanizace -- přesun obyvatel na okraje měst a do blízkého zázemí, důležité je dopravní spojení
		\end{itemize}
\end{itemize}



\section{Vývoj hospodářství}
\begin{itemize}
\item postavení českých zemí v rámci Rakouska-Uherska
	\begin{itemize}
	\item důležitá součást, 80\% průmyslového potenciálu, těžba uhlí, stříbra
	\end{itemize}
\item prvorepublikové Československo -- patří mezi průmyslové země světa, prosazovalo se na světových trzích (Baťa, strojírenství, zbraně, sklo, pivo)
\item Slovensko zaostává za Českými zeměmi (od první republiky)
\item po roce 1948 (nástup komunismu)
	\begin{itemize}
	\item znárodnění průmyslu a kolektivizace zemědělství, zrušen soukromý sektor
	\item centrální plánování ekonomiky
	\item preferování těžkého průmyslu
	\item orientace na východní blok \ra zaostávání za vypělými státy
	\item vysoká zaměstnanost žen, zhoršování ekologické situace
	\end{itemize}
\item 1989 -- pád komunistického režimu
	\begin{itemize}
	\item přechod na tržní hospodářství a celková transformace ekonomiky
	\item vybudování trhu, privatizace
	\end{itemize}


\paragraph{Transformace ekonomiky po roce 1989}
\begin{itemize}
\item návrat k tržní ekonomice, omezení role státu
\item rozpad Československa (1. 1. 1993) a východních trhů
\item hledání nových odbytišť, přeorientování na trhy EU
\item pokles státních dotací hospodářství
\item velká a malá privatizace, kupónová privatizace
\item restituce v zemědělství -- navrácení soukromého majetku (proces veden do dnes)
\item příliv zahraničního kapitálu
\item do roku 2004 byl mírně větší dovoz než vývoz (dnes obráceně)
	\begin{itemize}
	\item riziko snížení poptávky
	\end{itemize}
\item útlum těžkého průmyslu, zemědělské výroby \ra pokles zaměstnanosti
\item nezaměstnanost 1.9\% (Feb 2019) -- nejmenší v EU (nedostatek pracovní síly)
\item posílení sektoru služeb
\item rozšíření dopravního napojení na Rakousko a Německo
\item prioritou je dobudování dálniční sítě a železničních koridorů
\item rozvoj cestovního ruchu
	\begin{itemize}
	\item možnost cestování do zahraničí
	\end{itemize}
\item větší otevření trhu po vstupu do EU přináší své výhody i nevýhody
\item ekologická politika státu
\end{itemize}
\end{itemize}




\section{Průmysl}
\subsection{Transformace průmyslu po roce 1989}
\begin{itemize}

\item k hlavním průmyslovým oblastem patří: 
	\begin{itemize}
	\item Podkrušnohoří, pražská aglomerace, Plzeň, Jablonec-Liberec, Pardubice-Hradec, Ostravsko, Olomouc, Zlín, Brněnsko
	\end{itemize}
\item privatizace
\item změny v odvětvové a územní struktuře
\item rušení a útlum některých výrob
\item vstup zahraničního kapitálu, moderních technologií (EU, NATO)
\item využití levné, ale kvalifikované pracovní síly (nyní pracovní síla dražší)
\item budování průmyslových zón 
	\begin{itemize}
	\item Černovické terasy -- největší průmyslová zóna v ČR
	\end{itemize}
\item značné investice jdou do elektrotechniky a automobilového průmyslu \ra hlavní složka ekonomiky
\item 
\end{itemize}

\subsection{Těžební průmysl, energetika}
\begin{itemize}
\item pokles těžby surovin na objemu průmyslové výroby
\item zásoby uhlí, žuly, vápence, kaolínu, sklářských písků
\item černé uhlí -- Ostravsko karvinský revír
\item hnědé uhlí -- Podkrušnohoří -- Severočeský a Sokolovský revír (Mostecká pánev)
	\begin{itemize}
	\item limity, dnes překračovány
	\item využití v elektrárnách
	\end{itemize}
\item zastavena těžba uranu (Rožná) 
	\begin{itemize}
	\item dříve dodáván do Sovětského svazu
	\end{itemize}
\item  ropa a zemní plyn
	\begin{itemize}
	\item Hodonínsko, Břeclavsko
	\item dodávány z Německa a Ruska
	\end{itemize}
\item výroba elektrické energie
	\begin{itemize}
	\item odsíření tepelných el.
	\item dominantní postavení ČEZ (EON, Bohemia energy, Innogy, RWE)
	\item rozhodující podíl pevných paliv
	\item do r. 2020 13\% podíl obnovitelných zdrojů na výrobě el. en.
	\item 53\% energie z tepelných elektráren (Podkrušnohoří)
	\item 2 jaderné elektrárny -- Temelín, Dukovany (33\%)
	\item z obnovitelných zdrojů -- 12\% (hydro, větrné, biomasa, solární)
		\begin{itemize}
		\item spalování dřeva, rychle rostoucích dřevin, slámy
		\item obnovitelné zdroje nejsou v našich podmínkách příliš efektivní
		\end{itemize}
	\end{itemize}
\end{itemize}


\subsection{Hutnictví}
\begin{itemize}
\item po roce 1989 útlum a krize
\item těžba uhlí v Moravskoslezském kraji (Vítkovice, Nová huť, )
\end{itemize}

\subsection{Strojírenství}
\begin{itemize}
\item nejrozšířenšjší odvětví s mnohaletou tradicí
\item kvalifikovaná pracovní síla
\item přímé zahraniční investice
\item změny ve výrobním programu, útlum, nebo zánik výroby
\item dřívě hlavní centrum v Brně (Kuřim, Adamov)
	\begin{itemize}
	\item zbrojovka -- přesunuta do Uherského Brodu
	\item 1. brněnská strojírna -- stále funkční
	\item Zetor -- výroba utlumena
	\end{itemize}
\end{itemize}

\paragraph{Výroba dopravních prostředků}
\begin{itemize}
\item Škoda Mladá Boleslav (Volkswagen)
\item TPCA Kolín (Toyota(Jap), Peugeot(Fr), Citroen(Fr))
\item Hyundai Nošovice
\item Tatra Kopřivnice (Zikmund a Hanzelka)
\item celkově přes 1000 000 automobilů ročně
\end{itemize}


\section{Zemědělství}
\subsection{•}
\paragraph{Zemědělské výrobní oblasti}
\begin{itemize}
\item \textbf{Kukuřičná} -- Dyjskosvratecký a Dolnomoravský úval
	\begin{itemize}
	\item úrodné černozemě, teplé, ale suché klima
	\item pšenice, kukuřice, teplomilné druhy ovoce(meruňky, broskve), zeleniny a vinné révy
	\item sucho kvůli srážkovému stínu ČMV
	\end{itemize}
\item \textbf{Řepařská} -- Hornomoravský úval, Polabí, dolní Povltaví a dolní Pohoří
	\begin{itemize}
	\item černozemě, hnědozemě, teplé mírně vlhké klima
	\item pšenice, sladovnický ječmen, cukrovka, zelenina
	\end{itemize}
\item \textbf{Obilnářsko-bramborářská} -- vrchoviny především ČMV
	\begin{itemize}
	\item hnědé půdy, podzoly, chladnější vlhčí klima
	\item brambory, krmná pšenice a ječmen, oves, pícniny
	\end{itemize}
\item \textbf{Pícninářská} --  nejvyšší části vrchovin a horské oblasti
	\begin{itemize}
	\item hnědé půdy, podzoly, gleje (příliš mokré půdy), chladné vlhké klima
	\item převaha luk a pastvin pro živočišnou výrobu
	\end{itemize}
\end{itemize}


\section{Doprava a spoje}
\begin{itemize}
\item základní podmínka úspěšného hospodářského rozvoje (vnitřní doprava a napojení na mezinárodní)
\item nutnost modernizace (pomalá, drahá)
	\begin{itemize}
	\item dobudovat síť rychlostních železničních koridorů
	\item dobudovat a zmodernizovat dálniční síť		
	\end{itemize}
\end{itemize}

\subsection{Železniční doprava}
\begin{itemize}
\item nerovnoměrné rozmístění železniční sítě
\item značná hustota (9600km), ale málo elektrifikovaných a vícekolejných tratí (cca 1/3)
\item malá propustnost a nízké rychlosti
\item nerentabilní provoz na vedlejších tratích
\item nízká úroveň kultury cestování
\item pokles objemu přepravy osob a nákladů v důsledku útlumu těžkého průmyslu, poklesu ceny osobních automobilů a nárůstu ceny jízdného
\item v posledním roce dotace vlády na cestování studentů a seniorů \ra nárůst počtu osob
\item ekologičtější než automobilová doprava, ale dražší
\end{itemize}

\paragraph{Rychlostní koridory}
\begin{itemize}
\item modernizované nejdůležitější trasy
\item 1. Děčín-Praha-Pardubice-Brno-Břeclav
\item 2. Petrovice-Ostrava-Přerov-Břeclav
\item 3. Ostrava-Přerov-Praha-Plzeň-Cheb
\item 4. Děčín-Praha-Budějovice
\end{itemize}

\paragraph{Brno}
\begin{itemize}
\item důležitý dopravní uzel (průnik prvního a druhého koridoru)
\item problémy s nádražím
	\begin{itemize}
	\item nedostatečná kapacita
	\item nepřehlednost (v zatáčce)
	\end{itemize}
\end{itemize}

\paragraph{Konkurence}
\begin{itemize}
\item RegioJet
\end{itemize}

\subsection{Silniční doprava}
\begin{itemize}
\item hustá, ale zastaralá silniční síť
\item nutno dobudovat ucelený systém dálnic a rychlostních silnic
\item změny 1. 1. 2016 -- zrušeny rychlostní silnice, část převedena na dálnice a část na silnice 1. třídy
\item cca 1200km dálnic, výstavba bržděna vykupováním a ekologií
\item chybí obchvaty kolem měst
\item problémy s kamionovou dopravou -- silnice na ni nejsou stavěny
\end{itemize}

\subsection{Letecká doprava}
\begin{itemize}
\item nárůst přepravy osob po roce 1989
\item rozdíl mezi vnitrostátní a mezistátní přepravou
\item letiště v Praze, Brně, Ostravě a Karlových Varech
\item privatizace ČSA
\item letiště Brno-Tuřany
	\begin{itemize}
	\item modernizace po 1989
	\item charterové lety do středomoří
	\item linka Brno-Praha nefunguje
	\end{itemize}
\end{itemize}

\subsection{Říční doprava}
\begin{itemize}
\item podmínky pro provozování říční dopravy v ČR
\item nedostatečné parametry řek
\item využití Labe, Vltavy, Moravy
	\begin{itemize}
	\item Labe splavné od chvaltic
	\end{itemize}
\item Baťův kanál	
	\begin{itemize}
	\item vybudován před 2. sv. válkou podél moravy na přepravu lignitu do Baťových továren
	\item cca 50km
	\item po válce zabaven komunisty a přestal být provozuschopný
	\item nyní zrekonstruován, využíván pro turistický ruch
	\end{itemize}
\item cca 300km říčních tras
\item Česká plavba a labská, a.s.
\item trendy v EU a v ČR
\end{itemize}

\section{Kraje}
\subsection{Hlavní město Praha}
\begin{itemize}
\item poloha uprostřed České kotliny v části Pražské plošiny
\item reliéf se skládá z hluboce zaříznutých říčních údolí kopců a teras
\item jediné milionové město v ČR, statut kraje
\item hospodářské, politické, vzdělávací, kulturní, \ldots centrum 
\item nadnárodní význam -- mezinárodní akce, ambasády, \ldots
\item dopravní význam 
	\begin{itemize}
	\item metro
	\item dopravní uzel mezinárodního významu
	\item Letiště Václava Havla
	\end{itemize}
\item nejnavštěvovanější místo v ČR, historické centrum na seznamu UNESCO
\item historické jádro tvoří Hradčany, Malá Strana, Staré a Nové Město, Vyšehrad
\end{itemize}
\paragraph{Zajímavosti}
\begin{itemize}
\item Pražský hrad -- největší hradní komplex na světě
\item Karlův most -- jen kopie soch
\item Petřín -- rozhledna (60m), "pražská Eiffelova věž"
\item Staroměstské náměstí
	\begin{itemize}
	\item Orloj -- restaurován
	\item Chrám -- pochován Ticho de Brahe
	\end{itemize}
\item Václavské náměstí
	\begin{itemize}
	\item Národní muzeum
	\item socha sv. Václava
	\end{itemize}
\item Strahov -- sokolský sletový stadion (dříve spartakiády, nyní tréninkové středisko Sparty)
\item Vítkov -- mauzoleum (Gottwald)
\item Národní Divadlo
\item Vyšehrad		
\end{itemize}

\subsection{Jihomoravský kraj}
\begin{enumerate}
\item Kolik žije v Brně obyvatel (380 000)
\item Jaké postavení zaujímá Brno v rámci ČR?
	\begin{itemize}
	\item centrum vzdělání, vědy a výzkumu
		\begin{itemize}
		\item kampus v Bohunicích -- Masarykova univerzita
		\item komplex technologického parku VUT, Palackého vrch
		\end{itemize}
	\item dopravní uzel, 
	\end{itemize}
\item Jaké přírodní a kulturněhistorické zajímavosti Jihomoravského kraje znáš?
	\begin{itemize}
	\item Tugendhat, Špilberk, Lednice, 
	\end{itemize}
\end{enumerate}

\begin{itemize}
\item druhé největší město v ČR
\item statutární a krajské město (29 částí, 48 čtvrtí)
\item centrum strojírenského průmyslu, dříve také textilního a oděvního
	\begin{itemize}
	\item z velkých strojírenských podniků již fungují jen 1. Brněnská strojírna a Zetor
	\item útlumem prošli též Kuřim, Adamov, Blansko
	\end{itemize}
\item centrum vědy a vzdělání
\item sídlo Ústavního a Nejvyššího soudu a ombudsmana (Anna Šabatová)
\item významný dopravní uzel
\item veletržní město (otevřeny za první republiky (1925))
\item Velká cena -- motocyklové závody
\item 1292 -- Brno královským městem
\end{itemize}

\paragraph{Kulturní zajímavosti}
\begin{itemize}
\item Mahenovo divadlo -- 1. elektrifikované divadlo (Edison)
\item Náměstí svobody -- bývalý kostel, "orloj", \ldots
\item vila Tugendhat -- postavena židovskou rodinou před 2. sv. v., za války velitelství, za komunismu školka, po 89 památka
\item Petrov
\item Špilberk (dříve byl starý brněnský hrad)
\item Brněnská radnice -- pověsti
\item Měnínská brána -- jedna z pěti bran brněnského opevnění
\end{itemize}

\paragraph{Kraj}
\begin{itemize}
\item rozloha kraje cca 7000km$^2$, cca 1200000 obyvatel
\item významný podíl na tvorbě HDP
\item produkční zemědělská oblast Dyjsko-svrateckého a Dolnomoravského úvalu (kukuřičná oblast)
\item 
\end{itemize}

\subsection{Moravskoslezský kraj}
\begin{itemize}
\item průmyslový region s centry Ostrava, Opava, Karviná, Frýdek Místek, Bohumín, Třinec
\item těžba černého uhlí na Karvinsku, soustředění těžkého průmyslu
\item mezi největší podniky u nás patří Vítkovické železárny, Nová huť a železárny v Třinci a v  Bohumíně
\item vysoká míra nezaměstnanosti související s útlumem těžkého průmyslu
\item Tatra Kopřivnice
\item Nošovice -- Radegast, Hyundai
\item poškozené životní prostředí
\item historická moravskoslezská hranice
\item Jeseníky, Moravskoslezské Beskydy, Moravská brána
\item řeky Odra, Opava
\end{itemize}

\subsection{Jihočeský kraj}
\begin{itemize}
\item nejřídčeji osídlený kraj republiky
\item jaderná elektrárna Temelín
\item výroba piva Budvar
\item mléčné výrobky
\item význam v oblasti cestovního ruchu
	\begin{itemize}
	\item České Budějovice
	\item Tábor
	\item Písek
	\item Jindřichův Hradec
	\item Třeboň
	\item Český Krumlov
	\end{itemize}
\item selské baroko v Holačovicích -- UNESCO
\item biosferické rezervace Třeboňsko a Šumava
\item \textbf{CHKO Třeboňsko}
	\begin{itemize}
	\item největší rybníkářská oblast v ČR
	\item Jakub Krčín z Jelčan
	\item Rožmberk
	\item hnízdiště ptactva
	\item Lužnice, Nežárka
	\end{itemize}
\item 
\end{itemize}

\paragraph{NP Šumava}
\begin{itemize}
\item nejvyšší vrchol pPlechý 1378m
\item šumavské slatě a pláne
\item ledovcová jezera
\item prales Boubín
\item pramen Vltavy, Otavy (Vydra)
\item nádrž Lipno
\item problémy s kůrovcem
\end{itemize}

\end{document}