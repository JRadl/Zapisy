\title{Česká republika}
\documentclass[10pt,a4paper]{article}
\usepackage[utf8]{inputenc}
\usepackage[czech]{babel}
\usepackage{amsmath}
\usepackage{amsfonts}
\usepackage{amssymb}
\usepackage{chemfig}
\usepackage{geometry}
\usepackage{wrapfig}
\usepackage{graphicx}
\usepackage{floatflt}
\usepackage{hyperref}
\usepackage{fancyhdr}
\usepackage{tabularx}
\usepackage{makecell}
\usepackage{csquotes}
\usepackage{footnote}
\usepackage{movie15}
\MakeOuterQuote{"}

\renewcommand{\labelitemii}{$\circ$}
\renewcommand{\labelitemiii}{--}
\newcommand{\ra}{$\rightarrow$ }
\newcommand{\x}{$\times$ }
\newcommand{\lp}[2]{#1 -- #2}
\newcommand{\timeline}{\input{timeline}}


\geometry{lmargin = 0.8in, rmargin = 0.8in, tmargin = 0.8in, bmargin = 0.8in}
\date{\today}
\author{Jakub Rádl}

\makeatletter
\let\thetitle\@title
\let\theauthor\@author
\makeatother

\hypersetup{
colorlinks=true,
linkcolor=black,
urlcolor=cyan,
}



\begin{document}
\maketitle
\tableofcontents
\begin{figure}[b]
Toto dílo \textit{\thetitle} podléhá licenci Creative Commons \href{https://creativecommons.org/licenses/by-nc/4.0/}{CC BY-NC 4.0}.\\ (creativecommons.org/licenses/by-nc/4.0/)
\end{figure}
\newpage

\section{Základní informace}
\paragraph{Vyhledej v atlase:}
\begin{enumerate}
\item rozlohu a počet obyvatel ČR [78 866km$^2$, 10.5 mil]
\item okrajové obce státu a jejich souřadnice [S:51,03 Lobendava; Z:12,05 Krásná; J:48,33 Vyšší brod; V: 18,51 Bukovec]
\item vzdálenost S--J, Z--V [278km, 493km]
\item nejkratší vzdálenost od moře [Šluknov -- Štětínský záliv v Baltském moři -- 326km]
\item délku státní hranice se sousedními zeměmi [N:810km, P:762km, R:466km, S:252km]
\item geografický střed: u obce Čihošť (49,44 s.š. 15,20 v.d.) (u Ledeče nad Sázavou)
\end{enumerate}

\paragraph{Státní znak}
\begin{itemize}
\item husitská pavéza
\item dvakrát český lev, moravská orlice, slezská orlice
\end{itemize}

\subsection{Poloha}
\paragraph{Matematickogeografická}
\begin{itemize}
\item ČR leží na 50. rovnoběžce a na 15. poledníku (středočeské časové pásmu)
\end{itemize}

\paragraph{Fyzickogeografická poloha}
\begin{itemize}
\item vnitrozemský stát ležící na rozhraní mezi oceánským a kontinentálním klimatem
\item střední nadmořská výška -- 450m (Evropa 350m)
\item ČR leží na hlavním evropském rozvodí -- nedostatek velkých toků
\item poloha na styku Českého masívu a Karpat -- hranice vede mezi Znojmem, Brnem, Olomoucí a Ostravou
\item nejvyšší bod -- Sněžka (1603)
\item nejnižší bod (115m)
\end{itemize}

\paragraph{Geopolitická}
\begin{itemize}
\item poloha na střetu mocností a mocenských zájmů
\item součást Rakousko-Uherska, zabrání Sudet, vytvoření protektorátu, sovětský blok, návrat do Evropy po r. 1989
\end{itemize}

\section{Historie}
\subsection{Hranice státu}
\begin{itemize}
\item přirozená nebo umělá
\item patří mezi nejstarší (cca 1000 let) a nejstabilnější v Evropě
\item tvořena pohraničním pásmem hor
\item celková délka --  2290km 
\item dlouhá vzhledem k rozloze státu
\item členitá -- výběžky: ašský, šluknovský, frýdlandský, broumovský, rychlebský, osoblažský
\item současná hranice vymezena mírovými smlouvami 
	\begin{itemize}
	\item Versailleská -- Německo (1919)
	\item st. germainská -- Rakousko (1919)
		\begin{itemize}
		\item zisk kusu dolních Rakous 
		\end{itemize}
	\item mezinárodní arbitráž -- Polsko (1920)
		\begin{itemize}
		\item spory o Těšín \ra město rozděleno na Polský a Český Těšín
		\end{itemize}
	\item mezinárodní smlouva -- Slovensko (1997)
	\end{itemize}
\end{itemize}

\subsection{Územní vývoj státu}
\begin{itemize}
\item 7. stol. -- Sámova říše
\item 9. stol. -- Velká Morava (Mikulčice, Uherské Hradiště, \ldots)
\item 10. stol. -- základy přemyslovského státu v Čechách, Morava připojena v polovině 11. století
\item 1212 -- Zlatá bula sicilská
\item 13. stol. -- rozšiřování území za vlády Přemysla Otakara II. a Václava II.
\item 14. stol. -- po vymření přemyslovců Lucemburkové -- vznik zemí Koruny české
\item 1526 -- bitva u Moháče, česko se stává součástí Habsburské říše
\item 28. 10. 1918 -- vznik samostatného Československa
\item 1938 -- mnichovská dohoda (konec tzv. první republiky)
\item 1939 -- Slovenský štát, Protektorát Čechy a Morava
\item 1945 -- ztráta Podkarpatské rusi
\item 1993 -- rozdělení státu na Českou a Slovenskou republiku
\end{itemize}

\subsection{Mapování našeho státu}
\begin{itemize}
\item 1518 -- nejstarší tištěná mapa Čech od Mikuláše Klaudyána
	\begin{itemize}
	\item mnoho erbů a textu, málo mapy
	\end{itemize}
\item 1569 -- mapa Moravy od Pavla Fabricia
\item 1627 -- mapa Moravy od J. A. Komenského
\item v letech 1760--1780 probíhá úřední mapování vojenské a civilní
\item 1935 -- Atlas Republiky Československé
\item 1966 -- Atlas Československé socialistické republiky
\end{itemize}


\section{Geologie}
\subsection{Geomorfologické členění}
\begin{itemize}
\item věda o tvaru zemského povrchu
\end{itemize}

\begin{tabular}{|c|c|}
	\hline
	\textbf{Provincie} & \textbf{Subprovincie}\\
	\hline
	Česká vysočina & Šumavská\\
	& Krušnohorská\\
	& Krkonošsko-jesenická\\
	& Poberounská\\
	& Česká tabule\\
	\hline
	Středoevropská nížina & Středopolské nížiny\\
	\hline
	Západní karpaty & Vněkarpatské sníženiny \\
	& Vnější západní karpaty\\
	\hline
	Panonská pánev & Vídeňská pánev\\
	\hline

\end{tabular}


\subsection{Česká vysočina}
\begin{itemize}
\item V České vysočině převládají předhercynské útvary -- hlavně krystalické břidlice (ČMV, Šumava)
	\begin{itemize}
	\item starší břidlice -- metamorfované horniny
	\end{itemize}
\item Mezi nejstarší nepřeměněné útvary náleží oblast Barrandienu (Praha -- Plzeň)
	\begin{itemize}
	\item Joachym Barand -- francouzský geolog, objevil spoustu zkamenělin, bylo zde moře
	\end{itemize}
	\item největší mocnost zemské kůry je v okolí Sedlčan -- 42km
	\item tektonicky aktivní oblasti jsou na Chebsku, Náchodsku, Opavsku
	\item v horninovém složení převažují žuly, pískovce, vápence a vulkanické horniny
\end{itemize}

\paragraph{Geologický vývoj}
\begin{itemize}
\item \textbf{prahory} (archaikum) -- vznik ČMV, Šumavy, jižních Čech
\item \textbf{starohory} (proterozoikum) -- nejstarší mořské usazeniny v Barandienu
\item \textbf{prvohory} (paleozoikum) --  kaledonské vrásnění, část pzemí zalita mořem, vznik devonských vápenců Moravského krasu, hercynské vrásnění -- vysoká pohoří, ložiska černého uhlí, moře pouze na okrajích masívu
\item \textbf{druhohory} (mezozoikum)  
	\begin{itemize}
	\item snižování horstev
	\item mělké křídové moře, po jeho ústupu vznik České křídové tabule
	\item labské pískovce v Českém Švýcarsku, Adržpach, Český ráj
	\end{itemize}
\item \textbf{třetihory} (kenozoikum)
	\begin{itemize}
	\item vliv alpinského vrásnění -- neproběhlo přímo Českou vysočinou, ale Západními Karpaty \ra zdvih okrajových pohoří, zlomy, poklesy
	\item vulkanismus v Českém středohoří, Doupovských horách a Nízkém Jeseníku (Venušina sopka, Velký a Malý roudný)
	\item vývoj říční sítě, hnědouhelné pánve -- Podkrušnohorský zlom
	\end{itemize}
\item \textbf{čtvrtohory} (kvartér)
	\begin{itemize}
	\item doznívající vulkanismus
	\item pleistocenní zalednění -- lokální horské ledovce v Krkonoších a na Šumavě
	\item váté písky a spraše, dotváření říční sítě
	\item vliv člověka
	\end{itemize}
\end{itemize}

\subsection{Šumavská subprovincie}
\begin{itemize}
\item Šumava -- Velký Javor(1456, nejvyšší na Šumavě), Plechý(1378, nejvyšší na české části Šumavy)
\item Šumavské podhůří
\item Český les -- Čerchov(1042)
\item Novohradské hory
\end{itemize}

\paragraph{Šumava}
\begin{itemize}
\item hraniční pohoří s Německem a z části s Rakouskem
\item jedna z geologicky zaoblených oblastí -- prahory, starohory
\item vliv alpinského vrásnění
\item lokální zalednění v pleistocénu
	\begin{itemize}
	\item ledovcová jezera -- Černé, Plešné, Čertovo, Prášilské, Laka
\item kary, morény
	\end{itemize}
\item národní park Šumava
	\begin{itemize}
	\item (rašeliniště) a pláně(náhorní plošiny)
	\item největší český národní park
	\item NPR Boubín
	\item dlouhodobé problémy s kůrovci
	\end{itemize}
\item společně s Bavorským lesem se jedná o nejzalesněnější část střední Evropy
\item prameny Vltavy (pod Černá horou) a Otavy (Vydra), Lipenská nádrž (horní tok Vltavy)
\item hřebeny Šumavy prochází hlavní evropské rozvodí
\end{itemize}

\subsection{Krkonošsko-jesenická subprovincie}
\begin{itemize}
\item na svazích Krkonoš se vytvořili čtyři vegetační výškové stupně
	\begin{itemize}
	\item listnaté a smíšené lesy s kulturními loukami 400--800 m
	\item horské smrčiny 800--1200 m
	\item lišejníková tundra, kamenité sutě 1450--1602 m
	\end{itemize}
\item Krkonoše -- poškozená příroda, od 89 pokusy o nápravu
\item pohoří má pestrou skladbu rostlin a živočichů s řadou endemitů a glaciálních reliktů
\item příroda hřebenů Krkonoš s drsným klimatem se podobá přírodě severní Evropy -- severská tundra
\item 
\end{itemize}

\paragraph{Broumovská vysočina}
\begin{itemize}
\item součástí je CHKO Broumovsko s Adršpašsko teplickými skálami a Broumovskými stěnami
\item lokalita patří k největším skalním městům střední Evropy
\item členitý reliéf způsobuje klimatickou inverzi
\item regionem protéká řeka Metuje
\end{itemize}

\paragraph{Kralický Sněžník}
\begin{itemize}
\item rulová klenba, ze které vybíhají směrem k jihu dva hřbety
\item pramenná oblast řeky Moravy
\item centrální body hlavního evropského rozvodí  .......
\end{itemize}

\paragraph{CHKO Hrubý Jeseník}
\begin{itemize}
\item nejvyšší pohoří Moravy se zaoblenými hřbety a hlubokými údolími
\itme hřebeny Jeseníků patří k nejchladnějším oblastem v republice (Praděd 0.9)
\item vyvinuté výškové vegetační stupně
\item Ramzovské a Červenohorské sedlo
\item NPR Praděd -- vrchol Pradědu, Petrovy kameny, Velká a Malá kotlina, Bílá Opava
\item NPR Šerák-Keprník -- zbytky pralesních porostů horských smrčin
\item lázně Jeseník, Ramzová, Červenohorské sedlo, Malá Morávka, Karlov, Karlova Studánka
\end{itemize}

\subsection{Českomoravská subprovincie}
\begin{itemize}
\item Brněnská vrchovina, Českomoravská vrchovina (Javořice, 837), Středočeská pahorkatina, Jihočeské pánve
\end{itemize}

\paragraph{Brněnská vrchovina}
\begin{itemize}
\item Drahanská, Bobravská vrchovina, Boskovická brázda
\item součástí Drahanské vrchoviny je \textbf{Moravský kras}
	\begin{itemize}
	\item největší krasové území ČR, cca 100 km$^2$
	\item tvořen devonskými vápenci
	\item CHKO se sídlem v Blansku
	\item povrchové a podpovrchové krasové jevy (kaňony, údolí, závrty, propasti, jeskyně, ponory, 
vývěry, \ldots)
	\item nejnavštěvovanější místa v \textit{severní části}
		\begin{itemize}
		\item Sloupsko šošůvské jeskyně (Kůlna -- nálezy neandrtálců 120 000 let)
		\item Punkevní jeskyně (prof. Absolon, Masarykův dóm)
		\item Macocha -- 138 m
		\item Amatérská jeskyně (41 km, Punkva)
		\item Kateřinská jeskyně
		\item Balcarka
		\end{itemize}
	\item \textit{střední část} -- Rudické propadání, Býčí skála
	\item \textit{jižní část} -- Ochozská jeskyně, Pekárna -- rytiny koní a bizonů na kostech zvířat (13 000 let)
	\end{itemize}
\end{itemize}

\end{document}