\title{Česká meziválečná poezie}
\documentclass[10pt,a4paper]{article}
\usepackage[utf8]{inputenc}
\usepackage[czech]{babel}
\usepackage{amsmath}
\usepackage{amsfonts}
\usepackage{amssymb}
\usepackage{chemfig}
\usepackage{geometry}
\usepackage{wrapfig}
\usepackage{graphicx}
\usepackage{floatflt}
\usepackage{hyperref}
\usepackage{fancyhdr}
\usepackage{tabularx}
\usepackage{makecell}
\usepackage{csquotes}
\usepackage{footnote}

\MakeOuterQuote{"}

\renewcommand{\labelitemii}{$\circ$}
\renewcommand{\labelitemiii}{--}
\newcommand{\ra}{$\rightarrow$ }
\newcommand{\x}{$\times$ }
\newcommand{\lp}[2]{#1 -- #2}
\newcommand{\timeline}{\input{timeline}}


\geometry{lmargin = 0.8in, rmargin = 0.8in, tmargin = 0.8in, bmargin = 0.8in}
\date{\today}
\author{Jakub Rádl}

\makeatletter
\let\thetitle\@title
\let\theauthor\@author
\makeatother

\hypersetup{
colorlinks=true,
linkcolor=black,
urlcolor=cyan,
}



\begin{document}
\maketitle
\tableofcontents
\begin{figure}[b]
Toto dílo \textit{\thetitle} podléhá licenci Creative Commons \href{https://creativecommons.org/licenses/by-nc/4.0/}{CC BY-NC 4.0}.\\ (creativecommons.org/licenses/by-nc/4.0/)
\end{figure}
\newpage
\section{Úvod}
\begin{itemize}
\item výrazně ovlivněno světovou poezií
\item \textbf{proletářská poezie} 
	\begin{itemize}
	\item vyobrazuje problémy světa dělníků, nepsali ji ale dělníci, nýbrž levicově orientovaní spisovatelé
	\item kolektivní cítění
	\item snaha o zlepšení podmínek, výzva k revoluci
	\end{itemize}
\end{itemize}


\subsection{Jiří Wolker (1900--1924)}
\begin{itemize}
\item zemřel na tuberkulózu
\item představitel proletářské poezie
\item pocházel z dobré rodiny, Prostějov
\item sbírka \textbf{Host do domu}
	\begin{itemize}
	\item mnoho personifikací, obrazných pojmenování
	\item ukazuje krásu všedních věcí
	\end{itemize}
\item sbírka \textbf{Těžká hodina}
	\begin{itemize}
	\item proletářská sbírka
	\item obsahuje sociální balady (problémy jsou dány společenskými poměry, ne proviněním jednotlivce)
	\item báseň \textbf{Těžká hodina} 
		\begin{itemize}
		\item chlapec umírá, ale muž se ještě nezrodil
		\item doufá, že bude schopný mít budoucí život podle sebe, že bude dobrý člověk, přínosný pro ostatní
		\end{itemize}
	\item balada \textbf{Balada o očích topičových}
		\begin{itemize}
		\item silně metaforické
		\end{itemize}
	\end{itemize}
\item pásmo \textbf{Svatý kopeček}
	\begin{itemize}
	\item navrací se k tomu, co prožíval na Svatém kopečku na Olomoucku, když jezdil za prarodiči	
	\end{itemize}
\end{itemize}

\subsection{Josef Hora (1891--1945)}
\begin{itemize}
\item úvahová, proletářská poezie, pokládá mnoho otázek, ale nedává odpovědi
\item motiv odměřování času
\item snaží se zachytit, jak doba a životní podmínky, revoluce ovlivňují život jednotlivce
\item sociální nespravedlnost světa
\item sbírka\textbf{Pracující den}
	\begin{itemize}
	\item upozorňuje na důležitost práce, obdiv k civilizaci a k těm, co ji vytvořili
	\end{itemize}
\item sbírka \textbf{Srdce a vřava světa}
\item sbírka \textbf{Struny ve větru}
\item básnická povídka \textbf{Jan houslista}
	\begin{itemize}
	\item má příběh
	\item apostrofy -- básnické oslovení
	\end{itemize}
\end{itemize}

\section{Poetismus}
\begin{itemize}
\item chápán jako styl života spíše než umělecký směr
\item "umění žít a užívat", "metoda, jak nazírat svět, aby byl básní"
\item vznik v českých zemích 
\item chce poklidnou revoluci, básně mají ukazovat skutečnost, která nedrtí, ale nadchne a posiluje
\item k tvůrčímu činu nemá vést úvaha, ale city, fantazie, podvědomí, proud asociací
\item prolnout umění a život, vyjadřovací prostředky různých druhů umění, literární druhy a žánry
\item snaha oslovit široké vrstvy
\item v poezii hovorovější vyjádření, obraznost, lyrický humor
\item v próze malý význam
\item v dramatu oslabený děj, scénické básně, pásma komediálních výjevů
\item sdružení \textbf{Devětsil}
	\begin{itemize}
	\item levicový umělecký svaz působící ve 20. letech 
	\item literáti, herci, výtvarníci, hudebníci
	\item proletářské umění, magický realismus (prolínání skutečnosti s iluzivními prvky), poetismus
	\item 1922 -- sborník \textbf{Devětsil}
	\item věřili, že k revoluci je nutná organizace, že jedinec nemůže sám ničeho ani umělecky dosáhnout
	\end{itemize}
\item na přelomu 20. a 30. let poetismus vyhasíná
\end{itemize}


\subsection{Vítězslav Nezval (1900-1958)}
\begin{itemize}
\item dílo spjato s rodištěm na Vysočině, otec učitel
\item spoluzakladatel poetismu
\item v díle se projevuje kreativita, fantasie, spontánnost, přisuzoval velkou váhu dětství, spojování zdánlivě k sobě nepatřících věcí 
\item próza \textbf{Dolce far niente} -- evokace jednoho dne z dětství
\item báseň \textbf{Abeceda}
	\begin{itemize}
	\item popisuje písmenka abecedy a co si s nimi asociuje
	\end{itemize}
\item sbírka \textbf{Most}
\item sbírka \textbf{Pantomima}
	\begin{itemize}
	\item Papoušek na motocyklu
	\item báseň \textbf{Podivuhodný kouzelník}
		\begin{itemize}
		\item kouzelník ztělesňuje poetistické představy o budoucím člověku opojeném životem, toužícím po lásce, revolucionáři, bezmocným před smrtí
	\end{itemize}
	\end{itemize}
\item báseň \textbf{Akrobat}
	\begin{itemize}
	\item symbolizuje moderního básníka překonávajícího jakékoliv bariéry
	\item pád ústřední postavy naznačuje, že i poezie má své limity
	\end{itemize}
\item báseň \textbf{Edison}
	\begin{itemize}
	\item srovnává básníka s vynálezcem (oba potřebují odvahu, houževnatost, pochybují o významu svého díla, ale mají neuvěřitelnou radost z úspěchu, kterou jiní nepoznají)
	\item apollinairovská poezie
	\end{itemize}
\item po úpadku poetismu ovlivněn surrealismem
\item 1838 -- \textbf{Matka Naděje}
	\begin{itemize}
	\item napsal ji po období obav o matčin život
	\item vydáno před začátkem války -> symbolický význam
	\end{itemize}
\item 1839 -- \textbf{Historický obraz} -- reakce na Mnichov
\item po 1940 nemohl publikovat
\item na konci života se vrací k rodnému kraji a v díle se projevuje vize šťastnějších zítřků
\item sbírka \textbf{Z domoviny}
\end{itemize}

\subsection{Jaroslav Seifert (1901--1986)}
\begin{itemize}
\item z chudého prostředí
\item nejdříve novinář, angažoval se v komunistické straně, ale byl z ní vyloučen, význačný předseda Svazu českých spisovatelů, signatář Charty 77
\item z počátku psal pod vlivem proletářské poezie -- sbírka \textbf{Na vlnách T. F. S.}
\item následně komornější poezie, ve které kritizuje svou předchozí práci -- sbírka \textbf{Slavík zpívá špatně}
\item sbírka \textbf{Poštovní holub}
	\begin{itemize}
	\item z části poetismus, z části existencialismus
	\item vědomí pomíjejícího času
	\end{itemize}
\item na začátku války sbírka \textbf{Zhasněte světla}
	\begin{itemize}
	\item velký ohlas
	\item od přírodní a intimní lyriky k zobrazení národního údělu 
	\end{itemize}
\item během války se obrací k národním hodnotám -- sbírka \textbf{Vějíři Boženy Němcové}
\item po válce veršované doprovody ke kresbám Mikoláše Aleše a Josefa Lady
\item sbírka \textbf{Maminka}
	\begin{itemize}
	\item sentimentální citové verše, ale i mnoho děje, občas příběh a humor
	\item centrální postava matky symbolizující harmonické rodinné vztahy
	\end{itemize}
\item v padesátých letech nepíše (nemocný, pronásledovaný režimem)
\item v další poezii opět klíčový čas -- sbírka \textbf{Koncert na ostrově}
\item sbírka \textbf{Morový sloup}
	\begin{itemize}
	\item milostné motivy se propojují s vyhlížením smrti a vzpomínáním na přátele
	\item v zoufalství ze stáří rezignuje na psaní 
	\end{itemize}
\end{itemize}

\section{Reflexivní a meditativní lyrika}
\subsection{František Halas (1901--1949)}
\begin{itemize}
\item z chudé rodiny, vyučen knihkupcem, novinář, komunistické mládežnické hnutí
\item sbírka \textbf{Sípie} 
	\begin{itemize}
	\item první sbírka, název přesmyčkou slova "poesie" -> poetismus
	\end{itemize}
\item sbírka \textbf{Kohout plaší smrt} -- mnoho neologismů, "láska ke slovu", nemelodické
\item následují melodičtější sbírky, milostná lyrika
\item sbírka \textbf{Staré ženy}
	\begin{itemize}
	\item depresivní obrazy soustředěné kolem stáří, pro staré ženy už na světě nic není
	\end{itemize}
\item sbírka \textbf{Torzo naděje}	
	\begin{itemize}
	\item zklamání z vývoje dějin po Mnichově
	\item dodává národu odvahu
	\end{itemize}
\item sbírka \textbf{Naše paní Božena Němcová} -- vlastenecká
\item sbírka \textbf{Ladění} -- v době Heydrichiády
\item ke konci života sbírka \textbf{A co?}
	\begin{itemize}
	\item ztracená naděje v komunismus, ve význam poezie
	\item obavy z poválečného vývoje světa
	\end{itemize}
\end{itemize}

\subsection{Jan Zahradníček (1905--1960)}
\begin{itemize}
\item rolnická rodina, studium v Praze, překladatel, spisovatel, redaktor, po válce se oženil, 1952 odsouzen k 13 letům vězení
\item sbírka \textbf{Pokušení smrti}
	\begin{itemize}
	\item nelibozvučné, expresivní verše, konfrontace nestálosti všeho pozemského s křesťanským řádem světa
	\end{itemize}
\item sbírka \textbf{Jeřáby}
	\begin{itemize}
	\item melodičtější verš
	\item utrpení vyváženo vírou, zápas o hloubky víry, střety mezi smyslovým okouzlením a duchovním prožitkem
	\item vnímá život jako boží dar
	\end{itemize}
\item během okupace psal o smutku z podrobení vlasti odporu vůči okupantům, nutnosti návratu ke křesťanskému řádu 
\item sbírka \textbf{Znamení moci}
	\begin{itemize}
	\item 7 zpěvů o zrůdnostech komunistického režimu
	\end{itemize}
\item 
\end{itemize}

\subsection{Vladimír Holan (1905--1980)}
\begin{itemize}
\item gymnázium, úředník, domníval se, že básnickou tvorbou lze proniknout pod povrch skutečnosti a dobrat se její podstaty, přetvářel jazyk
\item sbírka \textbf{Vanutí}
	\begin{itemize}
	\item náměty a motivy jsou v pohybu, mění se
	\item pohyb je vyjádřen i deformací běžných jazykových prostředků
	\item básně jsou proudem obrazných vyjádření
	\item využívání protikladů, protiklad konkrétního a abstraktního
	\item abstraktnější výrazy, rýmová a rytmická propracovanost \ra intelektuální, konstruovaná poezie
	\end{itemize}
\item sbírka \textbf{Kameni, přicházíš\ldots}
	\begin{itemize}
	\item využití antických motivů i s křesťanskými mýty
	\item ovlivněno i Španělskou občanskou válkou
	\end{itemize}
\item reakce na Mnichov, okupaci -- sbírky \textbf{Září 1938}, \textbf{Sen}
	\begin{itemize}
	\item motivy pohřbívání a zmaru -- metafora tehdejší Prahy
	\end{itemize}
\item současně píše i intimní lyriku
\item báseň \textbf{Terezka Planetová}
	\begin{itemize}
	\item báseň s epickými prvky, existencialismus
	\item vnějším tématem je smrt mladíka, na kterého spadl strom
	\item vnitřní téma -- starý lékař vypráví příběh o věcech, které ho přinutili odejít z domova, o lásce k Terezce, po návratu do rodné vesnice už si na dívku nikdo nepamatuje
	\end{itemize}
\item sbírka \textbf{Rudoarmějci}
	\begin{itemize}
	\item návrat k politické lyrice po válce
	\item portréty vojáků, které poznal
	\item návrat k mluvené řeči, pozvolný přechod k volnému verši, větší důraz na obsah než na formu
	\end{itemize}
\item meditativní skladba \textbf{Noc s Hamletem}
	\begin{itemize}
	\item obdivoval Shakespeara
	\item vede pomyslný dialog s Hamletem a Shakespearem o umění, politice, úpadku člověka
	\item epické prvky, obrazy, city a úvahy se spojují do jednoho toku, který má vyslovit úděl člověka
	\item Hamlet je faustovská bytost vydaná na pospas tragickému údělu, je mu odepřena láska
	\end{itemize}
\item v polovině padesátých let napsal uměleckou závěť (báseň Toskána) kvůli zdravotním komplikacím, později se vyléčil
\item ke konci života psal spíše jako pohled na skutečnost a sebe sama s odstupem
\end{itemize}

\subsection{Jiří Orten (1919--1941)}
\begin{itemize}
\item studoval hereckou konzervatoř, publikoval v denících a časopisech
\item za okupace musel vydávat pod pseudonymy 
\item ve 22 letech ho přejela sanitka a zemřel
\item Ortenova poezie nepřekračuje tématicky intimitu svého nitra, zdrojem nápadů je sama báseň, proces její tvorby a jazyk
\item sbírka \textbf{Čítanka Jaro} -- první sbírka, něžné, důvěrné verše (pod. J. Wolker: Host do domu)
\item sbírka \textbf{Cesta k mrazu} -- úzkostné, osamělé motivy
\item sbírka \textbf{Jeremiášův pláč} -- rozpor se starozákonným Bohem, který trestá nevinné
\item sbírka \textbf{Elegie}
	\begin{itemize}
	\item devět žalozpěvů nad ztraceným domovem, dětstvím a životem
	\item Sedmá elegie -- dopis Karin Michaelisové, zpovídá se ze svých lásek, řeší vztah k Bohu, život mu připadá jako absurdní sen
	\end{itemize}
\end{itemize}

\paragraph{Skupina 42}
\begin{itemize}
\item zaměření ke konkrétním, odpatetizovaným jevům a k epičnosti
\item svět se jeví jako disharmoncký, skutečnost se rozpadá do bezpočtu výjevů
\item fragmentárnost, útžkovitost, dialogičnost, odklon od obrazných pojmenování, prozaika verše
\item příběhy vyplývají ze současného města, jeho drsné nestylizované podoby
\item orientace na anglosaské tvůrce, překládání (Whitman, Eliot)
\item 1940 -- program \textbf{Svět ve kterém žijeme}
\item veřejnost seznámena s aktivitami až po válce
\item Jindřich Chalupecký, Jiří Kolář, Josef Kainar, Jiřina Hauková, Ivan Blatný, Jan Hanč
\end{itemize}

\end{document}