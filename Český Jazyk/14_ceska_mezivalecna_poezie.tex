\title{Česká meziválečná poezie}
\documentclass[10pt,a4paper]{article}
\usepackage[utf8]{inputenc}
\usepackage[czech]{babel}
\usepackage{amsmath}
\usepackage{amsfonts}
\usepackage{amssymb}
\usepackage{chemfig}
\usepackage{geometry}
\usepackage{wrapfig}
\usepackage{graphicx}
\usepackage{floatflt}
\usepackage{hyperref}
\usepackage{fancyhdr}
\usepackage{tabularx}
\usepackage{makecell}
\usepackage{csquotes}
\usepackage{footnote}
\usepackage{movie15}
\MakeOuterQuote{"}

\renewcommand{\labelitemii}{$\circ$}
\renewcommand{\labelitemiii}{--}
\newcommand{\ra}{$\rightarrow$ }
\newcommand{\x}{$\times$ }
\newcommand{\lp}[2]{#1 -- #2}
\newcommand{\timeline}{\input{timeline}}


\geometry{lmargin = 0.8in, rmargin = 0.8in, tmargin = 0.8in, bmargin = 0.8in}
\date{\today}
\author{Jakub Rádl}

\makeatletter
\let\thetitle\@title
\let\theauthor\@author
\makeatother

\hypersetup{
colorlinks=true,
linkcolor=black,
urlcolor=cyan,
}



\begin{document}
\maketitle
\tableofcontents
\begin{figure}[b]
Toto dílo \textit{\thetitle} podléhá licenci Creative Commons \href{https://creativecommons.org/licenses/by-nc/4.0/}{CC BY-NC 4.0}.\\ (creativecommons.org/licenses/by-nc/4.0/)
\end{figure}
\newpage
\section{Úvod}
\begin{itemize}
\item výrazně ovlivněno světovou poezií
\item \textbf{poetismus} -- 
\item \textbf{proletářská poezie} 
	\begin{itemize}
	\item vyobrazuje problémy světa dělníků, nepsali ji ale dělníci, nýbrž levicově orientovaní spisovatelé
	\item kolektivní cítění
	\item snaha o zlepšení podmínek, výzva k revoluci
	\end{itemize}
\end{itemize}

\subsection{Jiří Wolker (1900--1924)}
\begin{itemize}
\item zemřel na tuberkulózu
\item představitel proletářské poezie
\item pocházel z dobré rodiny, Prostějov
\item sbírka \textbf{Host do domu}
	\begin{itemize}
	\item mnoho personifikací, obrazných pojmenování
	\item ukazuje krásu všedních věcí
	\end{itemize}
\item sbírka \textbf{Těžká hodina}
	\begin{itemize}
	\item proletářská sbírka
	\item obsahuje sociální balady (problémy jsou dány společenskými poměry, ne proviněním jednotlivce)
	\item báseň \textbf{Těžká hodina} 
		\begin{itemize}
		\item chlapec umírá, ale muž se ještě nezrodil
		\item doufá, že bude schopný mít budoucí život podle sebe, že bude dobrý člověk, přínosný pro ostatní
		\end{itemize}
	\item balada \textbf{Balada o očích topičových}
		\begin{itemize}
		\item silně metaforické
		\end{itemize}
	\end{itemize}
\item pásmo \textbf{Svatý kopeček}
	\begin{itemize}
	\item navrací se k tomu, co prožíval na Svatém kopečku na Olomoucku, když jezdil za prarodiči	
	\end{itemize}
\end{itemize}

\subsection{Josef Hora (1891--1945)}
\begin{itemize}
\item úvahová, proletářská poezie, pokládá mnoho otázek, ale nedává odpovědi
\item motiv odměřování času
\item snaží se zachytit, jak doba a životní podmínky, revoluce ovlivňují život jednotlivce
\item sociální nespravedlnost světa
\item sbírka\textbf{Pracující den}
	\begin{itemize}
	\item upozorňuje na důležitost práce, obdiv k civilizaci a k těm, co ji vytvořili
	\end{itemize}
\item sbírka \textbf{Srdce a vřava světa}
\item sbírka \textbf{Struny ve větru}
\item básnická povídka \textbf{Jan houslista}
	\begin{itemize}
	\item má příběh
	\item apostrofy -- básnické oslovení
	\end{itemize}
\end{itemize}

\section{Poetismus}
\subsection{Vítězslav Nezval}
\begin{itemize}
\item báseň \textbf{Abeceda}
	\begin{itemize}
	\item popisuje písmenka abecedy a co si s nimi asociuje
	\end{itemize}
\end{itemize}



\end{document}