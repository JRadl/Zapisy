\title{Národní obrození}

\documentclass[10pt,a4paper]{article}
\usepackage[utf8]{inputenc}
\usepackage[czech]{babel}
\usepackage{amsmath}
\usepackage{amsfonts}
\usepackage{amssymb}
\usepackage{chemfig}
\usepackage{geometry}
\usepackage{wrapfig}
\usepackage{graphicx}
\usepackage{floatflt}
\usepackage{hyperref}
\usepackage{fancyhdr}
\usepackage{tabularx}
\usepackage{makecell}
\usepackage{csquotes}
\usepackage{footnote}

\MakeOuterQuote{"}

\renewcommand{\labelitemii}{$\circ$}
\renewcommand{\labelitemiii}{--}
\newcommand{\ra}{$\rightarrow$ }
\newcommand{\x}{$\times$ }
\newcommand{\lp}[2]{#1 -- #2}
\newcommand{\timeline}{\input{timeline}}


\geometry{lmargin = 0.8in, rmargin = 0.8in, tmargin = 0.8in, bmargin = 0.8in}
\date{\today}
\author{Jakub Rádl}

\makeatletter
\let\thetitle\@title
\let\theauthor\@author
\makeatother

\hypersetup{
colorlinks=true,
linkcolor=black,
urlcolor=cyan,
}



\begin{document}
\maketitle
\tableofcontents
\begin{figure}[b]
Toto dílo \textit{\thetitle} podléhá licenci Creative Commons \href{https://creativecommons.org/licenses/by-nc/4.0/}{CC BY-NC 4.0}.\\ (creativecommons.org/licenses/by-nc/4.0/)
\end{figure}
\newpage

\section*{Úvod}
\paragraph{Národní obrození}
\begin{itemize}
\item 70. léta 18. stol -- $\frac{1}{2}$19.stol
\item proces utváření novodobého českého národa (obnova češtiny, kultury, tradic, \ldots)
\end{itemize}
\paragraph{Události umožňující NO}
\begin{itemize}
\item reformy \textbf{Marie Terezie} a \textbf{Josefa II.}
	\begin{itemize}
	\item povinná školní docházka
	\item výuka češtiny na základních školách
	\item zrušení nevolnictví \ra stěhování obyvatel do měst \ra návrat češtiny z venkova
	\item toleranční patent \ra návrat náboženských emigrantů
	\end{itemize}
\item 1774 -- založena Královská česká společnost nauk \ra Akademie věd \ra šíření vědy a vzdělanosti
\item zrušení Jezuitského řádu \ra povolení zakázaných děl
\end{itemize}

\section{Obranná fáze (1774--1805)}
\subsection{Autoři}
\begin{itemize}
\item \textbf{defensivní}
\item spojena s \textbf{osvícenstvím}, důležitá role panovníka, význam nastolení řádu
\end{itemize}

\paragraph{Jozef Dobrovský}(1753--1829)
\begin{itemize}
\item (pohřben na Ústředním hřbitově)
\item učebnice \textbf{Dějiny české řeči a literatury}
	\begin{itemize}
	\item periodizace a zachycení průběhu literatury
	\item zakládá obor literární historie
	\end{itemize}
\item učebnice \textbf{Zevrubná mluvnice jazyka českého}
	\begin{itemize}
	\item první kodifikační příručka češtiny
	\item psána německy (pro vzdělané)
	\end{itemize}
\item slovník \textbf{Německo-český} (2 dílný)
	\begin{itemize}
	\item shrnul gramatiku, vymýšlel vlastní slova, psáno německy
	\end{itemize}
\item \textbf{Základy jazyka staroslověnského}
	\begin{itemize}
	\item mylně považuje staroslověnštiny za původní jazyk Slovanů
	\item pokládá základy Slavistiky (studium slovanské kultury a jazyka)
	\item psáno latinsky
	\end{itemize}
\end{itemize}

\paragraph{Gelasius Dobner}(1719--1790)
	\begin{itemize}
	\item při překládání do latiny odhalil vady v kronice Václava Hájka z Libočan (zmanipulována šlechtici)
	\end{itemize}
	
\paragraph{František Martin Pelcl} (1734--1801)
	\begin{itemize}
	\item \textbf{Kronika Česká}
	\end{itemize}
	
\paragraph{Václav Matěj Kramerius} (1753--1808)
	\begin{itemize}
	\item věnoval se publicistice
	\item nakladatelství \textbf{Česká Expedice}
		\begin{itemize}
		\item vydávalo lidové knížky pro zábavu
		\item nejen vydavatelství, ale i půjčovna
		\item vydávalo též noviny
		\end{itemize}
	\end{itemize}

\paragraph{Václav Thám} (1765--1816)
\begin{itemize}
\item anakreonská poezie (oslavuje radosti života -- víno, zpěv, ženy, \ldots)
\item sborník \textbf{Básně v řeči vázané}
	\begin{itemize}
	\item propagace české poezie
	\end{itemize}
\item historické divadelní hry \textbf{Václav a Jitka}, \textbf{Břetislav a Šárka}
	\begin{itemize}
	\item napsány pro divadlo \textbf{Bouda}
	\end{itemize}
\end{itemize}

\paragraph{Bohuslav Balbín}
\begin{itemize}
\item \textbf{Rozprava na obranu jazyka slovanského, zvláště pak českého}
\end{itemize}

\subsection{Divadlo}
\paragraph{Bouda}
\begin{itemize}
\item první \textbf{čistě české} divadlo
\item postaveno ze dřeva na koňském trhu 
\item i na tehdejší dobu velmi \textbf{slabá hygiena}
\item v provozu v letech \textbf{1786--1789}
\end{itemize}

\paragraph{Stavovské divadlo}
\begin{itemize}
\item původně \textit{Nosticovo divadlo} \ra prodáno českým stavům \ra \textit{Stavovské divadlo}
\item za komunismu přejmenováno \textit{Tylovo divadlo}, po Sametové revoluci zpět
\item původně německé, vyhrazena nedělní odpoledne, kdy se hrálo jen česky
\item 1786 -- premiéra W. A. Mozart: Don Giovanni (\textit{„Mí Pražané mi rozumějí“})
\item 1834 -- premiéra J. K. Tyl: Fidlovačka aneb žádný hněv a žádná rvačka (poprvé zazněla píseň \textbf{Kde domov můj})
\end{itemize}

\paragraph{V kotcích}
\paragraph{U Hybernů}



\newpage
\section{Útočná fáze (1805--1830)}
\begin{itemize}
\item \textbf{ofensivní}
\item snaha více psát v češtině a tvořit kvalitnější literaturu
\item snaha o české vědecké publikace
\item ovlivněna preromantismem
\end{itemize}

\subsection{Autoři}
\paragraph{Josef Jungman} (1773--1847)
\begin{itemize}
\item psal česky, překládal, odborná činnost
\item \textbf{Rozmlouvání o jazyku českém}
	\begin{itemize}
	\item dialog Čecha, Němce a Daniela Adama z Veleslavína
	\item Daniel je zhrozen tím, jak Čech mluví více německy, než česky
	\end{itemize}
\item čítanka a učebnice slohu \textbf{Slovesnost}
\item \textbf{Historie literatury české}
	\begin{itemize}
	\item obraz vývoje české literatury, doteď platí periodizace	
	\end{itemize}
\item slovník \textbf{Česko-německý}
	\begin{itemize}
	\item cílem bylo ukázat, že každé německé slovo má český ekvivalent
	\item 5 dílů
	\end{itemize}
\end{itemize}

\paragraph{František Palacký} (1789-1876)
\begin{itemize}
\item kronika \textbf{Dějiny národa českého v Čechách i v Moravě}
	\begin{itemize}
	\item po nástupu Habsburků
	\item základ Historiografie
	\item začal psát německy, v půlce (3. díl) přešel do češtiny
	\end{itemize}
\end{itemize}

\paragraph{Jan Kollár}(1793 - 1852)
\begin{itemize}
\item sbírka sonetů \textbf{Slávy dcera}
	\begin{itemize}
	\item části pojmenované podle míst, která kdysi obývali Slované (\textit{Sála, Labe, Dunaj, \ldots})
	\item stopa -- časomíra
	\item dnes nesrozumitelný jazyk
	\end{itemize}
\end{itemize}

\paragraph{František Ladislav Čelakovskij}
\begin{itemize}
\item \textbf{ohlasová poezie}
	\begin{itemize}
	\item záměrně napodobuje lidovou poezii
	\end{itemize}
\item sbírka básní \textbf{Ohlas písní ruských}
	\begin{itemize}
	\item napodobuje ruskou lidovou slovesnost
	\item byliny -- hlavní postavou bohatýři (bojaři -- Ilja Muromec, Čurila Plenkovič)
	\item epika -- příběhy
	\end{itemize}
\item sbírka básní \textbf{Ohlas písní českých}
	\begin{itemize}
	\item balada \textbf{Toman a lesní panna}
	\item většina básní lyrických
	\end{itemize}
\end{itemize}

\subsection{Rukopisy}
\begin{itemize}
\item měly potvrdit vyspělost českého jazyka
\item koncem 19. století byla odhalena nepravost, ve 20. století potvrzena vědeckými metodami
\item vytvořeny \textbf{Václavem Hankou} a \textbf{Josefem Lindou} (s pomocí restaurátora Horčičky)
\end{itemize}

\paragraph{Rukopis Královédvorský} (1817)
\begin{itemize}
\item označen za literární památku z \textbf{13. století}
\item 6 lyrických, 6 epických a 2 lyricko-epické skladby
\item epická báseň \textit{Jaroslav} (údajně část eposu)
\end{itemize}

\paragraph{Rukopis Zelenohorský} (1818)
\begin{itemize}
\item označen za literární památku z \textbf{10. století}
\item obsahuje část básní \textit{Libušin soud, Český sněm}
\item napsán na seškrábaný pergamen z 10. století
\end{itemize}

\paragraph{Reakce na rukopisy}
\begin{itemize}
\item společnost rozdělena podle toho, kdo věřil v jejich pravost
\item posílení vlastenectví
\item v době, kdy byly považovány za pravé, inspirovali mnoho dalších autorů
	\begin{itemize}
	\item Bedřich Smetana: Libuše
	\item nástěnné malby v Praze
	\end{itemize}
\item podobné padělky se objevují i v jiných světových literaturách (Rusko: Slovo o pluku Igorově)
\end{itemize}

\subsection{Divadlo}
\begin{itemize}
\item stavovské divadlo pravidelně uvádí české hry
\item Klicpera: Hadrián z Římsů, Rohovín Čtverrohý, Divotvorný klobouk
\end{itemize}



\newpage
\section{3. fáze (1830--1848)}
\subsection{Charakteristika}
\begin{itemize}
\item cíle národního obrození jsou již splněny
\item začíná se rozvíjet krásná literatura
\item zájem především o českou kulturu a literaturu
\item mizí myšlenka austroslavismu (sjednocení slovanských zemí pod Rakouskem)
\item návrat k absolutistickým principům
	\begin{itemize}
	\item Čechy jsou jen část státu \ra Rakousko uplatňuje své principy (kancléř Metternich)
	\item \ra slabší projevy vlastenectví aby nebyly problémy (nešlo o samostatný stát)
	\end{itemize}
	\item biedermeier -- životní styl měšťanstva (důležitá rodina, výchova dětí, domácnost, spokojený život s prvky vlastenectví, ale neprotestuje proti Česku jako součásti Habsburské monarchie
\end{itemize}

\paragraph{Literatura}
\begin{itemize}
\item z počátku převažovaly výchovné cíle nad uměleckými
\item významné žánry
	\begin{itemize}
	\item novela
	\item kalendářová povídka -- na každém dnu kalendáře je krátký příběhu
	\item dramatická báchorka -- soudobý námět, nadpřirozené postavy, zpěv, hudba, napravení hlavní postavy
	\end{itemize}
\item vliv romantismu (K. H. Mácha)
\item začleňování společenských prvků do tvorby
\end{itemize}

\subsection{Starší generace}
\begin{itemize}
\item zastávají starší pozice
\item odsuzují polské povstání Slovanů proti Slovanům
\item Josef Jungman
\item Václav Hanka
\end{itemize}

\subsection{Mladá generace}
\begin{itemize}
\item zastává názor osamostatnění české kultury
\end{itemize}

\paragraph{Josef Kajetán Tyl (1808--1856)}
\begin{itemize}
\item ženatý s herečkou, měl 6 dětí s její sestrou
\item žurnalistická činnost
	\begin{itemize}
	\item časopis \textbf{Květy}, \textbf{Pražský posel}
	\item snaha o vzdělání a vychování čtenářů
	\item články o správném životním stylu, alkoholu, \ldots
	\item velmi plodný -- celé číslo časopisu mohlo být jeho dílem (měl také hodně dětí)
	\end{itemize}
\item činnost literární \textbf{prozaická}
	\begin{itemize}
	\item povídky: historické, vlastenecké, ze současnosti s idylickým koncem
	\item povídka \textbf{Poslední Čech}
		\begin{itemize}
		\item kritizováno Karlem Havlíčkem Borovským
		\end{itemize}
	\item povídka \textbf{Dekret Kutnohorský}
	\item povídka \textbf{Rozervanec}
		\begin{itemize}
		\item kritizuje, že Mácha v sobě má moc osobností, které si protiřečí 
		\item postavy Karel a Hynek se hádají (zanícený vlastenec $\times$ romantik)
		\end{itemize}
	\end{itemize}
\item \textbf{dramatická} činnost
	\begin{itemize}
	\item historické hry
		\begin{itemize}
		\item \textbf{Kutnohorští havíři} -- spor o práva horníků, milost od panovníka je udělena až po popravě
		\item \textbf{Jan Žižka}
		\item \textbf{Jan Hus}
		\end{itemize}
	\item soudobé hry
		\begin{itemize}
		\item \textbf{Fidlovačka aneb žádný hněv a žádná rvačka}
			\begin{itemize}
			\item \textbf{Kde domov můj?} (hudba František Škroup)
			\item fidlovačka -- ševcovská slavnost
			\item zamilovaní příslušníci dvou znepřátelených rodů \ra happy end
			\end{itemize}
		\item \textbf{Paličova dcera}
			\begin{itemize}
			\item popisuje chování na vesnici 
			\item není moc pozitivní
			\item 
			\end{itemize}
		\end{itemize}
	\item dramatické báchorky -- soudobý příběh obohacený o pohádkové motivy, doplněný zpěvy, hudbou a polepšením hlavní postavy
		\item \textbf{Strakonický dudák} (\textit{čít. 131})
		\begin{itemize}
		\item postavy
			\begin{itemize}
			\item Švanda -- naivní, nezodpovědný
			\item Pantaleon Vocílka -- vychytralý, výřečný, 
			\end{itemize}
		\item Švanda se vydal do světa vydělat peníze, aby si mohl vzít Dorotku, Vocílka se stane jeho  sekretářem, obírá ho o peníze, nakonec se Švanda vrátí domů a zjistí, že důležitá je hlavně poctivá práce.
		\end{itemize}
	\end{itemize}

\end{itemize}


\paragraph{Karel Havlíček Borovský}
\begin{itemize}
\item 
\end{itemize}

\subsubsection{Karel Hynek Mácha(1810--1836)}
\paragraph{Život:}
\begin{itemize}
\item typický romantik
\item umělecké nadání od matky
\item na gymnáziu získal vlastenecké myšlenky od Josefa Jungmana
\item student filosofie a práv v Praze
\item pomáhal polským revolucionářům, časté problémy s policií
\item působil v ochotnické družině J. K. Tyla
\item koníček cestování po hradech
\item zamilovaný do Marinky Štichové
\item později se seznámil s Eleonorou Šomkovou -- vztah byl nevyrovnaný, protože Mácha byl intelektuál a hodně žárlivý
\item syn s Lori zemřel v méně než 1 roku
\item zemřel při hašení požáru, pohřben byl ve 26 letech v den své plánované svatby
\item ostatky přeneseny na hřbitov na Vyšehradě na začátku okupace Německem za 2. sv. v.
\item nebyla zachována podoba \ra Emanuel Vlček vytvořil přibližnou podobiznu podle antropologických měření Máchovy lebky
\end{itemize}
\paragraph{Tvorba:}
\begin{itemize}
\item psaní pro něj bylo jen koníčkem
\item většina jeho děl byla vydána až po jeho smrti
\item na vlastní náklady vydal \textbf{Máj}
	\begin{itemize}
	\item za jeho života nebyl pochopen, jelikož stále šlo o vlastenectví více než o romantismus
	\end{itemize}
\item inspirován Shakespearem, Goethem, Byronem, Hájkem, Herderem
\item inspirací pro něj byla též příroda, především krajina okolo Bezdězu
\end{itemize}
\paragraph{Dílo:}
\begin{itemize}
\item poezie česká a německá
\item próza
	\begin{itemize}
	\item dvě povídky \textbf{Obrazy ze života mého}
		\begin{itemize}
		\item lyrizovaná próza
		\item Večer na Bezdězu
		\item Márinka
			\begin{itemize}
			\item inspirována Márinkou Štichovou
			\item v ich formě: o tom, jak se do ní zamiloval, jak odjel do Krkonoš a než dojel zpět ona zemřela
			\end{itemize}
		\end{itemize}
	\item povídka \textbf{Pouť krkonošská}
		\begin{itemize}
		\item lyrizovaná próza
		\item označováno za první český horor
		\item hrdina najde klášter na vrcholu Sněžky kde jsou mniši, kteří vstávají z mrtvých
		\end{itemize}
	\item román \textbf{Cikáni}
		\begin{itemize}
		\item odehrává se v okolí Kokořína
		\item děj připomíná Máj
		\item popisuje milostné problémy Starého a Mladého cikána
		\item vyjadřuje smutek a rozčarování ze života
		\end{itemize}
	\item tetralogie románů \textbf{Kat - Křivoklad}
		\begin{itemize}
		\item putování po hradech, za vlády Václava IV. 
		\item kat je šlechtického původu, ale je levoboček 
		\item zamýšlena tetralogie, stihl napsat jen první díl -- Křivoklad
		\end{itemize}
	\item básnická povídka \textbf{Máj}
		\begin{itemize}
		\item jediná kniha vydaná za jeho života
		\item na vlastní náklady vydal 600 výtisků, které se prodali
		\item česká společnost dílo nepochopila a odsoudila, jelikož nesplňovalo ideály naučného díla národního obrození
		\item přeložen do mnoha jazyků
		\end{itemize}
	\end{itemize}
\item hodně kreslil (především hrady)
\end{itemize}

\paragraph{Máchův kult}
\begin{itemize}
\item Karel Sabina
\item Jakub Arbes -- rozšifroval Máchovy deníky
\item Jan Neruda
\item František Xaver Šalda
\end{itemize}

\subsubsection{Karel Jaromír Erbern}
\begin{itemize}
\item inspirován lidovou tvořivostí (pověsti, ...)
\item pohádky Pták ohnivák, Zlatovláska, Dlouhý Široký a Bystrozraký
\end{itemize}




\section{4. fáze (1850--1860)}
\begin{itemize}
\item literatura se začíná vyvíjet od romantismu směrem k realismu
\item objevují se prvky z většiny směrů
\item počátek kritického realismu
\end{itemize}

\subsection{Božena Němcová}
\paragraph{Život}
\begin{itemize}
\item 1820? -- 1862, za svobodna Barbora Panklová
	\begin{itemize}
	\item pravděpodobně se narodila o dva roky dříve
	\item je možné, že byla nemanželskou dcerou ze šlechtického prostředí a byla rodinou Panklových pouze vychovávána
	\item otec pracoval jako panský kočí \ra spekulace, že byla dcerou někoho ze dvora
	\item problematika je rozebírána v knize Miroslava Ivanova: Zahrada života paní Betty
	\end{itemize}
\item v 17(19) se provdala za Josefa Němce
	\begin{itemize}
	\item výrazně starší, členem finanční správy
	\item uvedl ji do pražské vlastenecké společnosti
	\item byla inteligentnější a vzdělanější než on
	\item problematický vztah \ra mnoho afér
	\end{itemize}
\item syn Hynek zemřel ve 14 letech \ra motiv pro napsání Babičky
\item ke konci života velmi chudá
\end{itemize}

\paragraph{Tvorba}
\begin{itemize}
\item sbírala ústní slovesnost, sepisovala příběhy, pohádky, říkadla
\item vlastní tvorba -- básně dnes zapomenuty, próza
\item povídky -- snaha zachytit rozdíly mezi životem na venkově a ve městě (nepodléhá biedermaieru)
\item v tvorbě se prolíná romantismus s realismem
\item pokus o román, nebyl úspěšný
\end{itemize}

\paragraph{Dílo}
\begin{itemize}
\item povídky 
	\begin{itemize}
	\item \textbf{Pohorská vesnice} -- nedotažený pokus o román, směřuje k idyle
	\item \textbf{V zámku a podzámčí} -- porovnává život šlechty a poddaných, na konci náprava paní
	\item \textbf{Karla} -- Karel vychováván jako dívka aby nemusel na vojnu
	\item \textbf{Horská vesnice}
	\item \textbf{Pan učitel} -- idealizovaná představa učitele, který dělá vše pro druhé
	\item \textbf{Babička}
		\begin{itemize}
		\item rozděleno na čtyři roční doby
		\item děj se odehrává v průběhu deseti let
		\item zachytává vzpomínky na babičku Magdalenu Novotnou, která s rodinou Panklových žila na Starém Bělidle
		\item babička je idealizovaně vykreslována jako postava bez chyb (moudrá, milá, pomáhá všem ve svém okolí, nejen chudým, ale i šlechtě)
		\item vloženy epizodní příběhy (například o Viktorce)
		\end{itemize}

	\item \textbf{Divá Bára}
		\begin{itemize}
		\item velmi chudá dívka, dcera pastýře
		\item ničeho se nebojí (bere věci realisticky)
		\item pomáhá Elišce aby si nemusela vzít někoho, koho nemiluje, je odhalena \ra musí strávit den v márnici
		\item Eliška se nakonec nemusí brát, Bára potká lásku \ra happy end
		\end{itemize}
	\end{itemize}
\item \textbf{Obrazy z okolí domažlického}
	\begin{itemize}
	\item reportážní příběhy psané uměleckým jazykem ze zážitků na Domažlicku
	\end{itemize}
\end{itemize}

\subsection{Karel Havlíček Borovský (1821--1856)}
\paragraph{Život}
\begin{itemize}
\item kupecká rodina, Borová u Německého brodu \ra přídomek Borovský
\item studoval filozofii
\item nelákalo ho "prázdné vlastenectví"
\item kritizoval Tylova Posledního Čecha
\item zatčen -- souzen za "pobuřující" články kritizující ....
\item exil v Brixenu -- dobré prostředí, ale psychicky vyčerpávající, od ženy se nakazil Tuberkulózou
\item dcera Zdena po smrti vychovávána českou vlasteneckou společností
	\begin{itemize}
	\item odsuzována za zamilování se do rakouského vojáka
	\end{itemize}
\end{itemize}

\paragraph{Tvorba}
\begin{itemize}

\item epigramy, publicistická činnost
\item při návštěvě absolutistického Ruska opustil myšlenku, že by se slovanské státy měly sjednotit
\item v Rusku inspirován ke knize \textit{Obrazy z Rus} a k epigramům
\end{itemize}

\paragraph{Dílo}
\begin{itemize}
\item oddíly epigramů Církvi, Králi, Vlasti, Múzám, Světu
	\begin{itemize}
	\item inspirován Ruskem a Rakouskem
	\item \textit{Jehly, špičky, sochory a kůly\\
stesal, skoval, zostřil, sebral\\
kvůli vojně s hloupostí a zlobou místo šavel\\
Borovský Havel.}
	\end{itemize}
\item Epištoly kutnohorské, Duch Národních novin
\item satirická skladba \textbf{Tyrolské elegie}
	\begin{itemize}
	\item inspirováno pobytem v Tyrolsku
	\item kritizuje absolutismus, popisuje své zatčení a deportaci do Brixenu
	\item příhoda s koňmi
	\end{itemize}
\item satirická skladba \textbf{Křest svatého Vladimíra} -- nedokončené
	\begin{itemize}
	\item pojednává o rozšíření křesťanství na Kyjevské Rusi
	\end{itemize}
\item satirická skladba \textbf{Král Lávra} -- inspirováno antickým příběhem o králi Midasovi a oslích uších
	\begin{itemize}
	\item děj:
		\begin{itemize}
		\item král má oslí uši, snaží se to skrýt, zabíjí proto své holiče
		\item jednou povolá Kukulína, odsoudí ho k smrti, na přání jeho matky ho ušetřil
		\item Kukulín slíbil, že tajemství zachová, ale pošeptal ho vrbě
		\item basista si vyrobil kolíček z Vrby, ta zpívala, že král Lávra má oslí uši
		\end{itemize}
	\item kritika toho, že si vláda může dělat co chce
	\item kritika společnosti, že si nechá lecos líbit
	\end{itemize}
\item \textbf{Obrazy z Rus} -- inspirovány pobytem v Rusku
\end{itemize}

\end{document}