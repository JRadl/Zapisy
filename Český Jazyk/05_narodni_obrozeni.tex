\title{Národní obrození}
\documentclass[10pt,a4paper]{article}
\usepackage[utf8]{inputenc}
\usepackage[czech]{babel}
\usepackage{amsmath}
\usepackage{amsfonts}
\usepackage{amssymb}
\usepackage{chemfig}
\usepackage{geometry}
\usepackage{wrapfig}
\usepackage{graphicx}
\usepackage{floatflt}
\usepackage{hyperref}
\usepackage{fancyhdr}
\usepackage{tabularx}
\usepackage{makecell}
\usepackage{csquotes}
\usepackage{footnote}

\MakeOuterQuote{"}

\renewcommand{\labelitemii}{$\circ$}
\renewcommand{\labelitemiii}{--}
\newcommand{\ra}{$\rightarrow$ }
\newcommand{\x}{$\times$ }
\newcommand{\lp}[2]{#1 -- #2}
\newcommand{\timeline}{\input{timeline}}


\geometry{lmargin = 0.8in, rmargin = 0.8in, tmargin = 0.8in, bmargin = 0.8in}
\date{\today}
\author{Jakub Rádl}

\makeatletter
\let\thetitle\@title
\let\theauthor\@author
\makeatother

\hypersetup{
colorlinks=true,
linkcolor=black,
urlcolor=cyan,
}



\begin{document}
\maketitle
\tableofcontents
\begin{figure}[b]
Toto dílo \textit{\thetitle} podléhá licenci Creative Commons \href{https://creativecommons.org/licenses/by-nc/4.0/}{CC BY-NC 4.0}.\\ (creativecommons.org/licenses/by-nc/4.0/)
\end{figure}
\newpage

\section*{Úvod}
\paragraph{Národní obrození}
\begin{itemize}
\item 70. léta 18. stol -- $\frac{1}{2}$19.stol
\item proces utváření novodobého českého národa (obnova češtiny, kultury, tradic, \ldots)
\end{itemize}
\paragraph{Události umožňující NO}
\begin{itemize}
\item reformy \textbf{Marie Terezie} a \textbf{Josefa II.}
	\begin{itemize}
	\item povinná školní docházka
	\item výuka češtiny na základních školách
	\item zrušení nevolnictví \ra stěhování obyvatel do měst \ra návrat češtiny z venkova
	\item toleranční patent \ra návrat náboženských emigrantů
	\end{itemize}
\item 1774 -- založena Královská česká společnost nauk \ra Akademie věd \ra šíření vědy a vzdělanosti
\item zrušení Jezuitského řádu \ra povolení zakázaných děl
\end{itemize}

\section{Obranná fáze (1774--1805)}
\subsection{Autoři}
\begin{itemize}
\item \textbf{defensivní}
\item spojena s \textbf{osvícenstvím}, důležitá role panovníka, význam nastolení řádu
\end{itemize}

\paragraph{Jozef Dobrovský}(1753--1829)
\begin{itemize}
\item (pohřben na Ústředním hřbitově)
\item učebnice \textbf{Dějiny české řeči a literatury}
	\begin{itemize}
	\item periodizace a zachycení průběhu literatury
	\item zakládá obor literární historie
	\end{itemize}
\item učebnice \textbf{Zevrubná mluvnice jazyka českého}
	\begin{itemize}
	\item první kodifikační příručka češtiny
	\item psána německy (pro vzdělané)
	\end{itemize}
\item slovník \textbf{Německo-český} (2 dílný)
\item \textbf{Základy jazyka staroslověnského}
	\begin{itemize}
	\item mylně považuje staroslověnštiny za původní jazyk Slovanů
	\item pokládá základy Slavistiky (studium slovanské kultury a jazyka)
	\item psáno latinsky
	\end{itemize}
\end{itemize}

\paragraph{Gelasius Dobner}(1719--1790)
	\begin{itemize}
	\item odhalil vady v kronice Václava Hájka z Libočan
	\end{itemize}
	
\paragraph{František Martin Pelcl} (1734--1801)
	\begin{itemize}
	\item \textbf{Kronika Česká}
	\end{itemize}
	
\paragraph{Václav Matěj Kramerius} (1753--1808)
	\begin{itemize}
	\item věnoval se publicistice
	\item nakladatelství \textbf{Česká Expedice}
		\begin{itemize}
		\item vydávalo lidové knížky pro zábavu
		\item nejen vydavatelství, ale i půjčovna
		\item vydávalo též noviny
		\end{itemize}
	\end{itemize}

\paragraph{Václav Thám} (1765--1816)
\begin{itemize}
\item anakreonská poezie (oslavuje radosti života -- víno, zpěv, ženy, \ldots)
\item sborník \textbf{Básně v řeči vázané}
	\begin{itemize}
	\item propagace české poezie
	\end{itemize}
\item historické divadelní hry \textbf{Václav a Jitka}, \textbf{Břetislav a Šárka}
	\begin{itemize}
	\item napsány pro divadlo \textbf{Bouda}
	\end{itemize}
\end{itemize}

\subsection{Divadlo}
\paragraph{Bouda}
\begin{itemize}
\item první \textbf{čistě české} divadlo
\item postaveno ze dřeva na koňském trhu 
\item i na tehdejší dobu velmi \textbf{slabá hygiena}
\item v provozu v letech \textbf{1786--1789}
\end{itemize}

\paragraph{Stavovské divadlo}
\begin{itemize}
\item původně \textit{Nosticovo divadlo} \ra prodáno českým stavům \ra \textit{Stavovské divadlo}
\item za komunismu přejmenováno \textit{Tylovo divadlo}, po Sametové revoluci zpět
\item původně německé, vyhrazena nedělní odpoledne, kdy se hrálo jen česky
\item 1786 -- premiéra W. A. Mozart: Don Giovanni (\textit{„Mí Pražané mi rozumějí“})
\item 1834 -- premiéra J. K. Tyl: Fidlovačka aneb žádný hněv a žádná rvačka (poprvé zazněla píseň \textbf{Kde domov můj})
\end{itemize}
\paragraph{V kotcích}
\paragraph{U Hybernů}

\section{Útočná fáze}
\begin{itemize}
\item \textbf{ofensivní}
\item snaha více psát v češtině a tvořit kvalitnější literaturu
\item snaha o české vědecké publikace
\item ovlivněna preromantismem
\end{itemize}

\subsection{Autoři}
\paragraph{Josef Jungman} (1773--1847)
\begin{itemize}
\item psal česky, překládal, odborná činnost
\item \textbf{Rozmlouvání o jazyku českém}
	\begin{itemize}
	\item dialog Čecha, Němce a Daniela Adama z Veleslavína
	\item Daniel je zhrozen tím, jak Čech mluví více německy, než česky
	\end{itemize}
\item čítanka a učebnice slohu \textbf{Slovesnost}
\item \textbf{Historie literatury české}
	\begin{itemize}
	\item obraz vývoje české literatury, doteď platí periodizace	
	\end{itemize}
\item slovník \textbf{Česko-německý}
	\begin{itemize}
	\item cílem bylo ukázat, že každé německé slovo má český ekvivalent
	\item 5 dílů
	\end{itemize}
\end{itemize}

\paragraph{František Palacký} (1789-1876)
\begin{itemize}
\item kronika \textbf{Dějiny národa českého v Čechách i v Moravě}
	\begin{itemize}
	\item po nástupu Habsburků
	\item základ Historiografie
	\item začal psát německy, v půlce přešel do češtiny
	\end{itemize}
\end{itemize}

\paragraph{Jan Kollár}(1793 - 1852)
\begin{itemize}
\item sbírka sonetů \textbf{Slávy dcera}
	\begin{itemize}
	\item části pojmenované podle míst, která kdysi obývali Slované (\textit{Sála, Labe, Dunaj, \ldots})
	\item stopa -- časomíra
	\item dnes nesrozumitelný jazyk
	\end{itemize}
\end{itemize}

\paragraph{František Ladislav Čelakovskij}
\begin{itemize}
\item \textbf{ohlasová poezie}
	\begin{itemize}
	\item záměrně napodobuje lidovou poezii
	\end{itemize}
\item sbírka básní \textbf{Ohlas písní ruských}
	\begin{itemize}
	\item napodobuje ruskou lidovou slovesnost
	\item byliny -- hlavní postavou bohatýři (bojaři -- Ilja Muromec, Čurila Plenkovič)
	\item epika -- příběhy
	\end{itemize}
\item sbírka básní \textbf{Ohlas písní českých}
	\begin{itemize}
	\item balada \textbf{Toman a lesní panna}
	\item většina básní lyrických
	\end{itemize}
\end{itemize}

\subsection{Rukopisy}
\begin{itemize}
\item měly potvrdit vyspělost českého jazyka
\item koncem 19. století byla odhalena nepravost, ve 20. století potvrzena vědeckými metodami
\item vytvořeny \textbf{Václavem Hankou} a \textbf{Josefem Lindou} (s pomocí restaurátora Horčičky)
\end{itemize}

\paragraph{Rukopis Královédvorský} (1817)
\begin{itemize}
\item označen za literární památku z \textbf{13. století}
\item 6 lyrických, 6 epických a 2 lyricko-epické skladby
\end{itemize}

\paragraph{Rukopis Zelenohorský} (1818)
\begin{itemize}
\item označen za literární památku z \textbf{10. století}
\item obsahuje část básně \textit{Libušin soud}
\item napsán na seškrábaný pergamen z 10. století
\end{itemize}

\paragraph{Reakce na rukopisy}
\begin{itemize}
\item společnost rozdělena podle toho, kdo věřil v jejich pravost
\item posílení vlastenectví
\item v době, kdy byly považovány za pravé, inspirovali mnoho dalších autorů
	\begin{itemize}
	\item Bedřich Smetana: Libuše
	\item nástěnné malby v Praze
	\end{itemize}
\item podobné padělky se objevují i v jiných světových literaturách (Rusko: Slovo o pluku Igorově)
\end{itemize}

\subsection{Divadlo}
\begin{itemize}
\item stavovské divadlo pravidelně uvádí české hry
\item Klicpera: Hadrián z Římsů, Rohovín Čtverrohý 
\end{itemize}



\end{document}