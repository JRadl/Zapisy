\documentclass[10pt,a4paper]{article}
\usepackage[utf8]{inputenc}
\usepackage[czech]{babel}
\usepackage{amsmath}
\usepackage{amsfonts}
\usepackage{amssymb}
\usepackage{chemfig}
\usepackage{geometry}
\usepackage{wrapfig}
\usepackage{floatflt}
\usepackage{hyperref}
\geometry{lmargin = 0.8in, rmargin = 0.8in, tmargin = 0.8in, bmargin = 0.8in}
\title{Český jazyk -- 1. ročník}
\date{}
\author{}
\begin{document}
\maketitle
\tableofcontents
\newpage
\section{Sloh}
Umění vybrat adekvátní jazykové prostředky splňující cíl a srozumitelné příjemci komunikačně sdělit projev

\subsection{Slohotvorní činitelé}
\paragraph{Subjektivní}
\begin{itemize}
\item slovní zásoba, aktuální nálada, vzdělání, talent, účel, motivace
\end{itemize}

\paragraph{Objektivní}
\begin{itemize}
\item cíl projevu, společenské okolí, adresát, zamýšlená forma (mluvený, písemný), připravenost
\end{itemize}


\subsection{Slohové útvary}
\paragraph{Charakteristika}
\begin{itemize}
\item vnější -- popis vzhledu
\item vnitřní -- popis jeho vlastností
\item přímá -- přímo popisujeme
\item nepřímá -- popis pomocí příběhu
\end{itemize}

\paragraph{Vyprávění}
\begin{itemize}
\item často používá přímou řeč
\item \textbf{presens historický} -- přechod do přítomného času uprostřed minulého
\item \textbf{in media res} -- vhození čtenáře do děje (bez úvodu)
\item využívá krátké, úsečné věty / větné ekvivalenty v napínavých momentech
\item má zápletku a rozuzlení
\end{itemize}

\paragraph{Líčení}
\begin{itemize}
\item citově a subjektivně zabarvený básnický popis
\end{itemize}

\paragraph{Úvaha}
\begin{itemize}
\item subjektivní styl, obsahuje vlastní názory autora
\item může nabízet více úhlů pohledu
\item může uvažovat i o cizích nápadech
\item musí mít dokonalou kompozici
\end{itemize}

\paragraph{Popis děje}
\begin{itemize}
\item vyprávění bez citového zabarvení, napětí, zápletky, \ldots
\item výstižný
\item sleduje stejný děj/postavu od začátku do konce
\end{itemize}

\paragraph{Popis (předmětu)}
\begin{itemize}
\item cílem je popsat předmět tak, aby si ho adresát dokázal představit
\item postupný, přehledný, systematický, \ldots
\end{itemize}

\paragraph{Zpráva}
\begin{itemize}
\item informuje události (co, kdy, kde, jak, pro koho, \ldots)
\item úkolem je zaujmout
\item objektivní, psaná spisovnou češtinou
\item \textbf{zpráva s hodnocením} -- v závěru subjektivní zhodnocení události
\end{itemize}


\subsubsection{Funkční styly}
Každý styl má nějaká pravidla a funkci.
\paragraph{Prostěsdělovací}\mbox{}\\
\begin{itemize}
\item výměna informací běžného rázu
\item v každém útvaru se vyskytuje na základní úrovni
\item zpráva, oznámení, dopis, email, přání, inzerát, návod, pracovní, postup, SMS, chat, vypravování, úvaha, výklad, běžné tiskopisy
\item rozhovor, přípitek, představení se, omluva, telefonický rozhovor, blahopřání, diskuze
\end{itemize}

\paragraph{Publicistický}
\begin{itemize}
\item cílem je informovat, případně přesvědčit o nějakém názoru
\item může ho využít většina útvarů
\end{itemize}

\paragraph{Řečnický}
\begin{itemize}
\item cílem je informovat, poučit, pobavit, zaujmout, ovlivnit
\item primární využití pro mluvený projev
\end{itemize}

\paragraph{Odborný}
\begin{itemize}
\item vysvětlit
\end{itemize}

\paragraph{Administrativní}
\begin{itemize}
\item cílem je zprostředkovat komunikaci
\item využití při záznamech, soudních záznamech, daňových přiznáních, ...
\end{itemize}

\paragraph{Umělecký}
\begin{itemize}
\item funkce estetická -- cílem je vyvolat emoce
\end{itemize}

\paragraph{Esejistický}
\begin{itemize}
\item propojuje prvky jiných stylů
\item využití pro \textbf{esej} -- odborné vyjádření k dané problematice napsané uměleckou formou, případně využitím publicistických prostředků
\end{itemize}

\subsubsection{Slohové postupy}
\paragraph{Popisný}
\begin{itemize}
\item odborný styl
\end{itemize}

\paragraph{Úvahový}
\begin{itemize}
\item Závěr odborného textu
\end{itemize}

\paragraph{Výkladový}
\begin{itemize}
\item odborný styl
\end{itemize}

\newpage
\section{Struktura literárního díla}
\subsection{Jazyková vrstva}
\subsubsection{Slovní zásoba}
\begin{itemize}
\item[] \textbf{historismus} -- zastaralé, dnes již nepotřebné slovo (\textit{palcát, škroně})
\item[] \textbf{archaismus} -- zastaralé, nahrazené slovo (\textit{biograf, lučba, dívaje se})
\item[] \textbf{neologismus} -- vymyšlené slovo (\textit{snivec})
\item[] \textbf{poetismus} -- básnicky znějící slovo (\textit{luna})
\end{itemize}

\subsubsection{Zvuková stránka jazyka}
\begin{itemize}
\item[] \textbf{eufonie} (zvukosled) -- navození pocitu uspořádáním hlásek (\textit{r,č – chrčí, l – melancholie})
\item[] \textbf{kakofonie} -- nepříjemná eufonie
\item[] \textbf{onomatopoie} (zvukomalba) -- vyjádření konkrétního zvuku uspořádáním písmen (\textit{šustit})
\end{itemize}

\subsubsection{Grafická stránka jazyka}
ilustrace, nevyužívání interpunkce, formování odstavců do objektů, různé velikosti písmen
\begin{itemize}
\item[] \textbf{apoziopeze} -- neukončená výpověď
\end{itemize}

\subsubsection{Figury}
\begin{itemize}
\item[] \textbf{přesah} -- myšlenka není ukončena v rámci jednoho verše/sloky
\item[] \textbf{aliterace} -- opakování stejných hlásek na začátku slov/veršů
\item[] \textbf{anafora} -- opakování stejných slov na začátku veršů
\item[] \textbf{epifora} -- opakování stejných slov na konci veršů
\item[] \textbf{epanastrofa} -- opakování slova z konce verše na začátku dalšího verše
\item[] \textbf{inverze} -- zaměnění slovosledu
\item[] \textbf{apostrofa} -- básnické oslovení
\item[] \textbf{epizeuxis} -- opakování slov v rámci jednoho verše
\item[] \textbf{řečnická otázka} -- otázka bez čekané odpovědi
\item[] \textbf{elipsa}(výpustka) -- vypuštění slova, čtenář si ho má domyslet
\end{itemize}

\subsubsection{Tropy}
\begin{itemize}
\item[] \textbf{přirovnání} -- srovnává dva jevy se stejnou vlastností (\textit{moudrý jako sova})
\item[] \textbf{metafora} -- opis jevu smyslově vnímatelnou podobností (\textit{Jindra je sova})
\item[] \textbf{personifikace} -- předávání vlastností živých bytostí na neživé věci
\item[] \textbf{eufemismus} -- zjemnění (\textit{zemřel $\rightarrow$ usnul na vždy})
\item[] \textbf{litotes} -- zmírnění (\textit{nemá mě nerada})
\item[] \textbf{perifráze} -- básnický popis (\textit{básník - kdo v zlaté struny zahrát zná})
\item[] \textbf{metonýmie} -- přenesené pojmenování (\textit{celá tabule = všichni hosté})
\item[] \textbf{synekdocha} -- nahrazení celku částí (\textit{srdce = člověk})
\item[] \textbf{paronomázie} -- opakování slov stejného kořenu
\item[] \textbf{pleonasmus} -- zbytečné vyjádření jednoho významu vícekrát (\textit{couval dozadu})
\item[] \textbf{epiteton} -- básnický přívlastek, musí být shodný (\textit{živoucí den})
\item[] \textbf{oximóron} -- protiklad (\textit{zdravý nemocný, strhané struny zvuk, mrtvé milenky cit})
\item[] \textbf{hyperbola} -- zveličení (\textit{stokrát jsem ti řekl \ldots})
\item[] \textbf{synestézie} -- spojení dvou nesouvisejících smyslů/činností (\textit{cítím tu hudbu, namaloval verš})
\item[] \textbf{alegorie} -- jinotaj, skrytý význam
\item[] \textbf{podobenství} -- rozšířené přirovnání, moudro, ponaučení
\item[] \textbf{symbol} -- zastoupení
\item[] \textbf{ironie} -- opak, záleží na intonaci, kontextu
\item[] \textbf{sarkasmus} -- vyostření ironie
\end{itemize}


\subsection{Tématická vrstva}
\begin{itemize}
\item[] \textbf{fabule} -- výčet dějových prvků
\item[] \textbf{syžet} -- způsob uspořádání dějových prvků do celku
\item[] \textbf{časoprostor} -- kdy a kde se děj odehrává
\item[] \textbf{postavy} -- nejdůležitější osoby díla, ne nutně lidé
\begin{itemize}
\item charakteristika -- popisuje vzhled, povahu, minulost, vývoj postavy
\item vypravěč -- může být jednou z postav, oddělený, nebo více postav
\end{itemize}
\item[] \textbf{titul} -- nadpis, může, či nemusí naznačovat obsah
\end{itemize}

\subsection{Kompoziční vrstva}
\subsubsection{Epická díla}
\begin{itemize}
\item rozdělení do kapitol, úvod, doslov, názvy / číslování
\item plynutí času -- rozdělujeme několik kompozic
\begin{itemize}
\item \textbf{chronologická} -- děje jsou uspořádány od nejstarších po nejnovější
\item \textbf{retrospektivní} -- zpětné dozvídání se děje
\item \textbf{paralelní} -- více dějových linek
\item \textbf{rámcová} -- příběh v příběhu (\textit{Dekameron})
\item \textbf{řetězová} -- spojení několika různých událostí pomocí něčeho (\textit{povídky, hlavní postava})
\item mohou se navzájem prolínat
\end{itemize}
\end{itemize}

\subsubsection{Lyrická díla}
\begin{itemize}
\item[] \textbf{kontrast} -- (\textit{já u pramene jsem a žízní hynu})
\item[] \textbf{paralelismus} -- obdobná výstavba veršů
\item[] \textbf{opakování}(refrén)
\item[] \textbf{gradace} -- stupňování (možné uplatnit i v epických textech)
\item[] \textbf{teze, antiteze} -- tvrzení, protitvrzení
\item[] \textbf{pointa} -- překvapivý zvrat
\end{itemize}

\section{Zařazení literárního díla}
\subsection{Literární druhy}
\paragraph{epika} 
\begin{itemize}
\item založeno na příběhu
\item často psáno prózou
\item může obsahovat lyrické pasáže 
\end{itemize}

\paragraph{lyrika}
\begin{itemize}
\item není založena na příběhu, zachycuje náladu, okamžiky, emoce, dojmy, myšlenky, úvahy, \ldots
\item může obsahovat dějové úryvky, situace
\end{itemize}

\paragraph{drama}
\begin{itemize}
\item vytvořeno primárně pro divadlo
\item příběh, většinou založený na konfliktu
\item slabá funkce vyprávěče, většinou je kompletně nahrazen scénickými poznámkami
\end{itemize}


\subsection{Literární žánry}
\subsubsection{Epické}
\begin{itemize}
\item[] \textbf{bajka} -- krátký, alegorický příběh, ponaučení, často se objevují personifikovaná zvířata
\item[] \textbf{báje} -- vysvětlení jevů, domyšlená
\item[] \textbf{legenda} -- vyprávění o životě důležitých osob, hrdinové, dříve především svatí
\item[] \textbf{pověst} -- založeno na pravdě, domyšleno
\item[] \textbf{pohádka} -- pro děti, dobro proti zlu, z lidové slovesnosti
\item[] \textbf{povídka} -- kratší vypravování, málo postav, jednoduchý rychlý děj
\item[] \textbf{novela} -- jedna časová linka, pointa
\item[] \textbf{epos} -- román ve verších, rozsáhlé, mnoho dějových linek a postav
\item[] \textbf{romaneto} -- nepřirozené, mystické jevy, na konci logicky vysvětleno
\item[] \textbf{román} -- prozaická podoba eposu
\end{itemize}

\subsubsection{Lyrickoepické}
\begin{itemize}
\item[] \textbf{balada} -- tragická báseň, špatný konec
\item[] \textbf{romance} -- protipól balady, vesele, milostně laděná
\item[] \textbf{básnická povídka} -- rozsáhlejší, děj nedořečen, jen naznačen, otevřen čtenáři
\end{itemize}

\subsubsection{Lyrické}
\begin{itemize}
\item[] \textbf{píseň} -- opakování, silný rytmus, 
\item[] \textbf{elegie}(žalozpěv) -- tesklivá píseň, hlavní hrdina je lyrický subjekt
\item[] \textbf{óda} -- oslavuje obecné hodnoty člověka
\item[] \textbf{hymnus} -- oslavuje významné lidi, bohy
\item[] \textbf{epigram} -- krátká, satirická báseň, kritická ke společnosti, pointa v závěru
\item[] \textbf{pásmo} -- nepříliš přehledný proud mnoha témat a pohledů 
\item[] \textbf{sonet} -- 4 sloky po 4,4,3,3 verších, první dvě a druhé dvě sloky mají stejné rýmové téma
\begin{itemize}
\item 1. sloka teze, 2. sloka antiteze, 3. sloka syntéza, 4. sloka pointa
\end{itemize}
\end{itemize}

\subsubsection{Dramatické}
\begin{itemize}
\item[] \textbf{tragedie} -- vážné, chmurný děj, končí mnoho smrtmi, postavy řeší komplikované problémy, vyobrazuje vyšší společnost
\item[] \textbf{komedie} -- veselé a směšné situace, satirické, o nižší společnosti
\item[] \textbf{tragikomedie} -- směs tragedie a komedie
\item[] \textbf{činohra} -- řeší situace z běžného života
\item[] \textbf{muzikál} -- zpěv, hudba a dialog
\item[] \textbf{balet} -- pouze hudba a tanec
\item[] \textbf{opera} -- pouze zpěv, bez mikrofonu, méně náročná choreografie
\item[] \textbf{opereta} -- hudba a mluvené slovo, bez mikrofonu
\end{itemize}


\section{Výrazové formy}
\begin{itemize}
\item[] \textbf{poezie} -- verše
\item[] \textbf{próza} -- souvislý text
\end{itemize}

\subsection{Próza}
\paragraph{Vyprávěcí způsoby}
\begin{itemize}
\item[] \textbf{er forma} -- vyprávění z pohledu třetí osoby
\item[] \textbf{ich forma} -- vyprávění z pohledu 1. osoby
\end{itemize}

\paragraph{Druhy řeči}
\begin{itemize}
\item[] \textbf{monolog} -- projev jedné osoby
\item[] \textbf{dialog} -- rozhovor dvou a více lidí
\end{itemize}

\paragraph{Formy řeči}
\begin{itemize}
\item[] \textbf{přímá řeč} -- přesný zápis toho co někdo pronáší (\textit{maminka mi řekla "zajdi pro mléko"})
\item[] \textbf{nepřímá řeč} -- opis toho co bylo proneseno (\textit{maminka mi řekla, že mám zajít pro mléko})
\item[] \textbf{nevlastní přímá řeč}(neznačená) -- zachycení myšlenek postavy v první osobě
\item[] \textbf{polopřímá řeč} -- zachycení myšlenek postavy ve třetí osobě
\item[] \textbf{smíšená řeč} -- kombinace
\end{itemize}


\subsection{Poezie}
\begin{itemize}
\item[] \textbf{verš} -- jeden řádek
\item[] \textbf{vázaný verš} -- dodržuje pravidelný rytmus, střídání přízvučných a nepřízvučných dob, pravidelný počet slabik
\item[] \textbf{verš volný} -- nedodržuje výše uvedené, může se ale rýmovat
\item[] \textbf{sloka} -- skupina veršů
\item[] \textbf{stopa} -- vyjadřuje jakým způsobem se střídají přízvučné a nepřízvučné doby
\begin{itemize}
\item \textbf{trochej} -- přízvučná, nepřízvučná (\textit{doma})
\item \textbf{daktyl} -- přízvučná, nepřízvučná, nepřízvučná (\textit{návštěva})
\item \textbf{jamb} -- nepřízvučná, přízvučná (\textit{byl pozdní večer})
\end{itemize}
\item[] \textbf{rým} -- zvuková shoda na konci veršů
\begin{itemize}
\item \textbf{sdružený} -- AABB
\item \textbf{střídavý} -- ABAB
\item \textbf{obkročný} -- ABBA
\item \textbf{přerývaný} -- ABCA, ABAC
\item \textbf{postupný} -- ABCABC

\end{itemize}
\end{itemize}

\newpage
\section{Analýza uměleckého textu}

\section{Analýza rozbor neuměleckého textu}



\end{document}