\documentclass[10pt,a4paper]{article}
\usepackage[utf8]{inputenc}
\usepackage[czech]{babel}
\usepackage{amsmath}
\usepackage{amsfonts}
\usepackage{amssymb}
\usepackage{chemfig}
\usepackage{geometry}
\usepackage{wrapfig}
\usepackage{graphicx}
\usepackage{floatflt}
\usepackage{hyperref}
\usepackage{fancyhdr}
\usepackage{tabularx}
\usepackage{makecell}
\usepackage{csquotes}
\usepackage{marginnote}

\MakeOuterQuote{"}

\renewcommand{\labelitemii}{$\circ$}
\renewcommand{\labelitemiii}{--}
\newcommand{\ra}{$\rightarrow$ }
\newcommand{\x}{$\times$ }
\newcommand{\lp}[2]{#1 -- #2}
\newcommand{\timeline}{\input{timeline}}


\geometry{lmargin = 0.8in, rmargin = 0.8in, tmargin = 0.8in, bmargin = 0.8in}
\newcommand{\note}[1]{\marginnote{\hspace{-0.6\textwidth}#1}}

\date{}
\author{Jakub Rádl}
\title{Romain Rolland: Petr a Lucie -- Rozbor díla}

\begin{document}
\maketitle

\section*{Výňatek}
„Proč se máme tak velice rádi?“ zeptal se Petr.\\
„Ach Petře, jistě mě nemáš rád velice, když se ptáš proč.“\\
„Ptám se proto, abych od tebe slyšel to, co vím stejně dobře jako ty.“\\
„Ty chceš, abych tě chválila, abych ti řekla, proč tě mám ráda. Ale toho se nedočkáš, neboť víš-li ty, proč tě mám ráda, já to nevím.“\\
„Ty to nevíš?“ zaraženě se zeptal Petr.\\
„Nevím! (V duchu se zasmála.) Naprosto to nepotřebuji vědět. Ptáme-li se, proč je to nebo ono, znamená to, že si tím nejsme jisti a že je to špatné. Teď, když miluji, neznám již žádné proč! Již žádné \textbf{kde, neboť, nebo, jak}!\footnote{výčet} Má láska prostě je, opravdu je. Na ostatním mi nezáleží.“\\
Jejich tváře se políbily. \textbf{Déšť toho využil}\footnote{personifikace}, vklouzl pod \textbf{nemotorný deštník}\footnote{personifikace, epiteton} a poškádlil je svými prsty po vlasech a po lících; jejich semknuté \textbf{rty vsály}\footnote{synekdocha} studenou kapičku.\\
Petr řekl: „Ale co ti ostatní?“\\
„Kteří ostatní?“\\
„Ti ubožáci,“ odpověděl Petr. „Všichni \textbf{ti, kteří nejsou námi}\footnote{epiteton}?“\\
„Ať dělají totéž co my. Ať milují!“\\
„A nejsou milováni! To, Lucie, nemůže každý.“\\
„I může!“\\
„Nemůže Lucie. Neznáš cenu daru, který mi dáváš.“\\
„Dávat své srdce lásce, své rty milenci, \textbf{to je přece totéž jako dávat oči světlu}\footnote{přirovnání}; to není dávání, to je braní.“\\
„Ale mnozí \textbf{lidé jsou slepci}\footnote{metafora}.“\\
„My je nevyléčíme, Petříčku. Dívejme se místo nich!“\\

\newpage
\section*{Tématická stránka díla}
\begin{itemize}
\item \textbf{literární druh a žánr}: epický s velkým množstvím lyrických pasáží, milostná novela
	\begin{itemize}
	\item velká část díla je lyrická -- zachytává myšlenky a pocity postav
	\item pointa -- na konci jsou problémy "vyřešeny" tím, že oba překvapivě zahynou
	\end{itemize}
\item \textbf{téma a motiv}:
	\begin{itemize}
	\item \textbf{hlavní téma}: posedlost dvou mladých lidí láskou přehlušující strach strach z války, rozdíl mezi společenskými vrstvami Petra a Lucie, odsouzení války
	\item \textbf{další motivy v díle}:
		\begin{itemize}
		\item láska, důvěra, scházení se Petra a Lucie, Luciiny obrazy (úpadek umění za války)
		\item ulice Paříže, kašna u které sedávali, Luciin byt, Petrův pokoj
		\item válka, vojna, nálety na Paříž, utrpení lidí
		\end{itemize}
	\end{itemize}
\item \textbf{časoprostor}:
	\begin{itemize}
	\item "děj trvá od středy večer 30. ledna do velkého pátku 29. března 1918"
	\item děj se odehrává v ulicích Paříže, nebo u Lucie doma
	\end{itemize}
\item \textbf{zasazení výňatku do kontextu díla}:
	\begin{itemize}
	\item \textbf{časoprostor}:
		\begin{itemize}
		\item úryvek se odehrává v předposlední den díla, při jednom z každodenních setkání Petra s Lucií
		\end{itemize}
	\item \textbf{obsah}: 
		\begin{itemize}
		\item dialog mezi Petrem a Lucií o tom, proč se mají tak rádi
		\item neví proč se navzájem milují, důležité pro ně je, že se milují
		\item je zde patrný kontrast mezi Petrovým dobrosrdečným naivním pohledem na svět, který lituje a všechny ostatní, kteří nemohou prožít to, co oni a Luciiným realistickým pohledem
		\item opravdová láska je pocit a nepotřebuje se ptát proč
		\end{itemize}
	\end{itemize}
\end{itemize}
\section*{Kompozice, postavy}
\begin{itemize}
\item \textbf{kompoziční výstavba}
	\begin{itemize}
	\item chronologická kompozice, retrospektivní pohled na dětství a vztah s bratrem
	\item dílo je děleno na kapitoly, které nejsou nijak pojmenovány či číslovány, jen naznačeny předěly v textu
	\end{itemize}
\item vyprávěno v er-formě, vypravěč je nezúčastněný, popisuje děj i myšlenky postav
\item vyprávěcí způsoby:
	\begin{itemize}
	\item přímá řeč v dialozích Petra a Lucie
	\item nepřímá řeč popisuje jejich myšlenky
	\end{itemize}
\end{itemize}

\paragraph{Postavy}
\begin{itemize}
\item \textbf{Petr}
	\begin{itemize}
	\item osmnáctiletý hoch z bohaté rodiny, intelektuál
	\item za několik měsíců má být odveden na vojnu 
	\item vzhlíží ke svému bratrovi, ten se ale změnil Válkou a vztah mezi nimi zamrzl, v pozdější části díla se zase sbližují
	\item zamiluje se do Lucie při prvním velice krátkém setkání
	\item upřímný, milý, naivní, má idealizované představy, odpor k válce
	\item podporují válku, aby synové šli na vojnu, vlastenecké myšlení
	\end{itemize}
\item \textbf{Lucie}
	\begin{itemize}
	\item mladá dívka z chudé rodiny
	\item žije jen s matkou, se kterou v poslední době nemá příliš dobré vztahy
	\item živí se malováním a prodáváním kýčovitých obrazů, není příliš dobrá malířka
	\item má realistický, pragmatický pohled na svět
	\item zamiluje se do Petra
	\end{itemize}
\item Oba v lásce nachází únik před vlastními problémy a před utrpením a strachem z války. Jsou tedy zamilováni spíše do ideje lásky, jakožto nejčistší věci na světě v kontrastu s válkou, než do sebe navzájem. 
\item \textbf{Filip} 
	\begin{itemize}
	\item Petrův o dva roky starší bratr
	\item dřív si s ním býval velmi blízký, sdílel jeho pohled na svět
	\item boj ve válce ho připravil o ideály, ukázal mu reálný svět
	\item snaží se iluze vyvrátit i svému mladšímu bratrovi
	\item později si uvědomuje, že přišel o bratrovu důvěru a snaží se ji získat zpět
	\end{itemize}
\item Petrovi kamarádi
	\begin{itemize}
	\item poskytují náhled do pohledu mladých lidí na svět, zastupují různé názory
	\item Petr s nimi dříve rád polemizoval, nyní je ponořen do myšlenek na Lucii
	\end{itemize}
\end{itemize}
\section*{Jazyk}
\begin{itemize}
\item překlad Jaroslava Zaorálka
\item překlad je psán srozumitelnou, spisovnou češtinou, vyskytují se v něm občas příslušně zastaralé výrazy
\item mnoho uměleckých prostředků, především mnoho metafor, epitet, gradace řečnických otázek a zvolání vyjadřujících silu zamilovaných pocitů
\end{itemize}
\section*{Literárně historický kontext}
\begin{itemize}
\item Romain Rolland byl francouzský prozaik a dramatik
\item během války se odstěhoval do neutrálního Švýcarska
\item v roce 1915 získal Nobelovu cenu za literaturu
\item dílo je ovlivněno první světovou válkou \ra promítají se do něj myšlenky pacifismu
\item Petr a Lucie jsou příkladem charakterově vyhraněných postav s čistým srdcem v Rollandových dílech
\item mezi další Francouzské autory této doby patří
	\begin{itemize}
	\item Guillaume Apollinaire (Pásmo)
	\item Jacques Prévert (Slova)
	\item Henri Barbusse (Oheň, Peklo, Jasno)
	\end{itemize}
\item ve světové literatuře
	\begin{itemize}
	\item Erich Maria Remarque (Na západní frontě klid)
	\item Erza Pound (Zpěvy)
	\item Ernest Hemingway (Komu zvoní hrana, Stařec a moře, Sbohem armádo, \ldots)
	\end{itemize}

\end{itemize}
\section*{Zdroje}
\begin{itemize}
\item ROLLAND, Romain. Petr a Lucie [online]. V MKP 1. vydání. Praha : Městská knihovna v Praze, 2018 [cit. 2020-04-05]. Klasická světová próza. ISBN 978-80-7587-975-2. 

Dostupné z: http://search.mlp.cz/searchMKP.jsp?action=sTitul\&key=4409617
\item zápisy z hodin
\item učebnice
\end{itemize}
\end{document}