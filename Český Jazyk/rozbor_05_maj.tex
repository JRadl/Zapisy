\documentclass[10pt,a4paper]{article}
\usepackage[utf8]{inputenc}
\usepackage[czech]{babel}
\usepackage{amsmath}
\usepackage{amsfonts}
\usepackage{amssymb}
\usepackage{chemfig}
\usepackage{geometry}
\usepackage{wrapfig}
\usepackage{graphicx}
\usepackage{floatflt}
\usepackage{hyperref}
\usepackage{fancyhdr}
\usepackage{tabularx}
\usepackage{makecell}
\usepackage{csquotes}
\usepackage{marginnote}

\MakeOuterQuote{"}

\renewcommand{\labelitemii}{$\circ$}
\renewcommand{\labelitemiii}{--}
\newcommand{\ra}{$\rightarrow$ }
\newcommand{\x}{$\times$ }
\newcommand{\lp}[2]{#1 -- #2}
\newcommand{\timeline}{\input{timeline}}


\geometry{lmargin = 0.8in, rmargin = 0.8in, tmargin = 0.8in, bmargin = 0.8in}
\date{}
\author{Jakub Rádl}

\makeatletter
\let\thetitle\@title
\let\theauthor\@author
\makeatother

\newcommand{\note}[1]{\marginnote{\hspace{-0.6\textwidth}#1}}

\title{Karel Hynek Mácha: Máj -- Rozbor díla}

\begin{document}
\maketitle
\section*{Výňatek}
Klesla hvězda s nebes výše, \note{personifikace hvězdy} \\
mrtvá hvězda siný svit; 	\note{epiteton, oxymóron, přesah hvězdy z prvního verše} \\
padá v neskonečné říše 		\\
padá věčně v věčný byt.		\note{anafora, aliterace, paronomázie} \\
Její pláč zní z hrobu všeho,\\
strašný jekot, hrůzný kvíl.	\note{gradace, onomatopoie} \\
"Kdy dopadne konce svého?" 	\note{řečnická otázka} \\
Nikdy - nikde - žádný cíl.	\note{paronomázie} \\
Kol bílé věže větry hrají,	\note{aliterace} \\
při níž si vlnky šepotají.	\note{personifikace, onomatopoie}\\
Na bílé zdě stříbrnou zář	\note{zachycení atmosféry} \\
rozlila bledá lůny tvář;	\note{poetismus, eufonie} \\
však hluboko u věži je temno pouhé; \\
\\
(\ldots)\\
\\
Po měsíce tváři jak mračna jdou,\\
zahalil vězeň v ně duši svou; \note{metafora} \\
myšlenka myšlenkou umírá.\note{epizeuxis, eufonie(m)} \\
\\
(\ldots)\\
\\
"Hluboká noc - temná je noc! - \note{gradace, epizeuxis, eufonie(m)} \\
Temnější mně nastává - - -\\
Pryč, myšlenko!!" - A citu moc \note{apostrofa, expresivní vyjádření} \\
myšlenku překonává.\\
\\
Hluboké ticho. - Z mokrých stěn \note{epiteton, elipsa} \\
kapka za kapkou splyne, \note{epizeuxis, onomatopoie} \\
a jejich pádu dutý hlas \note{personifikace} \\
dalekou kobkou rozložen,  \\
jako by noční měřil čas, \\
zní - hyne - zní a hyne -\\
zní - hyne - zní a hyne zas. \note{anafora, epifora, epizeuxis, paralelismus} \\
\\
"Jak dlouhá noc - jak dlouhá noc - \note{gradace, opakující se motiv noci} \\
však delší mně nastává. - - -\\
Pryč, myšlenko!" - A hrůzy moc \note{apostrofa, personifikace, metafora} \\
myšlenku překonává. -\\
Hluboké ticho. - Kapky hlas \note{epiteton, elipsa, onomatopoie, personifikace, symbol pro měření času} \\ 
svým pádem opět měří čas. \note{eufonie(k,s)} \\

\section*{Tématická složka díla}
\begin{itemize}
\item \textbf{literární druh a žánr}: lyricko-epický, básnická povídka
	\begin{itemize}
	\item veršované, lyrika hraje velkou roli, lze vysledovat příběh
	\end{itemize}
\item \textbf{téma a motiv}:
	\begin{itemize}
	\item hlavní téma: osud a smrt Viléma, vůdce zbojníků, jeho láska k Jarmile
	\item další motivy v díle: májová příroda, kontrast temného příběhu a zamilované májové přírody, tragická láska, smrt, vina,  vražda otce, sebevražda Jarmily, loupežníci, máj, 
	\item kapání vody v druhém zpěvu symbolizuje čas zbývající do popravy
	\item lidové pověsti se promítají do sboru duchů 
	\item vlastenectví se promítá do lásky k přírodě
	\end{itemize}
\item \textbf{časoprostor}: údolí pod Bezdězem u Máchova jezera, 1., 2. květen, konec 18. století, návrat po sedmi letech
\item \textbf{zasazení výňatku do kontextu díla}: 
	\begin{itemize}
	\item časoprostor: druhý zpěv
	\item obsah: Vilém je ve vězení, snaží se nemyslet na smrt, uvědomuje si, že se blíží
	\end{itemize}
\item \textbf{kompoziční výstavba}:
	\begin{itemize}
	\item chronologická kompozice
	\item dělení: předzpěv (v původním vydání), zpěv, intermezzo, zpěv, zpěv, intermezzo, zpěv
		\begin{itemize}
		\item předzpěv -- nehodí se ke zbytku díla, původně ale byl považován za nejlepší část máje (po Nerudu)
		\item 1. zpěv -- popis prostředí, Jarmila se dozvídá, že Vilém byl odsouzen, protože ho podvedla s jeho otcem a on ho zabil.
		\item 1. intermezzo -- sbor duchů připravuje pohřeb
		\item 2. zpěv -- Vilém je ve věznici a přemýšlí nad smrtí, vinu dává svému otci (ne sobě či Jarmile), nejen za Jarmilu, ale také za to, že ho donutil stát se loupežníkem
		\item 3. zpěv -- poprava Viléma, nejvýraznější projev kontrastu mezi májovou přírodou a pochmurným dějem
		\item 4. zpěv -- Hynek se vrací k Vilémově popravišti, ztotožňuje se s jeho myšlenkami, naznačeno \textit{"Hynku! Viléme!! Jarmilo!!!"}
		\end{itemize}
	\end{itemize}
\end{itemize}
\section*{Kompozice, postavy}
\begin{itemize}
\item \textbf{vypravěč / lyrický subjekt}: 
	\item o vypravěči nejsou známy žádné informace
\item \textbf{vyprávěcí způsoby}:
	\begin{itemize}
	\item 1.--3. zpěv vyprávěn v er formě
	\item 4. zpěv je psán v ich formě
	\end{itemize}
\item \textbf{veršová výstavba}:	
	\begin{itemize}
	\item první zpěv -- verš vázaný, jamb, 8 slabik
	\item druhý zpěv -- vázaný verš
	\end{itemize}
\end{itemize}

\paragraph{Postavy}
\begin{itemize}
\item \textbf{Vilém}
	\begin{itemize}
	\item zamilovaný do Jarmily, žárlivý
	\item typický romantický hrdina
		\begin{itemize}
		\item na okraji společnosti
		\item nešťastná láska, smutek, 
		\item rozpor mezi temnými stránkami osobnosti (zabil otce, je loupežník) a láskou k Jarmile
		\end{itemize}
	\item nechce myslet na smrt, nevěří v posmrtný život
	\end{itemize}
\item \textbf{Jarmila}
	\begin{itemize}
	\item Vilémova milá, svedena jeho otcem
	\item neunesla vinu a spáchala sebevraždu
	\item není v díle příliš charakterizována, nejsou zmíněny okolnosti nevěry
	\end{itemize}
\item \textbf{Hynek}
	\begin{itemize}
	\item objevuje se jen na krátkou dobu na konci díla
	\item soucítí s Vilémem
	\end{itemize}
\item Vilémův otec -- záporná postava, nepřevzal odpovědnost
\item strážný -- soucítí s vězněm
\item sbor duchů -- těší se na Viléma, protože jako nový duch bude muset strážit
\end{itemize}
\section*{Jazyk}
V díle se vyskytuje mnoho \textbf{kontrastu} mezi májovou přírodou a pochmurným dějem.
\begin{itemize}
\item náznak pochmurného děje hned ze začátku \textit{"Jezero hladké v křovích stinných zvučelo temně tajný bol"}
\end{itemize}
Dílo je psáno archaickou češtinou. Často se vyskytují proudy oxymóronů. 
Zpěvný jazyk 
\begin{itemize}
\item eufonie
\item onomatopoie \textit{"řinčí řetězů hřmot"}
\end{itemize}
Mnoho inverze za cílem dodržení rytmu
\paragraph{Výňatek}
\begin{itemize}
\item jazykové prostředky a jejich funkce ve výňatku
	\begin{itemize}
	\item mnoho eufonie, onomatopoie a odmlk navozuje pocit beznaděje, ubíhajícího času
	\end{itemize}
\end{itemize}
\section*{Literárněhistorický kontext}
\paragraph{Mácha}
\begin{itemize}
\item studoval filosofii a práva v Praze, aktivista revolucí
\item vztah s Marinkou Štichovou, Eleonorou Šomkovou
\item zemřel na nemoc z vody při hašení požáru
\item typický zástupce světového romantismu
\item psal jako koníček, inspirován zahraničními autory
\item psal poezii a lyrizovanou prózu, snaha o dramata, kreslil\\
\item významná díla
	\begin{itemize}
	\item povídky \textbf{Obraz ze života mého}, \textbf{Pouť krkonošská}, román \textbf{Cikáni}, \textbf{Kat--Křivoklad}
	\end{itemize}
\end{itemize}

\paragraph{Kontext díla}
\begin{itemize}
\item spadá do 3. fáze národního obrození
\item dílo nebylo ve své době pochopeno, jelikož nereflektovalo ideály národního obrození
\item na začátku byla předmluva, která byla přehnaně vlastenecká a ve své době jedinou doceněná část díla
\end{itemize}

\paragraph{Současní autoři}
\begin{itemize}
\item Josef Kajetán Tyl
\item Karel Jaromír Erben
\end{itemize}
\section*{Zdroje}
\begin{enumerate}
\item MÁCHA, Karel Hynek. Máj. Praha : Dr. František Bačkovský, 1905. 48 s.
\item vlastní poznámky z hodin 
\end{enumerate}
\end{document}