\title{Česká próza a drama v první polovině 20. století}
\documentclass[10pt,a4paper]{article}
\usepackage[utf8]{inputenc}
\usepackage[czech]{babel}
\usepackage{amsmath}
\usepackage{amsfonts}
\usepackage{amssymb}
\usepackage{chemfig}
\usepackage{geometry}
\usepackage{wrapfig}
\usepackage{graphicx}
\usepackage{floatflt}
\usepackage{hyperref}
\usepackage{fancyhdr}
\usepackage{tabularx}
\usepackage{makecell}
\usepackage{csquotes}
\usepackage{footnote}
\usepackage{movie15}
\MakeOuterQuote{"}

\renewcommand{\labelitemii}{$\circ$}
\renewcommand{\labelitemiii}{--}
\newcommand{\ra}{$\rightarrow$ }
\newcommand{\x}{$\times$ }
\newcommand{\lp}[2]{#1 -- #2}
\newcommand{\timeline}{\input{timeline}}


\geometry{lmargin = 0.8in, rmargin = 0.8in, tmargin = 0.8in, bmargin = 0.8in}
\date{\today}
\author{Jakub Rádl}

\makeatletter
\let\thetitle\@title
\let\theauthor\@author
\makeatother

\hypersetup{
colorlinks=true,
linkcolor=black,
urlcolor=cyan,
}



\begin{document}
\maketitle
\tableofcontents
\begin{figure}[b]
Toto dílo \textit{\thetitle} podléhá licenci Creative Commons \href{https://creativecommons.org/licenses/by-nc/4.0/}{CC BY-NC 4.0}.\\ (creativecommons.org/licenses/by-nc/4.0/)
\end{figure}
\newpage

\section{Úvod}
\begin{itemize}
\item Československo po 1. sv. v. -- demokratické, rychle se rozvíjející
\item ve 30. letech hospodářská krize, národnostní spory, Mnichovská dohoda, protektorát
\item literatura -- impresionismus, realismus, expresionismus, kubismus, poetismus
\item architektura -- funkcionalismus
\item hudba -- Janáček, Dvořák (zhudebnil Čapkovu Věc Makropulos)
\item Jirásek a Čapek nominování na Nobelovu cenu za literaturu
\item hnutí Proletkult propagace komunismu a marxismu-leninismu \ra Čapek: Proč nejsem komunistou
\item Pátečníci -- skupina, která se scházela v pátek v Čapkově vile, novináři, demokraté
\item socialistický realismus -- odpovídá třídnímu pojetí společnosti a materialismu
\item levá fronta -- považovali za důležité posílit národní uvědomění a bojovat proti nacionalismu, fašismu a válce
\item drama -- expresionismus, růst počtu kabaretů a malých scén, avantgardní divadlo, odpor proti fašismu
	\begin{itemize}
	\item avantgardní scény -- Osvobozené divadlo (Voskovec, Werich), Divadlo D (E. F. Burian)
	\end{itemize}
\end{itemize}

\section{Reakce na 1. světovou válku}
\subsection{Jaroslav Hašek (1883--1923)}
\begin{itemize}
\item spisovatel, humorista, publicista
\item z gymnázia vyloučen za účast na demonstracei, vyučil se drogistou, vystudoval obchodní akademii, pracoval v bance, později spisovatel, publicista
\item bohémský způsob života, pokus o sebevraždu
\item politická Strana mírného pokroku v mezích zákona, kritika politiky
\item zakladatel žánru hospodská historka, cetopisné povídky, črty a humoresky, většina děl napsané v hospodě a publikována v časopisech
\item satirický román \textbf{Osudy dobrého vojáka švejka za světové války}
	\begin{itemize}
	\item satiricky popisuje 1. světovou válku
	\item hovorová čeština, vulgarismy, byrokratická a vojenská hantýrka, cizí výrazy, 
	\item nejasná dějová linie
	\item postava Švejka se vyvíjí z prosťáčka na mazaného šibala, který doslovným plněním rozkazů přivádí nadřízené k šílenosti
	\item Švejk nad válkou vítězí a ukazuje ji v celé její nesmyslnosti
	\item ironizace armády, církve, Rakouska-Uherska
	\end{itemize}
\item povídky \textbf{Velitelem města Bugulmy}
	\begin{itemize}
	\item popisuje fanatismus revoluční doby v Rusku a hloupost nových vládců
	\end{itemize}
\item \textbf{Dekameron humoru a satiry}
	\begin{itemize}
	\item povídka \textbf{Nemravné kalendáře} -- strážníci rozeberou nemravné kalendáře a tím ochrání veřejnost

	\item povídka \textbf{O básních} -- : kritika stereotypu lyrického subjektu, který neumí ve světě nalézt radost
	\end{itemize}
\end{itemize}

\section{Legionářská literatura}
\subsection{Rudolf Medek (1890--1940)}
\begin{itemize}
\item básník, prozaik, voják, učitel, 1913 narukoval, s Legiemi cestoval na Sibiř, po návratu vstoupil do československé armády, vyznamenán ve Francii a Anglii
\item za protektorátu a komunismu cenzurován
\item sbírka \textbf{Lví srdce} -- oslavuje legie
\item román \textbf{Veliké dny} -- o vzniku legií, bitva u Sborova
\item román \textbf{Anabase} -- poslední měsíc pobytu legií na východě
\item drama \textbf{Plukovník Švec} -- o veliteli, který se zastřelí, aby zastavil rozpad morálky svých vojáků
\end{itemize}

\subsection{Josef Kopta (1894--1962)}
\begin{itemize}
\item trilogie románů \textbf{Třetí rota, Třetí rota na magistrále, Třetí rota doma}
	\begin{itemize}
	\item líčí postup malé vojenské jednotky od Zborova přes Sibiř k Vladivostoku
	\end{itemize}
\end{itemize}

\subsection{František Langer (1888--1965)}
\begin{itemize}
\item spisovatel, dramatik, vojenský lékař, publikoval v časopisech, východní fronta, legie, po návratu pokračoval v práci vojenského lékaře
\item dramaturg Vinohradského divadla
\item za druhé světové války musel odejít kvůli židovskému původu
\item drama \textbf{Jízdní hlídka} -- o hrdinství legionářů
\item drama \textbf{Periférie} -- vykresluje pražskou periferii, motiv zločinu a trestu, číšník zabije milence své přítelkyně, ale nikdo mu to nevěří
\item komedie \textbf{Velbloud uchem jehly}, \textbf{Obrácení Ferdyše Pištory}
\item soubor povídek \textbf{Železný vlk}
\end{itemize}

\section{Katolicky orientovaná próza}

\subsection{Jaroslav Durych (1886--1962)}
\begin{itemize}
\item básník, prozaik, dramatik, novinářská rodina, brzy sirotek, přednosta vojenské nemocnice 
\item v tvorbě je významné baroko
\item \textbf{Jarmark života} (dvě prózy, dvě básně)
\item román \textbf{Na horách}
\item později historická forma spjatá s obdobím protireformace
\item historický román \textbf{Bloudění} -- období popravy českých pánů, hlavní hrdina ve službách Albrechta z Valdštejna se zamiluje do katoličky Španělky Andělky, před smrtí s ní má dítě, které se stává symbolem naděje
\end{itemize}

\subsection{Jan Čep (1902--1974)}
\begin{itemize}
\item esejista, prozaik, překladatel, 1934 šel propagovat československou kulturu do Paříže, 1948 tam emigroval
\item kratší epické útvary s jednoduchým dějem, hrdinové jsou spjati s rodným krajem
\item soubor existenciálních povídek \textbf{Zeměžluč} -- křesťansky orientované, hrdinové jsou tragicky lhostejní k životu
\item soubor povídek \textbf{Letnice} -- hrdinové překonávají životní tragedie návratem k víře
\item román \textbf{Hranice stínu} -- cesta moderního člověka k nalezení vlastní identity 
\item soubor povídek \textbf{Modrá a zlatá} -- motiv návratu do rodného kraje, vrcholné dílo
\end{itemize}

\subsection{Jakub Deml (1878--1961)} 
\begin{itemize}
\item kněz, básník, prozaik, ve dvanácti letech zemřela matka, později sourozenci, gymnázium, vysvěcen, ale knězem nebyl moc dlouho
\item předchůdce surrealismu, expresionismu, existencialismu, vzorem Otokar Březina
\item básnická sbírka \textbf{Notantur lumina} -- ovlivněno Otokarem Březinou
\item báseň v próze \textbf{Hrad smrti} -- popisuje úzkost a ohrožení smrtí
\item básnická sbírka \textbf{Moji přátelé} -- personifikace stromů, květin, hub jako jeho blízkých
\item soubor \textbf{Šlépěje} -- 26 svazků, deníky, úvahy, recenze, korespondence
\item próza \textbf{Zapomenuté světlo} -- soubor vzájemně propojených esejů sjednocený postavou básníka Bohumila Maliny Ptáčka, na jehož dopis autor odpovídá
\end{itemize}

\section{Ruralismus}
\begin{itemize}
\item  vesnická tematika, vztah lidí k půdě, vývoj vztahu venkova a města
\item nemilosrdný osud venkovského obyvatelstva v důsledku industrializace
\item Josef Knap, František Křelina
\end{itemize}



\end{document}