\title{Česká próza a drama v první polovině 20. století}
\documentclass[10pt,a4paper]{article}
\usepackage[utf8]{inputenc}
\usepackage[czech]{babel}
\usepackage{amsmath}
\usepackage{amsfonts}
\usepackage{amssymb}
\usepackage{chemfig}
\usepackage{geometry}
\usepackage{wrapfig}
\usepackage{graphicx}
\usepackage{floatflt}
\usepackage{hyperref}
\usepackage{fancyhdr}
\usepackage{tabularx}
\usepackage{makecell}
\usepackage{csquotes}
\usepackage{footnote}

\MakeOuterQuote{"}

\renewcommand{\labelitemii}{$\circ$}
\renewcommand{\labelitemiii}{--}
\newcommand{\ra}{$\rightarrow$ }
\newcommand{\x}{$\times$ }
\newcommand{\lp}[2]{#1 -- #2}
\newcommand{\timeline}{\input{timeline}}


\geometry{lmargin = 0.8in, rmargin = 0.8in, tmargin = 0.8in, bmargin = 0.8in}
\date{\today}
\author{Jakub Rádl}

\makeatletter
\let\thetitle\@title
\let\theauthor\@author
\makeatother

\hypersetup{
colorlinks=true,
linkcolor=black,
urlcolor=cyan,
}



\begin{document}
\maketitle
\tableofcontents
\begin{figure}[b]
Toto dílo \textit{\thetitle} podléhá licenci Creative Commons \href{https://creativecommons.org/licenses/by-nc/4.0/}{CC BY-NC 4.0}.\\ (creativecommons.org/licenses/by-nc/4.0/)
\end{figure}
\newpage

\section{Úvod}
\begin{itemize}
\item Československo po 1. sv. v. -- demokratické, rychle se rozvíjející
\item ve 30. letech hospodářská krize, národnostní spory, Mnichovská dohoda, protektorát
\item literatura -- impresionismus, realismus, expresionismus, kubismus, poetismus
\item architektura -- funkcionalismus
\item hudba -- Janáček, Dvořák (zhudebnil Čapkovu Věc Makropulos)
\item Jirásek a Čapek nominování na Nobelovu cenu za literaturu
\item hnutí Proletkult propagace komunismu a marxismu-leninismu \ra Čapek: Proč nejsem komunistou
\item Pátečníci -- skupina, která se scházela v pátek v Čapkově vile, novináři, demokraté
\item socialistický realismus -- odpovídá třídnímu pojetí společnosti a materialismu
\item levá fronta -- považovali za důležité posílit národní uvědomění a bojovat proti nacionalismu, fašismu a válce
\item drama -- expresionismus, růst počtu kabaretů a malých scén, avantgardní divadlo, odpor proti fašismu
	\begin{itemize}
	\item avantgardní scény -- Osvobozené divadlo (Voskovec, Werich), Divadlo D (E. F. Burian)
	\end{itemize}
\end{itemize}

\section{Reakce na 1. světovou válku}
\subsection{Jaroslav Hašek (1883--1923)}
\begin{itemize}
\item spisovatel, humorista, publicista
\item z gymnázia vyloučen za účast na demonstracei, vyučil se drogistou, vystudoval obchodní akademii, pracoval v bance, později spisovatel, publicista
\item bohémský způsob života, pokus o sebevraždu
\item politická Strana mírného pokroku v mezích zákona, kritika politiky
\item zakladatel žánru hospodská historka, cetopisné povídky, črty a humoresky, většina děl napsané v hospodě a publikována v časopisech
\item satirický román \textbf{Osudy dobrého vojáka švejka za světové války}
	\begin{itemize}
	\item satiricky popisuje 1. světovou válku
	\item hovorová čeština, vulgarismy, byrokratická a vojenská hantýrka, cizí výrazy, 
	\iem nejasná dějová linie
	\item postava Švejka se vyvíjí z prosťáčka na mazaného šibala, který doslovným plněním rozkazů přivádí nadřízené k šílenosti
	\item Švejk nad válkou vítězí a ukazuje ji v celé její nesmyslnosti
	\item ironizace armády, církve, Rakouska-Uherska
	\end{itemize}
\item povídky \textbf{Velitelem města Bugulmy}
	\begin{itemize}
	\item popisuje fanatismus revoluční doby v Rusku a hloupost nových vládců
	\end{itemize}
\item \textbf{Dekameron humoru a satiry}
	\begin{itemize}
	\item povídka \textbf{Nemravné kalendáře} -- strážníci rozeberou nemravné kalendáře a tím ochrání veřejnost

	\item povídka \textbf{O básních} -- : kritika stereotypu lyrického subjektu, který neumí ve světě nalézt radost
	\end{itemize}
\end{itemize}

\end{document}