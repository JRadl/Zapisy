\title{Naturalismus v české literatuře}
\documentclass[10pt,a4paper]{article}
\usepackage[utf8]{inputenc}
\usepackage[czech]{babel}
\usepackage{amsmath}
\usepackage{amsfonts}
\usepackage{amssymb}
\usepackage{chemfig}
\usepackage{geometry}
\usepackage{wrapfig}
\usepackage{graphicx}
\usepackage{floatflt}
\usepackage{hyperref}
\usepackage{fancyhdr}
\usepackage{tabularx}
\usepackage{makecell}
\usepackage{csquotes}
\usepackage{footnote}

\MakeOuterQuote{"}

\renewcommand{\labelitemii}{$\circ$}
\renewcommand{\labelitemiii}{--}
\newcommand{\ra}{$\rightarrow$ }
\newcommand{\x}{$\times$ }
\newcommand{\lp}[2]{#1 -- #2}
\newcommand{\timeline}{\input{timeline}}


\geometry{lmargin = 0.8in, rmargin = 0.8in, tmargin = 0.8in, bmargin = 0.8in}
\date{\today}
\author{Jakub Rádl}

\makeatletter
\let\thetitle\@title
\let\theauthor\@author
\makeatother

\hypersetup{
colorlinks=true,
linkcolor=black,
urlcolor=cyan,
}



\begin{document}
\maketitle
\tableofcontents
\begin{figure}[b]
Toto dílo \textit{\thetitle} podléhá licenci Creative Commons \href{https://creativecommons.org/licenses/by-nc/4.0/}{CC BY-NC 4.0}.\\ (creativecommons.org/licenses/by-nc/4.0/)
\end{figure}
\newpage
\section{Naturalismus v české literatuře}
\subsection{Karel Josef Šlejhar (1864--1914)}
\begin{itemize}
\item o osudu člověka rozhodují temné stránky lidského charakteru
\item \textbf{Kuře melancholik}
	\begin{itemize}
	\item mladé kuře bylo poraněno jestřábem
	\item chlapec ho zachránil a měl jako svého mazlíčka
	\item paralelní sledování příběhu kuřete a chlapce a jejich podobným osudem
	\end{itemize}
\end{itemize}

\subsection{Karel Matěj Čapek-Chod (1860--1927)}
\begin{itemize}
\item časté využívání ironie a grotesky (překvapivé zvraty, směšně hrůzné)
\item román \textbf{Kašpar Lén mstitel} \textit{(čít. 326)}
	\begin{itemize}
	\item psychologická studie o vzniku zločinu
	\item muž se chtěl pomstít muži, který mu svedl jeho milou \textit{(pod. Dekameron, Máj)}
	\end{itemize}

\item román \textbf{Jindrové}
	\begin{itemize}
	\item o zvláštním vztahu mezi otcem a synem (Jindrové)
	\item vztah byl nestabilní, někdy se měli rádi, někdy hádali
	\item starší Jindra seznámí mladšího s dívkou, mladší musí do války, kde skoro oslepne
	\item když se dostal domů Jiřina byla těhotná se starším Jindrou, protože domů přišla zpráva o smrti mladého 
	\item mladý a Jiřina se dají zpátky dohromady a vezmou se
	\item Jiřina krátce po porodu umírá
	\item už úplně slepý mladý Jindra je odrazen od sebevraždy pláčem syna (Jindry)
	\end{itemize}
\end{itemize}


\subsection{Vilém Mrštík (1863--1912)}
\begin{itemize}
\item studoval na Jarošce
\item silně spjat s jižní Moravou
\item pět let žil v Praze, přišel o prostředky, šel zpět na Moravu, žil u bratra Aloise (Diváky u Hustopečí)
\item neuspořádané vztahy s ženami
\item lyrizovaná próza \textbf{Pohádka Máje}
	\begin{itemize}
	\item idylický
	\item až impresionistické líčení psychologie, pocitů a přírody
	\item příběh Pražského studenta Ríši a jeho lásky Helenky z hájovny
	\item 
	\end{itemize}
\item naturalistický román \textbf{Santa Lucia}
	\begin{itemize}
	\item kriticky vzpomíná na středoškolská studia
	\item 
	\item 
	\item 
	\end{itemize}
\item kronika \textbf{Rok na vsi}
	\begin{itemize}
	\item z větší části napsal bratr Alois
	\item 
	\end{itemize}
\item drama \textbf{Maryša}
	\begin{itemize}
	\item 
	\end{itemize}
\end{itemize}

\section{Národní divadlo}
\subsection{Vývoj divadla u nás}
\begin{itemize}
\item do konce 14. století
	\begin{itemize}
	\item první divadlo v kostele -- výjevy z života Ježíše Krista
	\item postupně se v hrách objevovala ironie 
		\begin{itemize}
		\item Mastičkář -- vypráví o tom jak měli natřít Ježíšovo tělo mastičkami a zobrazuje výjevy z tržnice více než z Bible
		\end{itemize}
	\end{itemize}
\item 15. -- 18. století
	\begin{itemize}
	\item Matěj Kopecký -- loutkář, kočovné divadlo (18. stol.)
	\item Jan Ámos Komenský
	\item Divadlo v Kotcích
	\item Bouda
	\item Stavovské divadlo (Don Giovanni, Fidlovačka)
	\end{itemize}
\item 1. polovina 19. století
	\begin{itemize}
	\item Josef Kajetán Tyl (historické, současné, dramatické báchorky)
	\item Václav Kliment Klicpera
	\item 
	\end{itemize}
\item 2. polovina 19. století
\end{itemize}

\subsection{Vývoj národního divadla}
\begin{itemize}
\item 1848 propagace Tylem
\item finanční podpora velkých šlechtických rodů 
\item do postavení divadla -- Prozatimní divadlo
	\begin{itemize}
	\item 1862 otevřeno Králem Vukašinem (Hálek)
	\end{itemize}
\item 1861 -- položen základní kámen
\item v červnu 1881 otevřeno Smetanovou Libuší
\item 1883 -- požár
\item generace národního divadla -- umělci co pracovali v první etapě po otevření, podíleli se na stavbě a výzdobě
	\begin{itemize}
	\item architekt \textbf{Josef Zítek} -- před požárem
	\item architekt \textbf{Josef Šulc} -- po požáru
	\item malíř \textbf{Mikoláš Aleš} -- výzdoba foyer
	\item malíř \textbf{František Ženíšek} -- alegorie Múzy na stropě hlediště
	\item malíř \textbf{Vojtěch Hynais} -- opona
	\item sochař \textbf{Bohuslav Schmirch} -- triga nad vstupem 
	\item skladatelé \textbf{Bedřich Smetana}, \textbf{Antonín Dvořák}, \textbf{Zdeněk Fibich}
	\end{itemize}
\item \textbf{Jaromír John}: Rajský ostrov -- popisuje historii Národního divadla
\item realismus
\end{itemize}

\subsubsection{Ladislav Stroupežnický (1850--1892)}
\begin{itemize}
\item dramaturg Národního divadla (vybírá repertoár divadla)
\item dramatik 
\item dramata \textbf{Zvíkovský rarášek}, \textbf{Paní Mincmistrová}
	\begin{itemize}
	\item postava Mikuláše Dačického
	\end{itemize}
\item soudobá hra \textbf{Naši furianti}
	\begin{itemize}
	\item vesnický spor o zastávání funkce obecního ponocného
	\item furiant -- namyšlený, sebestředný, pyšný, nafoukaný \ra převedeno do venkovského prostředí (vdávání, velké věno)
	\item fraška -- komedie pracující s nadsázkou, zjednodušené postavy, poukazování na jednu vlastnost
	\end{itemize}
\end{itemize}

\subsubsection{Gabriela Preissová (1862--1946)}
\begin{itemize}
\item venkovské prostředí, nářečí, společenské motivy
\item tragedie \textbf{Gazdina roba}
	\begin{itemize}
	\item stále hrané
	\item inspirace Josefa Bohuslava Foerstera: opera Eva
	\item Roba -- hanlivé označení pro ženu
	\item \textbf{Evě} s manželem \textbf{Mánkem} zemřela dcera, protože manžel nešel pro doktora, protože měl jiné vyznání
	\item Eva manžela opustí a nakonec se utopí v řece
	\end{itemize}
\item tragedie Její pastorkyňa
	\begin{itemize}
	\item Janáčkova opera
	\item pastorkyňa = schovankyně
	\item dítě (Jenůfa) vychovávané kostelničkou
	\item Jenůfa má dítě, nesmí ho mít, dítě je zabito 
	\end{itemize}
\end{itemize}

\end{document}