\documentclass[10pt,a4paper]{article}
\usepackage[utf8]{inputenc}
\usepackage[czech]{babel}
\usepackage{amsmath}
\usepackage{amsfonts}
\usepackage{amssymb}
\usepackage{chemfig}
\usepackage{geometry}
\usepackage{wrapfig}
\usepackage{graphicx}
\usepackage{floatflt}
\usepackage{hyperref}
\usepackage{fancyhdr}
\usepackage{tabularx}
\usepackage{makecell}
\usepackage{csquotes}
\usepackage{marginnote}

\MakeOuterQuote{"}

\renewcommand{\labelitemii}{$\circ$}
\renewcommand{\labelitemiii}{--}
\newcommand{\ra}{$\rightarrow$ }
\newcommand{\x}{$\times$ }
\newcommand{\lp}[2]{#1 -- #2}
\newcommand{\timeline}{\input{timeline}}


\geometry{lmargin = 0.8in, rmargin = 0.8in, tmargin = 0.8in, bmargin = 0.8in}
\newcommand{\note}[1]{\marginnote{\hspace{-0.6\textwidth}#1}}

\date{}
\author{Jakub Rádl}
\title{Karel Jaromír Erben: Kytice -- Rozbor díla}

\begin{document}
\maketitle

\section*{Výňatek}


\section*{Tématická stránka díla}
\begin{itemize}
\item \textbf{literární druh a žánr}: poezie, lyrická báseň s epickými prvky
	\begin{itemize}
	\item epické části jsou reálné střípky z Edisonova života
	\item připomíná pásmo (chybí interpunkce, polytématické, proud vědomí), ale narozdíl od pásma je vázaný verš
	\end{itemize}
\item \textbf{téma a motiv}:
	\begin{itemize}
	\item \textbf{hlavní téma}: 
		\begin{itemize}
		\item hledání smyslu života, (smyslem života je pokrok, jakákoli tvůrčí činnost, která posouvá lidstvo dál)
		\item přirovnání vynálezce k hazardnímu hráči, lidi s cílem s lidmi na okraji společnosti -- kontrast chování smysluplného a nesmysluplného
		
		\end{itemize}				
		
	\item \textbf{další motivy v díle}:
		\begin{itemize}
		\item smrt, beznaděj, skepse
		\item vzpomínky, smysl života, vynálezy
		\end{itemize}
	\end{itemize}
\item \textbf{časoprostor}:
	\begin{itemize}
	\item život Edisona - spojené státy a Kanada na konci 19. století
	\item únor v Praze ve 20. století
	\end{itemize}
\item \textbf{zasazení výňatku do kontextu díla}:
	\begin{itemize}
	\item \textbf{časoprostor}: třetí zpěv
	\item \textbf{obsah}: 
	\end{itemize}
\item \textbf{kompoziční výstavba}
	\begin{itemize}
	\item rozděleno na 5 zpěvů
		\begin{enumerate}
		\item lyrický subjekt se prochází po Praze a potkává lidi bez smyslu života, potkává sebevraha, který je ale pouze přelud jeho vlastního vědomí, doma se z novin dozví o vynálezu žárovky, který mu dodá naději do života
		\item přemýšlí nad Edisonovým životem, popisuje čas kdy pracoval jako průvodčí, 
		\item pozitivnější část Edisonova života, vynálezy, oslava vědeckého pokroku
		\item oslava vynálezů, obdiv vynálezců, na konci návrat k pochmurným motivům, jen někteří lidé se mohou stát významnými, i někteří významní lidé nejsou pozitivní (Nero, který vypálil Řím)
		\item ve výsledku je život smysluplný, výsledkem vynalézání je krásná osvícená Praha, ve které se ale stejně objevují špatné věci, svět je v rovnováze
		\end{enumerate}
	\item rámcová kompozice
		\begin{itemize}
		\item rámec - Praha
		\item mezi tím retrospektivní prvky z Edisonova života
		\end{itemize}
	\item paralelismus
		\begin{itemize}
		\item opakování různě obměňovaného dvojverší "Bylo v tom však něco krásného co drtí \textbackslash odvaha a radost z života i smrti"
		\end{itemize}
	\end{itemize}
\end{itemize}
\section*{Kompozice, postavy}
\begin{itemize}
\item rámec vyprávěn v ich formě 
\item myšlenky o Edisonovi vyprávěny v er formě
\item typy promluv:
	\begin{itemize}
	\item neznačená/nevlastní přímá řeč, všechno jsou myšlenky lyrického subjektu
	\end{itemize}
\item \textbf{veršová výstavba}:	
	 \begin{itemize}
	 \item vázaný verš
	 \item střídání dvojic jedenácti a dvanácti slabičných veršů
	 \item trochej
	 \item sdružený rým
	 \end{itemize}
\end{itemize}

\paragraph{Postavy}
\begin{itemize}
\item
\end{itemize}
\section*{Jazyk}
\begin{itemize}
\item jazykové prostředky a jejich funkce ve výňatku
\item tropy a figury ve výňatku
\end{itemize}
\section*{Literárně historický kontext}
\begin{itemize}
\item současní autoři:
\item další autorova díla:
\end{itemize}
\section*{Zdroje}
\end{document}