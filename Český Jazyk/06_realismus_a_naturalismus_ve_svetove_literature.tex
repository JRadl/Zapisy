\title{Realismus a naturalismus ve světové literatuře}
\documentclass[10pt,a4paper]{article}
\usepackage[utf8]{inputenc}
\usepackage[czech]{babel}
\usepackage{amsmath}
\usepackage{amsfonts}
\usepackage{amssymb}
\usepackage{chemfig}
\usepackage{geometry}
\usepackage{wrapfig}
\usepackage{graphicx}
\usepackage{floatflt}
\usepackage{hyperref}
\usepackage{fancyhdr}
\usepackage{tabularx}
\usepackage{makecell}
\usepackage{csquotes}
\usepackage{footnote}

\MakeOuterQuote{"}

\renewcommand{\labelitemii}{$\circ$}
\renewcommand{\labelitemiii}{--}
\newcommand{\ra}{$\rightarrow$ }
\newcommand{\x}{$\times$ }
\newcommand{\lp}[2]{#1 -- #2}
\newcommand{\timeline}{\input{timeline}}


\geometry{lmargin = 0.8in, rmargin = 0.8in, tmargin = 0.8in, bmargin = 0.8in}
\date{\today}
\author{Jakub Rádl}

\makeatletter
\let\thetitle\@title
\let\theauthor\@author
\makeatother

\hypersetup{
colorlinks=true,
linkcolor=black,
urlcolor=cyan,
}



\begin{document}
\maketitle
\tableofcontents
\begin{figure}[b]
Toto dílo \textit{\thetitle} podléhá licenci Creative Commons \href{https://creativecommons.org/licenses/by-nc/4.0/}{CC BY-NC 4.0}.\\ (creativecommons.org/licenses/by-nc/4.0/)
\end{figure}
\newpage

\section{Úvod}
\begin{itemize}
\item druhá polovina 19. století
\end{itemize}
\paragraph{Prostředí a postavy}
\begin{itemize}
\item snaha zachytit \textbf{reálný a komplexní} obraz světa, takového jaký je
\item autoři se seznamují s prostředím o kterém píší a s postavami o kterých píší
	\begin{itemize}
	\item[\ra] vznik \textbf{typizovaných} postav daného prostředí, společenské vrstvy
	\item postavy mají typické i individuální rysy
	\item snaha vytvořit obecný charakterový typ
	\end{itemize}
\end{itemize}

\paragraph{Jazyk}
\begin{itemize}
\item vypravěč hovoří spisovně
\item postavy hovoří jazykem své společenské vrstvy (spisovné, uhlazené $\times$ nespisovné, vulgarismy, \ldots)
\end{itemize}

\paragraph{Kritický realismus}
\begin{itemize}
\item snaha hledat taková témata, která ukážou na nějaké \textbf{společenské problémy}
\item kritika společnosti, státu
\end{itemize}

\paragraph{Naturalismus}
\begin{itemize}
\item \textbf{vyhrocená forma} realismu
\item může docházet ke zkreslení
\item vychází ze \textbf{dvou předpokladů podstaty člověka} (dědičnost, výchova) -- naturalismus tvrdí, že není možné se z těchto vymanit
\item opomíjí volní složku osobnosti \ra postavy \textbf{nemají šanci se napravit}
\item \textbf{detailní popisy }i nepříjemných až nechutných skutečností a objektů
\end{itemize}


\section{Francie}
\subsection{Honoré de Balzac (1799--1850)}
\begin{itemize}
\item pokus o \textbf{zachycení všech vrstev} francouzské společnosti (aristokracie, měštanstvo, spodina)
\item hlavními postavami jsou muži i ženy
\item zobrazuje první polovinu devatenáctého století
\item přes 100 románů -- souhrnně nazváno \textbf{Lidská komedie}
	\begin{itemize}
	\item \textbf{Otec Goriot}
	\item \textbf{Lesk a bída kurtizán} -- ("kurtizána" = prostitutka)
	\item \textbf{Ztracené iluze}
	\item komedie \textbf{Evženie Grandeová}
	\end{itemize}
\item postavy se prolínají mezi romány, mohou se vyskytovat v různých etapách svých životů
\end{itemize}

\subsection{Gustave Flaubert (1821--1880)}
\begin{itemize}
\item venkovská, bohatá rodina, studoval medicínu
\item cílem zachytit lidské vášně 
\item historická tématika
\item román \textbf{Salambo} -- historické prostředí Kartága
\item soudobý román \textbf{Citová výchova}
\item román \textbf{Paní Bovaryová (Mravy francouzského venkova)}
	\begin{itemize}
	\item \textit{hlavní postava \textbf{Ema} -- romanticky zaměřená,  idealizuje si Paříž, neváží si Charlese, který pro ni chce udělat co může, povrchní, naivní, láska = přepych}
	\item \textit{provdá se za venkovského lékaře, život na venkově ji neuspokojuje \ra najde si milence \textbf{Rudolfa}}
	\item \textit{Rudolf nechce rozbít manželství a starat se o ni \ra rozejde se \ra přestěhování \ra opakování}
	\item \textit{nový vztah s milencem sama financuje \ra zrujnuje domácnost}
	\item veden soudní proces kvůli podtitulu \ra reklama 
	\item \textit{(čít.: 171)}
	\end{itemize}
\end{itemize}

\subsection{Emile Zola (1840--1902)}
\begin{itemize}
\item \textbf{zakladatel naturalismu}
\item literárně teoretické dílo \textbf{Experimentální román}
	\begin{itemize}
	\item román by měl být postaven na \textbf{vědeckých základech}
	\item hovoří o stěžejním vlivu \textbf{dědičnosti} a \textbf{prostředí} na člověka
	\item spisovatel má být \textbf{nestranný zprostředkovatel} toho, co se děje
	\end{itemize}
\item cyklus románů \textbf{Rougon-Macquartové (přírodopisná a sociální studie jedné rodiny za druhého císařství)}
	\begin{itemize}
	\item postavy se též vyskytují napříč romány
	\item \textbf{Zabiják}
		\begin{itemize}
		\item \textit{(čít.: 174)}
		\item zabiják -- personifikace alkoholu
		\end{itemize}
	\item \textbf{Štěstí Rougonů}
	\item \textbf{Nana} \textit{(čít.: 177)}
	\end{itemize}
\end{itemize}

\subsection{Guy de Maupassant (1850--1893)}
\begin{itemize}
\item sám se k naturalismu nehlásí
\item zemřel na syfilis
\item román \textbf{Miláček}
	\begin{itemize}
	\item hlavním hrdinou je muž, který přes ženy pronikne do vyšší společnosti (pod. Stendhal)
	\end{itemize}
\item povídka \textbf{Kulička}
	\begin{itemize}
	\item \textit{dostavník s prostitutkou je zastaven, důstojník chce za puštění dál, aby se mu podvolila, ostatní ji nakonec přesvědčí, ale následní jí ještě více opovrhují}
	\end{itemize}
\item povídka \textbf{Muška}
\end{itemize}

\section{Anglie}
\subsection{Charles Dickens (1812-1870)}
\begin{itemize}
\item tvorba na pomezí romantismu a kritického realismu
\item rodina ve finanční krizi, otec uvězněn \ra musel jako malý pracovat
\item \textbf{osud dětí} se stává významným tématem jeho tvorby
\item zobrazuje široký společenský obraz \textbf{viktoriánské Anglie}
\item postavy (děti i dospělí) jsou zachycovány jako oběti společenských sil, institucí
\item zobrazuje nedokonalost společenského systému, který nepomáhá potřebným
\item kromě kritických také humorná díla
\item román \textbf{Oliver Twist}
	\begin{itemize}
	\item příběh chlapce sirotka, žije v chudobincích a sirotčincích
	\item realisticky zachycuje jak je s dětmi zacházeno
	\item pracuje ve výrobě rakví
	\item část života žije ve společnosti zlodějů
	\item happy end (romantický prvek)
	\end{itemize}
\item román \textbf{David Coopperfield}
	\begin{itemize}
	\item osudy chlapce po smrti otce, vyrůstá s matkou a novým manželem
	\item \textit{(čít.: 92)}
	\item otec mu zakazuje scházet se s uklízečkami, být ve svém pokoji, číst dle svého výběru, \ldots, otec také brání matčinu vztahu k Davidovi \ra podřízené postavení matky, potlačování chlapcovi osobnosti
	\item žena ve viktoriánské Anglii je zcela odkázána na příjem svého muže
	\end{itemize}
\item román \textbf{Malá Borritka} (též dětský hrdina)
\item humorný román \textbf{Kronika Pickwickova klubu}
	\begin{itemize}
	\item \textit{(čít.: 89)}
	\end{itemize}
\end{itemize}

\section{Rusko}
\begin{itemize}
\item velmi \textbf{chudý venkov}
\item \textbf{byrokracie} ve městech, život podřadných \textbf{úředníků}
\item obecná \textbf{úplatnost}, \textbf{zneužívání moci} šlechtou
\item absence průmyslové revoluce vede k velkým \textbf{společenským rozdílům}
\item snaha též o \textbf{psychologickou analýzu} postav
\item \textbf{románová epopej} ("román řeka", "románová sága") -- obsáhlé, několika svazkové dílo, jedna hlavní dějová linie a mnoho bočních
\end{itemize}

\subsection{Lev Nikolajevič Tolstoj (1828--1910)}
\begin{itemize}
\item osvícený šlechtic, snaha o šíření vzdělanosti (škola na venkově)
\item novinář
\item románová epopej \textbf{Vojna a mír}
	\begin{itemize}
	\item 4 díly
	\item z období napoleonských válek
	\item paralelně líčí děj ve válce a za míru
	\end{itemize}
\item román \textbf{Anna Karenina}
	\begin{itemize}
	\item několikrát zfilmováno (oba romány), částečně jako divadelní představení
	\item \textit{mladá dívka je provdána za staršího muže}
	\item \textit{potká důstojníka \textbf{Vronskijho}, do kterého se zamiluje, přestože má dítě}
	\item \textit{čelí rozhodnutí mezi rodinou se synem a milencem, zvolila milence}
	\item \textit{s Vronskijm má ještě dceru}
	\item \textit{končí sebevraždou}
	\item \textit{porovnává s protipólným fungujícím vztahem}
	\item \textit{(čít.: 191)}
	\end{itemize}
\end{itemize}

\subsection{Fjodor Michajlovič Dostojevskij (1821--1881)}
\begin{itemize}
\item odsouzen k smrti, na popravišti mu byl trest změněn na uvěznění na Sibiři
\item životní osudy se promítají do tvorby
\item \textbf{psychologické romány} -- podrobně rozebírá myšlenky postav vedoucí k jejich činům
\item román \textbf{Zločin a trest}
	\begin{itemize}
	\item \textit{mladý student \textbf{Raskolnikov}, chytrý nadaný, z chudého prostředí }
	\item \textit{studuje v Petrohradě se slabou podporou od matky a sester}
	\item \textit{potřebuje získat peníze, rozhodne se zabít lichvářku, omlouvá to tím, že významné historické osobnosti provedly hrůzné věci pro větší dobro}
	\item \textit{nepovažuje zabití za zločin, protože \textbf{lichvářka} se chová nemorálně}
	\item \textit{prolíná se plánování vraždy s pasážemi, kde Raskolnikov ukazuje svou dobrou osobnost (např. pomůže rodině Soňi)}
	\item \textit{vražda se pokazí kde to jde a zabije také lichvářčinu sestru, z čehož je psychicky rozhozen}
	\item \textit{přemýšlí co udělal, proč to udělal, co na tom bylo špatně}
	\end{itemize}
\item román \textbf{Bratři Karamazovi}
\item román \textbf{Idiot}
	\begin{itemize}
	\item \textbf{Lev Myškin} -- má čisté srdce \ra jedná nezištně, okolí ho považuje za blázna
	\end{itemize}
\end{itemize}

\section{USA}
\subsection{Mark Twain (1835--1910)}
\begin{itemize}
\item román \textbf{Dobrodružství Toma Sawyera}
\item román \textbf{Dobrodružství Huckleberryho Finna}
\item romány a povídky
\item tématika dětských problému zastřešena problémy lidí dospělých
\end{itemize}

\section{Polsko}
\subsection{Henryk Sienkiewicz (1846--1916)}
\begin{itemize}
\item snaží se povzbudit Poláky, podpořit \textbf{vlastenectví}
\item ukazuje na kvalitu polského národa
\item podporuje \textbf{křesťanství}
\item prvky novoromantismu
\item román \textbf{Quo vadis}
\item román \textbf{Trilogie}
	\begin{itemize}
	\item \textbf{Ohněm a mečem}, \textbf{Potopa}, \textbf{Pan Wolodyjowski}
	\item historický román z 17. století
	\item boj poláků s ukrajinskými kozáky
	\end{itemize}
\item román \textbf{Křižáci}
	\begin{itemize}
	\item historický román z 15. století
	\item boj Poláků s německými křižáky
	\end{itemize}
\item román \textbf{Pouští a Pralesem}
	\begin{itemize}
	\item pro děti a mládež, dobrodružný
	\item \textit{uneseny 2 děti -- 14letý Poák a 8letá angličanka}
	\item \textit{hoch chrání holčičku, snaží se vrátit domů}
	\end{itemize}
\end{itemize}

\section{Drama}
\subsection{Henrik Ibsen (1828--1906)}
\begin{itemize}
\item Norsko
\item \textbf{dodnes hrané hry}
\item často dodržuje \textbf{Aristotelovu zásadu} jednoty místa času a děje
\item zapojuje \textbf{retrospektivní prvky}, kterými odkrývá minulosti psotav
\item proniká k \textbf{psychologii} postav
\item časté \textbf{ženské hlavní postavy}
\item poukazuje na to, že realita bývá jiná, než zdání
\item pesimistické, \textbf{negativní vyústění}
\item ideály přetrvávají, ale \textbf{nenaplněné}
\item drama \textbf{Nora} (Domov loutek, Dům pro panenky)
	\begin{itemize}
	\item \textit{\textbf{Nora} žije ve zdánlivě spokojeném a funkčním manželství se dvěma dětmi}
	\item \textit{\textbf{manžel} onemocní a jeho léčba je nákladná \ra nora zfalšuje manželův podpis na půjčce, manželovi to zamlčí a snaží se dluh sama splatit}
	\item \textit{bankéř zjistí falešnost podpisu a vydírá manžela \ra manžel se naštve, Noře zakáže vychovávání dětí, manželství je zachované jen z vnějšku}
	\item \textit{po vyřešení situace se manžel chce vrátit do původního stavu}
	\item \textit{Nora si uvědomí své postavení "loutky" \ra opustí manžela (a tím i děti) -- kontroverzní rozhodnutí}
	\end{itemize}

\item drama \textbf{Divoká Kachna}
	\begin{itemize}
	\item \textit{\textbf{Starý Ekdal} X \textbf{Werle} -- společníci ve firmě, Werle svedl své podvody na Ekdala, kterého tím dostal do vězení}
	\item \textit{Werle dostal \textbf{Ginu} -- svou služku do jiného stavu, provdal ji za Ekdalova syna \textbf{Hjalmara}. Dcera se jmenuje \textbf{Hedvika}}
	\item \textit{Werleho syn \textbf{Gregers} se dozvěděl pravdu a pokusil se situaci napravit tak, že pravdu řekne Hjalmarovi}
	\item \textit{Hjalmar chce opustit rodinu, přestane milovat Hedviku, myslí si, že to věděla a nikdy ho skutečně nemilovala}
	\item \textit{Gregers Hedvice namluví, že má otci obětovat divokou kachnu}
	\item \textit{Hedvika spáchá sebevraždu, aby Hjalmarovi dokázala jak moc ho miluje (lidský život má větší cenu, než zvířecí)}
	\item \textit{kachna byla součástí umělého lesa na půdě, umělá realita lesu na půdě je paralelní s životem ve lži}
	\item \textit{(čít.: 206)}
	\end{itemize}
	
\item drama \textbf{Peer Gynt}
	\begin{itemize}
	\item hlavní postava spadá do literárního typu zbytečného člověka (pod. Evžene Oněgin)
	\end{itemize}
	
\item drama \textbf{Helda Gablerová}
\end{itemize}

\subsection{Anton Pavlovič Čechov (1860--1904)}
\begin{itemize}
\item Rusko
\item dramatik a prozaik
\item humoristické příběhy / žertovné miniatury -- 600
	\begin{itemize}
	\item pod pseudonymem
	\item \textbf{Kniha Stížností}
		\begin{itemize}
		\item literárně zpracována kniha stížností z železniční stanice
		\end{itemize}	
	\end{itemize}
\item příběhy o ubohosti ruského úředníka (pod. Gogol)
	\begin{itemize}
	\item povídka \textbf{Úředníkova Smrt}
		\begin{itemize}
		\item absurdní příběh bezvýznamného úředníka
		\item \textit{šel do divadla, na sedadle před ním seděl vysoce postavený úředník}
		\item \textit{úředníček kýchnul a poprskal vysokého úředníka, omlouval se mu a po pár dnech si domluvil schůzku}
		\item \textit{úředník ho z kanceláře vyhnal, úředníček si myslel, že se na něj zlobí}
		\item \textit{úředníček ze strachu onemocní a zemře}
		\end{itemize}
	\end{itemize}
\item soudničky
	\begin{itemize}
	\item reportáže ze soudní síně
	\end{itemize}
\item psychologické povídky a novely
	\begin{itemize}
	\item \textbf{Dáma s Psíčkem}
		\begin{itemize}
		\item milostný příběh, flirt se promění v lásku
		\item některé části satirické
		\end{itemize}
	\end{itemize}
\item drama
	\begin{itemize}
	\item \textbf{tragikomické vyznění} -- "smích skrze slzy"
	\item specifická, \textbf{lyrizovaná dramata} (dějová stránka ustupuje do pozadí tragedie a dramatičnost je rozvedena ne příběhem, ale psychologií postav)
	\item v centru není vyhrocená zápletka 
	\item zobrazuje průměrné lidi a jejich zdánlivě všední život
	\item \textbf{vše má svůj účel}, řád, smysl -- "je-li na scéně puška, musí se z ní vystřelit"
	\item \textbf{MCHAT} -- Moskevké Umělecké Akademické Divadlo (režiséři Stanislavskij, Němirovič-Dančenko)
		\begin{itemize}
		\item ztvárňovalo mnoho Čechovových her
		\item Čechov nebyl příliš spokojený
		\end{itemize}
	\item \textbf{Tři sestry}, \textbf{Racek}, \textbf{Strýček Váňa}, \textbf{Višňový sad}
	\item komedie \textbf{Višňový sad}
		\begin{itemize}
		\item 1904, krátce před svou smrtí
		\item střet světa staré nepřizpůsobivé ruské šlechty a nové dravé generace, která se snaží nahlížet na život praktičtěji a využít možností, které se jim nabízejí
		\item \textit{odehrává se na panství kněžny \textbf{Raněvské}, která má dceru \textbf{Aňu} a bratra \textbf{Gajeva}}
		\item \textit{kněžna po smrti syna odjela do Francie, nyní se vrací na zadlužené panství spravované \textbf{Varjou}, její nevlastní dcerou}
		\item \textit{Varje se dvoří \textbf{Lopachin}, potomek generace nevolníků, člen mladé generace, která neuznává staré hodnoty }
		\item \textit{Lopachin navrhuje kněžně, aby rozparcelovala a prodala Višňový sad, ona ho ze sentimentality nechce prodat, prodá se celé panství a koupí ho Lopachin}
		\end{itemize}
	\end{itemize}
\item povídka \textbf{Myslivec} (čít.: 197)
\end{itemize}



\end{document}