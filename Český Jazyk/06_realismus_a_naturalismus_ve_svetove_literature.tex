\title{Realismus a naturalismus ve světové literatuře}
\documentclass[10pt,a4paper]{article}
\usepackage[utf8]{inputenc}
\usepackage[czech]{babel}
\usepackage{amsmath}
\usepackage{amsfonts}
\usepackage{amssymb}
\usepackage{chemfig}
\usepackage{geometry}
\usepackage{wrapfig}
\usepackage{graphicx}
\usepackage{floatflt}
\usepackage{hyperref}
\usepackage{fancyhdr}
\usepackage{tabularx}
\usepackage{makecell}
\usepackage{csquotes}
\usepackage{footnote}

\MakeOuterQuote{"}

\renewcommand{\labelitemii}{$\circ$}
\renewcommand{\labelitemiii}{--}
\newcommand{\ra}{$\rightarrow$ }
\newcommand{\x}{$\times$ }
\newcommand{\lp}[2]{#1 -- #2}
\newcommand{\timeline}{\input{timeline}}


\geometry{lmargin = 0.8in, rmargin = 0.8in, tmargin = 0.8in, bmargin = 0.8in}
\date{\today}
\author{Jakub Rádl}

\makeatletter
\let\thetitle\@title
\let\theauthor\@author
\makeatother

\hypersetup{
colorlinks=true,
linkcolor=black,
urlcolor=cyan,
}



\begin{document}
\maketitle
\tableofcontents
\begin{figure}[b]
Toto dílo \textit{\thetitle} podléhá licenci Creative Commons \href{https://creativecommons.org/licenses/by-nc/4.0/}{CC BY-NC 4.0}.\\ (creativecommons.org/licenses/by-nc/4.0/)
\end{figure}
\newpage

\section{Úvod}
\begin{itemize}
\item druhá polovina 19. století
\end{itemize}
\paragraph{Prostředí a postavy}
\begin{itemize}
\item snaha zachytit \textbf{reálný a komplexní} obraz světa, takového jaký je
\item autoři se seznamují s prostředím o kterém píší a s postavami o kterých píší
	\begin{itemize}
	\item[\ra] vznik \textbf{typizovaných} postav daného prostředí, společenské vrstvy
	\item postavy mají typické i individuální rysy
	\item snaha vytvořit obecný charakterový typ
	\end{itemize}
\end{itemize}

\paragraph{Jazyk}
\begin{itemize}
\item vypravěč hovoří spisovně
\item postavy hovoří jazykem své společenské vrstvy (spisovné, uhlazené $\times$ nespisovné, vulgarismy, \ldots)
\end{itemize}

\paragraph{Kritický realismus}
\begin{itemize}
\item snaha hledat taková témata, která ukážou na nějaké \textbf{společenské problémy}
\item kritika společnosti, státu
\end{itemize}

\paragraph{Naturalismus}
\begin{itemize}
\item vyhrocená forma realismu
\item může docházet ke zkreslení
\item vychází ze dvou předpokladů podstaty člověka (dědičnost, výchova) -- naturalismus tvrdí, že není možné se z těchto vymanit
\item opomíjí volní složku osobnost \ra postavy nemají šanci se napravit
\item detailní popisy i nepříjemných až nechutných skutečností a objektů
\end{itemize}


\section{Francie}
\paragraph{Honoré de Balzac (1799--1850)}
\begin{itemize}
\item pokus o zachycení všech vrstev francouzské společnosti (aristokracie, měštanstvo, spodina)
\item hlavními postavami jsou muži i ženy
\item zobrazuje první polovinu devatenáctého století
\item přes 100 románů -- souhrnně nazváno \textbf{Lidská komedie}
	\begin{itemize}
	\item \textbf{Otec Goriot}
	\item \textbf{Lesk a bída kurtizán} -- ("kurtizána" = prostitutka)
	\item \textbf{Ztracené iluze}
	\end{itemize}
\item postavy se prolínají mezi romány, mohou se vyskytovat v různých etapách svých životů
\item komedie
	\begin{itemize}
	\item \textbf{Evženie Grandeová}
	\end{itemize}
\end{itemize}


\end{document}