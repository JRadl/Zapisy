\title{Realismus a naturalismus ve světové literatuře}
\documentclass[10pt,a4paper]{article}
\usepackage[utf8]{inputenc}
\usepackage[czech]{babel}
\usepackage{amsmath}
\usepackage{amsfonts}
\usepackage{amssymb}
\usepackage{chemfig}
\usepackage{geometry}
\usepackage{wrapfig}
\usepackage{graphicx}
\usepackage{floatflt}
\usepackage{hyperref}
\usepackage{fancyhdr}
\usepackage{tabularx}
\usepackage{makecell}
\usepackage{csquotes}
\usepackage{footnote}

\MakeOuterQuote{"}

\renewcommand{\labelitemii}{$\circ$}
\renewcommand{\labelitemiii}{--}
\newcommand{\ra}{$\rightarrow$ }
\newcommand{\x}{$\times$ }
\newcommand{\lp}[2]{#1 -- #2}
\newcommand{\timeline}{\input{timeline}}


\geometry{lmargin = 0.8in, rmargin = 0.8in, tmargin = 0.8in, bmargin = 0.8in}
\date{\today}
\author{Jakub Rádl}

\makeatletter
\let\thetitle\@title
\let\theauthor\@author
\makeatother

\hypersetup{
colorlinks=true,
linkcolor=black,
urlcolor=cyan,
}



\begin{document}
\maketitle
\tableofcontents
\begin{figure}[b]
Toto dílo \textit{\thetitle} podléhá licenci Creative Commons \href{https://creativecommons.org/licenses/by-nc/4.0/}{CC BY-NC 4.0}.\\ (creativecommons.org/licenses/by-nc/4.0/)
\end{figure}
\newpage

\section{Úvod}
\begin{itemize}
\item druhá polovina 19. století
\end{itemize}
\paragraph{Prostředí a postavy}
\begin{itemize}
\item snaha zachytit \textbf{reálný a komplexní} obraz světa, takového jaký je
\item autoři se seznamují s prostředím o kterém píší a s postavami o kterých píší
	\begin{itemize}
	\item[\ra] vznik \textbf{typizovaných} postav daného prostředí, společenské vrstvy
	\item postavy mají typické i individuální rysy
	\item snaha vytvořit obecný charakterový typ
	\end{itemize}
\end{itemize}

\paragraph{Jazyk}
\begin{itemize}
\item vypravěč hovoří spisovně
\item postavy hovoří jazykem své společenské vrstvy (spisovné, uhlazené $\times$ nespisovné, vulgarismy, \ldots)
\end{itemize}

\paragraph{Kritický realismus}
\begin{itemize}
\item snaha hledat taková témata, která ukážou na nějaké \textbf{společenské problémy}
\item kritika společnosti, státu
\end{itemize}

\paragraph{Naturalismus}
\begin{itemize}
\item vyhrocená forma realismu
\item může docházet ke zkreslení
\item vychází ze dvou předpokladů podstaty člověka (dědičnost, výchova) -- naturalismus tvrdí, že není možné se z těchto vymanit
\item opomíjí volní složku osobnost \ra postavy nemají šanci se napravit
\item detailní popisy i nepříjemných až nechutných skutečností a objektů
\end{itemize}


\section{Francie}
\subsection{Honoré de Balzac (1799--1850)}
\begin{itemize}
\item pokus o zachycení všech vrstev francouzské společnosti (aristokracie, měštanstvo, spodina)
\item hlavními postavami jsou muži i ženy
\item zobrazuje první polovinu devatenáctého století
\item přes 100 románů -- souhrnně nazváno \textbf{Lidská komedie}
	\begin{itemize}
	\item \textbf{Otec Goriot}
	\item \textbf{Lesk a bída kurtizán} -- ("kurtizána" = prostitutka)
	\item \textbf{Ztracené iluze}
	\item komedie \textbf{Evženie Grandeová}
	\end{itemize}
\item postavy se prolínají mezi romány, mohou se vyskytovat v různých etapách svých životů
\end{itemize}

\subsection{Gustave Flaubert (1821--1880)}
\begin{itemize}
\item venkovská, bohatá rodina, studoval medicínu
\item cílem zachytit lidské vášně 
\item historická tématika
\item román \textbf{Salambo} -- historické prostředí Kartága
\item román \textbf{Citová výchova}
\item román \textbf{Paní Bovaryová (Mravy francouzského venkova)}
	\begin{itemize}
	\item \textit{hlavní postava \textbf{Ema} -- romanticky zaměřená}
	\item \textit{provdá se za venkovského lékaře, život na venkově ji neuspokojuje \ra najde si milence \textbf{Rudolfa}}
	\item \textit{Rudolf nechce rozbít manželství a starat se o ni \ra rozejde se \ra přestěhování \ra opakování}
	\item \textit{nový vztah s milencem sama financuje \ra zrujnuje domácnost}
	\item veden soudní proces kvůli podtitulu \ra reklama 
	\item z úryvku na str 171 charakterizuj paní Bovaryovou: idealizuje si Paříž, neváží si Charlese, který pro ni chce udělat co může, povrchní, naivní, láska = přepych
	\end{itemize}
\end{itemize}

\subsection{Emile Zola (1840--1902)}
\begin{itemize}
\item zakladatel naturalismu
\item literárně teoretické dílo \textbf{Experimentální román}
	\begin{itemize}
	\item román by měl být postaven na vědeckých základech
	\item hovoří o stěžejním vlivu dědičnosti a prostředí na člověka
	\item spisovatel má být nestranný zprostředkovatel toho, co se děje
	\end{itemize}
\item cyklus románů \textbf{Rougon-Macquartové (přírodopisná a sociální studie jedné rodiny za druhého císařství)}
	\begin{itemize}
	\item postavy se také vyskytují napříč romány
	\item \textbf{Zabiják}
		\begin{itemize}
		\item najdi naturalistické prvky v úryvku na str. 174, 177: mnoho detailního popisu, poslední věta Zabijáka
		\item proč se román jmenuje Zabiják?: personifikace alkoholu
		\end{itemize}
	\item \textbf{Štěstí Rougonů}, \textbf{Nana}
	\end{itemize}
\end{itemize}

\subsection{Guy de Maupassant (1850--1893)}
\begin{itemize}
\item uč. 54.
\item zemřel na syfilis
\item román \textbf{Miláček}
	\begin{itemize}
	\item hlavním hrdinou je muž, který přes ženy pronikne do vyšší společnosti (pod. Stendhal)
	\end{itemize}
\item povídka \textbf{Kulička}
	\begin{itemize}
	\item \textit{dostavník s prostitutkou je zastaven, důstojník chce za puštění dál, aby se mu podvolila, ostatní ji nakonec přesvědčí, ale následní jí ještě více opovrhují}
	\end{itemize}
\item povídka \textbf{Muška}
\end{itemize}

\section{Anglie}
\subsection{Charles Dickens (1812-1870)}
\begin{itemize}
\item tvorba na pomezí romantismu a realismu
\item rodina ve finanční krizi, otec uvězněn \ra musel jako malý pracovat
\item \textbf{osud dětí} se stává významným tématem jeho tvorby
\item zobrazuje široký společenský obraz viktoriánské Anglie
\item postavy (děti i dospělí) jsou zachycovány jako oběti společenských sil, institucí
\item zobrazuje nedokonalost společenského systému, který nepomáhá potřebným
\item kromě kritických také humorná díla
\item román \textbf{Oliver Twist}
	\begin{itemize}
	\item příběh chlapce sirotka, žije v chudobincích a sirotčincích
	\item realisticky zachycuje jak je s dětmi zacházeno
	\item pracuje ve výrobě rakví
	\item část života žije ve společnosti zlodějů
	\item happy end
	\end{itemize}
\item román \textbf{David Coopperfield}
	\begin{itemize}
	\item osudy chlapce po smrti otce, vyrůstá s matkou a novým manželem
	\item čít. 92:
		\begin{enumerate}
		\item vyhledej v textu místa, kde je patrné totální potlačování chlapcovi osobnosti a patrné podřízené postavení matky -- otec mu zakazuje scházet se s uklízečkami, být ve svém pokoji, číst dle svého výběru, \ldots, otec také brání matčinu vztahu k Davidovi
		\item zkuste vyvodit postavení ženy ve viktoriánské Anglii -- žena je zcela odkázána na příjem svého muže
		\end{enumerate}
	\end{itemize}
\item román \textbf{Malá Borritka} (též dětský hrdina)
\item humorný román \textbf{Kronika Pickwickova klubu}
	\begin{itemize}
	\item ukázka -- str. 89
	\end{itemize}
\end{itemize}

\section{Rusko}
\begin{itemize}
\item velmi chudý venkov
\item byrokracie ve městech, život podřadných úředníků
\item obecná úplatnost, zneužívání moci šlechtou
\item snaha též o psychologickou analýzu postav
\item románová epopej (román řeka, románová sága) -- obsáhlé, několika svazkové dílo, jedna hlavní dějová linie a mnoho bočních
\end{itemize}

\subsection{Lev Nikolajevič Tolstoj (1828--1910)}
\begin{itemize}
\item osvícený šlechtic, snaha o šíření vzdělanosti (škola na venkově)
\item románová epopej \textbf{Vojna a mír}
	\begin{itemize}
	\item 4 díly
	\item z období napoleonských válek
	\item paralelně líčí děj ve válce a za míru
	\end{itemize}
\item román \textbf{Anna Karenina}
	\begin{itemize}
	\item několikrát zfilmováno (oba romány), částečně jako divadelní představení
	\item mladá dívka je provdána za staršího muže
	\item potká důstojníka \textbf{Vronskijho}, do kterého se zamiluje, přestože má dítě
	\item čelí rozhodnutí mezi rodinou se synem a milencem, zvolila milence
	\item s Vronskijm má ještě dceru
	\item porovnává s protipólným fungujícím vztahem
	\end{itemize}
\end{itemize}

\subsection{Fjodor Michajlovič Dostojevskij (1821--1881)}
\begin{itemize}
\item odsouzen k smrti, na popravišti mu byl trest změněn na uvěznění na Sibiři
\item psychologické romány -- podrobně rozebírá myšlenky postav vedoucí k jejich činům
\item román \textbf{Zločin a trest}
	\begin{itemize}
	\item mladý student \textbf{Raskolnikov}, chytrý nadaný, z chudého prostředí 
	\item studuje v petrohradě se slabou podporou od matky a sester
	\item potřebuje získat peníze, rozhodne se zabít lichvářku, omlouvá to tím, že významnmé historické osobnosti provedly hrůzné věci pro větší dobro
	\item nepovažuje zabití za zločin, protože lichvářka se chová nemorálně
	\item při vraždě zabije také lichvářčinu sestru, z čehož je psychicky rozhozen
	\item prolíná se plánování veaždy s pasážemi, kde Raskolnikov ukazuje svou dobrou osobnost (např. pomůže rodině Soňi)
	\item přemýšlí co udělal, proč to udělal, co na tom bylo špatně
	\end{itemize}
\item román Bratři Karamazovi
\item román Idiot
	\begin{itemize}
	\item Lev Myškin -- má čisté srdce, jedná nezištně, okolí ho považuje za blázna
	\end{itemize}
\end{itemize}

\section{USA}
\subsection{Mark Twain (1835--1910)}
\begin{itemize}
\item román \textbf{Dobrodružství Toma Sawyera}
\item román \textbf{Dobrodružství Huckleberryho Finna}
\item romány a povídky
\end{itemize}

\section{Drama}
\subsection{Henrik Ibsen (1828--1906)}
\begin{itemize}
\item Norsko
\item dodnes hrané hry
\item drama \textbf{Nora}, Heda Gablerová, Divoká kachna, Peer Gynt
\item často dodržuje aristotelovu zásadu jednoty místa času a děje
\item zapojuje retrospektivní prvky, kterými odkrývá minulosti psotav
\item proniká k psychologii postav
\item časté ženské hlavní postavy
\item poukazuje na to, že realita bývá jiná, než zdání
\item pesimistické, negativní vyústění
\item ideály přetrvávají, ale nenaplněné
\item drama \textbf{Nora} (Domov loutek, Dům pro panenky)
	\begin{itemize}
	\item Nora žije ve zdánlivě spokojeném a funkčním manželství se dvěma dětmi
	\item manžel onemocní a jeho léčba je nákladná \ra nora zfalšuje manželův podpis na půjčce, manželovi to zamlčí a snaží se dluh sama splatit
	\item bankéř zjistí falešnost podpisu a vydírá manžela \ra manžel se naštve, Noře zakáže vychovávání dětí, manželství je zachované jen z vnějšku
	\item po vyřešení situace se manžel chce vrátit do původního stavu
	\item Nora si uvědomí své postavení "loutky" \ra opustí manžela (a tím i děti)
	\end{itemize}

\item drama \textbf{Divoká Kachna}
	\begin{itemize}
	\item \textbf{Starý Ekdal} X \textbf{Werle} -- společníci ve firmě, Werle svedl své podvody na Ekdala, kterého tím dostal do vězení
	\item Werle x Gina -- \textbf{Gina} je Werleho služka, dostal ji do jiného stavu, provdal ji za Ekdalova syna \textbf{Hjalmara}. Dcera se jmenuje \textbf{Hedvika}
	\item Werleho syn \textbf{Gregers} se dozvěděl pravdu a pokusil se situaci napravit tak, že pravdu řekne Hjalmarovi
	\item Hjalmar chce opustit rodinu, přestane milovat Hedviku, myslí si, že to věděla a nikdy ho skutečně nemilovala
	\item Gregers Hedvice namluví, že má otci obětovat divokou kachnu
	\item Hedvika spáchá sebevraždu, aby Hjalmarovi dokázala jak moc ho miluje (lidský život má větší cenu, než zvířecí)
	\item kachna byla součástí umělého lesa na půdě, umělá realita lesu na půdě je paralelní s životem ve lži
	\end{itemize}
\end{itemize}

\subsection{Anton Pavlovič Čechov (1860--1904)}
\begin{itemize}
\item Rusko
\item dramatik a prozaik
\item humoristické příběhy / žertovné miniatury -- 600
\item pseudonym
\item Kniha Stížností
	\begin{itemize}
	\item literárně zpracována kniha stížností z železniční stanice
	\end{itemize}
\item příběhy o ubohosti ruského úředníka (pod. Gogol)
\item povídka Úředníkova Smrt
	\begin{itemize}
	\item absurdní příběh bezvýznamného úředníka, který šel do divadla
	\item na sedadle před ním seděl vysoce postavený úředník
	\item úředníček kýchnul a poprskal vysokého úředníka, omlouval se mu a po pár dnech si domluvil schůzku
	\item úředník ho z kanceláře vyhnal, úředníček si myslel, že se na něj zlobí
	\item úředníček ze strachu onemocní a zemře
	\end{itemize}
\item soudničky
\item psychologické povídky a novely
	\begin{itemize}
	\item Dáma s Psíčkem
	\end{itemize}
\item drama
	\begin{itemize}
	\item tragikomické vyznění -- "smích skrze slzy"
	\item specifická, lyrizovaná dramata
	\item v centru není vyhrocená zápletka
	\item zobrazuje průměrné lidi a jejich zdánlivě všední život
	\item dějová stránka ustupuje do pozadí
	\item tragedie a dramatičnost je rozvedena ne příběhem, ale psychologií postav
	\item vše má svůj účel, řád, smysl -- "je-li na scéně puška, musí se z ní vystřelit"
	\item MCHAT -- Moskevké Umělecké Akademické Divadlo (režiséři Stanislavskij, Němirovič-Dančenko)
		\begin{itemize}
		\item ztvárňovalo mnoho Čechovových her
		\item Čechov nebyl příliš spokojený
		\end{itemize}
	\item Tři sestry, Racek, Strýček Váňa, Višňový sad
	\item komedie \textbf{Višňový sad}
		\begin{itemize}
		\item 1904, krátce před svou smrtí
		\item střet světa staré nepřizpůsobivé ruské šlechty a nové dravé generace, která se snaží nahlížet na život praktičtěji a využít možností, které se jim nabízejí
		\item odehrává se na panství kněžny \textbf{Raněvské}, která má dceru \textbf{Aňu} a bratra \textbf{Gajeva}
		\item kněžna po smrti syna odjela do Francie, nyní se vrací na zadlužené panství spravované \textbf{Varjou}, její nevlastní dcerou
		\item Varje se dvoří \textbf{Lopachin}, potomek generace nevolníků, člen mladé generace, která neuznává staré hodnoty 
		\item Lopachin navrhuje kněžně, aby rozparcelovala a prodala Višňový sad, ona ho ze sentimentality nechce prodat, prodá se celé panství a koupí ho Lopachin
		\end{itemize}
	\item povídka \textbf{Myslivec}
	\end{itemize}

\end{itemize}



\end{document}