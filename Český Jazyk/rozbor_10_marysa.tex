\documentclass[10pt,a4paper]{article}
\usepackage[utf8]{inputenc}
\usepackage[czech]{babel}
\usepackage{amsmath}
\usepackage{amsfonts}
\usepackage{amssymb}
\usepackage{chemfig}
\usepackage{geometry}
\usepackage{wrapfig}
\usepackage{graphicx}
\usepackage{floatflt}
\usepackage{hyperref}
\usepackage{fancyhdr}
\usepackage{tabularx}
\usepackage{makecell}
\usepackage{csquotes}
\usepackage{marginnote}

\MakeOuterQuote{"}

\renewcommand{\labelitemii}{$\circ$}
\renewcommand{\labelitemiii}{--}
\newcommand{\ra}{$\rightarrow$ }
\newcommand{\x}{$\times$ }
\newcommand{\lp}[2]{#1 -- #2}
\newcommand{\timeline}{\input{timeline}}


\geometry{lmargin = 0.8in, rmargin = 0.8in, tmargin = 0.8in, bmargin = 0.8in}
\newcommand{\n}[1]{\marginnote{\hspace{-0.6\textwidth}#1}}

\date{}
\author{Jakub Rádl}
\title{Alois, Vilém Mrštíkovi: Maryša}

\begin{document}
\maketitle

\section*{Tématická stránka díla}
\begin{itemize}
\item \textbf{literární druh a žánr}: realistické sociální drama, tragédie
	\begin{itemize}
	\item pochmurný děj, špatný konec
	\item rozebírá sociální problémy života na venkově
	\end{itemize}
\item \textbf{téma a motiv}: 
	\begin{itemize}
	\item \textbf{hlavní téma}: popis venkovského života v 19. století, postavení ženy na venkově
		\begin{itemize}
		\item poslušnost vůči rodičům, tradicím, víře
		\item upřednostňování peněz a zachování pověsti oproti blahu vlastních dětí
		\end{itemize}
	\item \textbf{další motivy v díle}:
		\begin{itemize}
		\item rozdělení vrstev podle majetnosti, sociální život
		\item donucený sňatek, nešťastné manželství, psychologie postav
		\item láska, hospoda, svatba, odvod na vojnu, otrava, domácí násilí na ženě, mlýn, peníze
		\end{itemize}
	\end{itemize}
\item \textbf{časoprostor}:
	\begin{itemize}
	\item stranou návsí nespecifikované moravské dědiny.
		\begin{itemize}
		\item v popiscích krojů jsou uvedeny vesnice
		\item smíchána různá nářečí
		\item inspirováno reálným příběhem
		\item[\ra] záměrně není určeno přesné místo děje, osoba kterou to bylo inspirováno
			\begin{itemize}
			\item dílo by mohlo být bráno jako pohoršující
			\item chtěli poukázat na obecnou problematiku venkova -- podobné věci se dějí všude
			\end{itemize}
		
		\end{itemize}
	\item v říjnu roku 1886. Neděle po mši -- začátek
	\item časový skok dvou let mezi 2. a 3. jednáním (Francek se vrátí z vojny)
	\end{itemize}
\item \textbf{zasazení výňatku do kontextu díla}:
	\begin{itemize}
	\item \textbf{časoprostor}:
		\begin{itemize}
		\item 5. jednání, 4. výstup
		\item u Vávrů doma
		\end{itemize}
	\item \textbf{obsah}: 
		\begin{itemize}
		\item Vávra se chce usmířit s Maryšou a začít smírné další manželství
		\item Maryša ho odmítne a řekne mu, aby dopil otrávené kafe
		\end{itemize}
	\end{itemize}
\item \textbf{kompoziční výstavba}
	\begin{itemize}
	\item kompozice chronologická
	\item dělení díla: 5 jednání dělených na výstupy
	\item extrémní množství scénických poznámek -- přesná představa jak má hra vypadat, nechtěli nechat prostor režii
	\item nedodržuje jednota místa času a děje
	\item antické dělení
		\begin{itemize}
		\item expozice -- úvod do děje, poznámky, Vávra se hádá o věno
		\item kolize -- Maryša Vávru nechce, nátlak rodičů
		\item krize -- Maryša si Vávru nakonec vezme, manželství nefunguje, Francek se vrátí z vojny, stále Maryšu miluje, chce aby s ním odjela do Brna, věřejně prohlásí, že za ní bude chodit
		\item peripetie -- několik možností vyřešení situace
			\begin{enumerate}
			\item Francek Maryši nabízí odchod do Brna
			\item Lízal nabízí Maryši vrácení se domů
			\item Vávra se pokouší o usmíření
			\end{enumerate}
		\item katastrofa -- Vávra umírá
		\end{itemize}
	\end{itemize}
\end{itemize}
\section*{Kompozice, postavy}
\begin{itemize}
\item \textbf{vypravěč} není \ra scénické poznámky
\item \textbf{vyprávěcí způsoby}:
	\begin{itemize}
	\item monology, dialogy (více)
	\item ich forma -- přímá řeč
	\end{itemize}
\item \textbf{veršová výstavba} -- verše jen v písních

\end{itemize}

\paragraph{Postavy}
\begin{itemize}
\item \textbf{Maryša} (dcera Lízalova)
	\begin{itemize}
	\item výrazná změna mezi šťastnou a plnou života před svatbou, od třetího jednání unavená, zamlklá, zničená dvěma lety manželství
	\item poslechla rodiče ze strachu vůči Bohu a veřejného mínění, finanční závislosti
	\item spor mezi poslušností (přijme sňatek, odmítne útěk) a rozhodnutím otrávit Vávru
		\begin{itemize}
		\item nejasný důvod změny -- všeho už měla dost, chtěla další den utéct s Franckem, Vávra Franckovi vyhrožoval smrtí, dochází k pocitu, že to je jediné řešení komplexní špatné situace
		\end{itemize}
	\end{itemize}
\item \textbf{Lízal} (otec Maryši)
	\begin{itemize}
	\item šedesátiletý sedlák
	\item hádá se s Vávrou o věnu, slíbil mu věno pokud se bud chovat slušně, nakonec mu je nedal \ra Vávra ho žaloval
	\item Lízal to ospravedlňuje tím, že je schopný s penězi lépe naložit
	\item na konci díla si uvědomuje, že udělal chybu
	\item před svatbou pochybuje o tom, zda je sňatek dobrý nápad, ale Lízalka ho přesvědčí
	\end{itemize}
\item \textbf{Lízalka} (matka Maryši)
	\begin{itemize}
	\item neměla žádné pochybnosti o sňatku, štěstí Maryši jí bylo ukradené, chtěla jen, aby si Maryša nevzala Francka
	\end{itemize}
\item \textbf{Vávra}
	\begin{itemize}
	\item špatný hospodář, zadlužený (půjčil si od Lízala), později alkoholik, domácí násilí, utýral první ženu
	\item 3 děti
	\item Maryšu si vzal kvůli věnu, výchově dětí, vedení domácnosti, sexu.
	\item za chyby viní Maryšu
	\item chtěl se s Maryšou na konci usmířit, těsně před tím než umírá
	\end{itemize}
\item \textbf{Francek}
	\begin{itemize}
	\item zamilovaný do Maryši
	\item přemýšlí především o sobě, dřív než o Maryše
	\end{itemize}
\item \textbf{Babička} -- soucítí s Maryšou, ale nemůže jí pomoct
\item \textbf{Rozára} -- Vávrova služebná
	\begin{itemize}
	\item snaží se Maryše pomáhat, protože nemá Vávru ráda
	\end{itemize}
\end{itemize}
\section*{Jazyk}
\begin{itemize}
\item smíchané tři moravské nářečí (zavříno, mó ženó, nemůže bét, ...)
\item vesnické výrazy, archaismy
\item mnoho scénických poznámek
\item tropy a figury ve výňatku
\end{itemize}
\section*{Literárně historický kontext}
\begin{itemize}
\item autoři českého realismu (až naturalismu)
	\begin{itemize}
	\item snaha zachytit reálný a komplexní obraz světa
	\item typizované postavy
	\item zachycuje společenské problémy
	\item spisovné vyjadřování vypravěče, postavy mluví jazykem své vrstvy
	\end{itemize}
\item 19. století 
	\begin{itemize}
	\item Alois Mrštík (1861--1925)	
		\begin{itemize}
		\item 
		\end{itemize}
	\item Vilém Mrštík (1863--1912)
		\begin{itemize}
		\item Jaorška -> nevystudoval práva -> bydlel s bratrem v Hustopečích
		\item téma sociálního života na vesnici
		\item román Santa Lucia
		\item Pohádka máje
		\end{itemize}
	\end{itemize}
\item další autoři realismu
	\begin{itemize}
	\item Cz.: Gabriela Preisová (Její pastorkyňa, Gazdina róba), Ladislav Stroupežnický (Naši furianti)
	\item Fr.: Balzac (Otec Goriot), Flaubert (Paní Bovaryová), Zola (Rougon-Macquartové), Maupassant (Miláček)
	\item Ang.: Charles Dickens (Oliver Twist, David Copperfield)
	\item Ru.: Tolstoj (Vojna a mír), Dostojevskij (Zločin a trest)
	\item USA: Twain (Dobrodružství Toma Sawyera)
	\item Pol.: Sienkiewicz (Quo Vadis)
	\end{itemize}
\end{itemize}
\section*{Zdroje}
\begin{itemize}
\item vlastní poznámky
\item MRŠTÍK, Alois, MRŠTÍK, Vilém. Maryša : drama v pěti jednáních [online]. V MKP 1. vyd. Praha : Městská knihovna v Praze, 2011 [cit. 2019-10-09]. Dostupné z WWW: \\ http://web2.mlp.cz/koweb/00/03/37/00/62/marysa.epub.
\end{itemize}


\newgeometry{rmargin=8cm}
\section*{Výňatek}
Výstup 4.  \\
VÁVROVÁ. – VÁVRA. \\
 \\
VÁVRA (vejde a věší pilku na hřebík). \n{scénická poznámka}\\
VÁVROVÁ (se \textbf{hřmotem} se probírá ve Ižičníku a pak nese Vávrovi hrnek se lžičkou). Chceš bílý nebo černý? Nesu ti černý. – \n{onomatopoie, elipsa (apoziopeze)}\\
VÁVRA. Černý mně dej.\\
VÁVROVÁ (staví na stůl před něho hrnek).\\
VÁVRA (mlčky sedne a položí čepici na stůl; ochotně). Sladila’s to?\\
VÁVROVÁ (sejme cukr s police, přinese na stůl a mlčky přisune k němu). Zapomněla sem. (Pak jde zase ke sporáku a sleduje stranou každý jeho pohyb).\\
VÁVRA (míchá lžičkou a srkne kávy). Od koho je káva? ‚\\
VÁVROVÁ. Od žida. \n{elipsa, anafora (od, od)} \\
VÁVRA. A proč ne ze spolku? \n{elipsa}\\
VÁVROVÁ. Měli zavříno.\\
VÁVRA. Tahle je ztuchlá, nebo co. (Rychle pije.) \n{elipsa}\\
VÁVROVÁ (jeví nepokoj a neví, jak a kde má stát. Ohlíží se stále po Vávrovi. S nápadným účastenstvím). Pojedeš na panský?\\
VÁVRA. Na panský. – Proč se ptáš? \n{elipsa}\\
VÁVROVÁ. Tak; aby dříví bylo suchý. Včera sme nemohly ani zatopit.\n{elipsa}\\
 \\
Oba mlčí.\\
 \\
VÁVRA (vstane a nerozhodně přistoupí k ženě, podívá se na ni, ale vida ji nevlídnou, vrátí se a pije. Přistoupí poznovu k ní. Vlídně). \textbf{Maryško}!\n{apostrofa}\\
VÁVROVÁ. Co chceš?\\
VÁVRA. Proč seš na mě taková zlá? Co nemůže bét u nás jinač? Ubyde ti, dyž se na mě vlídně podíváš? Poslechni! \textbf{Maryšo}! – Já na všecko zapomenu, nic ti nebudu vyčítat, nikomu nebudu hrozit, jen dyž trochu lepšů vůlu budu vidět u tebe. – Nevíš, jako to člověka bolí - - \n{apostrofa, řečnické otázky}\\
VÁVROVÁ (zakryje si rukama i zástěrou oči a tvář).\\
 \\
Ze dvora zvoní cepy.\\
 \\
VÁVRA. Tak sem ti sliboval, že až budeš mó ženo, všechno pro tebe udělám, co enom si budeš přát – jen abys byla u mně spokojena a šťastná. A zatím – podive se na mě – podive se na sebe. \textbf{Na hromech líháme, na hromech vstáváme.} – Je teho potřeba? Musí to bét? (Přivine ji vilně k sobě a chce ji líbat.) Co si \textbf{tvý srdce jenom žádá}, nic ti neodepřu. Všechno pro tebe udělám\textbf{•}. (Políbí ji.)\n{metafora, synekdocha, hyperbola}\\
VÁVROVÁ (polo dobrovolně, polo s nucením prohne se mu pod rukou a dá se políbit na tvář. Pojednou se zachvěje na celém těle, odpáčí mu ruku a s odporem uhne od něho).\\
\\	
VÁVRA (přemáhá zlost a dotkne se jejího ramena i doráží prosebně). Dyž ti pěkně prosím. Mám přece děti; co z nich bude, když budó vidět, jaké vedeme život?\\
VÁVROVÁ (rozhodí ruce a mírně jej odstrčí od sebe). Sedni a pi! (Jde pro hrnek.) Cos nedopil? (Postaví hrnek zpět.) \\
VÁVRA (vezme hrnek a dopije do dna).\\
 \\
Okamžik mlčení.\\
 \\
VÁVROVÁ. Které si vezmeš kožuch?\\
VÁVRA. Ten dlóhé mně přines.\\
VÁVROVÁ (jde do vedlejší světnice a ihned se vrátí s kožichem).\\
VÁVRA (oblékne kožich – dlouhý žlutý). Dohlídni na mlat a Rozára ať strace podestele novo slámu. (Odejde.)\\
 \\
Za scénou tepou cepy.\\
 \\
VÁVROVÁ (sama přemožena rozčilením zavrávorá – obepne lokte kolem hlavy a potácí se směrem, kde na zdi visí krucifix. Tiše stojí opřena o zeď a s hlavou vztyčenou leží nehybně na zdi). Sám temu chtěl.\\
  \\
Musí uplynouti delší chvíle, než cepy přestanou tepat, aby Vávrová měla dosti času vyhřátí všechny pocity, které se v ní ozývají této chvíle. Občas vzhlíží ke kříži. Když cepy přestaly tepat, \textbf{ozve se v ní strach}. Strašné ticho. Teprve až odehrála tento poslední pocit, který ji divoce zmítá, přiběhne Rozára. Cepy náhle přestanou tepat a ze dvora zaléhá sem hluk a ropot několika mužských i ženských hlasů. – Hluk roste a blíží se. \n{personifikace}\\
\\
\restoregeometry

\end{document}