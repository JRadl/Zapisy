\documentclass[10pt,a4paper]{article}
\usepackage[utf8]{inputenc}
\usepackage[czech]{babel}
\usepackage{amsmath}
\usepackage{amsfonts}
\usepackage{amssymb}
\usepackage{chemfig}
\usepackage{geometry}
\usepackage{wrapfig}
\usepackage{graphicx}
\usepackage{floatflt}
\usepackage{hyperref}
\usepackage{fancyhdr}
\usepackage{tabularx}
\usepackage{makecell}
\usepackage{csquotes}
\usepackage{marginnote}

\MakeOuterQuote{"}

\renewcommand{\labelitemii}{$\circ$}
\renewcommand{\labelitemiii}{--}
\newcommand{\ra}{$\rightarrow$ }
\newcommand{\x}{$\times$ }
\newcommand{\lp}[2]{#1 -- #2}
\newcommand{\timeline}{\input{timeline}}


\geometry{lmargin = 0.8in, rmargin = 0.8in, tmargin = 0.8in, bmargin = 0.8in}
\newcommand{\note}[1]{\marginnote{\hspace{-0.6\textwidth}#1}}

\date{}
\author{Jakub Rádl}
\title{Jan Neruda: Balady a Romance}

\begin{document}
\maketitle

\section*{Výňatek}
Bouř žene \textbf{koráb} u divokém běhu. \note{archaismus} \\ 
John lampu \textbf{klamnou} k skále přivěsil \note{epiteton, melancholická eufonie} \\
a dí: „Bůh žehnej břehu!“\\
 \\
A koráb k světlu žene se a v trysku\\
se náhle přes úskalí překotil,\\
a stěžněm vězí v písku.\\
 \\
John zavejsknul si ve syčící pěnu:\\
„Má dceruška si chystá veselku,\\
dnes pomohu jí k věnu!“\\
 \\
A člunek jeho jako liška běží \note{přirovnání} \\
po těžkých vlnách tam, kde zvrhlá loď\\
jak černá rakev leží.\\
 \\
John nenavykl marně tratit času,\\
svou sekyru hned v koráb zarazil,\\
v tom z nitra doslech‘ hlasu.\\
 \\
„„Jen \textbf{pospěš, pospěš}!““ zní to dutě zdůli, \note{epizeuxis} \\
„„a všeho zboží půli dostaneš,\\
i všeho zlata půli.““\\
 \\
John naslouchá a \textbf{vytřešťuje zraky} – \note{hyperbola}\\
„Aj \textbf{pakli} jedna půle bude má, \note{archaismus} \\
toť bude druhá taky!““\\
 \\
A rychle s člunkem ku břehu uhání.\\
Po celou noc se k lodi nevrátil,\\
až teprv po svítání.\\
 \\
Až po svítání, v bílé ranní době\\
zas sekyru svou v koráb zaráží,\\
a uvnitř již jak v hrobě.\\
 \\
Již voda otvorem si cestu \textbf{klestí}, \note{personifikace}\\
teď vyhoupla si první mrtvolu,\\
John rychle po ní pěstí. \note{elipsa}\\
 \\
\textbf{Tvář} mrtvou k sobě obrátil: „Eh kletě! \note{synekdocha}\\
je po svatbě – já tady za vlasy\\
mrtvého držím zetě!“\\
\\

\newpage
\section*{Tématická stránka díla}
\begin{itemize}
\item \textbf{literární druh a žánr}:
	\begin{itemize}
	\item básnická sbírka
	\item žánry: \textbf{balady} a \textbf{romance}, některé nazvané naopak
	\end{itemize}
\item \textbf{téma a motiv}:
	\begin{itemize}
	\item \textbf{hlavní téma}: ústřední téma není
	\item \textbf{další motivy v díle}: náboženské, historické, přírodní, o smrti
		\begin{itemize}
		\item Ballada pašijová -- Satan si stěžuje, že Ježíšova smrt není dostatečné k vykoupení, trpění matky je největší trest
		\item Balada horská -- (není balada), holčička vyléčí bylinkami Kristovy rány
		\item Balada dětská -- smrt nemocného dítěte
		\item Balada česká -- (romance) rytíř Rek (rádce Jiřího z Poděbrad) každý rok na osm dní ožije
		\item Romance o černém jezeře -- (balada) přemítá o zapomínání národní minulosti, hrdinů, pocit tíhy při pohledu jezera, v poslední sloce (podle interpretace) touha poznat hlubiny / utopit se
		\item Romance o Karlu IV. -- o povaze českého národa, přirovnávána k trpkému vínu, časem mu přišel na chuť
		\item Romance o jaře 1848 -- popisuje změnu myšlení lidí, šíření ideálů skrze společnost, lidé se konečně stali lidmi
		\item Romance italská -- (balada) skutečná historická postava, mnich odsouzen k popravě, 
		\item Romance helgolandská -- ; John vykrade loď, aby měl věno pro svou dceru, na lodi umírá jeho zeť (balada) 
		\item Balada zimní -- čaroděj prochází kolem šibeníce, kde visí tři zloději, oživí jejich mrtvoly, oni ukradnou chleba, víno, polštáře chudým lidem, čaroděj je pochválil
			\begin{itemize}
			\item na šibenici nakonec visí všichni
			\item trest za špatné činy
			\item nevyužití druhé šance
			\item motiv nadpřirozena
			\end{itemize}
		\item Balada stará -- (sociální balada) mlynář byl otcem dítěte, matka spáchala sebevraždu pod mlýnským kolem, aby zpytoval svědomí
		\item balada tříkrálová -- tři králové se jdou poklonit Ježíškovi, až bude velký, tak se budou bát a nezabrání ukřižování, poukazuje na pokrytectví
		\item Balada štědrovečerní -- Petr usne a přesune se k narození Ježíše, žárlí, že ho políbí dívka, která se mu líbí.
		\item Balada májová -- (romance) dívka se modlí, aby jí Panna Petronilla dala kluka, touží po lásce
		\item Balada rajská -- (romance) svatá Alžběta, patronka věrných žen, si stěžuje, že nemá koho chránit
		\item Balada o duši Karla Borovského -- (romance) Karla Borovského nechtějí pustit do nebe, pustí ho, když se pomodlí k svatém Janu z Nepomuku, aby nezanikl český jazyk
		\item Balada o polce -- (romance) personifikace polky, roztancování vesnice, radost z tance
		\item Balada o svatbě v Kannaán -- na svatbě dojde víno, Ježíš přemění vodu na víno, aby se mohli dál veselit; přání, aby Ježíš žil normální život – Marie je smutná neb má předtuchu, že to tak nebude
		\end{itemize}
	\end{itemize}
\item \textbf{časoprostor}: vesnice, přiroda, město
\item \textbf{zasazení výňatku do kontextu díla}: romance helgolandská
	\begin{itemize}
	\item \textbf{časoprostor}: neurčitý, 9. skladba
	\item \textbf{obsah}: John vykrade loď, aby měl věno pro svou dceru, na lodi umírá jeho zeť
	\end{itemize}
\item \textbf{kompoziční výstavba}
	\begin{itemize}
	\item chronologická kompozice
	\item dělení díla -- 12 balad, 6 romancí
	\end{itemize}
\end{itemize}
\section*{Kompozice, postavy}
\begin{itemize}
\item vypravěč
\item vyprávěcí způsob -- er forma
\item \textbf{veršová výstavba}:	
	\begin{itemize}
	\item romance helgolandská
		\begin{itemize}
		\item verš vázaný -- 11, 10, 7
		\item rým obkročný ABA
		\item mnoho inverzí, archaismy
		\item postavy
		\end{itemize}
		
	\end{itemize}
\end{itemize}

\paragraph{Postavy}
\begin{itemize}
\item romance helgolandská
	\begin{itemize}
	\item John -- kriminálník, nechá nabourat loď, aby získal peníze
	\item zeť, dcera
	\end{itemize}
\end{itemize}

			

\section*{Jazyk}
\begin{itemize}
\item archaismy, lidové výrazy (zavejsknul, eh Kletě, …) přesahy, inverze
\end{itemize}


\section*{Literárně historický kontext}
\begin{itemize}
\item česká literatura 2. poloviny 19. století, Jan Neruda (1834--1891)
\item řadí se k Májovcům
	\begin{itemize}
	\item cílem povznést kvalitativně českou literaturu na úroveň světové literatury, inspirováni světovými romantiky (Poe, Heine, Hugo) \ra překládání
	\item nová témata -- sociální problematika, postavení žen ve společnosti, věda; historickým tématům se vyhýbali
	\item Vítězslav Hálek (povídka Muzikantská Liduška, sbírka poezie V přírodě, drama Král Vukašín)
	\item Karolína Světlá (romány Kříž u potoka, Frantina, Vesnický román)
	\item Jakub Arbes (romaneta Newtonův mozek, Etiopská lilie, Svatý Xaverius)
	\end{itemize}
\item Nerudova tvorba
	\begin{itemize}
	\item z počátku pesimistická, kvůli životním osudům (smrt kamaráda, nešťastná láska), později se mění
	\item sbírky poezie Hřbitovní kvítí, Písně kosmické, Balady a Romance, Prosté motivy, Zpěvy páteční
	\item sbírky povídek Povídky malostranské, Arabesky
	\end{itemize}
\item v tomto období působí též Ruchovci
	\begin{itemize}
	\item škola národní -- cílem literatury je vychovávání národa, umění má sloužit aktuální společenské potřebě
	\item lidová poezie, slovesnost, vlastenectví
	\item Svatopluk Čech (sbírka Písně otroka, cyklus povídek Ve stínu lípy, epos Hanuman, Broučkiády)
	\item Eliška Krásnohorská 
	\end{itemize}
\item Lumírovci
	\begin{itemize}
	\item škola kosmopolitní -- pozvednout českou literaturu na světovou úroveň
	\item umění pro umění -- hlavním cílem je estetická funkce, dokonalá forma, poutavý příběh, nové žánry
	\item Josef Václav Sládek (překlady, sbírky Zlatý máj, Jiskry na moři, Selské písně, České znělky )
	\item Julius Zeyer (Román o věrném přátelství Amise a Amila)
	\item Jaroslav Vrchlický (básnický cyklus Zlomky epopeje, komedie Noc na Karlštejně)
	\end{itemize}
\end{itemize}
\section*{Zdroje}
\begin{itemize}
\item NERUDA, Jan. Balady a romance [online]. V MKP 1. vyd. Praha : Městská knihovna v Praze, 2011 [2019-11-05]. Dostupné z WWW:\\
http://web2.mlp.cz/koweb/00/03/65/85/53/balady\_a\_romance.epub.
\item poznámky z hodin
\end{itemize}
\end{document}
