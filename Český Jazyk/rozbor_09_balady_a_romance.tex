\documentclass[10pt,a4paper]{article}
\usepackage[utf8]{inputenc}
\usepackage[czech]{babel}
\usepackage{amsmath}
\usepackage{amsfonts}
\usepackage{amssymb}
\usepackage{chemfig}
\usepackage{geometry}
\usepackage{wrapfig}
\usepackage{graphicx}
\usepackage{floatflt}
\usepackage{hyperref}
\usepackage{fancyhdr}
\usepackage{tabularx}
\usepackage{makecell}
\usepackage{csquotes}
\usepackage{marginnote}

\MakeOuterQuote{"}

\renewcommand{\labelitemii}{$\circ$}
\renewcommand{\labelitemiii}{--}
\newcommand{\ra}{$\rightarrow$ }
\newcommand{\x}{$\times$ }
\newcommand{\lp}[2]{#1 -- #2}
\newcommand{\timeline}{\input{timeline}}


\geometry{lmargin = 0.8in, rmargin = 0.8in, tmargin = 0.8in, bmargin = 0.8in}
\newcommand{\note}[1]{\marginnote{\hspace{-0.6\textwidth}#1}}

\date{}
\author{Jakub Rádl}
\title{Jan Neruda: Balady a Romance}

\begin{document}
\maketitle

\section*{Výňatek}
Bouř žene \textbf{koráb} u divokém běhu. \note{archaismus} \\ 
John lampu \textbf{klamnou} k skále přivěsil \note{epiteton, melancholická eufonie} \\
a dí: „Bůh žehnej břehu!“\\
 \\
A koráb k světlu žene se a v trysku\\
se náhle přes úskalí překotil,\\
a stěžněm vězí v písku.\\
 \\
John zavejsknul si ve syčící pěnu:\\
„Má dceruška si chystá veselku,\\
dnes pomohu jí k věnu!“\\
 \\
A člunek jeho jako liška běží \note{přirovnání} \\
po těžkých vlnách tam, kde zvrhlá loď\\
jak černá rakev leží.\\
 \\
John nenavykl marně tratit času,\\
svou sekyru hned v koráb zarazil,\\
v tom z nitra doslech‘ hlasu.\\
 \\
„„Jen \textbf{pospěš, pospěš}!““ zní to dutě zdůli, \note{epizeuxis} \\
„„a všeho zboží půli dostaneš,\\
i všeho zlata půli.““\\
 \\
John naslouchá a \textbf{vytřešťuje zraky} – \note{hyperbola}\\
„Aj \textbf{pakli} jedna půle bude má, \note{archaismus} \\
toť bude druhá taky!““\\
 \\
A rychle s člunkem ku břehu uhání.\\
Po celou noc se k lodi nevrátil,\\
až teprv po svítání.\\
 \\
Až po svítání, v bílé ranní době\\
zas sekyru svou v koráb zaráží,\\
a uvnitř již jak v hrobě.\\
 \\
Již voda otvorem si cestu \textbf{klestí}, \note{personifikace}\\
teď vyhoupla si první mrtvolu,\\
John rychle po ní pěstí. \note{elipsa}\\
 \\
\textbf{Tvář} mrtvou k sobě obrátil: „Eh kletě! \note{synekdocha}\\
je po svatbě – já tady za vlasy\\
mrtvého držím zetě!“\\
\\
\section*{Tématická stránka díla}
\begin{itemize}
\item \textbf{literární druh a žánr}:
	\begin{itemize}
	\item básnická sbírka
	\item žánry: \textbf{balady} a \textbf{romance}, některé nazvané naopak
	\end{itemize}
\item \textbf{téma a motiv}:
	\begin{itemize}
	\item \textbf{hlavní téma}: ústřední téma není
	\item \textbf{další motivy v díle}: náboženské, historické, přírodní, o smrti
		\begin{itemize}
		\item Ballada pašijová -- Satan si stěžuje, že Ježíšova smrt není dostatečné k vykoupení, trpění matky je největší trest
		\item Balada horská -- (není balada), holčička vyléčí bylinkami Kristovy rány
		\item Balada dětská -- smrt nemocného dítěte
		\item Balada česká -- (romance) rytíř Rek (rádce Jiřího z Poděbrad) každý rok na osm dní ožije
		\item Romance o černém jezeře -- (balada) přemítá o zapomínání národní minulosti, hrdinů, pocit tíhy při pohledu jezera, v poslední sloce (podle interpretace) touha poznat hlubiny / utopit se
		\item Romance o Karlu IV. -- o povaze českého národa, přirovnávána k trpkému vínu, časem mu přišel na chuť
		\item Romance o jaře 1848 -- popisuje změnu myšlení lidí, šíření ideálů skrze společnost, lidé se konečně stali lidmi
		\item Romance italská -- (balada) skutečná historická postava, mnich odsouzen k popravě, 
		\item Romance helgolandská -- (balada)
		\item Balada zimní -- čaroděj prochází kolem šibeníce, kde visí tři zloději, oživí jejich mrtvoly, oni ukradnou chleba, víno, polštáře chudým lidem, čaroděj je pochválil
			\begin{itemize}
			\item na šibenici nakonec visí všichni
			\item trest za špatné činy
			\item nevyužití druhé šance
			\item motiv nadpřirozena
			\end{itemize}
		\item balada stará -- (sociální balada) mlynář byl otcem dítěte, matka spáchala sebevraždu pod mlýnským kolem, aby zpytoval svědomí
		\item balada tříkrálová -- tři králové se jdou poklonit Ježíškovi, až bude velký, tak se budou bát a nezabrání ukřižování, poukazuje na pokrytectví
		\end{itemize}
	\end{itemize}
\item \textbf{časoprostor}:
\item \textbf{zasazení výňatku do kontextu díla}:
	\begin{itemize}
	\item \textbf{časoprostor}:
	\item \textbf{obsah}: 
	\end{itemize}
\item \textbf{kompoziční výstavba}
	\begin{itemize}
	\item využity kompozice: 
	\item dělení díla
	\item jednota místa času a děje?
	\item antické dělení?
	\end{itemize}
\end{itemize}
\section*{Kompozice, postavy}
\begin{itemize}
\item vypravěč / lyrický subjekt:
\item vyprávěcí způsoby:
	\begin{itemize}
	\item 
	\item
	\end{itemize}
\item \textbf{veršová výstavba}:	
	\begin{itemize}
	\item romance helgolandská
		\begin{itemize}
		\item verš vázaný -- 11, 10, 7
		\item rým obkročný ABA
		\item mnoho inverzí, archaismy
		\item postavy
			\begin{itemize}
			\item John -- kriminálník, nechá nabourat loď, aby získal peníze
			\item zeť
			\item dcera
			\end{itemize}
		\end{itemize}
		
	\end{itemize}
\end{itemize}

\paragraph{Postavy}
\begin{itemize}
\item \textbf{Harpagon} --
\item \textbf{Kleant} --
\item \textbf{Elisa} --
\end{itemize}
\section*{Jazyk}
\begin{itemize}
\item jazykové prostředky a jejich funkce ve výňatku
\item tropy a figury ve výňatku
\end{itemize}
\section*{Literárně historický kontext}
\begin{itemize}
\item současní autoři:
\item další autorova díla:
\end{itemize}
\section*{Zdroje}
\end{document}
