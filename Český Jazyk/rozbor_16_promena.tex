\documentclass[10pt,a4paper]{article}
\usepackage[utf8]{inputenc}
\usepackage[czech]{babel}
\usepackage{amsmath}
\usepackage{amsfonts}
\usepackage{amssymb}
\usepackage{chemfig}
\usepackage{geometry}
\usepackage{wrapfig}
\usepackage{graphicx}
\usepackage{floatflt}
\usepackage{hyperref}
\usepackage{fancyhdr}
\usepackage{tabularx}
\usepackage{makecell}
\usepackage{csquotes}
\usepackage{marginnote}

\MakeOuterQuote{"}

\renewcommand{\labelitemii}{$\circ$}
\renewcommand{\labelitemiii}{--}
\newcommand{\ra}{$\rightarrow$ }
\newcommand{\x}{$\times$ }
\newcommand{\lp}[2]{#1 -- #2}
\newcommand{\timeline}{\input{timeline}}


\geometry{lmargin = 0.8in, rmargin = 0.8in, tmargin = 0.8in, bmargin = 0.8in}
\newcommand{\note}[1]{\marginnote{\hspace{-0.6\textwidth}#1}}

\date{}
\author{Jakub Rádl}
\title{Franz Kafka: Proměna -- Rozbor díla}

\begin{document}
\maketitle

\section*{Výňatek}
"Milí rodiče," řekla sestra a úvodem uhodila rukou do stolu, "takhle to dál nejde. Jestli vy to snad nechápete, já to chápu. Nechci před touto \textbf{obludou}\footnote{expresivní vyjádření} vyslovovat jméno svého bratra a řeknu tedy jen: musíme se jí pokusit zbavit. Zkusili jsme \textbf{vše, co je v lidských silách}\footnote{automatismus}, abychom se o ni starali a trpělivě ji snášeli, myslím, že nám nikdo nemůže ani to nejmenší vytknout." "Má \textbf{tisíckrát}\footnote{hyperbola} pravdu," řekl si otec pro sebe. Matka, která pořád ještě nemohla popadnout dech, se s pomateným výrazem v očích tlumeně rozkašlalá do dlaně. 

Sestra hned běžela k matce a položila jí ruku na čelo. Otce přivedla zřejmě sestřina slova na určitější myšlenky, napřímil se na židli, pohrával si se svou sluhovskou čepicí mezi talíři, které zůstaly na stole ještě od večeře pánů nájemníků, a chvílemi se podíval po tichém Řehořovi. 

"Musíme se \textbf{toho}\footnote{označení Řehoře za to} hledět zbavit," řekla teď sestra výslovně otci, neboť matka pro kašel neslyšela, "ještě vás oba umoří, vidím to už. Když jednou člověk musí tak těžce pracovat jako my všichni, nemůže mít přece doma tohle věčné soužení. Já už to také nevydržím." A rozplakala se tak usedavě, že jí slzy kanuly dolů na matčinu tvář, z níž je stírala mechanickými pohyby ruky. 

"Milé dítě," řekl otec soucitně a s nápadným porozuměním, "co ale máme dělat?" 

Sestra jen pokrčila rameny na znamení bezradnosti, jíž teď \textbf{plačíc}\footnote{přechodník} propadla přes všechnu dřívější jistotu. 

"Kdyby nám rozuměl," řekl otec napolo tázavě; sestra v slzách prudce zatřepala rukou naznačujíc, že to nepřichází v úvahu. 

"Kdyby nám rozuměl," opakoval otec a zamhouřením očí přijal sestřino přesvědčení, že je to nemožné, "bylo by snad možné nějak se s ním dohodnout. Ale takhle -" 

"Pryč musí," zvolala sestra, "to je jediný prostředek, tatínku. Musíš jen přestat myslet na to, že je to Řehoř.: Vždyť naše neštěstí je vlastně v tom, že jsme tomu tak dlouho věřili. Ale jakpak by to mohl být Řehoř?\footnote{řečnická otázka} Kdyby to byl Řehoř, dávno by už uznal, že lidé nemohou žít pohromadě s takovým zvířetem, a byl by dobrovolně odešel. Neměli bychom pak bratra, ale mohli bychom dál žít a chovat v úctě jeho památku. Takhle nás ale to zvíře pronásleduje, vypudí pány nájemníky, chystá se zřejmě zabrat celý byt a nás nechat nocovat na ulici. Podívej se, tatínku," vyřkla najednou, "už zase začíná!" A v hrůze, Řehořovi docela nepochopitelné, opustila sestra dokonce i matku, doslova se odrazila od její židle, jako by raději chtěla matku obětovat než zůstat v Řehořově blízkosti, běžela se schovat za otce, který, rozčilen toliko jejím počínáním, rovněž vstal a napolo před sestrou zdvihl ruce, jako by ji chtěl chránit.

\newpage
\section*{Tématická stránka díla}
\begin{itemize}
\item \textbf{literární druh a žánr}: 
	\begin{itemize}
	\item epika
	\item novela -- jedna hlavní dějová linka 
		\begin{itemize}
		\item pointa -- rodina se snaží problémy projít, nakonec ale otec Řehoře zabije a celá rodina je ve výsledku ráda
		\end{itemize}
	\end{itemize}
\item \textbf{téma a motiv}:
	\begin{itemize}
	\item \textbf{hlavní téma}: gradace rodinných vztahů do extrému po té co se jeden člen promění v brouka (stane tělesně postiženým)
	\item \textbf{další motivy v díle}:
		\begin{itemize}
		\item přeměna v brouka symbolizuje ochrnutí člena rodiny (dovedeno do absurdity)
		\item rodinné vztahy jsou špatně nastavené od začátku, situace je jen dohání do extrémů
		\item jablko, obraz v Řehořově pokoji, zástěra zakrývající Řehoře
		\end{itemize}
	\end{itemize}
\item \textbf{časoprostor}:
	\begin{itemize}
	\item počátek 20. století (Kafkova současnost)
	\item celý příběh se odehrává v bytě Samsových
	\end{itemize}
\item \textbf{zasazení výňatku do kontextu díla}:
	\begin{itemize}
	\item \textbf{časoprostor}: 3. kapitola, ke konci díla, doma jako celé dílo
	\item \textbf{obsah}: 
		\begin{itemize}
		\item Markétka se zlomí a prohlásí, že je nutné se Řehoře zbavit
		\item označuje ho za "to"
		\item 
		\end{itemize}
	\end{itemize}

\section*{Kompozice, postavy}
\item \textbf{kompoziční výstavba}
	\begin{itemize}
	\item in medias res
	\item chronologická kompozice
	\item dílo je rozděleno na 3 kapitoly
		\begin{itemize}
		\item v první kapitole proměna Řehoře a dozvědění se o ní rodiny
		\item v druhé reakce na proměnu a starání se o Řehoře
		\item ve třetí kapitole se vztahy zhoršují, rodina se chce Řehoře zbavit, Řehoř umírá
		\end{itemize}
	\end{itemize}
\item nezávislý objektivní vnější vypravěč, er forma
\item druhy promluv
	\begin{itemize}
	\item polopřímé řeči -- přemýšlení postav ve třetí osobě (Řehořovy myšlenky)
	\item přímá řeč -- dialogy mezi členy rodiny
	\end{itemize}
\end{itemize}

\paragraph{Postavy}
\begin{itemize}
\item \textbf{Řehoř Samsa}
	\begin{itemize}
	\item dříve cestovní obchodník, před proměnou zabezpečuje finančně celou rodinu, na sebe vůbec nemyslí
	\item má pocit, že se bez něj rodina neobejde, že musí řešit všechny problémy
	\item rodina se ani nesnaží zjisti, jestli jim Řehoř rozumí
	\item na konci jídla umírá z fyzických i psychických důvodů (přestává naschvál jíst)
	\end{itemize}
\item \textbf{Matka}
	\begin{itemize}
	\item tvrdí, že Řehoře miluje
	\item nebaví se s Řehořem, štítí se ho, bojí se ho, lituje ho zpovzdálí
	\end{itemize}
\item \textbf{Otec}
	\begin{itemize}
	\item před proměnou pohodlný člověk, nechodil do práce
	\item je k Řehořovi krutý, protože kvůli němu musí zpět do práce, nakonec Řehoře zabije jablkem
	\end{itemize}
\item \textbf{Markétka} (sestra)
	\begin{itemize}
	\item 17 let
	\item nejdříve se snaží o bratra starat, později říká, že by se ho rodina měla zbavit, protože na ní rodiče nechají všechny starosti o bratra a ona už to nezvládá
	
	\end{itemize}
\end{itemize}
\section*{Jazyk}
\begin{itemize}
\item jednoduchý jazyk, objektivní, nezabarvené jazykové prostředky, spisovná čeština
\item nevyskytuje se příliš mnoho uměleckých prvků (kontrast s metaforickým tématem)
\item dlouhá detailní souvětí, detailní popisy -- mnoho přídavných jmen
\end{itemize}
\section*{Literárně historický kontext}
\begin{itemize}
\item pro Kafku typická absurdita situace, se kterou si postavy neví rady (Řehoř jako první myslí na to, jak půjde do práce)
\item pražská německá literatura
\item další autorova díla: 
	\begin{itemize}
	\item povídka Před zákonem
	\item romány Proces, Zámek, Amerika
	\end{itemize}
\item současní autoři:
	\begin{itemize}
	\item Gustav Meyrink -- román Golem (židovské gheto)
	\item Max Brod -- román Hlídka
	\item Franz Werfel -- román Čtyřicet dnů
	\item Thomas Man  -- novela Smrt v Benátkách
	\item Heinrich Man -- román Poddaný
	\item Hermann Hesse -- novela Stepní vlk	
	\end{itemize}
\end{itemize}
\section*{Zdroje}
\end{document}