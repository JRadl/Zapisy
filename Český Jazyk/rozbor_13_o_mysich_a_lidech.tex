\documentclass[10pt,a4paper]{article}
\usepackage[utf8]{inputenc}
\usepackage[czech]{babel}
\usepackage{amsmath}
\usepackage{amsfonts}
\usepackage{amssymb}
\usepackage{chemfig}
\usepackage{geometry}
\usepackage{wrapfig}
\usepackage{graphicx}
\usepackage{floatflt}
\usepackage{hyperref}
\usepackage{fancyhdr}
\usepackage{tabularx}
\usepackage{makecell}
\usepackage{csquotes}
\usepackage{marginnote}

\MakeOuterQuote{"}

\renewcommand{\labelitemii}{$\circ$}
\renewcommand{\labelitemiii}{--}
\newcommand{\ra}{$\rightarrow$ }
\newcommand{\x}{$\times$ }
\newcommand{\lp}[2]{#1 -- #2}
\newcommand{\timeline}{\input{timeline}}


\geometry{lmargin = 0.8in, rmargin = 0.8in, tmargin = 0.8in, bmargin = 0.8in}
\newcommand{\note}[1]{\marginnote{\hspace{-0.6\textwidth}#1}}

\date{}
\author{Jakub Rádl}
\title{John Steinbeck: O myších a lidech -- Rozbor díla}

\begin{document}
\maketitle

\section*{Výňatek}

\section*{Tématická stránka díla}
\begin{itemize}
\item \textbf{literární druh a žánr}: epika, novela
	\begin{itemize}
	\item novela -- kratší vyprávění s jednou dějovou linií (rozdíl proti románu) a zvratem a pointou na konci
	\end{itemize}
\item \textbf{děj}:
	\begin{itemize}
	\item Lennie a George spolu putují po rančích za prací, aby si vydělali na svůj vlastní statek. Museli utéct z Weedu, kde naposled pracovali, protože Lennie byl obviněn ze znásilnění. Dojdou na ranč, kde jsou zaměstnáni. Zde se domluví s Candym, že se s nimi spojí, protože má nějaké peníze a koupí si statek dohromady. Lennie omylem ale zabije ženu statkářova syna a musí utéct. Chlapi ho jdou najít a zastřelit. George ho zastřelí sám, aby nemusel trpět.
	\end{itemize}
\item \textbf{téma a motiv}:
	\begin{itemize}
	\item \textbf{hlavní téma}: 
		\begin{itemize}
		\item Cesta za naplněním snu, která se náhle zhroutí, když už je na dosah ruky. 
		\item Společenské poměry na americkém venkově ve 30. letech 19. století.
		\end{itemize}			
	\item \textbf{další témata a motivy v díle}:
		\begin{itemize}
		\item paralela zastřelení Candyho psa a Lennieho, oba byli zabiti střelou do týla, aby nemuseli trpět
		\item osamění Curleyho ženy protože se s ní nikdo nechce bavit, Candyho bez svého psa, George a Lennieho, kteří spolu jsou, aby nebyli sami, Crookse izolovaného od ostatních dělníků ("Soledad" znamená španělsky "samota")
		\item přátelství mezi Georgem a Lenniem
		\item vysněný dům s králíky
		\item diskriminace černochů
		\end{itemize}
	\end{itemize}
\item \textbf{časoprostor}: 
	\begin{itemize}
	\item čas není přesně dán, pravděpodobně v době vydání 1937
	\item prostor -- na cestě na ranč a na ranči u města Soledad v Kalifornii
	\end{itemize}

\end{itemize}
\section*{Kompozice, postavy}
\begin{itemize}
\item dílo je vyprávěno v er-formě, vypravěč je nezaujatý a nezúčastněný v ději
\item typy promluv -- přímá řeč
\item \textbf{kompoziční výstavba}
	\begin{itemize}
	\item chronologická kompozice s občasnými retrospektivními prvky z historie George a Lennieho 
	\item dílo je děleno na 6 kapitol
	\end{itemize}
\end{itemize}

\paragraph{Postavy}
\begin{itemize}
\item \textbf{Lennie Small}
	\begin{itemize}
	\item obrovský, silný chlap
	\item mentálně postižený, je hodný, ale nedokáže ovládat své chování, zapomíná věci
	\item ve Weedu, odkud s Georgem putují sáhl nějakému děvčeti na šaty a pak se zarazil a nemohl je pustit, byl obviněn ze znásilnění a museli utéct
	\item je fixovaný na vidinu, že budou jednoho dne bydlet na svém statku a bude tam mít své králíky
	\item po cestě si našel myš na mazlení se, ale omylem ji zabil, později udělal to samé se štěnětem na ranči a stejně omylem zabil Curleyho ženu
	\end{itemize}
\item \textbf{George Milton}
	\begin{itemize}
	\item hodný, chytrý
	\item putuje s Lenniem po statcích za prací
	\item stará se o Lennieho, snaží se ho udržet mimo problémy
	\item dříve si z Lennieho hlouposti dělal srandu, ale později si uvědomil, že Lennie to nechápe a když se kvůli tomu Lennie málem utopil, nechal toho
	\item na konci Lennieho zastřelý ranou do týla, aby nemusel trpět od Curleyho
	\item Lennieho teta ho požádala, aby se o něj staral
	\end{itemize}
\item \textbf{Candy}
	\begin{itemize}
	\item postarší uklízeč, přišel o ruku při nehodě
	\item měl starého nemohoucího psa, kterého Carlson zastřelí, protože prý už ze světa stejně nic nemá a hrozně smrdí
	\item nabídne, že přihodí svých 350 dolarů a koupí statek s Lenniem a Georgem
	\end{itemize}
\item \textbf{Slim}
	\begin{itemize}
	\item kočí, pohledný, silný člověk
	\end{itemize}
\item \textbf{Curley}
	\begin{itemize}
	\item statkářův syn, bývalý boxer, dovoluje si na všechny okolo
	\item obzvláště má problém s Lenniem, zaútočí na něj a Lennie mu rozdrtí ruku
	\end{itemize}
\item \textbf{Curleyho žena}
	\begin{itemize}
	\item pohledná, mladá žena, není s Curleym šťastná, představovala si pro sebe jiný život
	\item flirtuje s muži na ranči, oni ji proto považují za "kurvu"
	\end{itemize}
\item \textbf{Crooks}
	\begin{itemize}
	\item černoch, koňák
	\item bydlí sám ve světnici u koní, vzájemně se nemá rád s ostatními, ale spřátelí se s Lenniem
	\item jméno Crooks vychází z jeho zkřivených zad, většinou se o něm mluví jako o "tom černým"
	\end{itemize}
\item \textbf{Majitel statku}
	\begin{itemize}
	\item přísný, umí se naštvat, ale jinak hodný na své zaměstnance
	\end{itemize}

\end{itemize}
\section*{Jazyk}
\begin{itemize}
\item překlad -- Vladimír Vendyš, Blanka Knotková-Čapková
\item překlad je psán spisovnou srozumitelnou češtinou, rozhovory postav jsou nespisovné a obsahují občas i vulgarismy
\end{itemize}
\section*{Literárně historický kontext}
\begin{itemize}
\item kniha byla často cenzurována kvůli vulgarismům a rasismu
\item John Steinbeck je autor, prozaik, dramatik americké meziválečné literatury
\item jeho současníci popisující život na americkém venkově v tomto období jsou například F. Scott Fidzgerald (Velký Gatsby) a William Faulkner (Jdi tam, Mojžíši)
\item další autorova díla:
	\begin{itemize}
	\item Pláň Tortilla
	\item Na plechárně
	\item Na východ od ráje
	\end{itemize}
\end{itemize}
\section*{Zdroje}
\begin{itemize}
\item STEINBECK, John. O myších a lidech: Plameny zářivé. 3. vyd. (O myších a lidech), 1. vyd. (Plameny zářivé), 1. vyd. v Nakl. Svoboda. Přeložil Vladimír VENDYŠ, přeložil Blanka KNOTKOVÁ-ČAPKOVÁ. Praha: Svoboda, 1994. ISBN 80-205-0409-5.
\item https://rozbor-dila.cz/o-mysich-a-lidech-rozbor-dila-k-maturite/
\item https://en.wikipedia.org/wiki/Of\_Mice\_and\_Men
\end{itemize}
\end{document}