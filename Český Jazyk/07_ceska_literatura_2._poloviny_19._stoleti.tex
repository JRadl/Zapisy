\title{Česká literatura druhé poloviny devatenáctého století}
\documentclass[10pt,a4paper]{article}
\usepackage[utf8]{inputenc}
\usepackage[czech]{babel}
\usepackage{amsmath}
\usepackage{amsfonts}
\usepackage{amssymb}
\usepackage{chemfig}
\usepackage{geometry}
\usepackage{wrapfig}
\usepackage{graphicx}
\usepackage{floatflt}
\usepackage{hyperref}
\usepackage{fancyhdr}
\usepackage{tabularx}
\usepackage{makecell}
\usepackage{csquotes}
\usepackage{footnote}
\usepackage{movie15}
\MakeOuterQuote{"}

\renewcommand{\labelitemii}{$\circ$}
\renewcommand{\labelitemiii}{--}
\newcommand{\ra}{$\rightarrow$ }
\newcommand{\x}{$\times$ }
\newcommand{\lp}[2]{#1 -- #2}
\newcommand{\timeline}{\input{timeline}}


\geometry{lmargin = 0.8in, rmargin = 0.8in, tmargin = 0.8in, bmargin = 0.8in}
\date{\today}
\author{Jakub Rádl}

\makeatletter
\let\thetitle\@title
\let\theauthor\@author
\makeatother

\hypersetup{
colorlinks=true,
linkcolor=black,
urlcolor=cyan,
}



\begin{document}
\maketitle
\tableofcontents
\begin{figure}[b]
Toto dílo \textit{\thetitle} podléhá licenci Creative Commons \href{https://creativecommons.org/licenses/by-nc/4.0/}{CC BY-NC 4.0}.\\ (creativecommons.org/licenses/by-nc/4.0/)
\end{figure}
\newpage

\section{Úvod}
\begin{itemize}
\item Česko je součástí Rakouska-Uherska
\item Bachův absolutismus -- přísná cenzura
\item počátkem 60. let Bach odvolán \ra uvolnění situace
\item 1855 -- Almanach Lada Niola
\begin{itemize}
\item Božena němcová: povídka Sestry
\item Česko chce prokázat svou důležitost v rámci literatury
\end{itemize}

\end{itemize}




\section{Májovci}
\paragraph{1858 -- Almanach Máj}
\begin{itemize}
\item Vítězslav Hálek
\item Karolína Světlá
\item Jan Neruda
\item Adolf Heyduk
\item za cíl vyzvednout Máchův Máj
\item Program
	\begin{itemize}
	\item povznést kvalitativně českou literaturu na úroveň světové literatury
	\item inspirací jsou romantici: E. A. Poe, Heinrich Heine, Viktor Hugo
	\item v počátku lehce zanedbávána národní specifika
	\item později řešili soudobou problematiku společnosti
	\item do tvorby pronikají realistické principy a kritika
	\item nová témata (sociální problematika dělníků, postavení žen ve společnosti, zájem o vědu)
	\item nechtěli ztvárňovat historická témata (důležitá jsou ta aktuální)
	\end{itemize}
\end{itemize}

\subsection{Vítězslav Hálek (1835--1874)}
\begin{itemize}
\item uznávaný již za svého života (především poezie)
\item rivalita s Nerudou (Neruda ve své době méně oblíbený)
\item povídka \textbf{Muzikantská Liduška}
	\begin{itemize}
	\item baladicky laděné
	\item Liduška se nakonec zblázní, protože jí rodina nutí sňatek s mužem, kterého nemiluje
	\end{itemize}
\item sbírky poezie \textbf{V přírodě}, \textbf{Večerní písně} \textit{(čít. 253)}
	\begin{itemize}
	\item přírodní, milostná, i ntimní lyrika
	\end{itemize}
\item drama \textbf{Král Vukašín}
	\begin{itemize}
	\item historické drama ze srbských dějin
	\item otevírací hra v Prozatímním divadle 
	\end{itemize}
\end{itemize}

\subsection{Jan Neruda (1834--1891)}
\begin{itemize}
\item nejvýznamnější z Májovců
\item narodil se na Malé straně, ovlivněn zdejšími lidmi a životem
\item nedostudoval VŠ, zemřel mu kamarád, nemohl si vzít svoji lásku \ra ze začátku pesimismus, později se mění
\item část života zamilován do Karolíny Světlé
\item žurnalista
	\begin{itemize}
	\item redakce Národních listů
	\item setkal se zde s Hálkem
	\item fejetony: Kam s ním
	\end{itemize}
\item próza
	\begin{itemize}
	\item povídky a črty
	\end{itemize}

\item sbírka \textbf{Hřbitovní kvítí} (1857)
	\begin{itemize}
	\item nepřijata kritikou (\ra do další sbírky čekal deset let)
	\item ironie, kritika lásky, citu, poezie, cynika, pochyby nad životem
	\end{itemize}
\item sbírka \textbf{Písně Kosmické}
	\begin{itemize}
	\item oslava řádu ve vesmíru
	\item víra v lidský rozum
	\item tvrdí, že se jednou do vesmíru podíváme
	\item personifikace vesmírných těles
	\item zachyceny osobní prožitky z milostného života
	\item čítanka: Jak lvové bijem o mříže
		\begin{itemize}
		\item víra v rozvoj vědy a techniky
		\end{itemize}
	\end{itemize}
\item sbírka \textbf{Balady a Romance}
	\begin{itemize}
	\item některé balady nazvány romancí a naopak
	\item ukázka: balada Májová
	\item ukázka: romance Helgolandská
	\end{itemize}
\item sbírka \textbf{Prosté motivy}
	\begin{itemize}
	\item intimní lyrika
	\item paralela mezi lidským životem a střídáním ročních dob
	\end{itemize}
\item sbírka \textbf{Zpěvy Páteční}
	\begin{itemize}
	\item deset elegických básních
	\item Moje barva červená a bílá
	\item Jen dál
	\item aluze na ukřižování Ježíše (naděje ze zmrtvýchvstání)
	\end{itemize}
\item sbírka \textbf{Povídky Malostranské}
	\begin{itemize}
	\item k maturitě nečíst první a poslední povídku
	\item inspirováno životem na Malé Straně
	\item \textbf{Přivedla žebráka na mizinu} \textit{(čít. 261)}
		\begin{itemize}
		\item negativní vliv pomluvy na život člověka
		\item baba Milionová rozšíří o žebráku Vojtíškovi, že je bohatý \ra lidé mu přestanou dávat peníze \ra  umře
		\item očima malého Jana Nerudy
		\end{itemize}

	\end{itemize}
\item sbírka \textbf{Arabesky}
	\begin{itemize}
	\item povídka \textbf{Byl darebákem} -- nejznámější
	\item povídka \textbf{Trhani} -- o dělnících na železnici
	\end{itemize}
\end{itemize}

\subsection{Karolína světlá (1830--1899)}
\begin{itemize}
\item Johana Rottová
\item povídky z pražského prostředí (Černý Petříček)
\item ještědské romány 
	\begin{itemize}
	\item černobílá psychologie postav
	\item jednoduchá chronologická kompozice vyprávění
	\item prvky tajemna
	\item kladné postavy bývají silné osobnosti, ženské hrdinky
	\item řeší dilema osobního štěstí vs. nemorální jednání
	\item vždy si zvolí čestné řešení situace, "Štěstí nelze dosáhnout porušováním mravních zákonitostí"
	\item \textbf{Kříž u Potoka}
	\item \textbf{Frantina}
		\begin{itemize}
		\item starostka obce
		\item v okolí se dějí loupeže, krádeže, zločiny
		\item potká muže, do kterého se zamiluje, později zjistí, že stojí za zločiny
		\item zabije ho, sama umírá
		\end{itemize}
	\item \textbf{Vesnický román}
	\end{itemize}
\end{itemize}

\subsection{Jakub Arbes (1840--1914)}
\begin{itemize}
\item rozluštil deníky K. H. Máchy
\item vězněn za politické názory
\item psal \textbf{Romaneta}
	\begin{itemize}
	\item vymyslel Neruda
	\item něco mezi novelou a povídkou
	\item záhadný, tajemný, mystický děj, v závěru vysvětlen pomocí logiky, vědy, techniky
	\item inspirací E. A. Poe
	\end{itemize}
\item romaneto \textbf{Newtonův mozek}
	\begin{itemize}
	\item dva kamarádi se zabývají kouzelnickými triky
	\item po škole se rozejdou, jeden padne ve válce
	\item později se znovu setkají, jelikož padlý byl zachráněn transplantací Newtonova mozku
	\end{itemize}
\item romaneto \textbf{Etiopská lilie}
	\begin{itemize}
	\item matematik učiní objev
	\item dostane děkovný dopis od přítele s lilií
	\item výpočty se mu ztratí
	\item po čase zjistí, že je použil na lisování lilie
	\end{itemize}
\item romaneto \textbf{Svatý Xaverius}
	\begin{itemize}
	\item obraz v kostele sv. Mikuláše na Malé straně
	\item má obsahovat šifru vedoucí k pokladu
	\end{itemize}
\end{itemize}


\section{Ruchovci}
\begin{itemize}
\item vznik o deset let po Májovcích
\item almanach \textbf{Ruch} (1686)
	\begin{itemize}
	\item sestavil ho Josef Václav Sládek (který spadá k Lumírovcům)
	\item propagace vlastenectví
	\item přírodní lyrika, politické výzvy, historické povídky, vlastenecké elegie
	\end{itemize}
\item škola národní
	\begin{itemize}
	\item cíl literatury -- vychovávání národa
	\item umění má sloužit aktuální společenské potřebě
	\item spisovatelé mají povinnost se vyjadřovat k boji za národ
	\item prvky lidové poezie a slovesnosti
	\item historické motivy
	\end{itemize}
\end{itemize}

\subsection{Svatopluk Čech (1846--1908)}
\begin{itemize}
\item lyrika, epika, poezie i próza
\item sbírka \textbf{Písně otroka}
	\begin{itemize}
	\item důraz na osvobození, právo, svobodu
	\item protest proti rakouské vládě
	\item kvůli cenzuře zasazeno do Afriky mezi černochy
	\end{itemize}
\item cyklus povídek \textbf{Ve stínu lípy}
	\begin{itemize}
	\item lyricko-epický, vyprávění příběhů pod lípou
	\end{itemize}
\item epos \textbf{Hanuman}
	\begin{itemize}
	\item opičí král, opice se snaží napodobovat lidi
	\item zesměšňuje lidské vlastnosti, kritika maloměšťáctví napodobujícího vysokou společnost
	\item epické
	\end{itemize}		
\item \textbf{Broučkiády} \textit{(čít. 281)}
	\begin{itemize}
	\item hlavní postava \textbf{Matěj Brouček} -- pivní vlastenec "spousta keců u piva, ale skutek utek"
	\item výlety zprostředkovány alkoholem
	\item román \textbf{Pravý výlet pana Broučka do měsíce}
		\begin{itemize}
		\item potká obyvatele měsíce
		\end{itemize}
	\item román \textbf{Nový epochální výlet pana Broučka, tentokráté do 15. století}
		\begin{itemize}
		\item období husitských válek
		\end{itemize}
	\end{itemize}
\end{itemize}

\subsection{Eliška Krásnohorská (1847--1926)}
\begin{itemize}
\item libreta ke Smetanovým operám
\end{itemize}

\paragraph{1868 -- almanach Ruch}
\begin{itemize}
\item Josef Václav Sládek
\item přírodní lyriky
\item politické výzvy k národu
\item historické povídky
\item vlastenecké elegie
\end{itemize}

\section{Lumírovci}
\begin{itemize}
\item publikovali v časopise Lumír
\item vznik v 70. letech krátce po ruchovcích
\item škola kosmopolitní
	\begin{itemize}
	\item snaha vidět širší souvislosti
	\item pozvednout českou literaturu na světovou úroveň
	\item seznámit české čtenáře se světovou literaturou
	\item umění má svůj účel samo v sobě -- "umění pro umění"
	\item dokonalá forma, poutavý příběh
	\item hl. cíle: estetická funkce, přinést nové žánry
	\end{itemize}
\item utkání v časopisech mezi kosmopolitní a národní školou
	\begin{itemize}
	\item časopisy Lumír, Osvěta
	\item polemiky
	\item spory časem otupeny \ra spolupráce \ra Národní divadlo
	\end{itemize}
\end{itemize}

\paragraph{Společné rysy ruchovců a lumírovců}
\begin{itemize}
\item snaha o velkolepé vyjádření 
\item zájem o historii
\item oživení básnické epiky
\item básník má v kultuře důležité postavení
\item novoromantická vlna
\item ruchovci více domácí, lumírovci více cizí náměty
\end{itemize}

\subsection{Josef Václav Sládek (1845--1912)}
\begin{itemize}
\item stále živá díla
\item sbírky poezie pro děti \textbf{Zlatý máj}, \textbf{Zvona a Zvonky}, \textbf{Skřivánčí písně}
\item překladatelská činnost
	\begin{itemize}
	\item 33 divadelních her Williama Shakespeara
	\item snaha o zachování obsahu
		\begin{itemize}
		\item těžké při překladu poezie
		\item upřednostňoval před zachováním formy
		\end{itemize}
	\item 
	\end{itemize}
\item cestoval na dva roky do USA
	\begin{itemize}
	\item poznávání historie místního původního obyvatelstva
	\end{itemize}
\item 19 sbírek
\item sbírka \textbf{Jiskry na moři}
	\begin{itemize}
	\item po smrti ženy \ra smutné vyznění
	\end{itemize}
\item sbírka \textbf{Selské písně}
	\begin{itemize}
	\item snaží se poukázat na hodnoty půdy
	\item český sedlák -- pozitivní lidské hodnoty
	\item idilizace venkova, odpor vůči šlechtě
	\item půda má patřit tomu, kdo na ní pracuje 
	\end{itemize}
\item sbírka\textbf{České znělky}
	\begin{itemize}
	\item vlastenecké, vydáváno se Selskými písněmi
	\item reagují na aktuální situace
	\item lyrické básně
	\end{itemize}
\end{itemize}

\subsection{Julius Zeyer (1841--1901)}
\begin{itemize}
\item bohatá rodina
\item náměty v mytologii, české i světové
\item román \textbf{Román o věrném přátelství Amise a Amila}
	\begin{itemize}
	\item kontroverzní -- upřednostněno přátelství před životem vlastních dětí
	\item Amis umírá, Amil obětuje své děti na jeho záchranu
	\item děti jsou oživeny Pannou Marií
	\item námět ze Starého Zákona (Abrahám) a Řecka (Agamemnon a Ifigenie)
	\end{itemize}
\item pohádka \textbf{Radúz a Mahulena}
	\begin{itemize}
	\item novoromantická scénická pohádka
	\item princ Radúz zabloudí, potká Mahulenu, slíbí že se vrátí, královna ho zakleje (ldyž políbí ženu, zapomene na Mahulenu), Mahulena se promění v topol, Radúz si vzpomene a jeho milá se promění zpátky
	\item divadelní hra, zfilmováno
	\end{itemize}
\end{itemize}

\subsection{Jaroslav Vrchlický (1853--1912)} 
\begin{itemize}
\item plodný autor, překladatel ( považoval formu za důležitější než obsah)
\item \textbf{parnasismus} -- umění pro umění (podle pohoří Parnas, kde v Řecku sídlily múzy)
\item uznávaný, sebekritický
\item mladší generace mu vyčítala lpění na cizích vzorech
\item poezie, milostná poezie
\item básnický cyklus \textbf{Zlomky epopeje}
	\begin{itemize}
	\item připomíná Legendu věků
	\item snaha zachytit velké okamžiky dějin
	\item není chronoligické 
	\item epické i lyrickoepické básně
	\end{itemize}
\item komedie \textbf{Noc na Karlštejně}
	\begin{itemize}
	\item žena Karla IV. se vplíží na Karlštejn
	\end{itemize}
\item \textit{(čít. 293, 294)}
\end{itemize}




\end{document}