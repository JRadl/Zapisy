\title{Česká literatura druhé poloviny devatenáctého století}
\documentclass[10pt,a4paper]{article}
\usepackage[utf8]{inputenc}
\usepackage[czech]{babel}
\usepackage{amsmath}
\usepackage{amsfonts}
\usepackage{amssymb}
\usepackage{chemfig}
\usepackage{geometry}
\usepackage{wrapfig}
\usepackage{graphicx}
\usepackage{floatflt}
\usepackage{hyperref}
\usepackage{fancyhdr}
\usepackage{tabularx}
\usepackage{makecell}
\usepackage{csquotes}
\usepackage{footnote}

\MakeOuterQuote{"}

\renewcommand{\labelitemii}{$\circ$}
\renewcommand{\labelitemiii}{--}
\newcommand{\ra}{$\rightarrow$ }
\newcommand{\x}{$\times$ }
\newcommand{\lp}[2]{#1 -- #2}
\newcommand{\timeline}{\input{timeline}}


\geometry{lmargin = 0.8in, rmargin = 0.8in, tmargin = 0.8in, bmargin = 0.8in}
\date{\today}
\author{Jakub Rádl}

\makeatletter
\let\thetitle\@title
\let\theauthor\@author
\makeatother

\hypersetup{
colorlinks=true,
linkcolor=black,
urlcolor=cyan,
}



\begin{document}
\maketitle
\tableofcontents
\begin{figure}[b]
Toto dílo \textit{\thetitle} podléhá licenci Creative Commons \href{https://creativecommons.org/licenses/by-nc/4.0/}{CC BY-NC 4.0}.\\ (creativecommons.org/licenses/by-nc/4.0/)
\end{figure}
\newpage

\section{Úvod}
\begin{itemize}
\item Česko je součástí Rakouska-Uherska
\item Bachův absolutismus -- přísná cenzura
\item počátkem 60. let Bach odvolán \ra uvolnění situace
\item 1855 -- Almanach Lada Niola
\begin{itemize}
\item Božena němcová: povídka Sestry
\item Česko chce prokázat svou důležitost v rámci literatury
\end{itemize}

\end{itemize}




\section{Májovci}
\paragraph{1858 -- Almanach Máj}
\begin{itemize}
\item Vítězslav Hálek
\item Karolína Světlá
\item Jan Neruda
\item Adolf Heyduk
\item za cíl vyzvednout Máchův Máj
\item Program
	\begin{itemize}
	\item povznést kvalitativně českou literaturu na úroveň světové literatury
	\item inspirací jsou romantici: E. A. Poe, Heinrich Heine, Viktor Hugo
	\item v počátku lehce zanedbávána národní specifika
	\item později řešili soudobou problematiku společnosti
	\item do tvorby pronikají realistické principy a kritika
	\item nová témata (sociální problematika dělníků, postavení žen ve společnosti, zájem o vědu)
	\item nechtěli ztvárňovat historická témata (důležitá jsou ta aktuální)
	\end{itemize}
\end{itemize}

\subsection{Vítězslav Hálek (1835--1874)}
\begin{itemize}
\item uznávaný již za svého života (především poezie)
\item rivalita s Nerudou (Neruda ve své době méně oblíbený)
\item povídka \textbf{Muzikantská Liduška}
	\begin{itemize}
	\item baladicky laděné
	\item dívka se nakonec zblázní, protože jí rodina nutí sňatek s mužem, kterého nemiluje
	\end{itemize}
\item sbírky poezie \textbf{Přírodě}, \textbf{Večerní písně}
	\begin{itemize}
	\item přírodní, milostná, intimní lyrika
	\end{itemize}
\item drama \textbf{Král Vukašín}
	\begin{itemize}
	\item historické drama ze srbských dějin
	\item hráno v Prozatímním divadle 
	\end{itemize}
\item povídka Myslivec
	\begin{itemize}
	\item Jegor Vlasyč -- namyšlený, k Pelageje se chová špatně, pije, opilý je agresivní
	\item Pelageja -- chudá, zamilovaná do Jegora, chce aby ji miloval zpět, ale je si vědoma, že ji nemiluje
	\item Pelageja vyká Jegorovi, on jí ne
	\end{itemize}
\end{itemize}

\subsubsection{Jan Neruda}
\begin{itemize}
\item sbírka \textbf{Písně Kosmické}
	\begin{itemize}
	\item oslava řádu ve vesmíru
	\item víra v lidský rozum
	\item tvrdí, že se jednou do vesmíru podíváme
	\item personifikace vesmírných těles
	\item zachyceny osobní prožitky z milostného života
	\item čítanka: 
	\end{itemize}
\item sbírka \textbf{Balady a Romance}
	\begin{itemize}
	\item některé balady nazvány romancí a naopak
	\item balada Májová
	\item romance Helgolandská
	\end{itemize}
\item sbírka \textbf{Prosté motivy}
	\begin{itemize}
	\item intimní lyrika
	\item paralela mezi lidským životem a střídáním ročních dob
	\end{itemize}
\item sbírka Zpěvy Páteční
	\begin{itemize}
	\item deset elegických básních
	\item Moje barva červená a bílá
	\item Jen dál
	\item aluze na ukřižování Ježíše (naděje ze zmrtvýchvstání)
	\end{itemize}
\item sbírka Povídky Malostranské
	\begin{itemize}
	\item k maturitě nečíst první a poslední povídku
	\item ukázka: 261 -- přivedla žebráka na mizinu
	\item inspirováno životem na Malé Straně
	\end{itemize}
\item sbírka \textbf{Arabesky}
	\begin{itemize}
	\item Byl darebákem
	\end{itemize}
\item povídka \textbf{Trhani} -- o dělnících na železnici
\end{itemize}

\section{Ruchovci}
\section{Lumírovci}
\end{document}