\documentclass[10pt,a4paper]{article}
\usepackage[utf8]{inputenc}
\usepackage[czech]{babel}
\usepackage{amsmath}
\usepackage{amsfonts}
\usepackage{amssymb}
\usepackage{chemfig}
\usepackage{geometry}
\usepackage{wrapfig}
\usepackage{graphicx}
\usepackage{floatflt}
\usepackage{hyperref}
\usepackage{fancyhdr}
\usepackage{tabularx}
\usepackage{makecell}
\usepackage{csquotes}
\usepackage{marginnote}

\MakeOuterQuote{"}

\renewcommand{\labelitemii}{$\circ$}
\renewcommand{\labelitemiii}{--}
\newcommand{\ra}{$\rightarrow$ }
\newcommand{\x}{$\times$ }
\newcommand{\lp}[2]{#1 -- #2}
\newcommand{\timeline}{\input{timeline}}


\geometry{lmargin = 0.8in, rmargin = 0.8in, tmargin = 0.8in, bmargin = 0.8in}
\newcommand{\note}[1]{\marginnote{\hspace{-0.6\textwidth}#1}}

\date{}
\author{Jakub Rádl}
\title{Guillaume Appolinaire: Pásmo -- Rozbor díla}

\begin{document}
\maketitle

\section*{Výňatek}

\section*{Tématická stránka díla}
\begin{itemize}
\item \textbf{literární druh a žánr}: původně psáno jako báseň, dalo vzniknout novému žánru - pásmo
	\begin{itemize}
		\item proud myšlenek bez větších spojitostí
		\item psáno bez interpukce
		\item volný verš, začíná velkým písmenem
		\item polytématická báseň
	\end{itemize}
\item \textbf{téma a motiv}: rekapitulace autorova života
	\begin{itemize}
	\item \textbf{hlavní téma}: sled kratších úvah
	\item \textbf{další motivy v díle}:
		\begin{itemize}
		\item náboženství
			\begin{itemize}
			\item svět se mění, křesťanství zůstává
			\item propojení náboženství s průmyslem -- létání ptáků, ikarus, letadlo
			\end{itemize}
		\item cestování místem i časem
			\begin{itemize}
			\item paříž -- popisuje život a průmyslový rozvoj v Paříži, promítá vzpomínky z dětství
			\item vzpomíná na první lásku a zlomené srdce
			\item navštívil prahu, napsal o ní povídku -- důvod proč považujeme lyrický subjekt za alterego autora
			\item stává se starším, uvědomuje si, že by měl něco dělat, rekapituluje si život
			\item mluví o konci života, 

			\end{itemize}
		\end{itemize}
	\end{itemize}
\item \textbf{časoprostor}: nejdříve antický řím, později moderní města (Praha, Paříž)
	\item \textbf{kompoziční výstavba}
	\begin{itemize}
		\item žádné dělení na kapitoly
		\item obsahově děleno podle jeho života - dětství, střední léta, konec života
	\end{itemize}
\end{itemize}
\section*{Kompozice, postavy}
\begin{itemize}
\item vyprávěno z první osoby, lyrický subjekt je alter egem autora 
\item vyprávěcí způsoby - bez přímé řeči
\item \textbf{veršová výstavba}:	
	\begin{itemize}
	\item verš volný, rým sdružený
	\end{itemize}
\end{itemize}

\paragraph{Postavy}
\begin{itemize}
\item nejsou
\end{itemize}
\section*{Jazyk}
\begin{itemize}
\item překlad Karel Čapek
\item hodně anafor (opakování slova na začátku verše)
\item chybí interpunkce (běžné v pásmu)
\item hodně metafor a personifikací

\end{itemize}
\section*{Literárně historický kontext}
\begin{itemize}
\item báseň patří do sbírky Alkoholy
\item dala vzniknout novému žánru nazvanému podle ní Pásmo
\item Apollinaire žil v letech 1880--1918
\item byl představitelem literálního kubismu -- rozkládání reality na části a obrazy, obtížná roztříštěná kompozice, splývání poezie a prózy, důležitost grafiky
\item další autorova díla: sbírka Kaligrami, hra Prsy Tiréziovy
\end{itemize}
\section*{Zdroje}
\begin{itemize}
\item https://cs.wikipedia.org/wiki/Guillaume\_Apollinaire
\item https://cs.m.wikisource.org/wiki/Pásmo\_(Apollinaire)
\item https://rozbor-dila.cz/pasmo-rozbor-dila-k-maturite/
\item vlastní zápisy z hodin
\end{itemize}
\end{document}