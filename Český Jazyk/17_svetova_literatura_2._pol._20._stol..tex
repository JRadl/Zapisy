\title{Světová literatura 2. poloviny 20. století}
\documentclass[10pt,a4paper]{article}
\usepackage[utf8]{inputenc}
\usepackage[czech]{babel}
\usepackage{amsmath}
\usepackage{amsfonts}
\usepackage{amssymb}
\usepackage{chemfig}
\usepackage{geometry}
\usepackage{wrapfig}
\usepackage{graphicx}
\usepackage{floatflt}
\usepackage{hyperref}
\usepackage{fancyhdr}
\usepackage{tabularx}
\usepackage{makecell}
\usepackage{csquotes}
\usepackage{footnote}

\MakeOuterQuote{"}

\renewcommand{\labelitemii}{$\circ$}
\renewcommand{\labelitemiii}{--}
\newcommand{\ra}{$\rightarrow$ }
\newcommand{\x}{$\times$ }
\newcommand{\lp}[2]{#1 -- #2}
\newcommand{\timeline}{\input{timeline}}


\geometry{lmargin = 0.8in, rmargin = 0.8in, tmargin = 0.8in, bmargin = 0.8in}
\date{\today}
\author{Jakub Rádl}

\makeatletter
\let\thetitle\@title
\let\theauthor\@author
\makeatother

\hypersetup{
colorlinks=true,
linkcolor=black,
urlcolor=cyan,
}



\begin{document}
\maketitle
\tableofcontents
\begin{figure}[b]
Toto dílo \textit{\thetitle} podléhá licenci Creative Commons \href{https://creativecommons.org/licenses/by-nc/4.0/}{CC BY-NC 4.0}.\\ (creativecommons.org/licenses/by-nc/4.0/)
\end{figure}
\newpage


\section{Úvod}
\begin{itemize}
\item reakce na válku, rozdělení světa na východní a západní blok, studená válka
\item revolta mladých vůči řízení shora
\item experimentace -- nové kompoziční postupy, proud vědomí, 
\item magický realismus -- mýty a legendy
\item rychlý rozvoj 
\item \textbf{pikareskní román} -- (mazaný šibal zkoušený osudem -- Gargantua a Pantagruel)
\end{itemize}

\section{Reakce na 2. sv. válku}
\begin{itemize}
\item rozdíl mezi východním a západním pojetím
	\begin{itemize}
	\item východ -- zaměřeno na oslavu vítězství, úspěchy v bitvách
		\begin{itemize}
		\item dokumentární charakter, patos, černobílé pojetí
		\end{itemize}
	\item západ -- ukazovány hrůzy války, zvenku i uvnitř (vojáci na frontě)
		\begin{itemize}
		\item zachycuje i civilisty žijící za války
		\item humoristické, satirické knihy
		\end{itemize}
	\end{itemize}
\item téma je dlouho aktuální (až do teď)
\end{itemize}

\subsection{Gunter Grass (1927--2015)}
\begin{itemize}
\item Německo, Nobelova cena za literaturu
\item silně demokraticky protiválečně zaměřený, proti rasismu, totalitarismu
\item z počátku absurdní drama, pak próza, (pikareskní) romány, novely
\item autobiografický román \textbf{Při loupání cibule}
	 \begin{itemize}
	 \item před vydáním přiznává, že byl příslušníkem SS (v 16 letech)
	 \end{itemize}
\item \textbf{Gdaňská trilogie}
	\begin{itemize}
	\item v pozadí válka
	\item román \textbf{Plechový bubínek}
		\begin{itemize}
		\item pikareskní -- hlavní postava zůstane v těle dítěte (ve třech letech skočí ze schodů a zraněním se zastaví růst)
		\item dostal k narozeninám magický bubínek, kterým manipuluje lidi
		\item ječením tříštil sklo -> manipulace
		\item autor přechází mezi ich formmou a er formou
		\item vraždí, potýká se s fašismem, má sexuální zkušenosti, sepisuje paměti s ústavu pro choromyslné
		\item ukázka (\textit{čít.: 111})
			\begin{itemize}
				\item první odstavec -- větné ekvivalenty, následně jedna věta přes půl stránky
				\item popis schopností -- mnoho personifikací
				\item 
			\end{itemize}
		\end{itemize}
	\item novela \textbf{Kočka a myš}
	\item román \textbf{Psí léta}
	\end{itemize}
\end{itemize}

\subsection{William Styron (1925--2006)}
\begin{itemize}
\item jižanský autor (jako W. Faulkner)
\item román \textbf{Sofiina Volba}
	\begin{itemize}
	\item děj se odehrává primárně ve Spojených státech
	\item složitá kompozice -- chronologicky v Americe, přeskáčkově retrospektivní v Sofiiných vzpomínkách
	\item 	
	\end{itemize}
\end{itemize}

\subsection{Michail Šolochov}
\begin{itemize}
\item ruský spisovatel
\item novela \textbf{Osud člověka}
\end{itemize}

\subsection{Joseph Heller (1923--1999)}
\begin{itemize}
\item Americký spisovatel 
\item román \textbf{Hlava 22} -- antimilitaristická satyra
	\begin{itemize}
	\item naturalistické výjevy, morbidní, černý humor, nadsázka
	\item "v hlavě 22 vyrůstá humor z krveprolití a utrpení" - Heller
	\item kapitán \textbf{Yosarian} pracuje na letecké základně, stále se zvyšuje počet letů nutný pro odchod do důchodu
	\item kapitán se snaží z této situace různými způsoby dostat
	\item hlava 22 vojenského zákona říká, že letecké povinnosti může být zproštěn člověk choromyslný, ale jen člověk který o zproštění požádá -- symbol nesmyslného zákona
	\item motiv degradace hodnot
	\item je proti sobě postaven americký voják a americká armáda
	\item ukázka (\textit{čít. 29}) 
		\begin{itemize}
		\item doktor je zapsán na listu letadla, které havarovalo a je veden jako mrtvý
		\item popis absurdní situace, kde ho všichni považují za mrtvého (nedostává plat, manželka dostává sociální příspěvky, ...)
		\item manželka nakonec přijme jeho smrt
		\item absurdita, černý humor, nadsázka
		\item ukazuje nesmyslnost byrokracie vítězící nad selským rozumem
		\end{itemize}
	\end{itemize}
\end{itemize}


\subsection{James Clavell (1924--1994)}
\begin{itemize}
\item narozen v Austrálii
\item román Král krysa
	\begin{itemize}
	\item v zajateckém táboře v Japonsku
	\item hlavní postava -- americký nízký armádní hodnostář 
	\item vždy přežije, dostane co chce
	\end{itemize}
\end{itemize}



\section{Existencialismus}
\begin{itemize}
\item vzniká po 1. sv. v. v Německu
\item člověk si neuvědomuje dostatečně svou existenci a možnost svobodné volby
\item až v krajní životní situaci si dokáže uvědomit odpovědnost za svá rozhodnutí a svou existenci
\item předchůdcem je Franz Kafka, v jeho dílech ale chybí moment uvědomění
\item hrdina 
	\begin{itemize}
	\item pociťuje odcizení od společnosti, necítí se v ní dobře, nelíbí se jim společenská pravidla
	\item vnímá absurditu života, klade si otázky smyslu života a cíle jednání
	\item snaží se smysl najít, ale pochybují o něm \ra úzkost, bezcílné proplouvání životem
	\item hrdinové se dostanou do vyhrocené mezní situace \ra uvědomí odpovědnost za svůj život a následky svých rozhodnutí \ra přijímají důsledky
	\end{itemize}
\end{itemize}


\subsection{Jean Paul Sartre (1905--1980)}
\begin{itemize}
\item francouzský dramatik, prozaik, esejista, filosof, průkopník existencialismu
\item pacifismus, proti válce ve Vietnamu, občanské válce v Alžírsku, levicové zaměření, spravedlivá společnost
\item získal Nobelovu cenu, ale nepřijal ji z politických důvodů
\item román \textbf{Hnus} (Nevolnost) -- filosofické dílo
\item drama \textbf{Za zavřenými dveřmi} (Vyloučení veřejnosti)
	\begin{itemize}
	\item tři lidé, kteří nedosáhli momentu uvědomění jsou po smrti zavřeni v hotelovém pokoj
	\item vedou spolu dialogy, v nichž se ukazují špatné vlastnosti jejich povah -> odsouzení k věčnému peklu (každý vidí v ostatních obraz svých špatných vlastností)
	\end{itemize}
\item soubor 5 novel \textbf{Zeď}
	\begin{itemize}
	\item symbolická zeď mezi hrdinou a společností
	\item novela \textbf{Místnost} -- manžel má poruchu, žena pozoruje degradaci v důsledku demence a agresivity, zasahují rodič, žena by měla dát muže do ústavu, ale rozhodne se, že ho zabije dřív, než to bude nezbytně nutné
	\item novela \textbf{Zeď} -- 
		\begin{itemize}
		\item Ibieta bojuje proti fašistům ve španělské občanské válce
		\item rozhodne se nezradit své druhy za příslib odložení popravy
		\item když už má umřít, chce se u toho pobavit
		\item 
		\end{itemize}
	\end{itemize}



\end{itemize}


\end{document}