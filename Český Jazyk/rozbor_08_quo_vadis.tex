\documentclass[10pt,a4paper]{article}
\usepackage[utf8]{inputenc}
\usepackage[czech]{babel}
\usepackage{amsmath}
\usepackage{amsfonts}
\usepackage{amssymb}
\usepackage{chemfig}
\usepackage{geometry}
\usepackage{wrapfig}
\usepackage{graphicx}
\usepackage{floatflt}
\usepackage{hyperref}
\usepackage{fancyhdr}
\usepackage{tabularx}
\usepackage{makecell}
\usepackage{csquotes}
\usepackage{marginnote}

\MakeOuterQuote{"}

\renewcommand{\labelitemii}{$\circ$}
\renewcommand{\labelitemiii}{--}
\newcommand{\ra}{$\rightarrow$ }
\newcommand{\x}{$\times$ }
\newcommand{\lp}[2]{#1 -- #2}
\newcommand{\timeline}{\input{timeline}}


\geometry{lmargin = 0.8in, rmargin = 0.8in, tmargin = 0.8in, bmargin = 0.8in}
\newcommand{\note}[1]{\marginnote{\hspace{-0.6\textwidth}#1}}

\date{}
\author{Jakub Rádl}
\title{Henryk Sienkiewicz: Quo Vadis? -- Rozbor díla}

\begin{document}
\maketitle

\section*{Výňatek}
Petronius svlékl svou bílou, šarlatem lemovanou tógu, zvedl ji nad hlavu a začal jí mávat na znamení, že chce promluvit.

“Mlčte! Mlčte!” ozývalo se ze všech stran.

Za chvíli nastalo skutečně ticho. Petronius se vzpřímil v sedle a promluvil zvučným, klidným hlasem:

“Občané! Ti, kdož mě uslyší, nechť opakují má slova těm, kdož stojí dál, avšak všichni nechť
se chovají jako lidé, a ne jako zvířata v arénách.”

“Posloucháme! Posloucháme!”

“Tedy slyšte. Město bude znovu postaveno. Lucullovy, Maecenatovy, Caesarovy a Agrippininy zahrady vám budou zpřístupněny! Od zítřka se začne rozdávat obilí, víno a olej, tak aby si každý mohl nacpat břicho až po krk! Pak vám caesar uchystá hry, jaké svět dosud neviděl a při nichž vás čekají hostiny a dary. Budete bohatší po požáru, než jste byli před požárem!”

Odpovědělo mu mručení, které se šířilo ze středu na všechny strany, tak jak se šíří kola na vodě, do které hodil někdo kámen: to ti, kdož byli blíže, sdělovali jeho slova dozadu. Pak se ozvaly tu a tam hněvivé anebo souhlasné výkřiky, které se nakonec změnily v jediné, mohutné volání všech:

“Panem et circenses!”

Petronius se zahalil do tógy a chvíli poslouchal, nehybný, podobný ve svém bílém oděvu mramorové soše. Volání sílilo, přehlušovalo hukot požáru, ozývalo se ze všech stran a stále z větších hloubek, ale mluvčí měl patrně ještě něco na srdci, protože čekal. A konečně, zjednav si opět zvednutou paží klid, zvolal:

“Slibuji vám panem et circenses, ale teď provolejte slávu caesarovi, který vás krmí a odívá, a pak spát, holoto, protože za chvíli se začne rozednívat!”

Po těchto slovech otočil koně, a zlehka ťukaje hůl401 čičkou po hlavách a tvářích těch, kdož
mu stáli v cestě, odjel zvolna do řad praetoriánů.

Zanedlouho byl pod vodovodem. Nahoře zastal téměř paniku. Nerozuměli tam výkřikům
“Panem et circenses” a domnívali se, že je to nový výbuch vzteku. Už ani nedoufali, že by se Petronius mohl zachránit, a proto, když ho Nero uviděl, přiběhl až ke schodům a začal se ho vyptávat, tvář pobledlou vzrušením:

“Co je? Co se tam děje? Už se strhla bitva?”

Petronius nabral do plic vzduch, zhluboka si oddechl a odpověděl:

“U Polluxe! Potí se a páchnou! Podejte mi někdo epilimmu, nebo omdlím.” Pak se otočil k caesarovi.

\newpage
\section*{Tématická stránka díla}
\begin{itemize}
\item \textbf{literární druh a žánr}:epický, historický román
	\begin{itemize}
	\item reálné postavy -- Nero, Patricius, apoštolové
	\end{itemize}
\item \textbf{téma a motiv}: 
	\begin{itemize}
	\item \textbf{hlavní téma}:Život v Římě za vlády císaře Nerona
		\begin{itemize}
		\item protichůdná láska Vinicia a Lygie
		\item počátky křesťanství (skrývání víry)
		\end{itemize}
	\item \textbf{další motivy v díle}: spasení a odpuštění, krutost císaře, pletichy, pochlebování, proměnlivost osobností postav, čistota Lygie proti zkaženosti společnosti, krása, historie, požár Říma, 
		\begin{itemize}
		\item ryba, otroctví, 
		\end{itemize}
	\end{itemize}
\item \textbf{časoprostor}: za vlády Nerona v Římě, Antilu, Achai (37--68 AD)

\item \textbf{zasazení výňatku do kontextu díla}:
	\begin{itemize}
	\item \textbf{časoprostor}: po hašení Říma
	\item \textbf{obsah}: Petronius uklidňuje obyvatelstvo, které se bouří, protože si myslí, že Nero podpálil řím
	\end{itemize}
\end{itemize}
\section*{Kompozice, postavy}
\begin{itemize}
\item chronologická a paralelní kompozice (vyprávěno z pohledů různých postav), retrospektivní ve vzpomínkách (minulost Lygie)
\item vypravěč / lyrický subjekt: nezaujatý, vnější vypravěč (dílo je silně prokřesťanské)
\item vyprávěcí způsoby:
	\begin{itemize}
	\item dialogy v ich formě
	\item vyprávění v er formě
	\end{itemize}
\item typy promluv:
	\begin{itemize}
	\item v úryvku -- přímá řeč
	\end{itemize}
\item \textbf{veršová výstavba}: próza
\end{itemize}

\paragraph{Postavy}
\begin{itemize}
\item \textbf{Petronius} --  
	\begin{itemize}
	\item arbiter elegancie
	\item pomohl Viniciovi (jeho synovci) a Lygii přežít vyvraždění křesťanů
	\item směřuje ke stoicismu, nemá rád nové věci, vyrušení ze svého klidu, přemýšlení o nových věce, změnu názoru
	\item bohém -- literatura, umění, krásné ženy, předměty, věci; estét 
	\item pochlebuje Neronovi, skutečně vůči němu má spíše záporný pohled, kritizuje jeho básnickou tvorbu \ra chytrý hráč -- riskoval, ale věděl, kdy se zastavit
	\item na konci díla pošle Neronovi upřímný, ironický, sarkastický dopis a spáchá s Euniké (otrokyně a později svobodná milenka) sebevraždu 
	\end{itemize}
\item \textbf{Vinicius}
	\begin{itemize}
	\item z počátku tvrdohlavý, zamiloval se do Lygie
	\item záchvaty vzteku, když mu Lygie utekla
	\item později se obrátí na křesťanství \ra pokora, láska (nejen fyzická)
	\end{itemize}
\item \textbf{Lygie} -- královská dcera, čistá postava, z počátku velmi naivní, představitelka křesťanů
\item \textbf{Nero} 
	\begin{itemize}
	\item císař
	\item samolibý, tvrdý, krutý, zavraždil matku a bratra, aby se dostal na trůn, dle díla nemá žádné dobré vlastnosti
	\item lehce manipulovatelný, hloupý
	\end{itemize}
\item \textbf{Chilon} -- filosof, cinik, snaží se co nejvíce vytěžit z každé situace, zrazuje lidi dříve, než stihnou oni zradit jeho
\end{itemize}
\section*{Jazyk}
\begin{itemize}
\item jazykové prostředky a jejich funkce ve výňatku
\item tropy a figury ve výňatku
\end{itemize}
\section*{Literárně historický kontext}
\begin{itemize}
\item snaha o povzbuzení utlačovaného polského národa
\item další díla -- trilogie: Ohněm a mečem, Potopa, Pan Wolodyjowski
\end{itemize}
\section*{Zdroje}
\end{document}