\title{Světová literatura na přelomu 19. a 20. století}
\documentclass[10pt,a4paper]{article}
\usepackage[utf8]{inputenc}
\usepackage[czech]{babel}
\usepackage{amsmath}
\usepackage{amsfonts}
\usepackage{amssymb}
\usepackage{chemfig}
\usepackage{geometry}
\usepackage{wrapfig}
\usepackage{graphicx}
\usepackage{floatflt}
\usepackage{hyperref}
\usepackage{fancyhdr}
\usepackage{tabularx}
\usepackage{makecell}
\usepackage{csquotes}
\usepackage{footnote}
\usepackage{movie15}
\MakeOuterQuote{"}

\renewcommand{\labelitemii}{$\circ$}
\renewcommand{\labelitemiii}{--}
\newcommand{\ra}{$\rightarrow$ }
\newcommand{\x}{$\times$ }
\newcommand{\lp}[2]{#1 -- #2}
\newcommand{\timeline}{\input{timeline}}


\geometry{lmargin = 0.8in, rmargin = 0.8in, tmargin = 0.8in, bmargin = 0.8in}
\date{\today}
\author{Jakub Rádl}

\makeatletter
\let\thetitle\@title
\let\theauthor\@author
\makeatother

\hypersetup{
colorlinks=true,
linkcolor=black,
urlcolor=cyan,
}



\begin{document}
\maketitle
\tableofcontents
\begin{figure}[b]
Toto dílo \textit{\thetitle} podléhá licenci Creative Commons \href{https://creativecommons.org/licenses/by-nc/4.0/}{CC BY-NC 4.0}.\\ (creativecommons.org/licenses/by-nc/4.0/)
\end{figure}
\newpage

\section{Literární moderna}
\begin{itemize}
\item nové směry vybočující z realismu jakožto hlavního určujícího směru
\item \textbf{původ ve Francii}, následně šíření v Evropě
\item projevuje se \textbf{parnasismus} ("umění pro umění" -- jde o více o formu než o obsah a vyjadřování se k~problematice)
\item důraz na jazyk, na výběr slov, vyjádření krásy jazyka
\item náměty z mytologie, exotiky, minulosti, ne v~soudobém světě
\item podobný program jako Lumírovci
\end{itemize}

\subsection{Impresionismus}
\paragraph{Impresionismus v malířství}
\begin{itemize}
\item \textbf{Claud Monet}: \textbf{Imprese -- východ slunce} \ra název "impresionismus"
\item \textbf{August Renoire}: V lese, U vody, \ldots
\item nemají ostré linie
\item tvořeny shlukem skvrn, krátkých tahů štětce
\item jemné pastelové barvy převládají nad ostrými živými
\item cílem zachytit náladu okamžiku \ra scenérie ve venkovním prostředí, nutno malovat v čas momentu (východ slunce)
\item snaha zachycení proměnlivosti \ra zachycení stejného místa v různých denních/ročních dobách
\item zachycována příroda, do ní zasazeni lidé
\end{itemize}

\paragraph{Impresionismus v literatuře}
\begin{itemize}
\item zachycení momentu, nálad \ra posílení lyrické složky 
\item projevuje se především v poezii, ale i v próze
	\begin{itemize}
	\item metaforika, zvukomalba (onomatopoie, eufonie) \ra působení na všechny smysly
	\end{itemize}
\item popis momentů, psychologie postav
\item melodičnost, hudebnost písní
\item \textbf{Antonín Sova}
\end{itemize}



\subsection{Symbolismus}
\begin{itemize}
\item využití symbolů (kříž -- smrt, hvězdička -- narození, dopravní značky)
	\begin{itemize}
	\item často nesrozumitelné, těžce dešifrovatelné, působí na pocitové vnímání
	\item jednoduché symboly -- Máj: kapání vody symbolizuje ubývání času
	\item složité symboly
	\end{itemize}
\item snaha o poznání světa i skrze jiné směry než techniku a vědu, ukázat čtenáři skryté významy
\item v poezii  dříve často verš vázaný \ra v symbolismu volný -- "uvolnění z pout rozumu"
\item stírání rozdílu mezi poezií a prózou
	\begin{itemize}
	\item[\ra] básně v próze
	\item[\ra] lyrizovaná próza
	\end{itemize}
\item symbolická díla jsou určena pro vlastní čtení, čtenář si dílo dotváří svou fantazií a svými zážitky
\item u nás -- \textbf{Otokar Březina}, Antonín Sova
\item ve světě -- \textbf{Arthur Rimbaud}
\end{itemize}


\subsection{Dekadence}
\begin{itemize}
\item ne jasně definovaný směr jakožto spíš životní styl
\item původní význam slova dekadence -- úpadek, "nálady konce století" 
	\begin{itemize}
	\item vědecký pokrok přinesl velké naděje a očekávání ve zlepšení života, které nebyly naplněny
	\item[\ra] špatná nálada ve společnosti
	\end{itemize}
\item vypjatý individualismus, zaměření na své ego
\item disharmonie mezi kolektivem a jedincem (pod. romantismus)
	\begin{itemize}
	\item hrdinové nejsou zapojeni do boje o ideály, šlechetné cíle
	\item přetrvávají úniky do exotiky, snů, symboliky
	\end{itemize}
\item tabu témata
	\begin{itemize}
	\item útěk od víry, až k vzývání satana
	\item erotika
	\item zlo v lidech
	\item smrt, kult smrti
	\end{itemize}
\item dekadence nelze charakterizovat jazykovými prostředky, používá impresionismus i symbolismus k vyjádření pocitů konce století
\end{itemize}

\section{Prokletí básníci}
\begin{itemize}
\item neakceptovatelný způsob života, nectili morálku společnosti
\item odmítali dobově platná pravidla, pobuřovali společnost
\item geniální básníci \ra výrazné, zajímavé, dodnes čtivé dílo
\item vyznávali čistou krásu poezie
\item řazen mezi ně Francois Vilon (z Renesance) -- předchůdce prokletých básníků
\end{itemize}

\subsection{Charles Baudelaire (1821--1867)}
\begin{itemize}
\item Francouz
\item před maturitou vyloučen ze střední školy
\item cestoval do Indie, kde potkal ženu, kterou přivedl zpátky
\item experimentoval s drogami \ra zadlužení
\item snaha o vydělávání literární kritikou a překlady z cizích jazyků (E. A. Poe)
\item umírá ochrnutý
\item tvorba osciluje mezi parnasismem a symbolismem
	\begin{itemize}
	\item myšlenky se nevylučují, ale je těžké v symbolismu přesně dodržet formu
	\end{itemize}
\item sbírka \textbf{Květy zla}
	\begin{itemize}
	\item podává realitu z co nejnegativnějšího úhlu pohledu
	\item květy zla \ra pokus o zmírnění a záblesky štěstí -- nalezeny ve vzpomínkách nebo v umění
	\item vzbudilo vlnu nechuti \ra zažalován cenzory, za negativitu a urážení víry \ra proces prohrál \ra musel vyřadit 11 básní, ale vytvořilo mu to reklamu
	\item 
	\end{itemize}
\end{itemize}


\subsection{Paul Verlaine (1844--1896)}
\begin{itemize}
\item původně psal v parnasismu
	\begin{itemize}
	\item \textbf{Saturnské básně}
	\item \textbf{Galantní slavnosti}
	\end{itemize}
\item ženatý, milenec Arthur Rimbaud
	\begin{itemize}
	\item Verlaine si nebyl jistý svou orientací
	\item bouřlivý vztah, Verlaine ho postřelil \ra vězení
	\end{itemize}
\item po seznámení s Rimbaudem symbolismus
	\begin{itemize}
	\item sbírka básní \textbf{Písně beze slov} -- symbolismus
		\begin{itemize}
		\item \textbf{Bělostný měsíc} (\textit{čít. 228})
		\end{itemize}
	\item sbírka básní \textbf{Moudrost}
		\begin{itemize}
		\item napsána ve vězení
		\item zpytování svědomí 
		\end{itemize}
	\end{itemize}
\end{itemize}

\subsection{Arthur Rimbaud (1854--1891)}
\begin{itemize}
\item snaha působit poezií na všechny smysly
\item zpracování zážitků, snů, halucinací
\item vyjádřit rozervanost světa a jeho vnímání
\item báseň \textbf{Samohlásky}
\item báseň \textbf{Opilý koráb}
	\begin{itemize}
	\item 
	\item vypravěč se stane korábem
	\end{itemize}
\end{itemize}

\subsection{Walt Whitman (1819--1892)}
\begin{itemize}
\item sbírka \textbf{Stébla trávy}
	\begin{itemize}
	\item celoživotní dílo -- 400 básní
	\item vyjadřuje víru v člověka, zachycuje člověka v obyčejném, všedním životě se vším, co k němu patří
	\item všímá si společenských problémů
	\item volný verš, prosté řazení motivů výčtem
	\end{itemize}
\end{itemize}

\section{Další autoři}
\subsection{Anton Pavlovič Čechov (1819--1892)}
\begin{itemize}
\item míjivý dialog -- dva lidé mluví o jiné věci a nereagují na promluvu druhého
\end{itemize}

\subsection{Oscar Wilde (1859--1900)}
\begin{itemize}
\item spjat s dekadentním stylem
\item román \textbf{Obraz Doriana Graye}
	\begin{itemize}
	\item zachycuje soudobou společnost v Londýně
	\item hlavní postavy Dorian Gray, 
	\end{itemize}
\end{itemize}


\end{document}