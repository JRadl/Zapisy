\title{Světová literatura na přelomu 19. a 20. století}
\documentclass[10pt,a4paper]{article}
\usepackage[utf8]{inputenc}
\usepackage[czech]{babel}
\usepackage{amsmath}
\usepackage{amsfonts}
\usepackage{amssymb}
\usepackage{chemfig}
\usepackage{geometry}
\usepackage{wrapfig}
\usepackage{graphicx}
\usepackage{floatflt}
\usepackage{hyperref}
\usepackage{fancyhdr}
\usepackage{tabularx}
\usepackage{makecell}
\usepackage{csquotes}
\usepackage{footnote}

\MakeOuterQuote{"}

\renewcommand{\labelitemii}{$\circ$}
\renewcommand{\labelitemiii}{--}
\newcommand{\ra}{$\rightarrow$ }
\newcommand{\x}{$\times$ }
\newcommand{\lp}[2]{#1 -- #2}
\newcommand{\timeline}{\input{timeline}}


\geometry{lmargin = 0.8in, rmargin = 0.8in, tmargin = 0.8in, bmargin = 0.8in}
\date{\today}
\author{Jakub Rádl}

\makeatletter
\let\thetitle\@title
\let\theauthor\@author
\makeatother

\hypersetup{
colorlinks=true,
linkcolor=black,
urlcolor=cyan,
}



\begin{document}
\maketitle
\tableofcontents
\begin{figure}[b]
Toto dílo \textit{\thetitle} podléhá licenci Creative Commons \href{https://creativecommons.org/licenses/by-nc/4.0/}{CC BY-NC 4.0}.\\ (creativecommons.org/licenses/by-nc/4.0/)
\end{figure}
\newpage

\section{Literární moderna}
\begin{itemize}
\item nové směry vybočující z \textbf{realismu} jakožto hlavního určujícího směru
\item \textbf{původ ve Francii}, následně šíření v Evropě
\item projevuje se \textbf{parnasismus} 
\begin{itemize}
	\item "umění pro umění" -- jde o více o formu než o obsah a vyjadřování se k~problematice
	\item důraz na jazyk, na výběr slov, vyjádření \textbf{krásy jazyka}
	\item náměty z \textbf{mytologie, exotiky, minulosti, ne v~soudobém světě}
	\item podobný program jako Lumírovci
\end{itemize}

\end{itemize}

\subsection{Impresionismus}
\paragraph{Impresionismus v malířství}
\begin{itemize}
\item \textbf{Claud Monet}: \textbf{Imprese -- východ slunce} \ra název "impresionismus"
\item \textbf{August Renoire}: V lese, U vody, \ldots
\item v obrazech se \textbf{nevyskytují ostré linie}, jsou tvořeny shlukem skvrn, krátkých tahů štětce \ra z blízka patlanice, z dálky se ukáže obraz
\item jemné pastelové barvy převládají nad ostrými živými
\item cílem \textbf{zachytit náladu okamžiku} \ra scenérie ve venkovním prostředí, nutno malovat v čas momentu (východ slunce)
\item snaha \textbf{zachycení proměnlivosti} \ra zachycení stejného místa v různých denních/ročních dobách
\item zachycována \textbf{příroda}, do ní zasazeni lidé
\end{itemize}

\paragraph{Impresionismus v literatuře}
\begin{itemize}
\item zachycení momentu, nálad \ra \textbf{posílení lyrické složky}
\item projevuje se především v poezii, ale i v próze
	\begin{itemize}
	\item metaforika, zvukomalba (onomatopoie, eufonie) \ra působení na všechny smysly
	\end{itemize}
\item popis momentů, přírodních motivůpsychologie postav
\item básník \textbf{Antonín Sova}
\item melodičnost, hudebnost písní

\end{itemize}



\subsection{Symbolismus}
\begin{itemize}
\item využití \textbf{symbolů} (kříž -- smrt, hvězdička -- narození, dopravní značky)
	\begin{itemize}
	\item složité symboly -- často nesrozumitelné, těžce dešifrovatelné, působí na pocitové vnímání
	\item jednoduché symboly -- Máj: kapání vody symbolizuje ubývání času
	\end{itemize}
\item snaha o poznání světa i skrze jiné směry než techniku a vědu, ukázat čtenáři skryté významy
\item v poezii  dříve často verš vázaný \ra v symbolismu \textbf{volný} -- "uvolnění z pout rozumu"
\item \textbf{stírání rozdílu} mezi poezií a prózou
	\begin{itemize}
	\item[\ra] básně v próze
	\item[\ra] lyrizovaná próza
	\end{itemize}
\item symbolická díla jsou určena \textbf{pro vlastní čtení}, čtenář si dílo dotváří svou fantazií a svými zážitky
\item u nás -- \textbf{Otokar Březina}, Antonín Sova
\item ve světě -- \textbf{Arthur Rimbaud}
\end{itemize}


\subsection{Dekadence}
\begin{itemize}
\item ne jasně definovaný směr jakožto spíš \textbf{životní styl}
\item původní význam slova dekadence -- \textbf{úpadek}, "nálady konce století" 
	\begin{itemize}
	\item vědecký pokrok přinesl velké naděje a očekávání ve zlepšení života, které nebyly naplněny
	\item[\ra] \textbf{špatná nálada} ve společnosti
	\end{itemize}
\item vypjatý \textbf{individualismus}, zaměření na své ego
\item \textbf{disharmonie} mezi kolektivem a jedincem (pod. romantismus)
	\begin{itemize}
	\item hrdinové ale nejsou zapojeni do boje o ideály, šlechetné cíle
	\item přetrvávají úniky do exotiky, snů, symboliky
	\end{itemize}
\item \textbf{tabu témata}
	\begin{itemize}
	\item útěk od víry, až k vzývání satana
	\item erotika
	\item zlo v lidech
	\item smrt, kult smrti
	\end{itemize}
\item dekadence \textbf{nelze charakterizovat jazykovými prostředky}, používá impresionismus i symbolismus k vyjádření pocitů konce století
\end{itemize}

\section{Prokletí básníci}
\begin{itemize}
\item francouzští autoři
\item \textbf{neakceptovatelný způsob života}, nectili morálku společnosti
\item odmítali dobově platná pravidla, pobuřovali společnost
\item geniální básníci \ra výrazné, zajímavé, dodnes čtivé dílo
\item vyznávali čistou krásu poezie
\item řazen mezi ně Francois Vilon (z Renesance) -- předchůdce prokletých básníků
\end{itemize}

\subsection{Charles Baudelaire (1821--1867)}
\begin{itemize}
\item před maturitou vyloučen ze střední školy
\item cestoval do Indie, kde potkal ženu, kterou přivedl zpátky
\item experimentoval s drogami \ra zadlužení
\item snaha o vydělávání literární kritikou a překlady z cizích jazyků (E. A. Poe)
\item umírá ochrnutý
\item tvorba osciluje mezi parnasismem a symbolismem
	\begin{itemize}
	\item myšlenky se nevylučují, ale je těžké v symbolismu přesně dodržet formu
	\end{itemize}
\item sbírka \textbf{Květy zla} (oxymóron)
	\begin{itemize}
	\item podává realitu z co nejnegativnějšího úhlu pohledu
	\item květy zla \ra pokus o zmírnění a záblesky štěstí -- nalezeny ve vzpomínkách nebo v umění
	\item vzbudilo vlnu nechuti \ra zažalován cenzory, za negativitu a urážení víry \ra proces prohrál \ra musel vyřadit 11 básní, ale vytvořilo mu to reklamu
	\item Mrcha (\textit{čít. 226})
		\begin{itemize}
		\item popis krásné mrtvoly -- naturalistické popisy 
		\item autor tvrdí, že milenka se nakonec změní na něco podobného
		\end{itemize}
	\end{itemize}
\item sbírka \textbf{Malé básně v próze}
	\begin{itemize}
	\item lyrizovaná próza 
	\item symbolismus -- opíjení 
	\item obrazné myšlení -- zachycení vnitřního světa člověka 
	\item \textbf{Opíjejte se!} (\textit{čít. 227}) -- musíme hledat krásné věci v realitě, obrazné myšlení snaha zachycení vnitřního života člověka
	\end{itemize}
\end{itemize}


\subsection{Paul Verlaine (1844--1896)}
\begin{itemize}
\item označil svou generaci za Prokleté básníky
\item původně psal v parnasismu
	\begin{itemize}
	\item sbírka \textbf{Saturnské básně}
		\item \textbf{Píseň podzimní}
	\item sbírka \textbf{Galantní slavnosti}
	\end{itemize}
\item ženatý, milenec Arthur Rimbaud
	\begin{itemize}
	\item Verlaine si nebyl jistý svou orientací
	\item bouřlivý vztah, Verlaine ho postřelil \ra vězení
	\end{itemize}
\item po seznámení s Rimbaudem symbolismus
	\begin{itemize}
	\item sbírka básní \textbf{Písně beze slov} -- symbolismus
		\begin{itemize}
		\item \textbf{Bělostný měsíc} (\textit{čít. 228})
		\end{itemize}
	\item sbírka básní \textbf{Moudrost}
		\begin{itemize}
		\item napsána ve vězení, zpytování svědomí
		\item uvolněnější verš 
		\item báseň \textbf{Bělostný měsíc} (\textit{čít. 228}) -- popis krásy okamžiku
		\item báseň \textbf{Spleen}
		\end{itemize}
	\end{itemize}
\end{itemize}

\subsection{Arthur Rimbaud (1854--1891)}
\begin{itemize}
\item snaha působit poezií na všechny smysly
\item všechno své dílo napsal mezi 16. a 18. rokem života
\item zpracování zážitků, snů, halucinací
\item vyjádřit rozervanost světa a jeho vnímání
\item symbolismus
\item báseň \textbf{Samohlásky}
	\item přiřazuje samohláskám barvy a emoce
\item báseň \textbf{Opilý koráb}
	\begin{itemize}
	\item popisuje svůj život
	\item korábu se urve kormidlo a unáší ho proud života
	\item vypravěč se ztotožňuje s korábem
	\item uvědomuje si, že není spokojen se životem, chce se vrátit do dětství
	\item náznak toho, že by raději ukončil svůj život, než v něm pokračoval
	\item bezradnost
	\item mnoho barevných přirovnání, metafor, symbolů, \ldots
	\end{itemize}
\item sbírka \textbf{Sezóna v pekle} 
	\begin{itemize}
	\item vydána za jeho života
	\item nemožnost vymanit se ze špatného života
	\end{itemize}
\item sbírka \textbf{Iluminace}
	\begin{itemize}
	\item vydána Verlainem po jeho smrti
	\item básnické prózy
	\end{itemize}
\item sbírka \textbf{Básně} -- vydána někým jiným
\end{itemize}


\section{Další autoři}
\subsection{Walt Whitman (1819--1892)}
\begin{itemize}
\item USA
\item sbírka \textbf{Stébla trávy}
	\begin{itemize}
	\item celoživotní dílo -- 400 básní
	\item vyjadřuje víru v člověka, zachycuje člověka v obyčejném, všedním životě se vším, co k němu patří
	\item všímá si společenských problémů
	\item volný verš, prosté řazení motivů výčtem, nepravidelný rytmus, počet slabik
	\item báseň  \textbf{Těmto Státům}
		\begin{itemize}
		\item k USA, poslouchat se nemá bez rozmyslu
		\end{itemize}
	\item báseň \textbf{Píseň širočiny}
		\begin{itemize}
		\item sekera a dřevo, vytváří věci pro člověka, díky kterým může plnit koloběh života
		\end{itemize}
	\end{itemize}
\end{itemize}


\subsection{Anton Pavlovič Čechov (1819--1892)}
\begin{itemize}
\item Rusko
\item dramatik a prozaik
\item humoristické příběhy / žertovné miniatury -- 600
	\begin{itemize}
	\item pod pseudonymem
	\item \textbf{Kniha Stížností}
		\begin{itemize}
		\item literárně zpracována kniha stížností z železniční stanice
		\end{itemize}	
	\end{itemize}
\item příběhy o ubohosti ruského úředníka (pod. Gogol)
	\begin{itemize}
	\item povídka \textbf{Úředníkova Smrt}
		\begin{itemize}
		\item absurdní příběh bezvýznamného úředníka
		\item \textit{šel do divadla, na sedadle před ním seděl vysoce postavený úředník}
		\item \textit{úředníček kýchnul a poprskal vysokého úředníka, omlouval se mu a po pár dnech si domluvil schůzku}
		\item \textit{úředník ho z kanceláře vyhnal, úředníček si myslel, že se na něj zlobí}
		\item \textit{úředníček ze strachu onemocní a zemře}
		\end{itemize}
	\end{itemize}
\item soudničky
	\begin{itemize}
	\item reportáže ze soudní síně
	\end{itemize}
\item psychologické povídky a novely
	\begin{itemize}
	\item \textbf{Dáma s Psíčkem}
		\begin{itemize}
		\item milostný příběh, flirt se promění v lásku
		\item některé části satirické
		\end{itemize}
	\end{itemize}
\item drama
	\begin{itemize}
	\item \textbf{tragikomické vyznění} -- "smích skrze slzy"
	\item specifická, \textbf{lyrizovaná dramata} (dějová stránka ustupuje do pozadí tragedie a dramatičnost je rozvedena ne příběhem, ale psychologií postav)
	\item v centru není vyhrocená zápletka 
	\item zobrazuje průměrné lidi a jejich zdánlivě všední život
	\item \textbf{vše má svůj účel}, řád, smysl -- "je-li na scéně puška, musí se z ní vystřelit"
	\item \textbf{MCHAT} -- Moskevké Umělecké Akademické Divadlo (režiséři Stanislavskij, Němirovič-Dančenko)
		\begin{itemize}
		\item ztvárňovalo mnoho Čechovových her
		\item Čechov nebyl příliš spokojený
		\end{itemize}
	\item \textbf{Tři sestry}, \textbf{Racek}, \textbf{Strýček Váňa}, \textbf{Višňový sad}
	\item komedie \textbf{Višňový sad}
		\begin{itemize}
		\item 1904, krátce před svou smrtí
		\item střet světa staré nepřizpůsobivé ruské šlechty a nové dravé generace, která se snaží nahlížet na život praktičtěji a využít možností, které se jim nabízejí
		\item \textit{odehrává se na panství kněžny \textbf{Raněvské}, která má dceru \textbf{Aňu} a bratra \textbf{Gajeva}}
		\item \textit{kněžna po smrti syna odjela do Francie, nyní se vrací na zadlužené panství spravované \textbf{Varjou}, její nevlastní dcerou}
		\item \textit{Varje se dvoří \textbf{Lopachin}, potomek generace nevolníků, člen mladé generace, která neuznává staré hodnoty }
		\item \textit{Lopachin navrhuje kněžně, aby rozparcelovala a prodala Višňový sad, ona ho ze sentimentality nechce prodat, prodá se celé panství a koupí ho Lopachin}
		\end{itemize}
	\end{itemize}
\item povídka \textbf{Myslivec} (čít.: 197)
\end{itemize}



\subsection{Oscar Wilde (1854--1900)}
\begin{itemize}
\item Irsko
\item spjat s dekadentním stylem
	\begin{itemize}
	\item Dorian Gray je přesvědčen, že důležité je umění, krása, mladost, povrchové jevy, oproti morálním hodnotám
	\end{itemize}
\item román \textbf{Obraz Doriana Graye}
	\begin{itemize}
	\item zachycuje soudobou společnost v Londýně
	\item hlavní postava Dorian Gray se stane modelem pro malíře
	\item je okouzlující, portrét je nádherný \ra debata o tom, že Dorian bude na obraze vždy krásný, ale ve skutečnosti ne
	\item sir Henri Wotton obviňuje Graye, ten se stává cynickým, nakonec se zachová špatně k dívce, změny se dějí v obraze a ne na něm  
	\end{itemize}
\item strávil nějaký čas ve vězení kvůli homosexualitě
\item drama -- konverzační a situační humor
\item komedie \textbf{Jak je důležité míti Filipa}
\item komedie \textbf{Vějíř lady Windermehrové}
\item komedie \textbf{Ideální manžel}
\end{itemize}


\end{document}