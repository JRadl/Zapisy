\title{Světová meziválečná literatura}
\documentclass[10pt,a4paper]{article}
\usepackage[utf8]{inputenc}
\usepackage[czech]{babel}
\usepackage{amsmath}
\usepackage{amsfonts}
\usepackage{amssymb}
\usepackage{chemfig}
\usepackage{geometry}
\usepackage{wrapfig}
\usepackage{graphicx}
\usepackage{floatflt}
\usepackage{hyperref}
\usepackage{fancyhdr}
\usepackage{tabularx}
\usepackage{makecell}
\usepackage{csquotes}
\usepackage{footnote}
\usepackage{movie15}
\MakeOuterQuote{"}

\renewcommand{\labelitemii}{$\circ$}
\renewcommand{\labelitemiii}{--}
\newcommand{\ra}{$\rightarrow$ }
\newcommand{\x}{$\times$ }
\newcommand{\lp}[2]{#1 -- #2}
\newcommand{\timeline}{\input{timeline}}


\geometry{lmargin = 0.8in, rmargin = 0.8in, tmargin = 0.8in, bmargin = 0.8in}
\date{\today}
\author{Jakub Rádl}

\makeatletter
\let\thetitle\@title
\let\theauthor\@author
\makeatother

\hypersetup{
colorlinks=true,
linkcolor=black,
urlcolor=cyan,
}



\begin{document}
\maketitle
\tableofcontents
\begin{figure}[b]
Toto dílo \textit{\thetitle} podléhá licenci Creative Commons \href{https://creativecommons.org/licenses/by-nc/4.0/}{CC BY-NC 4.0}.\\ (creativecommons.org/licenses/by-nc/4.0/)
\end{figure}
\newpage

\section{Umělecké směry}
\begin{itemize}
\item ovlivněno válkami -- mnoho mrtvých, přerozdělení světa, změna společnosti, španělská chřipka
\item literatura reaguje na rychlost doby
\item svět je roztříštěný, změněný, konec jistoty -- literatura hledá směry pro vyjádření, pochopení a popis světa
\item vznik \textbf{avantgardních směrů}
	\begin{itemize}
	\item avantgardní = experimentální, modernistický, výstřední
	\item projevy v obsahu -- nová témata, věci mimo realitu, z vědomí člověka
	\item projevy ve formě -- kompozice, chaotičnost, jazykové experimenty, propojování grafiky
	\end{itemize}
\end{itemize}

\paragraph{Impresionismus}
\begin{itemize}
\item není tak čistý jako dříve
\end{itemize}

\paragraph{Fauvismus}
\begin{itemize}
\item \textit{le fauve} = šelma
\item především malířství -- obrazy tvořené velkými barevnými plochami
\item barvy a tvary neodpovídají realitě, nemají detaily, útočí na smysly
\item Henri Matisse, Paul Gaugin
\end{itemize}

\paragraph{Expresionismus}
\begin{itemize}
\item vyjádření vnitřního psychického stavu \ra  citově zabarvená slova
\item malířství -- hlavně negativní emoce \ra deformace rysů, schématické kresby
	\begin{itemize}
	\item Edward Munch (Výkřik, Strach), Egon Schiele, Marc Chagall	
	\end{itemize}
\item literatura -- snaha o vyjádření pocitů člověka ztraceného v době, ztrácí se hranice mezi realitou a vědomím člověka
	\begin{itemize}
	\item Franz Kafka, Christian Morgenstern
	\end{itemize}
\end{itemize}

\paragraph{Kubismus}
\begin{itemize}
\item \textit{cubus} = krychle
\item malířství -- rozložení reality na pevné geometrické tvary a složení zpět ze střípků, mnoho hran
	\begin{itemize}
	\item velký odklon od reality, zobrazení více pohledů na jednu věc, někdy 2D
	\item Georges Braque, Pablo Picasso
	\end{itemize}
\item literatura -- mnoho úhlů pohledu do různých postav
	\begin{itemize}
	\item realita rozkládána na části a obrazy, znovu složena ze střípků, obtížné na pochopení
	\item roztříštěná kompozice, rušení hranic prózy a poezie, důležitost grafiky
	\end{itemize}
\end{itemize}


\paragraph{Futurismus}
\begin{itemize}
\item zaměřen na budoucnost, "nezjímá nás co bylo, ale co je a bude"
\item \textbf{Filippo Tommaso Marinetti}
	\begin{itemize}
	\item Francie
	\item \textbf{Manifest futurismu}
		\begin{itemize}
	 	\item 1909 -- publikován v časopisu Figaro
	 	\item provokativní a kontroverzní, vzbudil pozornost a emoce
		\item chce válku, opovrhovat vším co bylo, symbol oproštění od minulosti \ra zneužito fašisty v Itálii
		\end{itemize}
	\item básnická sbírka \textbf{Osvobozená slova}
		\begin{itemize}
		\item \textbf{Strašně rudé slunce} -- graficky nejednotné, slova nejsou ve větách, osvobozená od syntaxe, mnoho interpretací
			\begin{itemize}
			\item mnoho možných interpretací -- válka, soužití mužů a žen, cestování o samotě s podváděním, zmenšování
			\end{itemize}
		\end{itemize}
	\end{itemize}
\item Rusko
	\begin{itemize}
	\item distancuje se od Francie, ale projevuje stejné
	\item manifesty \textbf{Zdechlá luna}, \textbf{Políček veřejnému vkusu}
	\end{itemize}
	\item \textbf{Vladimir Chlebnikov}
	\item \textbf{Vladimir Majakovský}
\end{itemize}

\paragraph{Kubofuturismus}
\begin{itemize}
\item míchanice kubismu a futurismu
\end{itemize}


\paragraph{Dadaismus}
\begin{itemize}
\item reakce na první světovou válu, je potřeba odlehčit situaci, hrát si
\item Tristan Tzara -- náhodné slovo "dada" (= houpací koník) v Encyklopedii
\item umění je cokoli a umělcem může být kdokoli -- je nuté si hrát, náhoda, chaos (odráží chaos reálného světa)
\item tvorba rozstříháním náhodného textu
\item ovlivnění světa -- u nás Osvobozené divadlo, vznik poetismu
\end{itemize}


\paragraph{Surealismus}
\begin{itemize}
\item existuje svět nad realitou, který si můžeme uvědomit pod vlivem drog, hypnózy, meditace
\item odsuzují román
\item vychází z psychoanalýzy Zigmunda Freuda
\item francouzští básníci André Breton, Louis Aragon
\item španělský malíř Salvador Dalí, režisér Luis Buñuel
\end{itemize}

\paragraph{Civilismus}
\begin{itemize}
\item oslavuje civilizační výdobytky, které usnadňují člověku život
\end{itemize}

\paragraph{Vitalismus}
\begin{itemize}
\item oslavuje radosti života, tvrdí, že bychom si ho měli užívat
\end{itemize}

\paragraph{Naturalismus}
\begin{itemize}
\item oslavuje přírodní krásy, přirozený způsob života
\end{itemize}

\paragraph{Unanimismus}
\begin{itemize}
\item myšlenka kolektivního duchu skupiny lidí, kteří spolu tráví mnoho času
\end{itemize}

\paragraph{Pragmatismus}
\begin{itemize}
\item veškeré jednání, myšlení, poznání by mělo vést k praktickému cíli
\item důležité je to, co je prospěšné pro jedince (moderně deformováno do "účel světí prostředky")
\item Karel Čapek -- není jedna pravda, není jedna cesta
\end{itemize}

\paragraph{Marxismus}
\begin{itemize}
\item levicová orientace, vychází z ekonomické situace a vizí Karla Marxe
\item nesmiřitelné třídy společnosti -- nespravedlnost rozdělení
\end{itemize}

\paragraph{Secese}
\begin{itemize}
\item více názvů (Jugendstil, \ldots)
\item snaha o zkulturnění prostředí, návrat k přírodě, přírodním, rostlinným motivům, od nových technických
\item projevuje se nejen v literatuře, (pražské hl. Nádraží)
\end{itemize}

\paragraph{Akméismus}
\begin{itemize}
\item protipól moderny, reakce na množství nových směrů
\item ruská podoba neoklasicismu (vychází z klasicismu), snaha o jednotu, kázeň, povinnost, rozum, intelekt nad pocity
\end{itemize}

\paragraph{Imažinismus}
\begin{itemize}
\item básnický obraz, metafora, návrat k tradicím, hodnotám, víra, lidové písně
\item osobní pocity, bohémská témata, volný verš
\end{itemize}


\section{Rusko}

\subsection{Vladimir Majakovskij (1893--1930)}
\begin{itemize}
\item sebestředný, používá svoje jméno v básních
\item spoluautor manifestu \textbf{Políček veřejnému vkusu}
\item hlavní autor sovětského Ruska, oficiální básník revoluce
\item zvučná, deklamovatelná poezie
\item pracuje s jazykem, vymýšlí neologismy, používá ironii a sarkasmus
\item spáchal sebevraždu
\item poéma \textbf{Oblak v kalhotách} -- předpovídá revoluci
\item poéma \textbf{Vladimir Iljič Lenin}
\item milostné poémy \textbf{Miluji}, \textbf{O tom}
\item hra \textbf{Mystérie--Buffa}
	\begin{itemize}
	\item středověká lyrika, biblické téma, antická literatura -- šašek nutný pro kritiku a komedii
	\item příběh o potopě světa: 7 párů čistých, 7 nečistých \ra 7 nečistých přežije potopu, proletariát přebere vládu
	\end{itemize}
\end{itemize}

\subsection{Sergej Jesenin (1895--1925)}
\begin{itemize}
\item futurismus
\item pochází z venkova, známý s Majakovským (též spácahal sebevraždu), krátké manželství s tanečnicí, další manželství bez předchozího rozvodu, nesouhlasí se systémem
\item pomezí mezi tradiční tvorbou a uměleckou modernou, nesouhlasí se systémem
\item z počátku spojován s imažinismem \ra později cynismus, bouření se, nesouhlas se společností
\item poéma \textbf{Anna Sněginová}
	\begin{itemize}
	\item občanská válka zasahuje do milostného vztahu dvou mladých lidí
	\item několikrát se potkají, Anna pak utíká do Anglie
	\item autobiografické prvky
	\end{itemize}
\item \textbf{Moskevské hospody}, \textbf{Verše skandalistovy}
	\begin{itemize}
	\item personifikace s přírodními motivy
	\item milostná lyrika
	\item smutný osud jednotlivce, který je zmítán historickými událostmi
	\end{itemize}
\item báseň \textbf{Chuligánova zpověď}
	\begin{itemize}
	\item subjekt vzpomíná, jak dobře mu v dětství bylo na venkově
	\item teď už považuje venkov za zaostalý, aby něčeho dosáhl, musí odejít
	\item patos \ra shozen ironií
	\end{itemize}
\end{itemize}

\subsection{Michail Šolochov}
\begin{itemize}
\item vrcholný klasik sovětské literatury
\item románová epopej \textbf{Tichý Don}
	\begin{itemize}
	\item sleduje osudy kozáka Grigorije Melechova bojujícího v první světové válce
	\item Melechov přechází od Rudých k Bílím a zpět, nechce být ani na jedné straně
	\item popisuje hrůzy války a tragickou lásku mezi Grigorijem a Aksiňjou
	\end{itemize}
\item \textbf{Donské povídky}
\item novela \textbf{Osud člověka}
	\begin{itemize}
	\item hlavní hrdina přišel ve válce o rodinu, později nachází smysl života v péči o válečného sirotka+
	\end{itemize}
\end{itemize}

\subsection{Maxim Gorkij (1868--1936)}
\begin{itemize}
\item vlastním jménem Alexej Maximovič Peškov
\item osiřel, těžké dětství, od deseti let pracoval 
\item píše o životě "bosáků" - lidí žijících ne vlastní vinou na okraji společnosti
\item povídka \textbf{Makar Čudra} 
\item soubor \textbf{Črty a povídky} 
\item verše \textbf{Píseň sokolu}
\item dramata \textbf{Měšťáci}, \textbf{Na dně}
	\begin{itemize}
	\item zachycují svět v kontrastech, důraz na určitou společenskou skupinu, ne na individuální postavy
	\end{itemize}
\item román \textbf{Forma Gordějev}
	\begin{itemize}
	\item hlavním hrdinou je syn obchodníka, který se nechce přizpůsobit kupecké morálce a nakonec končí v blázinci.
	\end{itemize}
\item 1905 vyhnán do USA režimem (sympatizoval s levicovými radikály)
\item román \textbf{Matka}
	\begin{itemize}
	\item líčí proměnu obyčejné ženy v uvědomělou revolucionářku
	\item snaha o typizaci a schematizaci
	\end{itemize}
\item román \textbf{Podnik Artamonových} - zachycuje vzestup, stagnaci a pád tří generací podnikatelů
\item román \textbf{Život Klima Samgina}
	\begin{itemize}
	\item intelektuál, který sympatizuje s reformou, ale nedokáže se ztotožnit s radikálním hnutím
	\item "zbytečný člověk"
	\end{itemize}
\item drama (první díl zamýšlené pentalogie) \textbf{Jegor Bulyčov a ti druzí}
	\begin{itemize}
	\item těžce nemocný kupec, který se nemůže ztotožnit s hodnotami své sociální vrstvy
	\end{itemize}
\item díla byla interpretována zjednodušeně, pozdější tvorba označena za socialistický realismus, zvolen předsedou jednotné spisovatelské organizace
\item nemluvil o gulazích
\item pravděpodobně zavražděn na pokyn Stalina
\end{itemize}	
	
\subsection{Michail Bulgakov (1891--1940)}
\begin{itemize}
\item díla vznikala jako "tajná" literatura, protože bylo jasné, že za Stalina nebudou vydána \ra vydána desetiletí po smrti 
\item společně s Pasternakem patří k tzv. "vnitřní emigraci"
\item původně vystudoval medicínu, pak se stal spisovatelem
\item román \textbf{Bílá garda}
	\begin{itemize}
	\item popisuje tragicko-groteskní osudy obětí boje bílých proti rudým
	\end{itemize}
\item drama \textbf{Dni Turbinových} -- divadelní verze Bílé gardy
	\begin{itemize}
	\item jako jedna z mála her se hrála v divadle MCHAT za jeho života
	\end{itemize}
\item drama \textbf{Moliére}, \textbf{Puškin}, \textbf{Poslední dny}
	\begin{itemize}
	\item vrchol Bulgakovovy dramatické tvorby
	\item konflikt umělce s mocí
	\end{itemize}
\item novela \textbf{Psí srdce}
	\begin{itemize}
	\item vydaná 1987, zločinný experiment spočívající v implantaci psího srdce do člověka \ra "sovětský člověk" - zlověstná karikatura člověka
	\end{itemize}
\item román \textbf{Divadelní román}
	\begin{itemize}
	\item vyšel 1965, popisuje atmosféru v MCHAT
	\item příběh o spisovateli a jeho knize vydané napospas zvůli a náhodě
	\end{itemize}
\item román o ďáblovi \textbf{Mistr a Markétka}
	\begin{itemize}
	\item pracoval na něm dvanáct let, upravoval ho i v posledních dnech života
	\item zveřejněn 1966, silně cenzurovaný
	\item prolíná biblický příběh o Pilátovi s řáděním ďábla Wolanda v Moskvě na přelomu 20. a 30. let
	\item inspirace v Goethově Faustovi, u N. V. Gogola
	\end{itemize}
\end{itemize}

\subsection{Boris Pasternak (1890--1960)}
\begin{itemize}
\item legendární básník i prozaik, překladatel Shakespeara
\item rodina významného malíře, matka klavíristka, studoval hudbu
\item věnoval se filozofii
\item tvorba poezie z počátku kolísá mezi symbolismem a futurismem
\item sbírky \textbf{Život - má sestra}, \textbf{Druhé zrození}
	\begin{itemize}
	\item úvahovost, komorní ladění \ra protiváha k politické tvorbě Majakovského
	\end{itemize}
\item poema \textbf{Poručík Šmidt}
\item novela \textbf{Malá Luversová}
	\begin{itemize}
	\item představuje svět jako záznam nekonečně proměnlivých stavů hlavní postavy
	\end{itemize}	 
\item lyrická autobiografie \textbf{Glejt}
\item román \textbf{Doktor Živago}
	\begin{itemize}
	\item zachycuje složité postavení ruské inteligence, ztělesnění mužem, který protestuje proti ponižování lidskosti "všeho živého" \ra Živago
	\item sleduje osudy Živaga od dětství přes 30 let, vývoj jeho charakteru, názorů a vztahů
	\item Živago nakonec umírá, jeho láska Lara se umírá v koncentračním táboře
	\item mnoho lyrických pasáží
	\end{itemize}
\end{itemize}

\subsection{Ivan Bunin (1870--1953)}
\begin{itemize}
\item jeho krátké prózy jsou mozaikou systematicky poskládaných detailních "fotografií" 
\item jak v tvorbě tak v životě uznával pouze věci, které jsou vyjímečné, jedinečné, neopakovatelné
\item hlavním tématem tvorby je zanikání, hrdinové jsou zranitelní, mají sklon k tragickým činům, jsou poraženi, rezignují. Bunin je neomlouvá, je nemilosrdný
\item povídka \textbf{Antonovská jablka}
	\begin{itemize}
	\item popisuje mizející "šlechtická hnízda" 
	\end{itemize}
\item novely \textbf{Vesnice}, \textbf{Suchodol} -- zachycují obraz ruské vesnice
\item novela \textbf{Pán ze San Franciska}
	\begin{itemize}
	\item kritizuje podnikatele, kapitalisty
	\end{itemize}
\item 1920 odchází do exilu do Francie
\item milostné novely \textbf{Případ korneta Jelagina}, \textbf{Míťova láska}, \textbf{Temné aleje}

\end{itemize}


\section{Rakousko}
\subsection{Christian Morgenstern (1871--1914)}
\begin{itemize}
\item Rakušan, předchůdce dadaismu, surrealismu, expresionismu
\item zakladatel\textbf{ nonsensové poezie }\ra hravá poezie, nemá logiku
	\begin{itemize}
	\item hraje si se zvukem (onomatopoie)
	\item stylizace textu do obrazu \ra \textbf{kaligram}
	\end{itemize}
\item sbírka \textbf{Šibeniční písně}
	\begin{itemize}
	\item groteskní slovní hříčky
	\item grafické ztvárnění
	\item \textbf{Noční rybí zpěv}
		\begin{itemize}
		\item báseň poskládaná z pomlček a vlnek
		\end{itemize}
	\end{itemize}
\end{itemize}

\section{Německo}
\begin{itemize}
\item po nástupu Hitlera odešlo mnoho spisovatelů do exilu (i do ČSR)
\item snaha o vytvoření protiváhy k Hitlerovu režimu
\end{itemize}

\subsection{Erich Maria Remarque (1898--1970)}
\begin{itemize}
\item román \textbf{Na západní frontě klid}
	\begin{itemize}
	\item "pokus podat zprávu o generaci, která byla zničena válkou, i když unikla jejím granátům"
	\item vychází z vlastních zkušeností na vojně
	\item \textbf{Pavel Bäumer} pod vlivem propagandy dobrovolně se spolužáky narukoval, popisuje šikanující výcvik a průběh války, smrt kamarádů, nakonec sám umírá
	\end{itemize}
\item román \textbf{Cesta zpátky} -- volné pokračování
\item odchází do Švýcarska, později do USA
\item další romány věnované osudům emigrantů a obětem koncentračních táborů
\end{itemize}

\subsection{Heinrich Mann (1871--1950)}
\begin{itemize}
\item snaha o zachycení zhoubných rysů měšťácké mentality
\item snaha o rozvedení typu poddaného-tyrana
\item román \textbf{Profesor Neřád}
	\begin{itemize}
	\item \textbf{profesor Raata} je přísný profesor, který vyžaduje absolutní poslušnost a trestá za nejmenší prohřešky
	\item zamiluje se do podprůměrné zpěvačky s nevalnou morálkou
	\item po odhalení vztahu se s ní ožení a z cesty svádí i další občany města
	\end{itemize}
\item román \textbf{Poddaný}
	\begin{itemize}
	\item \textbf{Diederich Hessling} -- rozporný vztah s otcem -- bojí se ho, ale vzhlíží k němu
	\item podřizuje se přísnému režimu otce i školy, protože očekává, že tím na oplátku získá část moci
	\end{itemize}
\item romány \textbf{Mládí krále Jindřicha IV.}, \textbf{Zrání krále Jindřicha IV.}
	\begin{itemize}
	\item tolerantní humanisticky založený panovník, který dokázal ve Francii 16. století ukončit náboženské války a nastolit mír
	\end{itemize}
\end{itemize}

\subsection{Thomas Mann (1875--1955)}
\begin{itemize}
\item ovlivněn A. Schopenhauerem, F. Nietzschem a R. Wagnerem \ra pesimismus, spojení umění s chorobou, fascinace smrtí, tendence vnímat skutečnost v protikladech
\item román \textbf{Buddenbrookovi}
	\begin{itemize}
	\item líčí vývoj a úpadek jedné rodiny
	\item původně prosperující patricijská rodina, během dvou generací ztrácí obchodní a životní sílu, naopak ale kulturně a duchovně roste. Členové se zajímají o filozofii, hudbu a divadlo
	\item čím více se postavy upínají k duchovnu, tím více psychicky a fyzicky chřadnou
	\end{itemize}
\item v další tvorbě komplikovaná kombinace stylů a vyprávěcích prvků, hluboká psychologie, parodie
\item novela \textbf{Smrt v Benátkách}
	\begin{itemize}
	\item \textbf{Gustav Aschenbach} -- pracovitý a vážený spisovatel, umělec, jež opovrhuje průměrností a hledá opojení a vášeň
	\item setká se s chlapcem \textbf{Tadziem}, vnímá ho jako ztělesnění krásy
	\item touží po absolutním splynutí s ním, je pohlcen vášní
	\item s chlapcem nikdy ani nepromluví, zemře na choleru, ale šťastný a uvolněný
	\end{itemize}
\end{itemize}

\subsection{Hermann Hesse (1877--1962)}
\begin{itemize}
\item hrdina -- romantický poutník, který z měst prchá do lesů, kde hledá a nachází nachází pocit sounáležitosti se světem a všemi živými tvory
\item později ovlivnění východními naukami \ra smíření se smrtí
\item román \textbf{Stepní Vlk}
	\begin{itemize}
	\item \textbf{Harry Haller} -- nezakotvený spisovatel bojující s rozpolcením své osbnosti na spořádaného občana a nezkrotného vlka
	\item během pouti "chutnající po nesmyslu a zmatku" se setkává se svým druhým já -- Hermínou
	\item skrze ni poznává lákadla a svody světa
	\item nakonec se ocitá v Magickém divadle, díky kterému překoná svou životní krizi
	\end{itemize}
\item román-esej \textbf{Hra skleněných perel}
	\begin{itemize}
	\item zachycuje duchovní elitu, která si vytvořila ideální mravně přísné a ušlechtilé společenství, jež ale pomíjí to co se děje v normálním světě
	\end{itemize}
\end{itemize}

\subsection{Lion Feuchtwanger (1884--1958)}
\begin{itemize}
\item židovský romanopisec a dramatik, překladatel
\item za Hitlera emigroval od Francie
\item humanisticky a ostře protifašisticky orientovaný spisovatel, znamenitý vypravěč
\item román \textbf{Žid Süss}
	\begin{itemize}
	\item popisuje 30. léta 18. století v Německu rozdrobeném na řadu států, které mezi sebou bojují o moc
	\item hlavní postavou je schopný, ale bezvýznamný židovský obchodník Josef Openheimmer Süss
	\end{itemize}
\item \textbf{Židovka z Toleda}
	\begin{itemize}
	\item popisuje soužití Židů, křesťanů a muslimů ve středověkém Španělsku
	\item námětem byl milostný příběh kastilského krále Alfonse a židovský dívky Raquel, dcery králova ministra
	\end{itemize}
\end{itemize}



\section{Francie}
\subsection{Guillaume Apollinaire (1880--1918)}
\begin{itemize}
\item zemřel na španělskou chřipku po zásahu granátem
\item básník a prozaik, tvořil v duchu moderních směrů
\item ovlivněn Picasem, Braquem, Bretonem, inspiroval mnoho spisovatelů (i českých)
\item neodmaturoval, živil se vlastní tvorbou a úřednictví
\item povídky \textbf{Kacíř a spol}
	\begin{itemize}
	\item \textbf{Pražský chodec}
	\end{itemize}
\item román \textbf{Zahnívající kouzelník}
	\begin{itemize}
	\item[\ra] Vítězslav nezval: Podivuhodný kouzelník 
	\end{itemize}
\item sbírka \textbf{Alkoholik} (kubofuturismus)
	\begin{itemize}
	\item snaha o propojení minulosti, přítomnosti, různých míst ve světě \ra nemá jednotný motiv
	\item báseň \textbf{Pásmo}
		\begin{itemize}
		\item vychází z něj útvar pásmo
			\begin{itemize}
			\item lyrický žánr
			\item polytématické, časové skoky, prolínání časových linií, bez interpunkce
			\item proud vědomých myšlenek, řetězení metafor
			\end{itemize}
		\end{itemize}
	\end{itemize}
\item sbírka \textbf{Kaligramy} (kubofuturismus)
	\begin{itemize}
	\item vniká ve válečných letech
	\item střípky rozhovorů, situací z lidského života, inspirující věty vytržené z kontextu
	\end{itemize}
\item hra \textbf{Prsy Tirésiovy} (surealismus)
	\begin{itemize}
	\item odehrává se v Zanzibaru, hlavní postava Tereza se rozhodne emancipovat
	\item dohnáno k absurditě, kdy se Tereza rozhodne stát mužem Tirésiem
	\item manžel se rozhodne převzít naopak její roli
	\item nakonec se vrací do normálu  m
	\end{itemize}
\end{itemize}

\subsection{Jacques Prévert (1900--1977)}
\begin{itemize}
\item užíval si života \ra hravost \ra dadaismus (surealismus), nemá rád klišé, prázdné fráze
\item neměl zájem o uměleckou kariéru, psal pro radost \ra ztráta mnoha textů
\item sbírka \textbf{Slova}
	\begin{itemize}
	\item báseň \textbf{Snídaně}
		\begin{itemize}
		\item výčet akcí při snídani
		\item paralelismus -- "aniž by na mě promluvil, aniž by se na mě podíval"
		\end{itemize}
	\end{itemize}
\end{itemize}

\subsection{Romain Rolland (1866--1944)}
\begin{itemize}
\item studoval historii \ra profesor dějin hudby na Sorboně
\item v díle se vyskytuje patos, metafory, symboly, charakterově vyhraněné postavy, postavy s "čistým pohledem a čistým srdcem"
\item do díla se promítá pacifismus
\item divadelní trilogie \textbf{Divadlo Revoluce: Vlci}, \textbf{Danton}, \textbf{Čtrnáctý červenec}
\item životopisy \textbf{Beethoven}, \textbf{Gándhí}
\item román-řeka \textbf{Jan Kryštof}
	 \begin{itemize}
	 \item pojednává o životě hudebního skladatele, zahrnuje životopisné prvky z života Beethovena a dalších skladatelů
	 \item vyjadřuje se k současným společenským otázkám, socialistické myšlenky
	 \end{itemize}
\item válečná novela \textbf{Petr a Lucie}
\item román \textbf{Dobrý člověk ještě žije}
	\begin{itemize}
	\item líčí radosti a strasti života moudrého truhláře na přelomu 16. a 17. století
	\end{itemize}
\end{itemize}

\subsection{Henri Barbusse}
\begin{itemize}
\item prozaik, novinář, politik -- komunistická strana
\item původně literární kritika (Emile Zola) a romantické básně, později protiválečné naturalistické romány
\item založil protiimperialistickou organizaci Clarté (Jasnost), která existuje do dnes
\item román \textbf{Oheň}
	\begin{itemize}
	\item román o hrůzách války
	\item původně válku obdivoval jako boj za mír \ra narukoval \ra zjistil skutečnost
	\end{itemize}
\item román \textbf{Peklo}
	\begin{itemize}
	\item naturalistické zobrazení moderního velkoměsta
	\end{itemize}
\item román \textbf{Jasno}
	\begin{itemize}
	\item odsuzování války
	\end{itemize}
\item eseje \textbf{Světlo v propasti}, \textbf{Dopisy duševním pracovníkům}
\item reportáže \textbf{Gruzie}, \textbf{Rusko}, \textbf{Různé zprávy}
\item povídky \textbf{Šílenství lásky}
\end{itemize}



\section{USA}
\begin{itemize}
\item básnické uskupení modernistů z Británie a USA
\end{itemize}

\subsection{Ezra Pound (1885--1972)}
\begin{itemize}
\item Američan, který žil delší dobu v Evropě \ra obeznámen s místní tvorbou
\item obohacuje americkou literaturu evropskými myšlenkami
\item hlavní představitel hnutí
\item kontroverzní postava -- podporuje vyostřené nacionalistické ideály \ra fašismus, antisemitismus, účastnil se protiamerického rozhlasového vysílání v Itálii
\item označen za velezrádce (trest zmírněn kvůli nesvéprávnosti)
\item básně mají být přesné, jasné, nenarušované metaforami i přesto hledá hlubší významy za realitou
\item \textbf{vorticismus} -- vyjadřuje chaos světa, poezie mu má vtisknout nějaký řád
\item sbírka \textbf{Zpěvy} (Cantos)
	\begin{itemize}
	\item celoživotní dílo
	\item volný verš
	\item pohled na historii lidstva od starověku po moderní dějiny USA
	\end{itemize}
\end{itemize}

\subsection{Ernest Hemingway (1899--1961)}
\begin{itemize}
\item jeden z nejpopulárnějších spisovatelů 20. století
\item 1954 -- získal Nobelovu cenu za literaturu
\item otec lékař, milovník přírody, rybář, lovec, matka přísná, zbožná
\item novinář, dobrovolník Červeného kříže v Itálii, válečný korespondent ve Španělsku, po válce se přestěhoval do Paříže, později na Kubu, ztrácí tvůrčí schopnost, deprese, zastřelil se
\item psal hodně povídek -- jsou v nich patrné jeho záliby a názorový vývoj
\item povídky \textbf{Můj táta}, \textbf{Vojákův návrat}, \textbf{Kočka v dešti}, \textbf{Čistý, dobře osvětlený podnik}, \textbf{Krátké štěstí Felixe Macombera}, \textbf{Sněhy na Kilimandžáru}
\item  román \textbf{Sbohem armádo}
	\begin{itemize}
	\item poručík \textbf{Frederick Henry} je zmýlen za zběha a málem popraven, obrací se k válce zády, později žije ve Švýcarsku s Catherin, ta porodí mrtvého chlapce a sama brzy umírá
	\end{itemize}
\item román \textbf{Mít a nemít}
	\begin{itemize}
	\item popisuje ztroskotání muže snažícího se si zachovat nezávislost a slušnou existenci
	\item "Člověk sám nemá tu nejmizernější a nejzkurvenější šanci"
	\end{itemize}
\item román \textbf{Komu zvoní hrana}
	\begin{itemize}
	\item sleduje osudy dobrovolníka Roberta Jordana na republikánské straně občanské války ve Španělsku
	\item vztah s partyzánkou Marií
	\end{itemize}
\item novela \textbf{Stařec a Moře}
	\begin{itemize}
	\item líčí zápas kubánského rybáře s obrovskou rybou
	\item hodiny zápasí s zachycenou rybou, která odtáhla jeho loď na širé moře, pak bojuje se žraloky, nakonec dojede do přístavu s kostrou ryby
	\item symbolický příběh -- rybář = člověk, ryba = příroda, žraloci = zlo
	\item typické pro Hemingwaye:
		\begin{itemize}
		\item hlavní hrdina -- muž vystavený mravní i fyzické zkoušce
		\item přehledný děj se odehrává mimo Ameriku
		\item filosofické a společenské otázky
		\item důležitý vnitřní monolog hrdiny
		\end{itemize}
	\end{itemize}
\end{itemize}

\subsection{Francis Scott Fitzgerald (1896--1940)}
\begin{itemize}
\item román \textbf{Velký Gatsby}
	\begin{itemize}
	\item vyvrací to že bohatství otevírá "bezbřehé" možnosti
	\end{itemize}
\item román \textbf{Něžná je noc}
	\begin{itemize}
	\item zachycuje duševní rozklad mladého psychiatra, který si vzal bohatou pacientku
	\item proměna "amerického snu" v noční můru
	\end{itemize}
\item povídky \textbf{Povídky jazzového věku}
\end{itemize}

\subsection{John Steinbeck (1902--1968)}
\begin{itemize}
\item romance \textbf{Pláň Tortilla}
	\begin{itemize}
	\item o životě party dobrosrdečných povalečů
	\end{itemize}
\item romance \textbf{Na plechárně}
	\begin{itemize}
	\item navazuje
	\end{itemize}
\item novela \textbf{O myších a lidech}
	\begin{itemize}
	\item o zhroucení životního snu svérázné dvojice kovbojů
	\end{itemize}
\item román \textbf{Na východ od ráje}
	\begin{itemize}
	\item kombinace kroniky, autobiografie a vymyšleného příběhu
	\item obsahuje aluze na Bibli
	\end{itemize}
\end{itemize}

\subsection{William Faulkner}
\begin{itemize}
\item hlavní představitel "jižanské literatury" -- inspirován rodinnou historií
\item román \textbf{Sartoris}
	\begin{itemize}
	\item popisuje život v městě Jefferson v okrese Yoknapatawpha (inspirace Oxford, Mississippi)
	\item postavy přecházejí od dalších děl
	\item popisuje nejen problémy Jihu, ale také touhu vymanit se z konvencí a žít přirozený život
	\end{itemize}
\item román \textbf{Hluk a vřava}
	\begin{itemize}
	\item metoda proudu vědomí -- složité schéma více vyprávěcích rovin
	\end{itemize}
\item román \textbf{Když jsem umírala}
	\begin{itemize}
	\item přesun umírající matky k místu pochování svých příbuzných
	\end{itemize}
\item román \textbf{Svatyně}
	\begin{itemize}
	\item komerčně nejúspěšnější
	\end{itemize}
\end{itemize}


\section{Pražská německá literatura}
\item autoři německého původu, kteří se narodili v Česku, či sem odešli do exilu
\subsection{Franz Kafka (1883--1924)}
\begin{itemize}
\item židovského původu po matce, psal německy, špatný vztah s otcem, zemřel na tuberkulózu
\item vystudoval práva, psaní pro něj byla autoterapie, nechtěl svá díla vydat
\item čtyřikrát zasnoubený, ze strachu se ale ani jednou neoženil
\item za života publikoval jen několik povídek, zbytek publikoval po smrti Max Brod
\item nedokončený román \textbf{Amerika} 
	\begin{itemize}
	\item časové skoky -- chybějící kapitoly, které zamýšlel dopsat
	\end{itemize}
\item opakované motivy provinění a neúměrně vysokého trestu (pod. Erben), postava často sama nerozumí svému provinění
\item postavy mají špatný vztah ke společnosti, jsou někým utlačovány, neumí se bránit	

\end{itemize}

\subsection{Franz Werfel (1890--1945)}
\begin{itemize}
\item ovlivněn expresionismem, Kafkou
\item román \textbf{Čtyřicet dnů}
	\begin{itemize}
	\item vyjadřuje odpor vůči bezpráví 
	\item chtěl poukázat na utrpení křesťanů, Tureckou genocidu arménské menšiny \ra nezáměrná alegorie proti fašismu
	\end{itemize}
\item \textbf{Ne vrah, ale zavražděný je vinen}, \textbf{Sjezd abiturientů}
	\begin{itemize}
	\item kriticky popisuje morálku společnosti	
	\end{itemize}
\end{itemize}

\subsection{Gustav Meyrink (1868--1932)}
\begin{itemize}
\item román \textbf{Golem}
	\begin{itemize}
	\item odehrává se v židovském ghetu, přetváří prastarou legendu a rozvíjí oblíbené novoromantické téma dvojnictví
	\end{itemize}
\end{itemize}

\subsection{Max Brod (1884--1968)}
\begin{itemize}
\item vydával Kafkova díla
\end{itemize}

\section{Británie}
\subsection{Antoine de Saint-Exupéry (1900--1944)}
\begin{itemize}
\item letec, čerpal ze zkušeností
\item byl sestřelen v druhé světové válce při průzkumném letu, neútočil zpátky, pravděpodobně se nechal sestřelit, protože už neměl dál létat kvůli zdravotnímu stavu
\item filosofická pohádka \textbf{Malý princ}
	\begin{itemize}
	\item lehce uchopitelný 
	\item pilot havaruje v poušti a pokouší se opravit letadlo (z vlastní zkušenosti), potká Malého prince, z malé planetky, ze které odešel a hledá odpovědi
	\item princ postupně putuje po planetách a potkává různé lidi symbolizující různé vlastnosti
	\item popisuje rozpor mezi světem dospělých a světem dětí
	\item úryvek str. 97 -- 
	\end{itemize}	 
\end{itemize}

\subsection{David Herbert Lawrence (1885--1930)}
\begin{itemize}
\item román \textbf{Milenec lady Chatterleyové}
	\begin{itemize}
	\item kontroverzní sexuální téma \ra skandál
	\item manžel ochrnul \ra neschopný pohlavního styku \ra psychické dopady na něj i na manželku
	\end{itemize}
\end{itemize}

\newpage
\section{Reference}
\begin{enumerate}
\item KOMANEC, Jeroným. Poezie ve světě do 2. světové války [online]. Brno, 2020 [cit. 2020-3-5].
\item TEŠNAR, Michal. Poezie ve světě od počátku století do 2. světové války. In: Čeština [OneNote]. Brno, 2020 [cit. 2020-03-08].

\end{enumerate}


\end{document}	

