\title{Sloh -- 2. ročník}
\documentclass[10pt,a4paper]{article}
\usepackage[utf8]{inputenc}
\usepackage[czech]{babel}
\usepackage{amsmath}
\usepackage{amsfonts}
\usepackage{amssymb}
\usepackage{chemfig}
\usepackage{geometry}
\usepackage{wrapfig}
\usepackage{graphicx}
\usepackage{floatflt}
\usepackage{hyperref}
\usepackage{fancyhdr}
\usepackage{tabularx}
\usepackage{makecell}
\usepackage{csquotes}
\usepackage{footnote}
\usepackage{movie15}
\MakeOuterQuote{"}

\renewcommand{\labelitemii}{$\circ$}
\renewcommand{\labelitemiii}{--}
\newcommand{\ra}{$\rightarrow$ }
\newcommand{\x}{$\times$ }
\newcommand{\lp}[2]{#1 -- #2}
\newcommand{\timeline}{\input{timeline}}


\geometry{lmargin = 0.8in, rmargin = 0.8in, tmargin = 0.8in, bmargin = 0.8in}
\date{\today}
\author{Jakub Rádl}

\makeatletter
\let\thetitle\@title
\let\theauthor\@author
\makeatother

\hypersetup{
colorlinks=true,
linkcolor=black,
urlcolor=cyan,
}



\begin{document}
\maketitle
\tableofcontents
\begin{figure}[b]
Toto dílo \textit{\thetitle} podléhá licenci Creative Commons \href{https://creativecommons.org/licenses/by-nc/4.0/}{CC BY-NC 4.0}.\\ (creativecommons.org/licenses/by-nc/4.0/)
\end{figure}
\newpage


\section{Publicistický styl}


\subsection{Fejeton}
\subsection{Reportáž}
\paragraph{Rozbor: \textit{uč. 127: Kouzelné kameny angkorské}}
\begin{itemize}
\item kompozice: 
\begin{itemize}
	\item gradace (opakování akgkor, \ldots)
	\item in medias res, prézens historický
	\item ze začátku považováno za bájné místo
\end{itemize}
\item střídá objektivní a subjektivní prvky
	\begin{itemize}
	\item objektivní nutné pro reportáž
	\item subjektivní zachycují atmosféru
	\end{itemize}
\item slovní zásoba
\begin{itemize}
	\item bohatá slovní zásoba
	\item archaismy (překládáno v 30. letech 20. stol.)
	\item hodně přídavných jmen
	\item umělecké prostředky(tropy): personifikace, metafory, přirovnání, hyperboly, \ldots
	\item termíny z architektury, historie: pagody, terasy, reliéf, ochoz, basreliéf, 
	\item tropy a figury
		\begin{itemize}
		\item metafory: \textit{"posvátná hrůza"}, \textit{"žlutá skvrna"} \ldots
		\item přirovnání: \textit{"jako ohromný motýl"}, \ldots
		\item aliterace(enumerace): \textit{"Zkomolen, ztroskotán, zarostlý"}
		\item personifikace: "ze stínů vylézali smrt a zmar"
		\item perifráze: "v houstnoucím šeru, poletovali duchové"
		\item \ldots
		\end{itemize}
	\end{itemize}	
\end{itemize}


\paragraph{Reportáž}
\begin{itemize}
\item cílem je informovat, zaujmout, (ovlivnit), vyvolat emoce
\item čtenář by měl mít pocit, že se události účastní
\item fyzická přítomnost reportéra (měl by psát o něčem čeho se zúčastnil)
\item nutná faktografie, objektivní
\item zachytit atmosféru
\item zapsat dialogy
\item publicistický jazyk (s uměleckými prvky)
	\begin{itemize}
	\item spisovná čeština (kromě např. citací)
	\item promyšlená kompozice (in media res)
	\item prézens historický
	\item větné ekvivalenty
	\item obrazné vyjadřování, metaforika
	\end{itemize}
\item random poznámky
	\begin{itemize}
	\item pokud někoho cituju, uvedu jméno, případně titul
	\item nepsat fejeton! reportáž by měla být vyvážená
	\item vysvětlovat neznámé zkratky
	\item nezapomenout na závěr, odůvodnit ho
	\item držet se tématu, nezaměřit se na jeden aspekt, ale na všechny detaily celku
	\end{itemize}
\end{itemize}

\paragraph{D.ú.:} učebnice 134 úlohy 1-6 k textu
\begin{enumerate}
\item B
\item D
\item B
\item D
\item A
\item C
\end{enumerate}


\section{Lexikologie}
\begin{itemize}
\item \textbf{frazeologie} -- zkoumá ustálená slovní spojení
\item \textbf{sémantika} -- 
\item \textbf{pojmenování} -- skupina slov formulující význam (\textit{četl jsem}\textit{byl bych přestal})
	\begin{itemize}
	\item \textbf{sousloví} -- spojení více slov k označení jednoho konkrétního pojmu (\textit{tigr usurijský)}
	\item \textbf{složené pojmenování} -- širší označení(\textit{sibiřská fauna})
	\end{itemize}
\item \textbf{slovo} -- uskupení hlásek/písmen (\textit{četl}, \textit{jsem}, \textit{byl}, ...)
\item[] \textbf{multiverbizace} -- rozdělení slova na dvě (\textit{prošetřit $>$ provést šetření})
\item[] \textbf{univerbizace} -- zkrácení sousloví na jedno (\textit{obývací pokoj $>$ obývák})
\end{itemize}

\subsection{Větné pojmenovací jednotky}
\begin{itemize}
\item pochází z lidové slovesnosti, jedná se o zobecněnou lidskou zkušenost
\item \textbf{pořekadlo} -- konstatuje opakující se skutečnost (\textit{Vrána k vráně sedá, rovný rovného si hledá})
\item \textbf{pranostika} -- pořekadlo o počasí, často ve vazbě na úrodu, vychází z opakované zkušenosti s určitým obdobím (\textit{Březen, za kamna vlezem, duben, ještě tam budem.})
\item \textbf{přísloví} -- výchovný záměr (\textit{Bez práce nejsou koláče})
\item \textbf{rčení} -- slovní spojení zapojené do věty s obrazným významem (\textit{Za tebe bych dal ruku do ohně.})
\end{itemize}

\subsection{Podstatná jména}
\begin{itemize}
\item \textbf{abstraktní} -- nefyzické věci (\textit{láska}, \textit{smutek}, \textit{přátelství}, \ldots)
\item \textbf{konkrétní} -- fyzické věci (\textit{dopis}, \textit{kámen}, \ldots)
\item \textbf{látková} -- pojmenování materiálu bez zřetele na množství (\textit{mouka}, \textit{písek}, \textit{voda}, \ldots)
\item \textbf{hromadná} -- mnoho něčeho pohromadě (\textit{listí}, \textit{dobytek}, \textit{křoví}, \ldots)
\item \textbf{pomnožná} -- v základním tvaru množného čísla, význam pouze jednoho kusu, pojí se s druhovými číslovkami (\textit{nůžky}, \textit{kalhoty}, \textit{dveře}, \ldots)
\item \textbf{mnohoznačná} -- jedno pojmenování má více významů
\end{itemize}

\subsection{Čeština}
\paragraph{Spisovná}
\begin{itemize}
\item vysoká formální $\times$ hovorová
\item více, mléko, \ldots $\times$ víc, mlíko, \ldots
\end{itemize}

\paragraph{Nespisovná}
\begin{itemize}
\item obecná čeština 
\item nářečí, nadnářečí, slang, argot, hantec (propojení nářečí, němčiny a argotu)
\end{itemize}

\paragraph{Pojmy}
\begin{itemize}
\item \textbf{archaismus} -- nahrazené pojmenování
\item \textbf{historismus} -- již nepotřebné pojmenování
\item \textbf{eufemismus} -- zjěmnění
\item \textbf{disfemismus} -- opak eufemismu
\item \textbf{vulgarismus} 
\item \textbf{familiární a domácká pojmenování} -- miláčku, \ldots (též zdrobněliny)
\item \textbf{hanlivá (pejorativní) pojmenování} -- vyjadřuje záporný vztah (panděro, kokos, čokl)
\item \textbf{hyperbola} -- nadsázka
\item \textbf{ironie} -- sděluje opak, závisí na intonaci
\item \textbf{zvukově expresivní pojmenování} -- působí citově svou zvukomalbou (ťulpas, ňouma, frňák, blablanina)
\end{itemize}

\subsection{Druhy pojmenování podle významu}
\begin{itemize}
\item \textbf{homonyma} -- stejně se píší a vyslovují, ale mají úplně jiný význam (rys -- zvíře, obraz)
	\begin{itemize}
	\item úplná homonimie -- stejný slovní druh (travička -- tráva, žena, která někoho otrávila)
	\item částečný homonimie -- jiný slovní druh (pila -- nástroj, sloveso)
	\end{itemize}
\item \textbf{mnohoznačná} -- stejně se píší a vyslovují, mají různý, ale propojený význam (rys -- obraz, znak, tah obličeje)
\item \textbf{homofonie} -- stejně zní, jinak se píší (být $\times$ bít)
\item \textbf{synonyma} -- slova stejného nebo podobného významu (rychlý, svižný, hbitý)
\item \textbf{paronyma} -- zvukově podobná, s jiným významem
\end{itemize}

\end{document}