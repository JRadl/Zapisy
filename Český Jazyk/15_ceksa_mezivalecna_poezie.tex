\title{Česká meziválečná próza}
\documentclass[10pt,a4paper]{article}
\usepackage[utf8]{inputenc}
\usepackage[czech]{babel}
\usepackage{amsmath}
\usepackage{amsfonts}
\usepackage{amssymb}
\usepackage{chemfig}
\usepackage{geometry}
\usepackage{wrapfig}
\usepackage{graphicx}
\usepackage{floatflt}
\usepackage{hyperref}
\usepackage{fancyhdr}
\usepackage{tabularx}
\usepackage{makecell}
\usepackage{csquotes}
\usepackage{footnote}

\MakeOuterQuote{"}

\renewcommand{\labelitemii}{$\circ$}
\renewcommand{\labelitemiii}{--}
\newcommand{\ra}{$\rightarrow$ }
\newcommand{\x}{$\times$ }
\newcommand{\lp}[2]{#1 -- #2}
\newcommand{\timeline}{\input{timeline}}


\geometry{lmargin = 0.8in, rmargin = 0.8in, tmargin = 0.8in, bmargin = 0.8in}
\date{\today}
\author{Jakub Rádl}

\makeatletter
\let\thetitle\@title
\let\theauthor\@author
\makeatother

\hypersetup{
colorlinks=true,
linkcolor=black,
urlcolor=cyan,
}



\begin{document}
\maketitle
\tableofcontents
\begin{figure}[b]
Toto dílo \textit{\thetitle} podléhá licenci Creative Commons \href{https://creativecommons.org/licenses/by-nc/4.0/}{CC BY-NC 4.0}.\\ (creativecommons.org/licenses/by-nc/4.0/)
\end{figure}
\newpage

\section{Úvod}
\begin{itemize}
\item Československo po 1. sv. v. -- demokratické, rychle se rozvíjející
\item ve 30. letech hospodářská krize, národnostní spory, Mnichovská dohoda, protektorát
\item literatura -- impresionismus, realismus, expresionismus, kubismus, poetismus
\item architektura -- funkcionalismus
\item hudba -- Janáček, Dvořák (zhudebnil Čapkovu Věc Makropulos)
\item Jirásek a Čapek nominování na Nobelovu cenu za literaturu
\item hnutí Proletkult propagace komunismu a marxismu-leninismu \ra Čapek: Proč nejsem komunistou
\item Pátečníci -- skupina, která se scházela v pátek v Čapkově vile, novináři, demokraté
\item socialistický realismus -- odpovídá třídnímu pojetí společnosti a materialismu
\item levá fronta -- považovali za důležité posílit národní uvědomění a bojovat proti nacionalismu, fašismu a válce
\item drama -- expresionismus, růst počtu kabaretů a malých scén, avantgardní divadlo, odpor proti fašismu
	\begin{itemize}
	\item avantgardní scény -- Osvobozené divadlo (Voskovec, Werich), Divadlo D (E. F. Burian)
	\end{itemize}
\end{itemize}

\section{Reakce na 1. světovou válku}
\subsection{Jaroslav Hašek (1883--1923)}
\begin{itemize}
\item spisovatel, humorista, publicista
\item z gymnázia vyloučen za účast na demonstracei, vyučil se drogistou, vystudoval obchodní akademii, pracoval v bance, později spisovatel, publicista
\item bohémský způsob života, pokus o sebevraždu
\item politická Strana mírného pokroku v mezích zákona, kritika politiky
\item zakladatel žánru hospodská historka, cetopisné povídky, črty a humoresky, většina děl napsané v hospodě a publikována v časopisech
\item satirický román \textbf{Osudy dobrého vojáka švejka za světové války}
	\begin{itemize}
	\item satiricky popisuje 1. světovou válku
	\item hovorová čeština, vulgarismy, byrokratická a vojenská hantýrka, cizí výrazy, 
	\item nejasná dějová linie
	\item postava Švejka se vyvíjí z prosťáčka na mazaného šibala, který doslovným plněním rozkazů přivádí nadřízené k šílenosti
	\item Švejk nad válkou vítězí a ukazuje ji v celé její nesmyslnosti
	\item ironizace armády, církve, Rakouska-Uherska
	\item děj na sebe příliš nenavazuje, spíše sled dějových epizod s prvky lidového humoru a dokumentaristiky
	\item citace z vyhlášek, dobových dokumentů, písní
	\end{itemize}
\item povídky \textbf{Velitelem města Bugulmy}
	\begin{itemize}
	\item popisuje fanatismus revoluční doby v Rusku a hloupost nových vládců
	\end{itemize}
\item \textbf{Dekameron humoru a satiry}
	\begin{itemize}
	\item povídka \textbf{Nemravné kalendáře} -- strážníci rozeberou nemravné kalendáře a tím ochrání veřejnost

	\item povídka \textbf{O básních} -- : kritika stereotypu lyrického subjektu, který neumí ve světě nalézt radost
	\end{itemize}
\end{itemize}

\section{Legionářská literatura}
\begin{itemize}
\item čeští vojáci (zajatci, nebo dezertéři), kteří se přidali na druhou stranu a vytvořili Legie
\item někteří podporovali Bolševiky, většina legií podporovala Menševiky
\end{itemize}
\subsection{Rudolf Medek (1890--1940)}
\begin{itemize}
\item básník, prozaik, voják, učitel, 1913 narukoval, s Legiemi cestoval na Sibiř, po návratu vstoupil do československé armády, vyznamenán ve Francii a Anglii
\item za protektorátu a komunismu cenzurován
\item sbírka \textbf{Lví srdce} -- oslavuje legie
\item román \textbf{Veliké dny} -- o vzniku legií, bitva u Sborova
\item román \textbf{Anabase} -- poslední měsíc pobytu legií na východě, cesta domů
\item drama \textbf{Plukovník Švec} -- o veliteli, který se zastřelí, aby zastavil rozpad morálky svých vojáků
\end{itemize}

\subsection{Josef Kopta (1894--1962)}
\begin{itemize}
\item trilogie románů \textbf{Třetí rota, Třetí rota na magistrále, Třetí rota doma}
	\begin{itemize}
	\item líčí postup malé vojenské jednotky od Zborova přes Sibiř k Vladivostoku
	\end{itemize}
\end{itemize}

\subsection{František Langer (1888--1965)}
\begin{itemize}
\item spisovatel, dramatik, vojenský lékař, publikoval v časopisech, východní fronta, legie, po návratu pokračoval v práci vojenského lékaře
\item dramaturg Vinohradského divadla
\item za druhé světové války musel odejít kvůli židovskému původu
\item drama \textbf{Jízdní hlídka} -- o hrdinství legionářů
\item drama \textbf{Periférie} -- vykresluje pražskou periferii, motiv zločinu a trestu, číšník zabije milence své přítelkyně, ale nikdo mu to nevěří
\item komedie \textbf{Velbloud uchem jehly}, \textbf{Obrácení Ferdyše Pištory}
\item soubor povídek \textbf{Železný vlk}
\end{itemize}

\section{Katolicky orientovaná próza}

\subsection{Jaroslav Durych (1886--1962)}
\begin{itemize}
\item básník, prozaik, dramatik, novinářská rodina, brzy sirotek, přednosta vojenské nemocnice 
\item v tvorbě je významné baroko
\item \textbf{Jarmark života} (dvě prózy, dvě básně)
\item román \textbf{Na horách}
\item později historická forma spjatá s obdobím protireformace
\item historický román \textbf{Bloudění} -- období popravy českých pánů, hlavní hrdina ve službách Albrechta z Valdštejna se zamiluje do katoličky Španělky Andělky, před smrtí s ní má dítě, které se stává symbolem naděje
\end{itemize}

\subsection{Jan Čep (1902--1974)}
\begin{itemize}
\item esejista, prozaik, překladatel, 1934 šel propagovat československou kulturu do Paříže, 1948 tam emigroval
\item kratší epické útvary s jednoduchým dějem, hrdinové jsou spjati s rodným krajem
\item soubor existenciálních povídek \textbf{Zeměžluč} -- křesťansky orientované, hrdinové jsou tragicky lhostejní k životu
\item soubor povídek \textbf{Letnice} -- hrdinové překonávají životní tragedie návratem k víře
\item román \textbf{Hranice stínu} -- cesta moderního člověka k nalezení vlastní identity 
\item soubor povídek \textbf{Modrá a zlatá} -- motiv návratu do rodného kraje, vrcholné dílo
\end{itemize}

\subsection{Jakub Deml (1878--1961)} 
\begin{itemize}
\item kněz, básník, prozaik, ve dvanácti letech zemřela matka, později sourozenci, gymnázium, vysvěcen, ale knězem nebyl moc dlouho
\item předchůdce surrealismu, expresionismu, existencialismu, vzorem Otokar Březina
\item básnická sbírka \textbf{Notantur lumina} -- ovlivněno Otokarem Březinou
\item báseň v próze \textbf{Hrad smrti} -- popisuje úzkost a ohrožení smrtí
\item básnická sbírka \textbf{Moji přátelé} -- personifikace stromů, květin, hub jako jeho blízkých
\item soubor \textbf{Šlépěje} -- 26 svazků, deníky, úvahy, recenze, korespondence
\item próza \textbf{Zapomenuté světlo} -- soubor vzájemně propojených esejů sjednocený postavou básníka Bohumila Maliny Ptáčka, na jehož dopis autor odpovídá
\end{itemize}

\section{Ruralismus}
\begin{itemize}
\item  vesnická tematika, vztah lidí k půdě, vývoj vztahu venkova a města
\item nemilosrdný osud venkovského obyvatelstva v důsledku industrializace
\item Josef Knap, František Křelina
\end{itemize}

\section{Psychologická próza}
\begin{itemize}
\item snaží se prozkoumat důvody které vedly k chování postav
\item většinou negativní pozadí
\end{itemize}

\subsection{Egon Hostovský (1908--1973)}
\begin{itemize}
\item prozaik, novinář, židovská rodina, gymnázium, filozofie, redaktor, během 2 sv. v. emigroval do Ameriky, pak se vrátil, za komunismu opět emigroval \ra Dánsko \ra Norsko \ra USA
\item tvorba psychologická, expresivní, filozofická a existenciální
\item častým tématem jsou lidé vykořenění ze svého prostředí, tvorba ovlivněna židovstvím
\item novela \textbf{Ghetto v nich} -- psychologická analýza vyděděnosti a odlišnosti
	\begin{itemize}
	\item židovský chlapec se snaží pomoci svému otci překonat osamělost
	\end{itemize}
\item román \textbf{Případ profesora Körnera} -- intelektuál se snaží překonat komplex méněcennosti
\item román \textbf{Žhář} -- patnáctiletý chlapec ventiluje milostné problémy do paličských dopisů, někdo je však zneužije k opravdovému žhářství
\item 
\end{itemize}

\subsection{Jaroslav Havlíček (1896--1943)}
\begin{itemize}
\item prozaik, nejvýznamnější představitel domácí psychologické prózy, rodina učitele, ČVUT, po válce úředníkem, zároveň psal
\item psychologické romány, děj zasazen do maloměsta na přelomu 19. a 20. století, tragické osudy, degenerace člověka
\item naturalistické líčení duševního stavu, narušených lidí, extrémních situací
\item motiv viny a trestu, osudu, postavy vykazují velkou mravní a duševní sílu
\item román \textbf{Vyprahlé touhy / Petrolejové lampy}
	\begin{itemize}
	\item Vyprahlé touhy -- pouze první vydání
	\item symbol touhy ženy po obyčejném lidském štěstí, po rodině, po dětech
	\item hlavní hrdinka je dlouho nevdaná, nakonec se provdá za bratrance, do kterého se i zamiluje, ale on je alkoholik a později zešílí kvůli syfilisu a ona se o něj musí starat
	\end{itemize}
	
\item román \textbf{Helimadoe}
	\begin{itemize}
	\item otec a pět dcer, dcery mu musí pomáhat, nechce, aby se samy vzdělávaly
	\item Dora uteče s kouzelníkem -> musí pomáhat i nejmladší Emma
	\end{itemize}
	
\item sbírka fantastických povídek \textbf{Zánik městečka Olšiny} -- sleduje podivné udlosti v městečku po oznámení toho, že Bůh zemřel

\item povídka \textbf{Skleněný vrch} -- příběh ředitele záložny, který se dopustí zpronevěry kvůli manželce
\end{itemize}

\subsection{Jarmila Glazarová (1901--1977)}
\begin{itemize}
\item prozaička, publicistka, v šestnácti osiřela, po válce dostudovala a vzala si lékaře staršího o 30 let, komunisty oslavována
\item po 2. sv. válce politické působení, zvolena do Národního shromáždění, titul národní umělkyně
\item kronika \textbf{Roky v kruhu} -- o jejím šťastném manželství
\item román \textbf{Vlčí jáma} -- soužití despotické ženy, jejího mladšího manžela a jejich schovanky
\item baladický román \textbf{Advent}
	\begin{itemize}
	\item během 24 hodin v Beskydech
	\item matka hledá a zachraňuje synka, při tom zpětně hodnotí své nevydařené manželství
	\end{itemize}
\end{itemize}

\section{Avantgardní (Imaginativní) próza}
\begin{itemize}
\item 
\end{itemize}

\subsection{Vladislav Vančura (1891--1942)}
\begin{itemize}
\item prozaik, režisér, intelektuál, vystudoval medicínu, zabit při heydrichiádě, předseda Devětsilu, vyloučen z komunistické strany
\item psal dramata a scénáře, romány a povídky
\item průkopník experimentálních a avantgardních postupů, poetismus, usiloval vždy o sdělení myšlenky, ne jen o hraní si se slovy
\item nezvykle pojatý vypravěč, silní lyrizace próz
\item román \textbf{Pekař Jan Marhoul} -- sociální problematika, polemizace s křesťanstvím
\item historický román \textbf{Markéta Lazarová} -- oslava života a lásky navzdory nepřízni osudu
	\begin{itemize}
	\item 13. století, 2 znepřátelené rody loupeživých rytířů se  snaží klást odpor králi, jeden z rytířů ukořistí princeznu Markétu, která měla jít do kláštera, ale nakonec se do rytíře zamiluje
	\item děj upozaděn, silně lyrické
	\item láska je silnější než smrt, život beze strachu, revolta mladých v kontrastu s dobou
	\end{itemize}
\item novela \textbf{Rozmarné léto}
	\begin{itemize}
	\item zachycuje idylu českého maloměsta
	\item satira omezenosti maloměšťáků a jejich touhy po jiném způsobu života
	\end{itemize}
\item postupně dospívá k realismu
\item román \textbf{Útěk do Budína} -- o tragické lásce Češky a Slováka
\item milostné povídky
\item za okupace píše o dějinách
\end{itemize}

\section{Socialistický realismus}
\begin{itemize}
\item schválená směrnice pro umění komunisty, vychází v vládnoucího režimu, poukazuje na jeho úspěchy v každodenním životě
\end{itemize}

\subsection{Ivan Olbracht (1882--1952)}
\begin{itemize}
\item prozaik a novinář, levicově orientovaný, česko-židovská rodina, práva, historie, zeměpis, studia nedokončil, redaktor, vězněn za politickou činnost, protinacistický odboj, poslanec ÚNS
\item psychologický román \textbf{O zlých samotářích} -- povídky z cirkusu a o tulácích, kteří bojují o důstojnost a svobodu
\item psychologický román \textbf{Žalář nejtemnější} -- o policejním komisaři Machovi, který týrá svou ženu žárlivostí, oslepne \ra ještě horší
\item tvorba ovlivněna pobytem na Podkarpatské Rusi
\item sbírka reportáží \textbf{Hory a staletí} -- fascinován krajinou
\item román \textbf{Nikola Šuhaj loupežník}
	\begin{itemize}
	\item inspirovaný životem skutečného loupežníka
	\item předloha pro muzikál Balada pro banditu
	\item řazeno také k imaginativní próze
	\end{itemize}
\item soubor povídek \textbf{Golet v údolí} -- popisuje život nejchudších židovských obyvatel
\item psal i pro děti, upravil staré české pověsti a převyprávěl Starý zákon
\end{itemize}

\subsection{Marie Majerová (1882--1967)}
\begin{itemize}
\item spisovatelka, novinářka, překladatelka, z chudých poměrů, služebná, písařka
\item levicové postoje, anarchistka, sociální demokratka, komunistka
\item román \textbf{Panenství} -- šestnáctiletá Hana se zamiluje do Jimeše s tuberkulózou, snaží se pro něj získat peníze výměnou za své panenství, nakonec ale uteče, v závěru Jimeš umírá, Hana se vdá za majitele restaurace kde pracuje
\item utopický román \textbf{Přehrada} -- zachycuje budoucí socialistickou revoluci v Praze
\item generační román \textbf{Siréna} -- zachytila osudy čtyř generací rodiny v prostředí dolů a hutí
\item román pro dívky \textbf{Robinsonka} -- dívka je po smrti matky nucena rychle zaujmout její místo v domácnosti
\item protikomunisticky zaměřené reportáže \textbf{Zpívající Čína}, \textbf{Vítězný pochod}
\end{itemize}

\subsection{Marie Pujmanová (1893--1958)}
\begin{itemize}
\item spisovatelka, novinářka, národní umělkyně, autorka sociální prózy, intelektuální prostředí
\item rozdíl názorů se svou společenskou třídou, zajímala se o dělnické prostředí, zastánkyně komunistického režimu a míru, požadovala smrt Milady Horákové
\item kladla důraz na vykreslení psychologie postav, později typizace kvůli komunistické ideologii
\item psychologická novela \textbf{Předtucha} -- nejistoty dospívání
\item román \textbf{Pacientka doktora Hegla} -- svobodné mateřství
\item románová trilogie \textbf{Lidé na křižovatce}, \textbf{Hra s ohněm}, \textbf{Život proti smrti}
	\begin{itemize}
	\item 1. díl se odehrává v "Baťově" továrně, líčí všechny vrstvy tehdejší průmyslové společnosti
	\item 2. díl -- třicátá léta až protektorát
	\item 3. díl uzavírá osudy postav na konci války
	\item poznamenáno socialistickým schematismem
	\end{itemize}
\end{itemize}

\section{Demokratický proud}
\begin{itemize}
\item silně se vymezuje proti fašismu a komunismu, diktatuře
\item spisovatelé fungují i jako novináři v lidových novinách
\end{itemize}
\subsection{Karel Čapek (1890--1938)}
\begin{itemize}
\item * Malé Svatoňovice, otec lékař, dobré rodinné vztahy, osmileté gymnázium v Hradci, pak Jaroška, maturoval v Praze, dějiny umění a jazykověda na FFUK, člen skupiny výtvarných umělců (moderní umění, kubismus)
\item nemusel na frontu kvůli vadě páteře, redaktorem Národních listů, Lidových novin
\item 1935 se oženil, 1938 nominován na Nobelovu cenu, o Vánocích ale podlehl zánětu plic
\item první díla psal už za studia s bratrem Josefem -- \textbf{Zářivé hlubiny}, \textbf{Krakonošova zahrada}
\item v próze pokládá svoje otázky, ale netlačí na čtenáře vlastní názory, vnímá, že různí lidé mohou mít různé názory
\item dialogy převažují nad monology, používá prvky běžného hovoru
\item souhrnným tématem bývá prosazování jediné vlastní pravdy a vynálezy	
	\begin{itemize}
	\item lidská morálka se vyvíjí pomaleji než technika \ra etické otázky ohledně přelomových vynálezů
	\end{itemize}
\item sbírka filosofických povídek \textbf{Boží muka} 
	\begin{itemize}
	\item první samostatná sbírka
	\item ovlivněno zážitky z válečných let
	\item povídka Šlépěj -- problém osamocené stopy ve sněhové závěji
	\item myšlenky rozvíjí po celý život
	\end{itemize}

\item sbírka obrazů \textbf{Trapné povídky}
	\begin{itemize}
	\item vzniká těsně po válce
	\item obrazy zautomatizovaného živoření
	\end{itemize}


\item stoupenec pragmatismu -- tvrdí že ten, kdo vidí jen jednu pravdu zamítá všechny ostatní názory \ra spory, válka
	\begin{itemize}
	\item všechno lidské konání by mělo spět k praktickému řešení
	\end{itemize}

\item román-fejeton \textbf{Továrna na Absolutno}
	\begin{itemize}
	\item vynálezce vytvoří stroj na rozklad hmoty na energii, každý kdo se k němu přiblíží se promění ve světce, který touží konat dobré skutky
	\item každý považuje jen svoji představu o světě za správnou \ra řada satyrických scén s obchodníky, bankéři, politiky
	\item konflikty vedou k válce, která málem vyhladí lidstvo
	\end{itemize}
	
\item román \textbf{Krakatit}
	\begin{itemize}
	\item nejrozsáhlejší román, utopický námět
	\item téma vynálezu supertřaskaviny o kterou se přetahují různé vlivné skupiny
	\item ústřední téma je ale láska Prokopa a pyšné princezny Wilhelminy
	\end{itemize}

\item fantasticko-utopická hra \textbf{RUR}
\item dvacátá léta -- nejplodnější
\item drama \textbf{Loupežník} (+ Josef)
	\begin{itemize}
	\item téma konfliktu generací a přístupu k životu
	\item vyzývavé a bezstarostné mládí loupežníka vítězí nad životní zkušeností a opatrností profesora
	\end{itemize}

\item utopická komedie \textbf{Věc Makropulos}
	\begin{itemize}
	\item krásná zpěvačka testuje elixír života než ho dostane Rudolf II., dožije se 300 let, je přesycena zážitky, citově otupená, poznává, že krása života je podmíněna vědomím jeho konce \ra zřekne se nesmrtelnosti
	\end{itemize}
\item během 20. let se často stýkal s T. G. Masarykem \ra publikoval jejich hovory \textbf{Hovory s T. G. Masarykem}
\item kriminální povídky \textbf{Povídky z jedné kapsy} a \textbf{Povídky z druhé kapsy}
	\begin{itemize}
	\item o obyčejných lidech, kteří se dostali mimo zákon kvůli osudu či náhodě
	\end{itemize}

\item noetická románová trilogie \textbf{Hordubal}, \textbf{Povětroń}, \textbf{Obyčejný život}
	\begin{itemize}
	\item společné téma -- v každém člověku se skrývá více osobností
	\item Hadrubal -- horal se po návratu z Ameriky do Podkarpatské Rusi snaží sehnat peníze pro rodinu, žena se mu odcizila a je s čeledínem, Hadrubal je nakonec zabit.
	\item Povětroń -- příběh letce, který havaroval a je v kómatu, pohled lidí, kteří neví kdo je, podle jeho jizev lékař, sestra a další odhadují co v životě dělal a vytváří si každý svůj vlastní příběh o muži
	\item Obyčejný život -- příběh zesnulého muže, který si psal záznamy o svém životě, více pohledů na sebe sama
	\end{itemize}

\item v době nástupu nacismu v Německu se vymezuje proti totalitním režimům
\item román \textbf{Válka s mloky}
	\begin{itemize}
	\item 
	\end{itemize}
\item drama \textbf{Matka}
	\begin{itemize}
	\item matka se jmenuje Dolores (el dolor -- bolest)
	\item manžel i synové umírají pro nějakou důležitou věc, prožije si mnoho bolesti
	\item snaží se uhájit nejmladšího posledního syna, nakonec si uvědomí, že je potřeba bojovat a posílá i nejmladšího syna do boje -- reakce i na Mnichov, pacifismus není vždy řešení
	\end{itemize}
\item drama \textbf{Bílá nemoc}
	\begin{itemize}
	\item 
	\end{itemize}
\item pohádky \textbf{Devatero pohádek} -- zaměřeno na nějaké povolání
\item pohádka \textbf{Dášenka čili život štěněte}
\item cestopisy \textbf{Anglické listy}, \textbf{Italské listy}, \textbf{Obrázky z Holandska}, \ldots
\end{itemize}

\subsection{Josef Čapek (1877--1945)}
	\begin{itemize}
	\item malíř, ilustrátor, prozaik, uměleckoprůmyslová škola
	\item redaktor časopisů Umělecký měsíčník, Volné směry
	\item esej \textbf{Tvořivá povaha moderní doby} -- vyjadřuje svůj názor na moderní umění
	\item maloval pod vlivem kubismu, expresionismu a civilismu, navrhl kostýmy pro R. U. R.
	\item pohádky \textbf{Povídání o pejskovi a kočičce} -- pro dceru, maloval ilustrace
	\item satirická alegorie \textbf{Ze života hmyzu}
	\item sbírka filosofických esejů \textbf{Kulhavý poutník}
	\item baladický příběh \textbf{Stín kapradiny} -- nejceněnější dílo
		\begin{itemize}
		\item dva pytláci zastřelí lesníka, který je přistihl, na útěku ještě četníka a dělníka, nakonec jsou dopadeni, jeden je zastřelen četníky, druhý spáchá sebevraždu
		\item důraz na psychiku, a pocity, krása lesa, svoboda, později podléhají v důsledku okolností beznaději
		\end{itemize}
	\item zatčen za protinacistické názory, zemřel v Bergen-Belsenu na tyfus
	\end{itemize}

\subsection{Eduard Bass (1888--1946)}
\begin{itemize}
\item spisovatel, novinář, zpěvák, fejetonista, herec, recitátor, konferenciér, textař
\item vystupoval v kabaretu U Bílé labutě, později ředitelem kabaretů, vydával kabaretní texty
\item sbírka satir, veršů a písniček \textbf{Letáky}
\item humoristický román \textbf{Klapzubova jedenáctka}
	\begin{itemize}
	\item příběh otce, který má 11 synů, ze kterých vytvoří profesionální fotbalový tým
	\item humorné zamyšlení nad etikou a profesionalitou sportu
	\end{itemize}
\item román \textbf{Cirkus Humberto}
	\begin{itemize}
	\item cílem povzbudit český národ, snaží se motivovat vyzdvihováním kladných vlastností
	\item sleduje tři generace cirkusáků, syn zedníka se ožení s dcerou majitele cirkusu, později se stane ředitelem
	\end{itemize}
\end{itemize}


\subsection{Karel Poláček (1892--1945)}
\begin{itemize}
\item spisovatel, humorista, novinář, scénárista
\item právnická fakulta, za první světové v armádě, 
\item Lidové noviny -- sloupkař, fejetonista, soudní zpravodaj, redaktor Tvorby, vydavatelství Melantrich
\item v románech zachycuje směsnost a tragiku života běžného člověka
\item humoristický román \textbf{Muži v Offsidu} -- v prostředí fanoušků FK Viktoria Žižkov a SK Slavia Praha
\item sbírka humoru \textbf{Židovské anekdoty}
\item povídka pro děti \textbf{Edudant a francimor} (-> večerníček)
\item román \textbf{Hlavní přelíčení} 
	\begin{itemize}
	\item inspirován případem loupežné vraždy
	\item příběh slabocha, snílka, který se v touze po majetku dopustí podvodu a pak vraždy
	\end{itemize}
\item humoristický román \textbf{Bylo nás pět}
	\begin{itemize}
	\item vyobrazuje dění na maloměstě očima dítěte
	\item významná jazyková komika, příhody pěti chlapců jsou psány groteskní kombinací hovorového jazyka, obratů z dobrodružné četby a slohu povinných školních prací
	\end{itemize}
\end{itemize}

\end{document}