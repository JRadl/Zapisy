\documentclass[10pt,a4paper]{article}
\usepackage[utf8]{inputenc}
\usepackage[czech]{babel}
\usepackage{amsmath}
\usepackage{amsfonts}
\usepackage{amssymb}
\usepackage{chemfig}
\usepackage{geometry}
\usepackage{wrapfig}
\usepackage{graphicx}
\usepackage{floatflt}
\usepackage{hyperref}
\usepackage{fancyhdr}
\usepackage{tabularx}
\usepackage{makecell}
\usepackage{csquotes}
\usepackage{marginnote}

\MakeOuterQuote{"}

\renewcommand{\labelitemii}{$\circ$}
\renewcommand{\labelitemiii}{--}
\newcommand{\ra}{$\rightarrow$ }
\newcommand{\x}{$\times$ }
\newcommand{\lp}[2]{#1 -- #2}
\newcommand{\timeline}{\input{timeline}}


\geometry{lmargin = 0.8in, rmargin = 0.8in, tmargin = 0.8in, bmargin = 0.8in}
\newcommand{\note}[1]{\marginnote{\hspace{-0.6\textwidth}#1}}

\date{}
\author{Jakub Rádl}
\title{Viktor Dyk: Krysař -- Rozbor díla}

\begin{document}
\maketitle

\section*{Výňatek}

\section*{Tématická stránka díla}
\begin{itemize}
\item \textbf{literární druh a žánr}: epika (s mnoha lyrickými pasážemi), novela (relativně krátký a jednoduchý příběh s jednou dějovou linií)
\item \textbf{téma a motiv}:
	\begin{itemize}
	\item poenta:
		\begin{itemize}
		\item Krysař putuje světem, nikde nezůstává
		\item V momentu kdy se poprvé v životě rozhodl usadit, o vše přišel
		\item Sepp Jörgen s dítětem se jako jediný zachrání
		\end{itemize}
	\item \textbf{hlavní téma}: 
		\begin{itemize}
		\item Krysař přišel do města Hameln, zbavit ho krys, byl okraden o plat, zamiloval se, milá mu zemřela, Krysař se rozhodl ukončit svůj život a vzít s sebou celé město.
		\end{itemize}
	\item \textbf{další motivy v díle}:láska, píšťala, hospoda, ďábel, krysy, Sedmihradská země --symbol posmrtného ráje
	\end{itemize}
\item \textbf{časoprostor}: hanzovní město Hameln, středověk (dílo vychází ze středověké pověsti)
\item \textbf{zasazení výňatku do kontextu díla}:
	\begin{itemize}
	\item \textbf{časoprostor}: 18 kapitola
	\item \textbf{obsah}: Krysař se dozvěděl o tom, že Agnes čeká Krystiánovo dítě. Probíhá boj Krysařových myšlenek, jestli chce zůstat, či odejít. Už je rozhodnut, že zůstane, vzpomíná na to, jak odešel poprvé a vrátil se. Agnes se ho snaží přemluvit, aby odešel. 
	\end{itemize}
\item \textbf{kompoziční výstavba}
	\begin{itemize}
	\item chronologická kompozice celého příběhu, s retrospektivními prvky z Krysařovy historie
	\item dílo rozděleno na 26 kapitol
	\item na konci před smrtí Agnes píseň Sedmihradská
	\end{itemize}
\end{itemize}
\section*{Kompozice, postavy}
\begin{itemize}
\item nezúčastněný, nezaujatý vypravěč (3. osoba -- er forma)
\item vyprávěcí způsoby:
	\begin{itemize}
	\item přímá řeč v dialozích a monolozích
	\item krysař o sobě někdy mluví ve třetí osobě
	\item často popisovány krysařovy myšlenky, myšlenky ostatních postav 
	\end{itemize}

\end{itemize}

\paragraph{Postavy}
\begin{itemize}
\item \textbf{Krysař}
	\begin{itemize}
	\item nemá či nesděluje své jméno, cestuje světem, nikde nezůstává, má svou kouzelnou píšťalu jejíž moci se sám bojí
	\item během díla můžeme pozorovat změnu jeho myšlenek a názorů
		\begin{itemize}
		\item nejprve se usadit nechce, poté se zamiluje do Agnes a chce zůstat, nakonec neovládne své city a pomstí se celému městu
		\end{itemize}
	\item neprokazuje respekt konšelům
	\item postava na okraji společnosti, odlišný, neznámý, strašidelný, tajemný
	\item jeho neurčitá identita slouží konšelům jako výmluva pro nevyplacení odměny
	\item zamilovaný do Agnes, kvůli ní ušetří město
	\item sám sobě nerozumí -> vnitřní boj, když odejde
	\item 
	\end{itemize}
\item \textbf{Agnes}
	\begin{itemize}
	\item nebojí se krysaře
	\end{itemize}
\item \textbf{Krystián}
	\begin{itemize}
	\item běžný člověk
	\item chodí s Agnes
	\item přemýšlí o penězích, raduje se ze smrti strýčka (pragmatické uvažování)
	\item 
	\end{itemize}
\item \textbf{Sepp Jörgen}
	\begin{itemize}
	\item pomalý, prostý, naivní člověk, není schopný okamžitě analyzovat situaci, z toho pramení jeho nevinnost
	\item dítě je k němu na konci díla přirovnáno
	\end{itemize}
\item \textbf{Faustus}
	\begin{itemize}
	\item ztělesnění ďábla, potká Krysaře v hospodě, snaží se ho zlákat aby upsal své srdce ďáblu
	\end{itemize}
\item \textbf{konšelé} --
\end{itemize}
\section*{Jazyk}
\begin{itemize}
\item jazykové prostředky a jejich funkce ve výňatku
\item tropy a figury ve výňatku
\end{itemize}
\section*{Literárně historický kontext}
\begin{itemize}
\item původně vydáno pod názvem Pravdivý příběh
\end{itemize}
\section*{Zdroje}
\begin{itemize}
\item https://cs.wikipedia.org/wiki/Krysa\%C5\%99\_(kniha)
\item DYK, Viktor. Krysař. V Praze: Fortuna Libri, 2016. ISBN 978-80-7546-022-6.
\end{itemize}
\end{document}