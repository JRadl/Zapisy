\documentclass[10pt,a4paper]{article}
\usepackage[utf8]{inputenc}
\usepackage[czech]{babel}
\usepackage{amsmath}
\usepackage{amsfonts}
\usepackage{amssymb}
\usepackage{chemfig}
\usepackage{geometry}
\usepackage{wrapfig}
\usepackage{graphicx}
\usepackage{floatflt}
\usepackage{hyperref}
\usepackage{fancyhdr}
\usepackage{tabularx}
\usepackage{makecell}
\usepackage{csquotes}
\usepackage{marginnote}

\MakeOuterQuote{"}

\renewcommand{\labelitemii}{$\circ$}
\renewcommand{\labelitemiii}{--}
\newcommand{\ra}{$\rightarrow$ }
\newcommand{\x}{$\times$ }
\newcommand{\lp}[2]{#1 -- #2}
\newcommand{\timeline}{\input{timeline}}


\geometry{lmargin = 0.8in, rmargin = 0.8in, tmargin = 0.8in, bmargin = 0.8in}
\newcommand{\note}[1]{\marginnote{\hspace{-0.6\textwidth}#1}}

\date{}
\author{Jakub Rádl}
\title{Viktor Dyk: Krysař -- Rozbor díla}

\begin{document}
\maketitle


\section*{Tématická stránka díla}
\begin{itemize}
\item \textbf{literární druh a žánr}: epika (s mnoha lyrickými pasážemi), novela (relativně krátký a jednoduchý příběh s jednou dějovou linií)
\item \textbf{téma a motiv}:
	\begin{itemize}
	\item poenta:
		\begin{itemize}
		\item Krysař putuje světem, nikde nezůstává
		\item V momentu kdy se poprvé v životě rozhodl usadit, o vše přišel
		\item Sepp Jörgen s dítětem se jako jediný zachrání
		\end{itemize}
	\item \textbf{hlavní téma}: 
		\begin{itemize}
		\item Krysař přišel do města Hameln, zbavit ho krys, byl okraden o plat, zamiloval se, milá mu zemřela, Krysař se rozhodl ukončit svůj život a vzít s sebou celé město.
		\end{itemize}
	\item \textbf{další motivy v díle}
		\begin{itemize}
	\item láska, píšťala, hospoda, ďábel, krysy, Sedmihradská země --symbol posmrtného ráje
		\end{itemize}
	\end{itemize}
\item \textbf{časoprostor}: hanzovní město Hameln, středověk (dílo vychází ze středověké pověsti)
\item \textbf{zasazení výňatku do kontextu díla}:
	\begin{itemize}
	\item \textbf{časoprostor}: 18 kapitola
	\item \textbf{obsah}: Krysař se dozvěděl o tom, že Agnes čeká Krystiánovo dítě. Probíhá boj Krysařových myšlenek, jestli chce zůstat, či odejít. Už je rozhodnut, že zůstane, vzpomíná na to, jak odešel poprvé a vrátil se. Agnes se ho snaží přemluvit, aby odešel. 
	\end{itemize}
\item \textbf{kompoziční výstavba}
	\begin{itemize}
	\item chronologická kompozice celého příběhu, s retrospektivními prvky z Krysařovy historie
	\item dílo rozděleno na 26 kapitol
	\item na konci před smrtí Agnes píseň Sedmihradská
	\end{itemize}
\end{itemize}
\section*{Kompozice, postavy}
\begin{itemize}
\item nezúčastněný, nezaujatý vypravěč (3. osoba -- er forma)
\item vyprávěcí způsoby:
	\begin{itemize}
	\item přímá řeč v dialozích a monolozích
	\item krysař o sobě někdy mluví ve třetí osobě
	\item často popisovány krysařovy myšlenky, myšlenky ostatních postav 
	\end{itemize}

\end{itemize}

\paragraph{Postavy}
\begin{itemize}
\item \textbf{Krysař}
	\begin{itemize}
	\item nemá či nesděluje své jméno, cestuje světem, nikde nezůstává, má svou kouzelnou píšťalu jejíž moci se sám bojí
	\item během díla můžeme pozorovat změnu jeho myšlenek a názorů
		\begin{itemize}
		\item nejprve se usadit nechce, poté se zamiluje do Agnes a chce zůstat, nakonec neovládne své city a pomstí se celému městu
		\end{itemize}
	\item neprokazuje respekt konšelům
	\item postava na okraji společnosti, odlišný, neznámý, strašidelný, tajemný
	\item jeho neurčitá identita slouží konšelům jako výmluva pro nevyplacení odměny
	\item zamilovaný do Agnes, kvůli ní ušetří město
	\item sám sobě nerozumí \ra vnitřní boj, když odejde
	\end{itemize}
\item \textbf{Agnes}
	\begin{itemize}
	\item nebojí se krysaře
	\item mladá, krásná, zdánlivě nevinná dívka, má se vdávat za Kristiána, ale zamiluje se do Krysaře
	\item po otěhotnění s Kristiánem spáchá sebevraždu
	\item jediná věří Krysaři a nemá proti němu předsudky
	\end{itemize}
\item \textbf{Dlouhý Krystián}
	\begin{itemize}
	\item běžný člověk toužící po běžném lidském štěstí
	\item chodí s Agnes, neví, že ho podvádí s Krysařem
	\item přemýšlí o penězích, raduje se ze smrti strýčka (pragmatické uvažování)
	\end{itemize}
\item \textbf{Sepp Jörgen}
	\begin{itemize}
	\item rybář z okolí města, pohledný
	\item zavraždil svého ptáčka v zuřivosti nad odchodem dívek
	\item pomalý, prostý, naivní člověk, není schopný okamžitě analyzovat situaci, vše si uvědomuje až druhý den, z toho pramení jeho nevinnost
	\item dítě je k němu na konci díla přirovnáno
	\end{itemize}
\item \textbf{Faustus}
	\begin{itemize}
	\item ztělesnění ďábla, potká Krysaře v hospodě, snaží se ho zlákat aby upsal své srdce ďáblu
	\item krysař se nenechá přesvědčit
	\end{itemize}
\item \textbf{konšelé (Strum, Frosh)}
	\begin{itemize}
	\item slíbili krysaři velkou odměnu za zbavené města krys, později mu ji odmítli vyplatit, protože krysař nemůže potvrdit svou identitu
	\item argument použit protože krysař byl odlišný -- společnost se bojí neznámého a tajemného
	\end{itemize}
\end{itemize}
\section*{Jazyk}
\begin{itemize}
\item v díle využita stará, umělecká, poetická čeština
\item mnoho metafor, symbolů, archaismů, řečnických otázek
\item výňatek
	\begin{itemize}
	\item průlet Krysařovými myšlenkami o jeho předešlém životě urychlen pomocí výčtů, větných ekvivalentů a krátkých vět, paralelismů (typické pro Dykovu tvorbu)
	\item opakování motivu opustit Hameln
	\end{itemize}
\end{itemize}
\section*{Literárně historický kontext}
\begin{itemize}
\item Viktor Dyk žil v letech 1877--1931
\item řadí se k anarchistickým buřičům
	\begin{itemize}
	\item mladá generace konce 19. století, podobná dekadentům, která otevřeně protestuje proti Rakousku-Uhersku
	\item soustředí se na obsah před formou, snaží se ho přiblížit co nejširší skupině lidí \ra navazují na ÚLS  a folklór
	\end{itemize}
\item psal nejen prózu, ale i básně (sbírka Satiry a sarkasmy, balada Milá sedmi loupežníků, \ldots) a drama (Zmoudření dona Quijota)
\item novela Krysař byla původně vydána pod názvem Pravdivý příběh, je inspirována středověkou pověstí
\item mezi anarchistické buřiče řadíme též F. Gellnera, K. Tomana, S. K. Neumana, F. Šrámka, P. Bezruče
\end{itemize}
\section*{Zdroje}
\begin{itemize}
\item https://cs.wikipedia.org/wiki/Krysa\%C5\%99\_(kniha)
\item DYK, Viktor. Krysař. V Praze: Fortuna Libri, 2016. ISBN 978-80-7546-022-6.
\item vlastní zápisky z hodin
\end{itemize}

\newpage
\section*{Výňatek}
Krajiny míhaly se před krysařovýma očima; překonávaly se krásou. Měly \textbf{smavý}\footnote{epiteton (smavý od smát se)} půvab jihu a přísnou krásu severu. Plynuly řeky, týčily se hory, hlučela města, bouřila moře. A pak bylo prázdno, strašné prázdno.

Tváře žen se zamíhaly v dáli; žen bílé i tmavé pleti, žen různých plemen a zemí. Šly, \textbf{šeptajíce něhu a lásku}.\footnote{synestézie} – Šly, kynouce na pozdrav. – A potom přišla noc.

Úžasný a překotný let!

Vše zmizelo, a krysař necítil; všechny odešly, a krysař nelitoval. Daleko, daleko, k novým obzorům \textbf{letělo jeho pyšné a prudké srdce}.\footnote{metafora, synekdocha, metonýmie} Nezakotvit! Nestanout!

\textbf{Agnes! Agnes!}\footnote{epizeuxis} Čím upoutala Agnes krysaře? Byly krásnější ženy, byly opojivější ženy.

Spojuje hoře více nežli rozkoš? \footnote{řečnická otázka}

Oči Agnes hleděly rezignovaně a sklesle. Neopakovala už své „\textbf{Zab}!“\footnote{archaismus} Ale v jejím pohledu nebylo radosti a naděje. Pouze bol.

Ale bol ten byl nehlasný. A tělo Agnes bylo chladné a nehybné jako tělo ženy, kterou ubili.

Krysař stál nad touto Agnes. Hory valily se na jeho srdce. Těžké byly rozpomínky. Vteřiny byly dlouhé, ale přece míjely. Počínalo se šeřiti. A krysař vzpomínal ještě na svůj odchod a na setkání s dlouhým Kristiánem u vrátek zahrady za časného jitra.

Odešel, aby mu učinil místo, není-li pravda?\footnote{řečnická otázka}

A krysař snažil se urovnati své myšlenky, násilím zmoci odboj svého srdce.

Zajisté že je nutno opustiti Hammeln. Za hradbami města osvobodí se jeho srdce. Ptáci zpívají tak vesele na cestu těm, kdo jdou v stínu starých dubů a v úpalu polních cest. Mnoho krásných a podivuhodných věcí čeká a uvítá. Krysař viděl mnoho, neviděl však vše. Nač vzpomínati na hoře, jež je za námi, nač mysliti na dveře, které se za námi zavřely?

Opustit Hammeln!

Ale v jeho vzpomínce ožil nedávný den, kdy také opouštěl Hammeln... příliš vhod dlouhému Kristiánu. Odešel domněle svobodný, a vracel se spoutaný touhou, nejasnou zprvu, vždy však mocnější a krutější, která ho hnala zpět, zpět k protivným hradbám hnusného města, města malých životů a malých srdcí, kam zabloudila krysařova láska. Zajisté že je také dnes možno opustiti brány Hammeln. Ale kam dojde krysař? Na jak dlouho dovede utéci před svými vzpomínkami?

Zůstati. Jen jak možno zůstati?

Nesnese pohledu na Agnes. Nesnese pohledu na \textbf{onoho}\footnote{archaismus} muže. Nelze mu žít v Hammeln vedle dlouhého Kristiána. A krysař – co chce ještě krysař v městě, jehož krysy, neopatrný, příliš záhy vyhubil? Jaký to krysař, nesvobodný a přikovaný nepochopitelnými pouty?

Ale co zbývalo, nebylo-li možno odejíti z Hammeln ani zůstati?

Agnes hleděla na krysaře svýma rezignujícíma očima. Četla v jeho myšlenkách a v jeho srdci.

Nečekala ničeho; chtěla jen \textbf{zahleděti}\footnote{archaismus} se ještě na okamžik na tuto drahou a zachmuřenou tvář.

Čtla v jeho myšlenkách a jeho srdci.

„Jdi,“ pravila smutně a tiše.

Krysař odmítl posuňkem. Ale Agnes stála tvrdošíjně na svém. Prosila o jeho odchod jako o milost.



\end{document}