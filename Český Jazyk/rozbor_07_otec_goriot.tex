\documentclass[10pt,a4paper]{article}
\usepackage[utf8]{inputenc}
\usepackage[czech]{babel}
\usepackage{amsmath}
\usepackage{amsfonts}
\usepackage{amssymb}
\usepackage{chemfig}
\usepackage{geometry}
\usepackage{wrapfig}
\usepackage{graphicx}
\usepackage{floatflt}
\usepackage{hyperref}
\usepackage{fancyhdr}
\usepackage{tabularx}
\usepackage{makecell}
\usepackage{csquotes}
\usepackage{marginnote}

\MakeOuterQuote{"}

\renewcommand{\labelitemii}{$\circ$}
\renewcommand{\labelitemiii}{--}
\newcommand{\ra}{$\rightarrow$ }
\newcommand{\x}{$\times$ }
\newcommand{\lp}[2]{#1 -- #2}
\newcommand{\timeline}{\input{timeline}}


\geometry{lmargin = 0.8in, rmargin = 0.8in, tmargin = 0.8in, bmargin = 0.8in}
\newcommand{\note}[1]{\marginnote{\hspace{-0.6\textwidth}#1}}
\newcommand{\fnote}[1]{\footnote{#1}}

\date{}
\author{Jakub Rádl}
\title{Honoré de Balzac: Otec Goriot -- Rozbor díla}

\begin{document}
\maketitle

\section*{Výňatek}
Hle, \textbf{stojíte na rozcestí života}\footnote{metafora}, mládenče,
vyberte si! Už jste si vybral: šel jste k našemu bratranci de
Beauséant a tam jste zvětřil přepych. Šel jste k paní de Restaud,
dceři otce Goriota, a tam jste zvětřil Pařížanku. Onoho dne, kdy
jste se vrátil, \textbf{měl jste napsáno na čele heslo}\footnote{metafora}, jež jsem uměl snadno
rozluštit: Prorazit! Prorazit! Prorazit stůj co stůj! „Výborně!“ řekl
jsem, „ten chlapík se mi líbí.“ Potřeboval jste peníze. \textbf{Kde je vzít?}\footnote{řečnická otázka}
Vytáhl jste je ze svých sester. Všichni bratři více méně obírají své
sestry. Po vašich patnácti stech francích, vydobytých bůhvíjak z
\textbf{kraje, kde je víc kaštanů než peněz}\footnote{přirovnání, hyperbola, perifráze, aliterace}, se zapráší jako po vojácích,
když jdou loupit. A potom, co uděláte? Budete pracovat? Práce,
tak jak vy ji nyní chápete, vynese na stará kolena byt u matky
Vauquerové takovým mládencům, jako je Poiret. Jak rychle získat
jmění, je otázka, kterou si klade padesát tisíc mladých mužů, kteří
jsou v takové situaci jako vy. Vy jste jedním z nich. Mějte na
zřeteli úsilí, jež budete musit vynaložit, a \textbf{dravost boje}\footnote{personifikace}. Budete 
90
muset pozřít jeden druhého, \textbf{jako se pozřou pavouci vtěsnaní v
nějakém kořenáči}\footnote{přirovnání}, protože padesát tisíc dobrých míst není. Víte,
jak se z toho vymotat? Geniálním důvtipem nebo chytrou korupcí.
Do toho množství lidí se musí \textbf{vpadnout jako koule z děla, nebo se
tam vplížit jako mor}.\footnote{přirovnání, perifráze} Poctivost není k ničemu. Lidé se koří síle
génia, nenávidí ho, snaží se ho pomlouvat, protože bere, ale
nerozdílí se. Koří se mu však, jen je-li houževnatý. Jedním slovem,
koří se mu na kolenou, když ho nemohli pohřbít v blátě. Korupce
je v přesile, talent je vzácný. A tak korupce je zbraní prostřednosti,
jíž je ažaž, a všude \textbf{pocítíte její hrot}\footnote{metafora}. Uvidíte ženy, jejichž manželé
mají celkem šest tisíc franků příjmu a které utratí víc než deset
tisíc franků jen za svou parádu. Uvidíte úředníčky s dvanácti sty
franky, jak kupují pozemek. Uvidíte ženy, jak se prodávají, aby
mohly jet v kočáře syna některého francouzského \textbf{paira}\footnote{[péra], francouzský šlechtický titul}, který smí
v Longchampu projíždět prostředkem cesty. Viděl jste
\textbf{zbědovaného ubožáka}\footnote{pleonasmus} otce Goriota, který byl nucen proplatit
směnku uvalenou mu na krk jeho dcerou, jejíž manžel má padesát
tisíc liber důchodu. Sázím se, že se \textbf{na každém druhém kroku}\footnote{hyperbola}
potkáte v Paříži s nějakými ďábelskými uskoky. \textbf{Svou hlavu bych
vsadil}\footnote{synekdocha, hyperbola} proti hlávce tohoto salátu, že \textbf{píchnete do vosího hnízda}\footnote{metafora} u
první bohaté, krásné mladé ženy, jež se vám zalíbí. Všechny \textbf{jsou
na štíru} se zákony a pro všechno \textbf{jsou se svými manžely na válečné
noze}\footnote{metafora}. Nikdy bych neskoncoval, kdybych vám musel vykládat o
nekalých obchodech, které provádějí pro milence, pro hadříky, pro
děti, pro domácnost, nebo pro ješitnost, zřídka pro ctnost, tím si
buďte jist!


\newpage
\section*{Tématická stránka díla}
\begin{itemize}
\item \textbf{literární druh a žánr}: epický, kriticky realistický román
\item \textbf{téma a motiv}: 
	\begin{itemize}
\item \textbf{hlavní téma}: Mladý student práv Evžen se vydá studovat do Paříže, kde poznává klady a zápory společenského života. Zároveň poznává otce Goriota, který miluje své dcery, ale ony pro něj neudělají nic.
	\item \textbf{další motivy v díle}: 
		\begin{itemize}
		\item společenské vrstvy
		\item láska otcovská, láska Evžena k Delfíně
		\item peníze, zločiny, penzion
		\end{itemize}
	\end{itemize}
\item \textbf{časoprostor}: Paříž, 1. pol. 19. století
	\begin{itemize}
	\item dům paní Vauquerové, sídla Delfíny a Anastázie, byt Delfíny a Evžena
	\end{itemize}
\item \textbf{zasazení výňatku do kontextu díla}:
	\begin{itemize}
	\item \textbf{časoprostor}: v domě paní Vauquerové, cca třetina díla
	\item \textbf{obsah}: Vautrin vede monolog o tom, že se Evžen nemůže navždy držet svých morálních zásad, protože s nimi v Paříži nezbohatne. Mluví o loajalitě ke svým přátelům. Snaží se přesvědčit Evžena, aby se oženil s Viktorínou pro peníze.
	\end{itemize}
\item \textbf{kompoziční výstavba}
	\begin{itemize}
	\item využity kompozice
		\begin{itemize}
		\item chronologická 
		\item retrospektivní -- informace o otci Goriotovi, o Vautrinovi
		\end{itemize}
	\item dílo není členěno na kapitoly, jen odstavce
	\end{itemize}
\end{itemize}
\section*{Kompozice, postavy}
\begin{itemize}
\item vypravěč / lyrický subjekt: vnější, nezúčastněný vypravěč, er forma
\item vyprávěcí způsoby: 
	\begin{itemize}
	\item popisy, dialogy, monology
	\end{itemize}
\item \textbf{veršová výstavba}: psáno prózou
\end{itemize}

\paragraph{Postavy}
\begin{itemize}
\item \textbf{Evžen de Rastignac}
	\begin{itemize}
	\item student práv, pochází z chudého prostředí, do Paříže cestuje za vzděláním za podpory
	\item postupně objevuje jaký je život ve vysokých kruzích v Paříži, poznává povahy lidí a ztrácí své morální hodnoty
	\item v pozdější části knihy přestane mít studium prioritu
	\item zamiluje se do Goriovy dcery Delfíny
	\item má dobré srdce, stará se o Goriota, když na něj dcery zapomenou, zaplatí za něj pohřeb 
	\item lehce manipulovatelný, Delfína ho přesvědčila aby šel na ples, přijal od otce Goriota byt, \ldots
	\item během díla lze pozorovat výrazný vývoj osobnosti postavy
	\end{itemize}
	
\item \textbf{Bianchon} -- Evženův nejlepší přítel, student medicíny

\item \textbf{Otec Goriot} 
	\begin{itemize}
	\item otec Delfíny a Anastázie, bezmezně je miluje, dcery mu nejsou za to vděčné, protože je vždy rozmazloval
	\item bývalý bohatý výrobce nudlí, dobrý obchodník, dal každé ze svých dcer 800000 franků jako věno
	\item na konci si uvědomí, že ho dcery jen využívali kvůli penězům
	\end{itemize}
\item \textbf{Delfína de Nucingen} 
	\begin{itemize}
	\item dcera otce Goriota, provdaná za barona bankéře, nemají děti
	\item má otce raději než Anastázie
	\item miluje Evžena
	\end{itemize}	 
\item \textbf{Anastázie de Restaurd}
	\begin{itemize}
	\item dcera otce Goriota, provdaná za hraběte, žije s ním, má milence, některé děti jsou jeho
	\item má nedostatek finnančních prostředků a snaží se od otce Goriota získat více peněz pro Maxima, který je prohraje
	\end{itemize}
\item \textbf{paní de Beausánt}
	\begin{itemize}
	\item Evženova sestřenice, vysvětluje mu jak se chovat ve francouzské společnosti
	\end{itemize}
\item \textbf{Viktorína Taileferová} -- zamilovaná do Evžena, vnímána jako čistě pozitivní postava
\item \textbf{Vautrin}
	\begin{itemize}
	\item nehledí na morální hodnoty, snaží se udělat, co je pro něj nejvýhodnější
	\item v závěru díla se čtenář dozvídá jeho minulost
	\end{itemize}	 
\item \textbf{paní Vauquerová}
	\begin{itemize}
	\item majitelka penzionu, penězi-chtivá
	\item rozděluje hosty podle peněz do lepších a horších pokojů a pater
	\end{itemize}
\end{itemize}
\section*{Jazyk}
\begin{itemize}
\item zdlouhavé popisy
\item složitá souvětí
\item jazyk čitelný, s malým množstvím archaických výrazů
\item jazykové prostředky a jejich funkce ve výňatku
\item tropy a figury ve výňatku
\end{itemize}
\section*{Literárně historický kontext}
\begin{itemize}
\item Honoré de Balzac je autorem francouzského realismu. Žil v letech 1799--1850. Dílo spadá do cyklu románů \textit{Lidská komedie}, jehož cílem je zachytit život všech společenských vrstev ve Francii první poloviny devatenáctého století. Mezi další díla v cyklu patří \textit{Lesk a bída kurtizán}, \textit{Ztracené iluze}, \textit{Evženie Grandeová}, \ldots
\end{itemize}
\section*{Zdroje}
\begin{itemize}
\item DE BALZAC, Honoré. Otec Goriot. Mladá Fronta, 1970. Dostupné také z: https://cestina.xyxy.cz/wp-content/uploads/2014/02/Otec-Goriot-CZ.pdf
\item http://www.cesky-jazyk.cz/ctenarsky-denik/honore-de-balzac/otec-goriot-rozbor.html
\item https://cs.wikipedia.org/wiki/Otec\_Goriot
\item vlastní poznámky z hodin
\end{itemize}

\end{document}