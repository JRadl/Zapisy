\documentclass[10pt,a4paper]{article}
\usepackage[utf8]{inputenc}
\usepackage[czech]{babel}
\usepackage{amsmath}
\usepackage{amsfonts}
\usepackage{amssymb}
\usepackage{chemfig}
\usepackage{geometry}
\usepackage{wrapfig}
\usepackage{graphicx}
\usepackage{floatflt}
\usepackage{hyperref}
\usepackage{fancyhdr}
\usepackage{tabularx}
\usepackage{makecell}
\usepackage{csquotes}
\usepackage{marginnote}

\MakeOuterQuote{"}

\renewcommand{\labelitemii}{$\circ$}
\renewcommand{\labelitemiii}{--}
\newcommand{\ra}{$\rightarrow$ }
\newcommand{\x}{$\times$ }
\newcommand{\lp}[2]{#1 -- #2}
\newcommand{\timeline}{\input{timeline}}


\geometry{lmargin = 0.8in, rmargin = 0.8in, tmargin = 0.8in, bmargin = 0.8in}

\date{}
\author{Jakub Rádl}
\title{Karel Jaromír Erben: Kytice -- Rozbor díla}


\newcommand{\note}[1]{\marginnote{\hspace{-0.6\textwidth}#1}}

\begin{document}
\maketitle

\section*{Výňatek}
\textbf{Zlatý kolovrat}\\
\\
I\\
Okolo lesa pole lán, \note{historismus}\\
hoj jede, jede z lesa pán,		\note{zvolání, epizeuxis}\\
na vraném bujném jede koni, \note{epiteta, eufonie(e,o)}\\
vesele podkovičky zvoní,\\
jede sám a sám. \note{epizeuxis}\\
\\
A před chalupou s koně hop\\
a na chalupu: klop, klop, klop! \note{epizeuxis, onomatopoie}\\
"Hola hej! otevřte mi dvéře, 	\note{zvolaní}\\
zbloudil jsem při lovení zvěře, \\
dejte vody pít!"\\
\\
Vyšla dívčina jako květ, 		\note{přirovnání}\\
neviděl také krásy svět; 		\\
přinesla vody ze studnice,\\
stydlivě sedla u přeslice,		\note{aliterace}\\
předla, předla len. \note{epizeuxis}\\
\\
Pán stojí, nevěda, co chtěl, \note{archaický přechodník}\\
své velké žízně zapomněl; \note{archaismus(zapomenul + 3.p.)}\\
diví se tenké, rovné niti,\\
nemůže očí odvrátiti\\
z pěkné přadleny.\\
\\
"Svobodna-li jest ruka tvá,\\
ty musíš býti žena má!"\\
dívčinu k boku svému vine -\\
"Ach pane, nemám vůle jiné,\\
než jak máti chce."\\
\\
"A kde je, děvče, máti tvá?\\
Nikohoť nevidím tu já." -\\
"Ach pane, má nevlastní máti\\
zejtra se s dcerou domů vrátí,\\
vyšly do města."\\
\\
II\\
Okolo lesa pole lán,\\
hoj jede, jede zase pán;\\
na vraném bujném jede koni,\\
vesele podkovičky zvoní,\\
přímo k chaloupce.\\
\\
A před chalupou s koně hop\\
a na chalupu: klop, klop, klop! \note{epizeuxis, onomatopoie}\\
"Hola! otevřte, milí lidi, \\
ať oči moje brzo vidí\\
potěšení mé!"\\
\\
Vyšla babice, kůže a kost:\\
"Hoj, co nám nese vzácný host?"\\
"Nesu ti, nesu v domě změnu,\\
chci tvoji dceru za svou ženu,\\
tu tvou nevlastní."\\
\\
\section*{Tématická stránka díla}
\begin{itemize}
\item \textbf{literární druh a žánr}: lyricko-epický, sbírka především balad, pohádka (Zlatý kolovrat), legenda (Záhořovo lože), pověst (věštkyně)
	\begin{itemize}
	\item balada (většina díla, typická Polednice, Vodník, \ldots) -- lyricko-epická báseň, s pochmurným dějem a většinou tragickým koncem
	\item pohádka (Zlatý kolovrat) -- nadpřirozené jevy, živá voda, mrtvá voda, macecha + vlastní a nevlastní dcera, dobrý konec
	\item pověst 
	\item legenda
	\end{itemize}
\item \textbf{téma a motiv}: 
	\begin{itemize}
	\item \textbf{hlavní téma}:
		\begin{itemize}
		\item lidé poruší pravidla, morální zásady, jsou příliš chamtiví, nevěří v boha což se jim později vymstí a skončí špatně. Kladné postavy jsou naopak nakonec zachráněny
		\item v úvodní a závěrečné básni se silně projevuje \textbf{vlastenectví}
		\item častým tématem je \textbf{mateřská láska}, vztah matky a dítěte, ve \textit{Vodníkovy} zdvojené (láska matky k dceři a zároveň dcery k jejímu dítěti, v protikladu), v \textit{Dceřině kletbě} dcera proklíná svou matku, že ji nezastavila ve vztahu, v \textit{Polednici} dítě matka k sobě tiskne tak moc, až ho udusí
		\end{itemize}		 
	\item \textbf{další motivy v díle}:
		\begin{itemize}
		\item projevují se prvky \textbf{ústní lidové slovesnosti} -- polednice, vodník, oživlí mrtví
		\item motiv \textbf{viny a trestu} - i za malou chybu je člověk nelítostně a nepoměrně potrestán
		\item \textbf{pokání} -- uvědomí li si hříšník své chyby, je mu odpuštěno a Bůh ho ochrání od neštěstí
		\item \textbf{hororové motivy}
		\end{itemize}
	\end{itemize}
\item \textbf{časoprostor}: 
	\begin{itemize}
	\item neurčitá doba, pravděpodobně doba, kdy bylo dílo psáno
	\item neurčité venkovské prostředí (rybník, les, \ldots)
	\item neurčitost časoprostoru je dána stále přepracovávanou a oživovanou ústní lidovou slovesností
	\end{itemize}
\item \textbf{zasazení výňatku do kontextu díla}:
	\begin{itemize}
	\item \textbf{časoprostor}: počátek básně \textit{Zlatý kolovrat}
	\item \textbf{obsah}: Láska na první pohled -- princ se zamiluje do dívky, kterou potká v lese
	\end{itemize}
\item \textbf{kompoziční výstavba}
	\begin{itemize}
	\item dělení díla
		\begin{itemize}
		\item jednotlivé básně mají stejné téma symetricky podle středu díla (1. a 12., 2. a 11., \ldots)
		\item kytice a věštkyně jsou vlastenecké
		\item Zlatý kolovrat a Záhořovo lože mají dobrý konec
		\end{itemize}
	\item kompozice úryvku
		\begin{itemize}
		\item pět částí, každá dělena na sloky
		\item kompozice chronologická
		\item mnoho paralelismů, opakování a gradace
		\item častý motiv tří, trojího opakování
		\end{itemize}
	\end{itemize}
\end{itemize}
\section*{Kompozice, postavy}
\paragraph{Zlatý kolovrat}
\begin{itemize}
\item vypravěč / lyrický subjekt: vnější, nezúčastněný vypravěč, er forma
\item vyprávěcí způsoby: důrazné dialogy, také popisy
\item typy promluv: přímá řeč
\item \textbf{veršová výstavba}:
	\begin{itemize}
	\item verš vázaný -- 8, 8, 9, 9, 5 slabik
	\item stopa -- střídá se trochej a jamb
	\end{itemize}
\end{itemize}

\paragraph{Postavy}
\begin{itemize}
\item hlavní postavou je většinou dívka provázaná s matkou/dítětem mateřskou láskou
\item nadpřirozené postavy -- polednice, vodník, mrtvý ze \textit{Svatební košile}
\item Zlatý kolovrat
	\begin{itemize}
	\item pán -- dobrý král, zamiluje se do dívky
	\end{itemize}	 
\end{itemize}

\paragraph{Jednotlivé skladby}
\begin{itemize}
\item 
\textbf{Kytice}, 
\textbf{Poklad}, 
\textbf{Svatební košile}, 
\textbf{Polednice}, 
\textbf{Zlatý kolovrat}, 
\textbf{Štědrý den}, 
\textbf{Holoubek}, 
\textbf{Záhořovo lože}, 
\textbf{Vodník}, 
\textbf{Vrba}, 
\textbf{Lilie} (zařazena až později), 
\textbf{Svatební košile}, 
\textbf{Věštkyně}, 
\end{itemize}


\section*{Jazykové prostředky ve výňatku}

	\begin{itemize}
	\item jednoduchý jazyk, nevyskytuje se mnoho archaismů ani inverzí, vyskytují se venkovské výrazy
	\end{itemize}

\section*{Literárně historický kontext}
\begin{itemize}
\item Karel Hynek Mácha (1810--1836) byl českým autorem 3. fáze českého národního obrození
\item  inspirován zahraničními autory (Shakespeare, Goethe, \ldots) byl typickým představitelem světového romantismu, za své doby ale nepochopen
\item psal především prózu (povídky, romány, obrazy), také poezii (Máj) 
\item inspirován krajinou
\item většina jeho děl byla vydána až po jeho smrti ve 26 letech na zápal plic
\item další díla:
	\begin{itemize}
	\item Kat: Křivoklad
	\item Obrazy ze života mého
	\item Pouť krkonošská
	\item Cikáni
	\end{itemize}
\end{itemize}
\section*{Zdroje}
\begin{itemize}

\item https://cs.wikipedia.org/wiki/Kytice\_(sb\%C3\%ADrka) \\
\item Karel Jaromír Erben: Kytice (https://web2.mlp.cz/koweb/00/03/37/00/42/kytice.pdf)
\item poznámky z hodin
\end{itemize}
\end{document}