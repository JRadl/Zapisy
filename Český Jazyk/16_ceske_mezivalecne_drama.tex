\title{České meziválečné drama}
\documentclass[10pt,a4paper]{article}
\usepackage[utf8]{inputenc}
\usepackage[czech]{babel}
\usepackage{amsmath}
\usepackage{amsfonts}
\usepackage{amssymb}
\usepackage{chemfig}
\usepackage{geometry}
\usepackage{wrapfig}
\usepackage{graphicx}
\usepackage{floatflt}
\usepackage{hyperref}
\usepackage{fancyhdr}
\usepackage{tabularx}
\usepackage{makecell}
\usepackage{csquotes}
\usepackage{footnote}
\usepackage{movie15}
\MakeOuterQuote{"}

\renewcommand{\labelitemii}{$\circ$}
\renewcommand{\labelitemiii}{--}
\newcommand{\ra}{$\rightarrow$ }
\newcommand{\x}{$\times$ }
\newcommand{\lp}[2]{#1 -- #2}
\newcommand{\timeline}{\input{timeline}}


\geometry{lmargin = 0.8in, rmargin = 0.8in, tmargin = 0.8in, bmargin = 0.8in}
\date{\today}
\author{Jakub Rádl}

\makeatletter
\let\thetitle\@title
\let\theauthor\@author
\makeatother

\hypersetup{
colorlinks=true,
linkcolor=black,
urlcolor=cyan,
}



\begin{document}
\maketitle
\tableofcontents
\begin{figure}[b]
Toto dílo \textit{\thetitle} podléhá licenci Creative Commons \href{https://creativecommons.org/licenses/by-nc/4.0/}{CC BY-NC 4.0}.\\ (creativecommons.org/licenses/by-nc/4.0/)
\end{figure}
\newpage

\paragraph{Divadla}
	\begin{itemize}
	\item \textbf{Osvobozené divadlo}
		\begin{itemize}
		\item Jiří Voskovec, Jan Werich, Jaroslav Ježek -- hudebník
			\begin{itemize}
			\item po Mnichovu divadlo nemohlo pokračovat \ra emigrovali do Ameriky
			\item Ježek v Americe umírá
			\item 48 Werich se nakonec vrátil do Česka, Voskovec nakonec zpět do Ameriky \ra herec
			\item  
			\end{itemize}
		\item osvobozeno z tradičních divadelních forem \ra začali nové formy \ra avantgardní divadlo
		\item navazuje na dadaismus -- hravost, poetismus -- asociace, návaznosti, cirkusové prostředí -- klauni, 
		\item velikou roli hraje hudba
		\item \textbf{autorské divadlo} -- hry vznikaly v rámci divadla
		\item původně hravé hry
		\item později společenská funkce -- varování, vyjadřování obav formou satyry
			\begin{itemize}
			\item forbíny -- předscéna (před zataženou oponou)
			\item s vyostřením situace se v předscénách diskutovalo o problematice
			\end{itemize}
		\item \textbf{caesar}
		\item \textbf{osel a stín}
		\item \textbf{těžká barbora}
		\end{itemize}
	\end{itemize}
\item \textbf{D 34}
	\begin{itemize}
	\item název podle roku (D35, D36, \ldots) -- vznik 1934
	\item \textbf{režisérské divadlo} -- hry zde nevznikají, jsou pouze adaptovány
	\item \textbf{Emil František Burian} -- zakladatel, režisér
	\item uvádí buďto původní dramata, nebo díla, která původně dramata -> převedeno
		\begin{itemize}
		\item Máj, Evžen Oněgin, Krysař
		\item Matka, Bílá nemoc, RUR, hry Františka Langera
		\end{itemize}
	\end{itemize}
\end{document}