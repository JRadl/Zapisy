\title{Česká literatura 19./20. stol. -- 1. sv. v.}
\documentclass[10pt,a4paper]{article}
\usepackage[utf8]{inputenc}
\usepackage[czech]{babel}
\usepackage{amsmath}
\usepackage{amsfonts}
\usepackage{amssymb}
\usepackage{chemfig}
\usepackage{geometry}
\usepackage{wrapfig}
\usepackage{graphicx}
\usepackage{floatflt}
\usepackage{hyperref}
\usepackage{fancyhdr}
\usepackage{tabularx}
\usepackage{makecell}
\usepackage{csquotes}
\usepackage{footnote}

\MakeOuterQuote{"}

\renewcommand{\labelitemii}{$\circ$}
\renewcommand{\labelitemiii}{--}
\newcommand{\ra}{$\rightarrow$ }
\newcommand{\x}{$\times$ }
\newcommand{\lp}[2]{#1 -- #2}
\newcommand{\timeline}{\input{timeline}}


\geometry{lmargin = 0.8in, rmargin = 0.8in, tmargin = 0.8in, bmargin = 0.8in}
\date{\today}
\author{Jakub Rádl}

\makeatletter
\let\thetitle\@title
\let\theauthor\@author
\makeatother

\hypersetup{
colorlinks=true,
linkcolor=black,
urlcolor=cyan,
}



\begin{document}
\maketitle
\tableofcontents
\begin{figure}[b]
Toto dílo \textit{\thetitle} podléhá licenci Creative Commons \href{https://creativecommons.org/licenses/by-nc/4.0/}{CC BY-NC 4.0}.\\ (creativecommons.org/licenses/by-nc/4.0/)
\end{figure}
\newpage

\section{Úvod}
\begin{itemize}
\item střed mladé a starší generace
\item tvůrce má právo na vyjádření, nemusí se podřizovat vlastenectví
\item společenský svět není ideální místo, tak proč by mu měla literatura sloužit
\item[\ra] individualizace literatury, důraz na osobnost, osobní prožitek 
\item vliv světové literatury \ra symbolismus, impresionismus, dekadence
\end{itemize}

\section{Autorská uskupení 90. let}
\subsection{Dekadenti}
\begin{itemize}
\item sdružování kolem časopisu \textbf{Moderní revue}
	\begin{itemize}
	\item filosofické, psychologické texty, ilustrace
	\end{itemize}
\item umění má sledovat především estetické cíle (nevyhraněný parnasismus)
\end{itemize}

\subsubsection{Jiří Karásek ze Lvovic (1871--1951)}
\begin{itemize}
\item sbírka \textbf{Sodoma}
	\begin{itemize}
	\item aluze na Sodomu a Gomoru z Bible
	\item 
	\end{itemize}
\end{itemize}


\subsection{Karel Hlaváček (1874--1898)}
\begin{itemize}

\item sbírka \textbf{Pozdě k ránu}
	\begin{itemize}
	\item neurčité (i z názvu), smutné básně, melancholická hudebnost, tajemnost
	\item \textbf{Hrál kdosi na hoboj} (\textit{Čít. 333})
		\begin{itemize}
		\item smutná, osamělá, malátná, neurčitá atmosféra
		\item mnoho smutně zabarvených slov, často použity hlásky o, u, ou, m, l, i
		\end{itemize}
	\item báseň \textbf{Dva hlasy}
	\end{itemize}
\item sbírka \textbf{Mstivá kantiléna}
	\begin{itemize}
	\item 
	\end{itemize}
\end{itemize}

\subsection{Česká Moderna}
\begin{itemize}
\item hlásí se k směrům světové literární moderny
\item mladší generace 
\item \lp{1895}{Manifest České moderny}
	\begin{itemize}
	\item[1.] umělecké názory
		\begin{itemize}		
		\item publikován v časopise Rozhledy
		\item tvůrci mají právo na individuální vyjádření, na svobodný, nikomu nepodřizovaný projev (reakce na přílišné vlastenectví)
		\item důležitá je individualita autora \ra díla by měla být různorodá
		\item důležitost zahraniční literatury -- inspirovat, ale nenapodobovat
		\item literární kritika je vlastní druh umění
		\end{itemize}
	\item[2.] politické názory
		\begin{itemize}
		\item všeobcné volební právo
		\item národnostní tolerance
		\item zlepšování sociálních podmínek nižších vrstev
		\item kritika některých společenských stran
		\end{itemize}
	\item Karel Josef Šlejhar, Vilém Mrštík, František Xaver Šalda, Antonín Sova, Otokar Březina, Josef Svatopluk Machar
\end{itemize}
\item uskupení dlouho nevydrželo, ale myšlenka manifestu byla důležitá
\end{itemize}

\subsubsection{Josef Svatopluk Machar (1864--1942)}
\begin{itemize}
\item zatčen pro podezření z protirakouské tvorby
\item oceňoval Nerudu, zavrhoval Hálka
\item civilnější forma vyjadřování v poezii, realistický dojem, méně figur a tropů
\item sbírka \textbf{Confiteor} -- lyrická
	\begin{itemize}
	\item báseň \textbf{Lístek} (\textit{čít. 336})
	\item podstata básně až na konci -- postupně se buduje zvědavost
	\item poměrně přímočará metaforika
	\end{itemize}
\item sbírka \textbf{Zde by měly kvést růže}
	\begin{itemize}
	\item věnována ženám, poukazuje na špatné společenské poměry, ironie
	\item epická, 9 veršovaných příběhů
	\item báseň \textbf{Idyla} (\textit{čít. 336})
	\end{itemize}
\item román ve verších \textbf{Magdalena}
	\begin{itemize}
	\item hlavní postava Magdalena je prostitutka, nemá šanci se vrátit do normální společnosti \ra naturalistický podtext
	\end{itemize}
\item sbírka \textbf{Tristium Vindobona}
	\begin{itemize}
	\item žalozpěvy z Vídně
	\item politická lyrika, ironická, sarkastická, kritizuje společenskou morálku jak v Čechách tak v Rakousku
	\end{itemize}
\item sbírka \textbf{Satirikon}
	\begin{itemize}
	\item báseň \textbf{V bezmyšlenkovém hovoru} -- krátká ironická báseň
	\end{itemize}
\end{itemize}

\subsubsection{Antonín Sova (1869--1928)}
\begin{itemize}
\item zemřela mu matka, kvůli syfilidě skončil na vozíčku
\item impresionistické sbírky \textbf{Květy intimních nálad}, \textbf{Z mého kraje}, \textbf{Soucit i vzdor}
\item symbolistické sbírky \textbf{Vytoužené smutky}, \textbf{Údolí nového království}, \textbf{Ještě jednou se vrátíme}
\item báseň \textbf{Ještě jednou se vrátíme} (\textit{čít. 339})
	\begin{itemize}
	\item symbolismus -- popisuje život, vzpomínku na mládí
	\end{itemize}
\item báseň \textbf{U řek} (\textit{čít. 339})
	\begin{itemize}
	\item Květy intimních nálad
	\item impresionismus -- popisuje jeden moment u řeky
	\end{itemize}
\end{itemize}

\end{document}