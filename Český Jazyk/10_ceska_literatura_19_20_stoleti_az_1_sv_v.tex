\title{Česká literatura 19./20. stol. -- 1. sv. v.}
\documentclass[10pt,a4paper]{article}
\usepackage[utf8]{inputenc}
\usepackage[czech]{babel}
\usepackage{amsmath}
\usepackage{amsfonts}
\usepackage{amssymb}
\usepackage{chemfig}
\usepackage{geometry}
\usepackage{wrapfig}
\usepackage{graphicx}
\usepackage{floatflt}
\usepackage{hyperref}
\usepackage{fancyhdr}
\usepackage{tabularx}
\usepackage{makecell}
\usepackage{csquotes}
\usepackage{footnote}

\MakeOuterQuote{"}

\renewcommand{\labelitemii}{$\circ$}
\renewcommand{\labelitemiii}{--}
\newcommand{\ra}{$\rightarrow$ }
\newcommand{\x}{$\times$ }
\newcommand{\lp}[2]{#1 -- #2}
\newcommand{\timeline}{\input{timeline}}


\geometry{lmargin = 0.8in, rmargin = 0.8in, tmargin = 0.8in, bmargin = 0.8in}
\date{\today}
\author{Jakub Rádl}

\makeatletter
\let\thetitle\@title
\let\theauthor\@author
\makeatother

\hypersetup{
colorlinks=true,
linkcolor=black,
urlcolor=cyan,
}



\begin{document}
\maketitle
\tableofcontents
\begin{figure}[b]
Toto dílo \textit{\thetitle} podléhá licenci Creative Commons \href{https://creativecommons.org/licenses/by-nc/4.0/}{CC BY-NC 4.0}.\\ (creativecommons.org/licenses/by-nc/4.0/)
\end{figure}
\newpage

\section{Úvod}
\begin{itemize}
\item střed mladé a starší generace
\item tvůrce má právo na vyjádření, nemusí se podřizovat vlastenectví
\item společenský svět není ideální místo, tak proč by mu měla literatura sloužit
\item[\ra] individualizace literatury, důraz na osobnost, osobní prožitek 
\item vliv světové literatury \ra symbolismus, impresionismus, dekadence
\end{itemize}

\section{Autorská uskupení 90. let}
\subsection{Dekadenti}
\begin{itemize}
\item sdružování kolem časopisu \textbf{Moderní revue}
	\begin{itemize}
	\item \item filosofické, psychologické texty, ilustrace
	\end{itemize}
\item umění má sledovat především estetické cíle (nevyhraněný parnasismus)
\end{itemize}

\subsubsection{Jiří Karásek ze Lvovic (1871--1951)}
\begin{itemize}
\item sbírka \textbf{Sodoma}
	\begin{itemize}
	\item aluze na Sodomu a Gomoru z Bible
	\item 
	\end{itemize}
\end{itemize}


\subsection{Karel Hlaváček (1874--1898)}
\begin{itemize}

\item sbírka \textbf{Pozdě k ránu}
	\begin{itemize}
	\item neurčité, smutné básně
	\item \textbf{Hrál kdosi na hoboj} (\textit{Čít. 333})
		\begin{itemize}
		\item smutná, osamělá, malátná, neurčitá atmosféra
		\item mnoho smutně zabarvených slov, často použity hlásky o, u, ou, m, l, i
		\end{itemize}
	\end{itemize}
\end{itemize}

\end{document}