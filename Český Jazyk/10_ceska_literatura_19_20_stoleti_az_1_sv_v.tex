\title{Autorská uskupení 90. let}
%Česká literatura od přelomu 19. a 20. století do 1. světové války
\documentclass[10pt,a4paper]{article}
\usepackage[utf8]{inputenc}
\usepackage[czech]{babel}
\usepackage{amsmath}
\usepackage{amsfonts}
\usepackage{amssymb}
\usepackage{chemfig}
\usepackage{geometry}
\usepackage{wrapfig}
\usepackage{graphicx}
\usepackage{floatflt}
\usepackage{hyperref}
\usepackage{fancyhdr}
\usepackage{tabularx}
\usepackage{makecell}
\usepackage{csquotes}
\usepackage{footnote}

\MakeOuterQuote{"}

\renewcommand{\labelitemii}{$\circ$}
\renewcommand{\labelitemiii}{--}
\newcommand{\ra}{$\rightarrow$ }
\newcommand{\x}{$\times$ }
\newcommand{\lp}[2]{#1 -- #2}
\newcommand{\timeline}{\input{timeline}}


\geometry{lmargin = 0.8in, rmargin = 0.8in, tmargin = 0.8in, bmargin = 0.8in}
\date{\today}
\author{Jakub Rádl}

\makeatletter
\let\thetitle\@title
\let\theauthor\@author
\makeatother

\hypersetup{
colorlinks=true,
linkcolor=black,
urlcolor=cyan,
}



\begin{document}
\maketitle
\tableofcontents
\begin{figure}[b]
Toto dílo \textit{\thetitle} podléhá licenci Creative Commons \href{https://creativecommons.org/licenses/by-nc/4.0/}{CC BY-NC 4.0}.\\ (creativecommons.org/licenses/by-nc/4.0/)
\end{figure}
\newpage

\paragraph{Autorská uskupení 90. let}
\begin{itemize}
\item střet mladé generace se starší (názory národního obrození)
\item tvůrce má právo na vyjádření, nemusí se podřizovat vlastenectví
\item společenský svět není ideální místo, tak proč by mu měla literatura sloužit
\item[\ra] individualizace literatury, důraz na osobnost, osobní prožitek 
\item vliv světové literatury \ra symbolismus, impresionismus, dekadence
\end{itemize}

\section{Dekadenti}
\begin{itemize}
\item umění má sledovat především estetické cíle (nevyhraněný parnasismus)
\item sdružování kolem časopisu \textbf{Moderní revue}
	\begin{itemize}
	\item náročný obsah -- filosofické, psychologické texty, ilustrace, překlady zahraničních děl
	\end{itemize}

\end{itemize}

\subsection{Jiří Karásek ze Lvovic (1871--1951)}
\begin{itemize}
\item sbírka \textbf{Sodoma}
	\begin{itemize}
	\item otevřeně popisuje tabu témata -- smrt, sex, homosexualita
	\item aluze na Sodomu a Gomoru z Bible (Sodoma \ra nesmíme se na to dívat)
	\item sbírka byla zakázána cenzory
	\item báseň \textbf{Smrt} -- nevyhnutelnost smrti, využití eufonie
	\end{itemize}
\end{itemize}


\subsection{Karel Hlaváček (1874--1898)}
\begin{itemize}
\item nejvýraznější představitel české dekadence -- smutná hudebnost
\item psal a kreslil pro Moderní revue
\item sbírka \textbf{Pozdě k ránu}
	\begin{itemize}
	\item smutek, melancholie, neurčitost (i v názvu), nejasnost -- typické pro dekadenci
	\item \textbf{Hrál kdosi na hoboj} (\textit{Čít. 333})
		\begin{itemize}
		\item smutná, osamělá, malátná, neurčitá atmosféra
		\item mnoho smutně zabarvených slov, často použity hlásky o, u, ou, m, l, i
		\end{itemize}
	\item báseň \textbf{Dva hlasy}
		\begin{itemize}
		\item symbolismus – dva hlasy = dvě osoby, dvorce ani klekání nezavřeli – nešli spát ani v noci
		\end{itemize}
	\end{itemize}
\item sbírka \textbf{Mstivá kantiléna}
	\begin{itemize}
	\item v náznacích zmíněno povstání v Nizozemí proti Španělsku v 17. století
	\end{itemize}
\end{itemize}

\section{Česká Moderna}
\begin{itemize}
\item hlásí se k směrům světové literární moderny
\item mladší generace 
\item \lp{1895}{\textbf{Manifest České moderny}}
	\begin{itemize}
	\item zveřejněn v časopise Rozhledy
	\item[1.]\textbf{ umělecké názory}
		\begin{itemize}		
		\item publikován v časopise Rozhledy
		\item tvůrci mají právo na individuální vyjádření, na svobodný, nikomu nepodřizovaný projev (reakce na přílišné vlastenectví)
		\item důležitá je individualita autora \ra díla by měla být různorodá
		\item důležitost zahraniční literatury -- inspirovat, ale nenapodobovat
		\item literární kritika je vlastní druh umění
		\end{itemize}
	\item[2.] \textbf{politické názory}
		\begin{itemize}
		\item všeobecné volební právo
		\item národnostní tolerance
		\item zlepšování sociálních podmínek nižších vrstev
		\item kritika některých společenských stran
		\end{itemize}
	\item Karel Josef Šlejhar, Vilém Mrštík, František Xaver Šalda, Antonín Sova, Otokar Březina, Josef Svatopluk Machar
\end{itemize}
\item uskupení dlouho nevydrželo, ale myšlenka manifestu byla důležitá
\end{itemize}

\subsection{Josef Svatopluk Machar (1864--1942)}
\begin{itemize}
\item vězněn pro podezření z protirakouské tvorby, podporoval republiku, Masaryka,později kritika republiky 
\item oceňoval Nerudu, zavrhoval Hálka
\item civilnější forma vyjadřování v poezii, realistický dojem, méně figur a tropů
\item sbírka \textbf{Confiteor} -- lyrická
	\begin{itemize}
	\item v předmaturitním věku
	\item báseň \textbf{Lístek} (\textit{čít. 336})
		\begin{itemize}
			\item podstata básně až na konci -- postupně se buduje zvědavost
		\item poměrně přímočará metaforika
		\end{itemize}
	\end{itemize}
\item sbírka \textbf{Zde by měly kvést růže}
	\begin{itemize}
	\item věnována ženám, poukazuje na špatné společenské poměry, ironie
	\item epická, 9 veršovaných příběhů
	\item báseň \textbf{Idyla} (\textit{čít. 336})
	\end{itemize}
\item román ve verších \textbf{Magdalena}
	\begin{itemize}
	\item hlavní postava Magdalena je prostitutka, nemá šanci se vrátit do normální společnosti \ra naturalistický podtext
	\end{itemize}
\item sbírka \textbf{Tristium Vindobona}
	\begin{itemize}
	\item žalozpěvy z Vídně
	\item politická lyrika, ironická, sarkastická, kritizuje společenskou morálku jak v Čechách tak v Rakousku
	\end{itemize}
\item sbírka \textbf{Satirikon}
	\begin{itemize}
	\item báseň \textbf{V bezmyšlenkovém hovoru} -- krátká ironická báseň
	\end{itemize}
\end{itemize}

\subsection{Antonín Sova (1869--1928)}
\begin{itemize}
\item zemřela mu matka, kvůli syfilidě skončil na vozíčku	
\item impresionismus
	\begin{itemize}
	\item sbírky \textbf{Z mého kraje}, \textbf{Soucit i vzdor}
	\item sbírka \textbf{Květy intimních nálad}
		\begin{itemize}
		\item \textbf{U řek} (\textit{čít. 339})
			\begin{itemize}
			\item impresionismus -- popis procházek při večeru u řeky, nálada a atmosféra, eufonie a onomatopoie, zasazeno v přírodě
			\end{itemize}
		\end{itemize}
	\end{itemize}
\item symbolismus
	\begin{itemize}
	\item sbírka \textbf{Ještě jednou se vrátíme}
		\begin{itemize}
		\item \textbf{Ještě jednou se vrátíme} (\textit{čít. 339})
			\begin{itemize}
			\item symbolismus -- popisuje život, vzpomínku na mládí
			\item popisuje nejrůznější podoby lásky
			\end{itemize}
		\item \textbf{Kdo vám tak zcuchal tmavé vlasy} (\textit{čít. 341})
		\end{itemize}
	\item sbírka \textbf{Vytoužené smutky}
		\begin{itemize}
		\item vzdor, protest proti nespravedlivosti společnosti
		\item básník se uchyluje do samoty
		\end{itemize}
	\item \textbf{Údolí nového království}
		\begin{itemize}
		\item víra ve společnost, symbol naděje
		\end{itemize}
	\end{itemize}
\item lyrizovaná próza
	\begin{itemize}
	\item román \textbf{Ivův román}
	\end{itemize}
\end{itemize}

\subsection{Otokar Březina (1868--1929)}
\begin{itemize}
\item Václav Jebavý
\item největší český symbolista 
\item postupně píše volnějším veršem
\item sbírka \textbf{Tajemné dálky}
	\begin{itemize}
	\item pesimistické, ukazuje lidské smutky a trápení, trpění následující generace kvůli předchozí
	\item \textbf{Moje matka}
		\begin{itemize}
		\item matka žije smutný život, těší se na smrt, umírá, subjekt na ni vzpomíná, uvědomuje si, že z ní je stvořen, také je smutný, paralela životů
		\end{itemize}
	\end{itemize}
\item sbírka \textbf{Svítání na západě} -- trochu více naděje
\item sbírka \textbf{Větry od Pólů}
	\begin{itemize}
	\item zobrazuje různé protipóly života
	\item \textbf{Příroda} -- volný verš
	\end{itemize}
\item sbírky \textbf{Stavitelé chrámu}, \textbf{Ruce}
	\begin{itemize}
	\item naděje, pocit lidské sounáležitosti, symbol spojených rukou okolo planety
	\end{itemize}
\item \textbf{alexandrin} -- 13-slabičný rýmovaný verš
\item kniha esejů \textbf{Kniha esejů}
	\begin{itemize}
	\item odborný text podaný publicistickou formou
	\end{itemize}
\end{itemize}

\subsection{František Xaver Šalda (1867--1937)}
\begin{itemize}
\item nejvýznamnější český literární kritik
	\begin{itemize}
	\item kritika má být vášnivá, zaujatá, individuální, kritika je na hranici umění \ra esej
	\end{itemize}

\item esej \textbf{Syntetism v novém umění}
\item soubor esejí \textbf{Boje o zítřek}
\item soubor esejí \textbf{Duše a dílo}
	\begin{itemize}
	\item věnováno zahraničním i českým podobám romantismu a realismu (Rouseau, Falubert, Mácha, Němcová, Sova)
	\end{itemize}
\item soubor esejí \textbf{O nejmladší poezii české}
	\begin{itemize}
	\item věnováno meziválečným autorům (Seifrt, Wolker, Nezval)
	\end{itemize}
\item vlastní dílo nezajímavé
\end{itemize}


\section{Anarchističtí buřiči}
\begin{itemize}
\item anarchobohéma
\item mladší generace 90. let 19. stol.
\item bouření se proti Rakousku-Uhersku 
\item podobní jako prokletí básníci a dekadenti, na rozdíl od nich je revolta otevřená
\item antimilitarismus -- nechtěli armádu
\item vitalismus -- oslava života, láska
\item časopis \textbf{Nový kult} (Stanislav Kostka Neuman)
\item snaha o přiblížení tvorby co nejširší vrstvě obyvatelstva \ra menší množství květnatých výrazů a symboliky
\item záměrně navazují na ÚLS, folklór
\item důležitý je obsah sdělení a aby byl všemi pochopen, není důležitá forma 
\end{itemize}

\subsection{Viktor Dyk (1877--1931)}
\begin{itemize}
\item aforistická úsečnost (krátká průpovídka na pomezí poezie a prózy, vtipné, stručné, od důležité osoby s pointou)
\item poezie
	\begin{itemize}
	\item sbírka \textbf{Satiry a sarkasmy}
		\begin{itemize}
		\item útočná, výrazná sbírka
		\item ironizuje dobové dění, prodejnost politiků
		\end{itemize}
	\end{itemize}
	\item balada \textbf{Milá sedmi loupežníků}
		\begin{itemize}
		\item o lásce ženy k loupežníkům, jednoho má nejraději
		\end{itemize}
	\item válečná tetralogie sbírek \textbf{Lehké těžké kroky}, \textbf{Anebo}, \textbf{Okno}, \textbf{Poslední rok}
		\begin{itemize}
		\item inspirace 1. světovou válkou
		\item \textbf{Země mluví}
		\end{itemize}
	\item alegorická sbírka \textbf{Devátá vlna}
		\begin{itemize}
		\item předtucha a předzvěst smrti -- infarkt v moři
		\end{itemize}
	\item sbírka \textbf{Síla života}
		\begin{itemize}
		\item \textbf{Plíží se večery, plíží se teskna}
		\end{itemize}
\item próza
	\begin{itemize}
	\item novela \textbf{Krysař}
	\end{itemize}
\item drama
	\begin{itemize}
	\item pikareskní román \textbf{Zmoudření dona Quijota}
	\begin{itemize}
	\item (renesance -- Miquel de Servantes: Důmyslný rytíř Don Quijot de la Mancha)
	\item pikaro -- někdo si umí vždy poradit
	\end{itemize}
	\end{itemize}
\end{itemize}
\subsection{František Gellner (1881--1914)}
\begin{itemize}
\item padl v prvním roce 1. sv. války
\item malíř, karikaturista, kapulety (zábavná kabaretní vystoupení), šansony (smutné písně), městské obrazy
\item inspirace v lidové tvorbě
\item skrývá se za maskou lyrického subjektu na okraji společnosti -- opilce, zhýralce
\item sbírky \textbf{Po nás přijde potopa}
	\begin{itemize}
	\item \textbf{Vzpomínka} -- sbírání prvních sexuálních zkušeností
	\item \textbf{Perspektiva} -- lidová píseň o užívání si života
	\item \textbf{Pozdrav rodnému kraji}
		\begin{itemize}
		\item podobné jako \textbf{Zdechlina} (krásný popis začátku, najednou zdechlina)
		\end{itemize}
	\end{itemize}
\item sbírka \textbf{Radosti života}
	\begin{itemize}
	\item životní skepse a rezignace ústící v alkoholismus
	\item krátké komické básničky (\textbf{Přetékající pohár}, \textbf{Což, páni spisovatelé!}, \textbf{Radosti života}, \textbf{Všichni mi lhali})
	\end{itemize}
\end{itemize}

\subsection{Karel Toman (1877--1946)}
\begin{itemize}
\item vyjadřuje nespokojenost se světem, který ničí a omezuje individualitu \ra cestuje
\item sbírka \textbf{Pohádky krve}
	\begin{itemize}
	\item symbolismus, dekadence
	\item kritika měšťáckého života, bez vyšších cílů a vizí
	\item \textbf{Píseň} -- týpek chce sex, týpka ne, týpek je grr
	\end{itemize}
\item sbírka \textbf{Měsíce}
	\begin{itemize}
	\item popisuje přírodu s přesahem do problémů společnosti
	\item \textbf{Únor} – popis února; vzpomínka na válku, dějiny lidstva = dějiny válek; chvilka naděje \ra přesah, po únoru jaro
	\end{itemize}
\end{itemize}

\subsection{Stanislav Kostka Neuman (1875--1947)}
\begin{itemize}
\item vystřídal mnoho směrů psaní (dekadence, symbolismus, proletářská poezie, ...)
\item sbírka \textbf{Satanova sláva mezi námi} -- dekadence, 
\item sbírka \textbf{Sen o zástupu zoufajících} -- civilismus
\item sbírka \textbf{Nemesis bonorum eustos} -- anarchismus
	\begin{itemize}
	\item vychází z vlastní zkušenosti ve vězení
	\item Spravedlnost ochránkyně dobrých (ironické)
	\item i milostné verše
	\end{itemize}
\item sbírka \textbf{Kniha lesů vod a strání}
	\begin{itemize}
	\item přestěhoval se do Bílovic nad Svitavou
	\item většina básní formou modlitby
	\item \textbf{Improvizace} (paralelismus)
	\end{itemize}
\item sbírka \textbf{Nové zpěvy} -- civilismus 
	\begin{itemize}
	\item oslava civilizace, techniky, vliv W. Whitmana
	\item \textbf{Zpěvy drátů} (\textit{čít. 229})
		\begin{itemize}
		\item personifikace drátů -- slouží lidem, vyprávějí, zpívají si
		\item mnoho výčtů
		\item pravidelné rýmové schéma ABAB -- střídavé
		\end{itemize}
	\end{itemize}
\item sbírka \textbf{Rudé zpěvy} -- proletářská poezie
	\begin{itemize}
	\item \textbf{František Halas: Staré ženy}
		\begin{itemize}
		\item zachycuje jak fyzické, tak psychické stárnutí, osamělost ve stáří
		\end{itemize}
	\item \textbf{Staří dělníci}
		\begin{itemize}
		\item protipól
		\item vyzdvihuje, co zůstane po dělnících
		\end{itemize}
	\end{itemize}
\item \textbf{Bezedný rok} -- reakce na Mnichov
\end{itemize}

\subsection{Fráňa Šrámek (1877--1952)}
\begin{itemize}
\item sbírka \textbf{Modrý a rudý}
	\begin{itemize}
	\item anarchistická, antimilitaristická, proti Rakousku
	\item modrá (rakouští rezervisté), rudá (anarchismus, krev)
	\item \textbf{Raport}
		\begin{itemize}
		\item paralelismus nevinného člověka a zvířete, kůň a vojáci, nevinnost, nevědomost
		\end{itemize}
	\end{itemize}
\item sbírka \textbf{Curriculum vitae}
	\begin{itemize}
	\item vitalismus -- snaha užívat si života, milovat, ne někomu sloužit, lidová forma, dobře zapamatovatelná
	\end{itemize}
\item sbírka \textbf{Splav}
	\begin{itemize}
	\item milostná lyrika
	\end{itemize}
\item lyrizovaný román \textbf{Stříbrný vítr}
	\begin{itemize}
	\item název -- symbolika touhy někam jít, věčná touha a věčné mládí
	\item dospívající kluk, nechuť k společnosti a životu
	\end{itemize}
\item drama \textbf{Měsíc nad řekou}
\end{itemize}

\subsection{Petr Bezruč (1867--1958)}
\begin{itemize}
\item vlastním jménem Vladimír Vašek
\item pochybnosti o autorství jeho básní (možná je psal Ondřej Boleslav Petr, jméno inspirováno Dodou Besrutschovou)
\item řazen k anarchistům
\item studoval na Slovanském gymnáziu -- české vyučování
\item jeho otec (učitel) kritizoval RKZ, propagoval češství
\item strávil nějakou dobu ve Frýdku-Místku jako poštovní úředník \ra 
\item sbírka \textbf{Slezské písně}
	\begin{itemize}
	\item národnostní a sociální útlak slezského lidu (míšení s Poláky, Slováky, Němci)
	\item intimní lyrika -- obavy ze sblížení, nedostatek odvahy 
	\item místopis Slezska a širšího okolí
	\item sociální balady -- za špatný konec mohou špatné společenské poměry (nemůže zvrátit jednotlivec)
	\item sociální balada \textbf{Maryčka Magdónova}
		\begin{itemize}
		\item pěti sourozencům umřou rodiče \ra Maryčka se o ně musí starat \ra krade dřevo \ra má jít do vězení \ra skočí ze skály a zabije se (Cestou do vězení)
		\end{itemize}
	\item báseň \textbf{Motýl}
		\begin{itemize}
		\item intimní lyrika
		\item břemy -- modříny
		\item odhání od sebe motýla protože už si s ním neužívá
		\item slezské nářečí
		\end{itemize}
	\item báseň \textbf{Škaredý zjev}
		\begin{itemize}
		\item utrpení, vzdor, výhrůžka
		\item kakofonie -- r, l
		\item nezřejmé aluze
		\end{itemize}
	\item sociální balada \textbf{Kantor Halfar}
		\begin{itemize}
		\item učil česky \ra nemohl si najít stálé místo \ra nemohl si vzít svou milou \ra oběsil se \ra dostal stálé místo na hřbitově
		\end{itemize}
	\item báseň \textbf{Jen jedenkrát}
		\begin{itemize}
		\item intimní lyrika, ale s příběhem
		\item popisuje téma promarněné šance
		\end{itemize}
	\end{itemize}
\item báseň \textbf{Stužkonoska modrá}
\end{itemize}

\subsection{Jakub Deml (1878--1961)}
\begin{itemize}
\item vystudovaný kněz, spory s církví (nechtěl celibát)
\item inteligentní člověk, špatné mezilidské vztahy
\item mnoho hraničních názorů -- zarytý antifašista i antisemita
\item psal prózu plnou lyrických pasáží
	\item próza \textbf{Hrad smrti}, \textbf{Tanec smrti}
		\begin{itemize}
		\item depresivní
		\item nevidí ve smrti vykoupení
		\end{itemize}
	\item básně v próze \textbf{Moji přátelé}
		\begin{itemize}
		\item próza s básnickými prvky
		\item rozmluvy s květinami
		\end{itemize}
	\item báseň v próze \textbf{Miriam}
		\begin{itemize}
		\item propojeno s filosofickými úvahami
		\item Miriam je ztělesněním lásky, všech vztahů mužů k ženám (matka, sestra, dcera, manželka, milenka)
		\end{itemize}
	\item próza \textbf{Zapomenuté světlo}
		\begin{itemize}
		\item farář nedokáže pomoct umírající selce, v jejím bezvědomí si promítá v hlavě události svého života
		\end{itemize}
	\item deník \textbf{Šlépěje}
		\begin{itemize}
		\item obsahuje dopisy, básně, prozaické texty, komentáře, útočné pamflety 
		\end{itemize}
\end{itemize}

\end{document}