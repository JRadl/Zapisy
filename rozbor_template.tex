\documentclass[10pt,a4paper]{article}
\usepackage[utf8]{inputenc}
\usepackage[czech]{babel}
\usepackage{amsmath}
\usepackage{amsfonts}
\usepackage{amssymb}
\usepackage{chemfig}
\usepackage{geometry}
\usepackage{wrapfig}
\usepackage{graphicx}
\usepackage{floatflt}
\usepackage{hyperref}
\usepackage{fancyhdr}
\usepackage{tabularx}
\usepackage{makecell}
\usepackage{csquotes}
\usepackage{marginnote}

\MakeOuterQuote{"}

\renewcommand{\labelitemii}{$\circ$}
\renewcommand{\labelitemiii}{--}
\newcommand{\ra}{$\rightarrow$ }
\newcommand{\x}{$\times$ }
\newcommand{\lp}[2]{#1 -- #2}
\newcommand{\timeline}{\input{timeline}}


\geometry{lmargin = 0.8in, rmargin = 0.8in, tmargin = 0.8in, bmargin = 0.8in}
\newcommand{\note}[1]{\marginnote{\hspace{-0.6\textwidth}#1}}

\date{}
\author{Jakub Rádl}
\title{Karel Jaromír Erben: Kytice -- Rozbor díla}

\begin{document}
\maketitle

\section*{Výňatek}

\section*{Tématická stránka díla}
\begin{itemize}
\item \textbf{literární druh a žánr}:
\item \textbf{téma a motiv}:
	\begin{itemize}
	\item \textbf{hlavní téma}:
	\item \textbf{další motivy v díle}:
	\end{itemize}
\item \textbf{časoprostor}:
\item \textbf{zasazení výňatku do kontextu díla}:
	\begin{itemize}
	\item \textbf{časoprostor}:
	\item \textbf{obsah}: 
	\end{itemize}
\item \textbf{kompoziční výstavba}
	\begin{itemize}
	\item využity kompozice: 
	\item dělení díla
	\item jednota místa času a děje?
	\item antické dělení?
	\end{itemize}
\end{itemize}
\section*{Kompozice, postavy}
\begin{itemize}
\item vypravěč / lyrický subjekt:
\item vyprávěcí způsoby:
	\begin{itemize}
	\item 
	\item
	\end{itemize}
\item \textbf{veršová výstavba}:	
\end{itemize}

\paragraph{Postavy}
\begin{itemize}
\item \textbf{Harpagon} --
\item \textbf{Kleant} --
\item \textbf{Elisa} --
\end{itemize}
\section*{Jazyk}
\begin{itemize}
\item jazykové prostředky a jejich funkce ve výňatku
\item tropy a figury ve výňatku
\end{itemize}
\section*{Literárně historický kontext}
\begin{itemize}
\item současní autoři:
\item další autorova díla:
\end{itemize}
\section*{Zdroje}
\end{document}