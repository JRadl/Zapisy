\title{Politologie}
\documentclass[10pt,a4paper]{article}
\usepackage[utf8]{inputenc}
\usepackage[czech]{babel}
\usepackage{amsmath}
\usepackage{amsfonts}
\usepackage{amssymb}
\usepackage{chemfig}
\usepackage{geometry}
\usepackage{wrapfig}
\usepackage{graphicx}
\usepackage{floatflt}
\usepackage{hyperref}
\usepackage{fancyhdr}
\usepackage{tabularx}
\usepackage{makecell}
\usepackage{csquotes}
\usepackage{footnote}

\MakeOuterQuote{"}

\renewcommand{\labelitemii}{$\circ$}
\renewcommand{\labelitemiii}{--}
\newcommand{\ra}{$\rightarrow$ }
\newcommand{\x}{$\times$ }
\newcommand{\lp}[2]{#1 -- #2}
\newcommand{\timeline}{\input{timeline}}


\geometry{lmargin = 0.8in, rmargin = 0.8in, tmargin = 0.8in, bmargin = 0.8in}
\date{\today}
\author{Jakub Rádl}

\makeatletter
\let\thetitle\@title
\let\theauthor\@author
\makeatother

\hypersetup{
colorlinks=true,
linkcolor=black,
urlcolor=cyan,
}



\begin{document}
\maketitle
\tableofcontents
\begin{figure}[b]
Toto dílo \textit{\thetitle} podléhá licenci Creative Commons \href{https://creativecommons.org/licenses/by-nc/4.0/}{CC BY-NC 4.0}.\\ (creativecommons.org/licenses/by-nc/4.0/)
\end{figure}
\newpage

\section{Stát}
\textbf{stát} -- forma organizace lidské společnosti; sdružení obyvatel určitého území v právní celek
\paragraph{Historie}
\begin{itemize}
\item Sumerská říše, Egypt (3000 BC)
\item Městské státy v Řecku a Římě
	\begin{itemize}
	\item \textbf{polis} -- Řecko (Athény, Théby, Korynt)
	\item \textbf{civitas} -- Řím 
	\item \textbf{impérium}
	\end{itemize}
\item \textbf{prvotní} tvoření -- usazování
\item \textbf{druhotné} tvoření -- sjednocování a rozpadání (ČSR \ra ČR + SR)
\item vznik pojmu \textit{stát} v renesanci  -- Nicolo Machiavelli

\end{itemize}

\paragraph{Věda}
\begin{itemize}
\item stát zkoumá mnoho vědních disciplín z různých úhlů pohledu
\item politologie, sociologie, historie, kartografie, státověda, etika, filosofie, ekonomie, demografie
\end{itemize}

\subsection{Teorie vzniku státu}
\paragraph{Náboženská}
\begin{itemize}
\item panovník je bůh
\item Egypt, \ldots
\end{itemize}

\paragraph{Patriarchát}
\begin{itemize}
\item vznik z rodiny -- panovník se stará o své poddané (jako otec o rodinu)
\end{itemize}

\paragraph{Mocenská teorie}
\begin{itemize}
\item silní vládnou slabším
\item vychází z Darwina
\end{itemize}

\paragraph{Teorie společenské smlouvy}
\begin{itemize}
\item důvodem vzniku státu je smlouva mezi lidmi
\item lidé se vzdají části své svobody ve prospěch celku
\item stát jim za to poskytne bezpečí
\end{itemize}


\subsection{Definice státu}
\paragraph{Každý stát musí mít}
\begin{itemize}
\item obyvatelstvo
\item území
	\begin{itemize}
	\item půda -- 10km
	\item vzdušný prostor 30--100km
	\item pobřežní vody -- 22km
	\end{itemize}
\item právo
\item státní aparát
	\begin{itemize}
	\item organizace politické scény
	\item zákonodárná, výkonná soudní moc
	\end{itemize}
\item státní suverenita
\end{itemize}

\paragraph{Vnitřní funkce státu}
\begin{itemize}
\item reprezentovány ministerstvy (vzdělání, zdravotnictví, ekonomie, doprava, životní prostředí, \ldots)
\item hlavní funkcí státu je \textbf{ochrana obyvatelstva}
\end{itemize}

\paragraph{Vnější funkce}
\begin{itemize}
\item mezinárodní mír, bezpečnost
\item hospodářská, politická spolupráce
\end{itemize}


\subsection{Formy státu}
\paragraph{Podle vládce}
\begin{itemize}
\item \textbf{monarchie} -- vláda jednoho panovníka
\item \textbf{diktatura} -- vláda diktátora (tyranie -- tyran)
\item \textbf{aristokracie} -- vláda šlechty
\item \textbf{oligarchie} -- vláda bohatých
\item \textbf{sofokracie} -- vláda moudrých
\item \textbf{technokracie} -- vláda inteligence
\item \textbf{byrokracie} -- vláda úředníků
\item \textbf{demokracie} -- vláda lidu
\end{itemize}

\paragraph{Podle přenosu moci}
\begin{itemize}
\item \textbf{monarchie} -- dědičná vláda
\item \textbf{republika} -- volená vláda
	\begin{itemize}
	\item bez panovníka
	\item tři nezávislé složky: zákonodárná, výkonná, soudní
	\end{itemize}
\end{itemize}

\paragraph{Podle formy režimu}
\begin{itemize}
\item \textbf{demokracie} -- dodržování lidských práv a svobod, pravidelná změna vláda
\item \textbf{diktatura} -- potlačování práv, opozičních názorů
\end{itemize}

\paragraph{Podle výkonné a zákonodárné moci}
\begin{itemize}
\item \textbf{prezidentský} -- přímo volený prezident, výkonná moc nezávislá na zákonodárné (USA)
\item \textbf{polo-prezidentský} -- přímo volený prezident, více pravomocí, vládu kontroluje prezident i parlament (Francie)
\item \textbf{parlamentní} -- vládu kontroluje parlament (Česko)
\end{itemize}

\paragraph{Podle územně-správního členění}
\begin{itemize}
\item \textbf{konfederace} -- volné sdružení několika samostatných států (EU)
\item \textbf{federace}  -- jeden stát s více vládami (Německo, USA, Rusko)
\item \textbf{unitární stát} -- jednoduchý stát (Česko)
\end{itemize}

\paragraph{Podle výkonu státní moc}
\begin{itemize}
\item \textbf{centralizované} -- řízen z jednoho místa
\item \textbf{decentralizované} -- řízen lokálně
\end{itemize}
\end{document}