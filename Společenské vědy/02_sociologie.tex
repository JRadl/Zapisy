\title{Sociologie}
\documentclass[10pt,a4paper]{article}
\usepackage[utf8]{inputenc}
\usepackage[czech]{babel}
\usepackage{amsmath}
\usepackage{amsfonts}
\usepackage{amssymb}
\usepackage{chemfig}
\usepackage{geometry}
\usepackage{wrapfig}
\usepackage{graphicx}
\usepackage{floatflt}
\usepackage{hyperref}
\usepackage{fancyhdr}
\usepackage{tabularx}
\usepackage{makecell}
\usepackage{csquotes}
\usepackage{footnote}

\MakeOuterQuote{"}

\renewcommand{\labelitemii}{$\circ$}
\renewcommand{\labelitemiii}{--}
\newcommand{\ra}{$\rightarrow$ }
\newcommand{\x}{$\times$ }
\newcommand{\lp}[2]{#1 -- #2}
\newcommand{\timeline}{\input{timeline}}


\geometry{lmargin = 0.8in, rmargin = 0.8in, tmargin = 0.8in, bmargin = 0.8in}
\date{\today}
\author{Jakub Rádl}

\makeatletter
\let\thetitle\@title
\let\theauthor\@author
\makeatother

\hypersetup{
colorlinks=true,
linkcolor=black,
urlcolor=cyan,
}



\begin{document}
\maketitle
\tableofcontents
\begin{figure}[b]
Toto dílo \textit{\thetitle} podléhá licenci Creative Commons \href{https://creativecommons.org/licenses/by-nc/4.0/}{CC BY-NC 4.0}.\\ (creativecommons.org/licenses/by-nc/4.0/)
\end{figure}
\newpage

\section{Sociologie jako věda}
Sociologie se snaží podat celkový obraz společnosti, její struktury, jejích zákonitostí.
\begin{itemize}
\item \textit{societos} -- společnost
\item \textit{logis} -- věda
\item 1. pol. 19. stol. Auguste Conto (1798 - 1857)
\end{itemize}
Předmětem zkoumání sociologie je společnost a její struktura. Není možné brát sociologii jako přírodní vědu, její výzkumy omezenou platnost, jelikož se liší podmínky po celém světě. Povaha poznatků a zjištění je pouze pravděpodobnostní.\\
Sociologie úzce souvisí s jinými disciplínami (psychologie, historie, ekologie, politologie, ...).\\
Obsahem sociologie jsou skutečnosti všeobecně známé (zdravý rozum).

\textbf{sociologická imaginace} -- snaha o zobecnění výzkumu

\subsection{Předmět sociologie}
\paragraph{Emile Durkheim}
\begin{itemize}
\item sociologie je věda o \textbf{sociálních faktech}
\item sociální fakta jsou vnější, nezávislé na psychice a působí na člověka nátlakem (\textit{móda, jazyk})
\item \textbf{sociální realismus}
\end{itemize}

\paragraph{Max Weber}
\begin{itemize}
\item sociologie je věda o sociálním jednání
\item sociální jednání (oblékání, česání, mluvení)
\item \textbf{sociální nominalismus}
\end{itemize}

\paragraph{Sociální vs. sociologický problém}
\begin{itemize}
\item \textbf{sociální problém} -- problém, který společnost považuje za problém (určitá věc)
\item \textbf{sociologický problém} -- týká se sociologie jako vědy, ptá se proč něco funguje určitým způsobem (obecná věc)
\end{itemize}

\subsection{Metody sociologického výzkumu}
\paragraph{Kvantitativní}
\begin{itemize}
\item mnoho informací a málo jedinců
\end{itemize}
\paragraph{Kvalitativní}
\begin{itemize}
\item málo informací, mnoho jedinců
\end{itemize}

\end{document}