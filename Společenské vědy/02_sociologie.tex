\title{Sociologie}
\documentclass[10pt,a4paper]{article}
\usepackage[utf8]{inputenc}
\usepackage[czech]{babel}
\usepackage{amsmath}
\usepackage{amsfonts}
\usepackage{amssymb}
\usepackage{chemfig}
\usepackage{geometry}
\usepackage{wrapfig}
\usepackage{graphicx}
\usepackage{floatflt}
\usepackage{hyperref}
\usepackage{fancyhdr}
\usepackage{tabularx}
\usepackage{makecell}
\usepackage{csquotes}
\usepackage{footnote}
\usepackage{movie15}
\MakeOuterQuote{"}

\renewcommand{\labelitemii}{$\circ$}
\renewcommand{\labelitemiii}{--}
\newcommand{\ra}{$\rightarrow$ }
\newcommand{\x}{$\times$ }
\newcommand{\lp}[2]{#1 -- #2}
\newcommand{\timeline}{\input{timeline}}


\geometry{lmargin = 0.8in, rmargin = 0.8in, tmargin = 0.8in, bmargin = 0.8in}
\date{\today}
\author{Jakub Rádl}

\makeatletter
\let\thetitle\@title
\let\theauthor\@author
\makeatother

\hypersetup{
colorlinks=true,
linkcolor=black,
urlcolor=cyan,
}



\begin{document}
\maketitle
\tableofcontents
\begin{figure}[b]
Toto dílo \textit{\thetitle} podléhá licenci Creative Commons \href{https://creativecommons.org/licenses/by-nc/4.0/}{CC BY-NC 4.0}.\\ (creativecommons.org/licenses/by-nc/4.0/)
\end{figure}
\newpage

\section{Sociologie jako věda}
Sociologie se snaží podat celkový obraz společnosti, její struktury, jejích zákonitostí. Předmětem zkoumání sociologie je společnost a její struktura.
\begin{itemize}
\item \textit{societos} -- společnost
\item \textit{logis} -- věda
\item zakladatel sociologie: \textbf{Auguste Conto} (1798--1857) (1. pol. 19. stol.)
\end{itemize}
Sociologii \textbf{nelze brát jako exaktní vědu}. Její výzkumy omezenou platnost, jelikož se po celém světě liší podmínky. Povaha poznatků a zjištění je tedy pouze \textbf{pravděpodobnostní}.
\begin{itemize}
\item \textbf{sociologická imaginace} -- snaha o zobecnění výzkumu
\end{itemize}
Sociologie úzce souvisí s jinými disciplínami (psychologie, historie, ekologie, politologie, ...).\\



\subsection{Historie a předmět sociologie}
\paragraph{Emile Durkheim}
\begin{itemize}
\item sociologie je věda o \textbf{sociologických faktech}
\item sociální fakta jsou vnější, nezávislé na psychice a působí na člověka nátlakem (\textit{móda, jazyk})
\item zastává kvantitativní
\item \textbf{sociální realismus}
\end{itemize}

\paragraph{Max Weber}
\begin{itemize}
\item sociologie je věda o \textbf{sociálním jednání} (oblékání, česání, mluvení)
\item zastává kvalitativní výzkum
\item \textbf{sociální nominalismus}
\end{itemize}

\paragraph{Karl Marx, Friedrich Engelsem}
\begin{itemize}
\item kapitalismus je problémem
\item společnost nemůže fungovat jako velký celek, ale spíše jako více celků s vlastními zájmy
\end{itemize}

\paragraph{Sociální vs. sociologický problém}
\begin{itemize}
\item \textbf{sociální problém} -- problém, který společnost považuje za problém (určitá věc)
\item \textbf{sociologický problém} -- týká se sociologie jako vědy, ptá se proč něco funguje určitým způsobem (obecná věc)
\end{itemize}

\subsection{Metody sociologického výzkumu}
\paragraph{Kvantitativní}
\begin{itemize}
\item mnoho informací a málo jedinců
\end{itemize}
\paragraph{Kvalitativní}
\begin{itemize}
\item málo informací, mnoho jedinců
\end{itemize}

\subsection{Kvantitativní dotazníček: \textit{Jak jsi spokojený s Jaroškou?}}
\paragraph{Ohodnoť 1--10, 1 je nejhorší, 10 je nejlepší}
\begin{enumerate}
\item Jak rád chodíš na Jarošku?
\item Jak jsi spokojený se svým rozvrhem?
\item Jak jsi spokojený s tím, kolik se toho naučíš?
\item Jak jsi spokojený se svými vyučujícími?
\item Jak jsi spokojený s metodami výuky na Jarošce?
\item Jak jsi spokojený s možností výběru předmětů?
\item Jak jsi spokojený s výukou matematiky?
\item Jak moc jsi někdy uvažoval o změně školy?
\item Jak ti chutnají školní obědy?
\item Jak rád chodíš na Jarošku?
\end{enumerate}




\section{Kultura jako způsob života}
\textbf{Kulturou} rozumíme zakotvené \textbf{sociální vzorce vnímání a jednání}, které si lidé osvojili a které jsou pozorovatelné v jejich jednání. Podstata veškerého lidského jednání je kulturně podmíněna.
\begin{itemize}
\item člověk -- tvůrce i produkt kultury
\item příroda -- opak kultury
\end{itemize}

\paragraph{Prvky kultury}
\begin{itemize}
\item \textbf{vědění a znalosti} -- každodenně potřebné vědomosti, poznávací stránka kultury
\item \textbf{normy a hodnoty} -- shoduje se na nich většina společnosti
\item \textbf{materiální statky} -- fyzické předměty (artefakty) specifické pro určitou kulturu
\item \textbf{jazyk} -- symbolická komunikace v kultuře
\end{itemize}

\paragraph{Kulturní difuze}
\begin{itemize}
\item promíchání kultur
\item časem se mění lidské vědomosti, normy, hodnota a tím celá kultura společnosti
\end{itemize}

\paragraph{Useless pojmy}
\begin{itemize}
\item \textbf{instituce} -- typy, vzorce jednání vymezené společností, vzorové chování
\item \textbf{sociální organizace} -- způsob provázání lidí, kteří jednají určitým způsobem (škola, vězení, rodina)
\end{itemize}




\section{Komunikace}
Přenos významu prostřednictvím symbolů, získávání informací od ostatních lidí.
\begin{itemize}
\item \textbf{sociální komunikace} -- veškeré mezilidské předávání 
\end{itemize}

\paragraph{Dělení}
\begin{itemize}
\item verbální -- slovní
\item neverbální -- pomocí symbolů
\end{itemize}

\paragraph{Formy komunikace}
\begin{itemize}
\item primitivní -- tváří v tvář
\item tradiční -- psané texty
\item moderní -- sdělovací prostředky, masová komunikace
\end{itemize}


\paragraph{Masová komunikace}
\begin{itemize}
\item informační, orientační, socializační, interpretační, veřejně kontrolní,
 zábavná funkce
\item umožňuje rychlé šíření informací
	\begin{itemize}
	\item dobře i špatně -- informace $\times$ manipulace
	\end{itemize}
\item anonymní příjemci
\item jednosměrná -- neočekává zpětnou vazbu
\item noviny, televize, rádio, sociální sítě
\end{itemize}

\paragraph{Another useless pojmy}
\begin{itemize}
\item \textbf{investigativní žurnalistika} -- 
\item \textbf{veřejnoprávní média} -- financována koncesionářskými poplatky 
\end{itemize}




\section{Socializace}
Proces poznávání a \textbf{osvojování si kulturních norem}.
\begin{itemize}
\item \textbf{resocializace} -- přizpůsobení se nové společnosti
\item \textbf{výchova} -- cílená socializace
\item nejdůležitější úlohu hrají nejbližší příbuzní $>$ vrstevníci $>$ škola
\end{itemize}

\paragraph{Rodina}
\begin{itemize}
\item základní sociální a ekonomická jednotka
\item povinnost výchovy dětí
\item \textbf{neúplná} $\times$ \textbf{úplná}
\item \textbf{nukleární} $\times$ \textbf{široká}
\item \textbf{orientační} -- původní rodina, ze které člověk pochází
\item \textbf{prokreační} -- založená rodina
\item 
\item \textbf{monogamie}(1), \textbf{bigamie}(2), \textbf{polygamie}(3+)
\begin{itemize}
\item \textbf{polygynie} -- muž má více žen
\item \textbf{polyandrie} -- žena má více mužů
\end{itemize}
\item rozvodovost v Česku -- 66\%
\end{itemize}

\paragraph{Sociální deviace}
\begin{itemize}
\item odchylování chování od společenských norem
\item \textbf{sociální patologie} -- zaměřuje se na příčiny deviantního chování
\end{itemize}




\section{Sociální vztahy a sociální struktura}
\begin{itemize}
\item \textbf{sociální struktura} -- síť vztahů, jedinců, pozic a rolí
\item \textbf{sociální pozice} (status) -- hodnota postavení ve, které člověk zaujímá
\item \textbf{sociální role} -- soubor očekávaného chování člověka
\item sociální kontrola
\begin{itemize}
\item \textbf{svědomí} -- vnitřní kontrola společnosti
\item \textbf{sankce} -- vnější kontrola sociálního pořádku (pozitivní či negativní)
\end{itemize}
\end{itemize}

\paragraph{Typy společností}
\begin{itemize}
\item \textbf{archaická} -- pravěk
\item \textbf{tradiční} -- sociální diferenciace, religionizace, agrární společnost
\item \textbf{moderní} -- industrializace, urbanizace, sekularizace
\item \textbf{postmoderní} -- služby, globalizace
\end{itemize}

\subsection{Sociální stratifikace}
\begin{itemize}
\item rozvrstvení
\item rozdělení podle nerovností (pohlaví, náboženství, ...)
\item určena a určuje majetek, vzdělání a moc
\item rozlišované vrstvy
	\begin{itemize}
	\item[a)] vyšší vrstva -- elita (bohatí lidé, aristokraté)
	\item[b)] vyšší střední vrstva -- odborníci
	\item[c)] nižší střední -- úředníci
	\item[d)] vyšší dělnická -- odborná manuální práce
	\item[e)] nižší dělnická -- neodborná manuální práce
	\item[f)] deklasovaní lidé
	\end{itemize}
\end{itemize}

\paragraph{Sociální mobilita}
\begin{itemize}
\item horizontální -- přesun v rámci jedné vrstvi
\item vertikální -- mezi vrstvami
\item intergenerační -- jedinec je jinde než jeho předci
\item intragenerační -- jedinec se přesouvá v rámci svého života
\end{itemize}

\paragraph{Typy stratifikačních systémů}
\begin{itemize}
\item otrokářský -- svobodní a nesvobodní
\item kastovní -- neprostupné (zakázané sňatky, ...)
\item stavovský -- stavy mají svá práva, řídí spol. život, prostupnější než kasty
\item třídní systém -- 	např. vrstvy zmíněné výše
\end{itemize}

\subsection{Sociální útvary}
\begin{itemize}
\item skupiny lidí, ve kterých lidé kolektivně organizují své činnosti
\item formální $\times$ neformální vztahy
	\begin{itemize}
	\item tykání $\times$ vykání
	\item \ldots
	\end{itemize}

\end{itemize}

\paragraph{Sociální skupiny}
\begin{itemize}
\item \textbf{sociální skupina} -- 3 a více lidí spojených vztahy
\item formální skupiny: škola, úřad, politika, práce
\item neformální skupiny: vztah, rodina, kamarádi
\item \textbf{referenční skupina} -- skupina do které se člověk chce zařadit	
\item \textbf{primární} skupiny -- neformální, citové, rodina
\item \textbf{sekundární} skupiny -- formální, pracovní, politické
\item \textbf{malé} skupiny
\item \textbf{velké} skupiny
\end{itemize}

\subsection{Subkultury}
\begin{itemize}
\item \textbf{geek} -- zapálený do informatiky, není módní
\item \textbf{skate} -- volné oblečení, drogy, pizza
\end{itemize}

\newpage
\section{Náboženství}
\begin{itemize}
\item označuje projevy lidského vztahu k vyšší nadlidské moci
\item provází civilizaci od jejího počátku
\item všem náboženstvím jsou společné používání symbolů a kolektivní rituály
\item \textbf{panteismus} -- bůh sloučený s přírodou
\item \textbf{polyteismus} $\times$ \textbf{monoteismus} (více, jeden bůh)
\item \textbf{deismus} -- Bůh stvořil svět a více se o něj nestaral
\item \textbf{atheismus} -- nevíra
\item \textbf{teologie} -- věda studující konkrétního  náboženství
\item \textbf{religionistika} -- věda studující a porovnávající různá náboženství
\end{itemize}

\paragraph{Západní náboženství}
\begin{itemize}
\item monoteistické	
\item judaismu, křesťanství, islám
\end{itemize}

\paragraph{Východní náboženství}
\begin{itemize}
\item polyteistická, víra v životní cyklus
\item budhismus, hinduismus, konfucianismus, \ldots
\end{itemize}

\subsection{Judaismus}
\begin{itemize}
\item \textbf{židovská diaspora} rozšíření židů okolo středozemního moře
\item \textbf{Tóra} -- svatá kniha
\item 2 důležití proroci, kdo dal židovskému lidu desatero? Mojžíš, Abrahám
\item \textbf{Izrael} -- z hebrejštiny "zápasí bůh"
\item \textbf{šabat} -- den odpočinku (od pátečního západu slunce do sobotního večera)
\item \textbf{pesach}  -- jarní svátek, oslava vysvobození židů z Egypta
\item \textbf{košér jídlo} -- přežvýkaví sudokopytníci, šupinaté ryby
\item \textbf{synagoga} -- židovský kostel
\item \textbf{šalom} -- mír, pokoj v duši
\item základní židovské symboly: hvězda, menora (sedmiramenný svícen)
\item kdy a kde vznikl židovský stát
\item židovské rituály: nejméně 3 motlitby, skupinové, obřízka, škola judaismu, 18 věk na svatbu, zpěv na pohřbu, bez rakve
\end{itemize}

\subsection{Islám}
\begin{itemize}
\item \textbf{islám} = oddanost
\item \textbf{burka} -- svrchní vrstva ženského oděvu
\item \textbf{mešita} -- muslimský kostel
\item \textbf{minaret} -- věž u mešity -- svolávání na modlitby
\item \textbf{muezín} -- svolává na motlitby
\item \textbf{Korán} -- posvátná kniha
\item zákaz pití vína, jezení vepřového masa a masa ze zdechlin
\item 622 -- Mohamed vyhnán do Mediny \ra počátek muslimského kalendáře
\item povinnosti: 1 za život podniknout pouť do Mekky, 5x denně modlení k Mekce, nepít alkohol
\end{itemize}

\subsection{Křesťanství}
\begin{itemize}
\item \textbf{Bible}
	\begin{itemize}
	\item pentateuch, evangelium
	\item starý zákon hebrejsky, nový zákon řecky
	\end{itemize}
\item\textbf{ založeno na judaismu}
\item Konstantin 313 -- edikt milánský -- povoleno křesťanství
\item spojeno s osobou \textbf{Ježíše Krista} -- spasitel
\item trojjedinost Boha -- otec, syn i duch svatý
\item ekumenický pohled na církev -- spojení všech větví
\end{itemize}

\subsection{Budhismus}
\begin{itemize}
\item jádro pudla
\item vznik z hinduismu, v 6. stol. př. n. l. v Indii
\item jóga, meditace, nirvana, reinkarnace, \ldots
\item rozdělení světu na čtyři říše
\item čtyři vznešené pravdy (o utrpení, vznikání utrpení, zaniknutí utrpení, stezce vedoucí k zaniknutí)
\item patero etických rozhodnutí -- nekrást, nezabíjet, necizoložit, nelhat, nepoužívat alkohol/drogy
\item \textbf{sangha} -- mniši, mnišky, \textbf{viháry} -- kláštery
\item \textbf{hinajána} -- konzervativní, jen mnichové dosáhnou nirvany
\item \textbf{mahajáva} -- liberální, kdokoliv
\end{itemize}

\subsection{Činská lidová náboženství}
\begin{itemize}
\item směs mahájánového budhismu, lidového taoismu a konfucianismu i některých menších místních kultů
\item \textbf{konfucianismus}
	\begin{itemize}
	\item náboženství = způsob vzdělání
	\item uctívání nebes, neexistuje struktura, důležité jsou morální hodnoty důraz na přesné vykonávání obřadů
	\item vnější struktura, praktická stránka, tradice, mravy, patriarchát, konzervativní
	\end{itemize}
\item \textbf{taoismus}
	\begin{itemize}
	\item splynutí člověka s přírodou pomocí meditací
	\item nezasahovat do přirozeného chodu věcí
	\item jin -- tma, měkkost, ženskost, jang -- světlo, tvrdost, mužskost
	\item vnitřní struktura, duševní stránka, volnost, klid, příroda, liberální
	\end{itemize}
\end{itemize}

\subsection{Řecká a římská mytologie}
\begin{itemize}
\item hodně bohů a incestu
\end{itemize}

\subsection{Hinduismus}
\begin{itemize}
\item Indie
\item základní jazyk -- \textbf{sanskrt}
\item každodenní život propojený s náboženstvím
\item bohové -- brahma, višna, šivu, kalí
\item \textbf{ganeša} -- bůh v podobě slona
\item \textbf{hanuman} -- bůh v podobě opice
\item \textbf{mahabháráta}, \textbf{rámájána} -- eposy
\item modlí se, aby měli dobrý příští život, nebo se osvobodili z koloběhu životů
\item neměnný kastovní systém
\end{itemize}


\end{document}