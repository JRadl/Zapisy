\title{Hudební výchova -- 4. pololetí}
\documentclass[10pt,a4paper]{article}
\usepackage[utf8]{inputenc}
\usepackage[czech]{babel}
\usepackage{amsmath}
\usepackage{amsfonts}
\usepackage{amssymb}
\usepackage{chemfig}
\usepackage{geometry}
\usepackage{wrapfig}
\usepackage{graphicx}
\usepackage{floatflt}
\usepackage{hyperref}
\usepackage{fancyhdr}
\usepackage{tabularx}
\usepackage{makecell}
\usepackage{csquotes}
\usepackage{footnote}
\usepackage{movie15}
\MakeOuterQuote{"}

\renewcommand{\labelitemii}{$\circ$}
\renewcommand{\labelitemiii}{--}
\newcommand{\ra}{$\rightarrow$ }
\newcommand{\x}{$\times$ }
\newcommand{\lp}[2]{#1 -- #2}
\newcommand{\timeline}{\input{timeline}}


\geometry{lmargin = 0.8in, rmargin = 0.8in, tmargin = 0.8in, bmargin = 0.8in}
\date{\today}
\author{Jakub Rádl}

\makeatletter
\let\thetitle\@title
\let\theauthor\@author
\makeatother

\hypersetup{
colorlinks=true,
linkcolor=black,
urlcolor=cyan,
}



\begin{document}
\maketitle
\tableofcontents
\begin{figure}[b]
Toto dílo \textit{\thetitle} podléhá licenci Creative Commons \href{https://creativecommons.org/licenses/by-nc/4.0/}{CC BY-NC 4.0}.\\ (creativecommons.org/licenses/by-nc/4.0/)
\end{figure}
\newpage

\section{Český Romantismus}
\subsection{Bedřich Smetana}
\paragraph{Poslech: symfonie}


\subsection{Antonín Dvořák}
\begin{itemize}
\item *1841 Nelahozeves
\item otec řezník \ra chtěl aby jeho syn převzal řeznictví
\item odešel studovat na konzervatoř do prahy -- klavír, varhany, viola
\item Johannes Brahms -- vydavatel
\item odjel do Ameriky
\item po návratu se stává ředitelem konzervatoře
\item zemřel 1904
\end{itemize}

\paragraph{Dílo:}
\begin{itemize}
\item opery: \textbf{Rusalka} (libreto Jaroslav Kvapil)
\item symfonie(9): \textbf{9. Novosvětská}, 
	\begin{itemize}
	\item píseň z indiánského pohřbu
	\end{itemize}
\item \textbf{Slovanské tance}
\item \textbf{Humoreska}
\item poslech -- Novosvětská, Slovanské tance (1), Humoreska
\end{itemize}

\subsection{Zdeněk Fibich}
\paragraph{Život}
\begin{itemize}
\item studoval Smetanův hudební ústav v Praze
\item konzervatoř v Lipsku
\item Paříž, Manheim
\item dirigent, dramaturg národního divadla
\end{itemize}

\paragraph{Tvorba}
\begin{itemize}
\item opery nemají dobrá libreta
\end{itemize}

\paragraph{Dílo}
\begin{itemize}
\item opery: \textbf{Eva}, \textbf{Šárka}, \textbf{Hedy}, \textbf{Pád Arkuna}
\item selanka: \textbf{V podvečer}
	\begin{itemize}
	\item známá část -- Poem
	\end{itemize}
\item melodram: \textbf{Štědrý den}, \textbf{Vodník}, Hippodamie
\end{itemize}

\subsection{Josef Suk}
\paragraph{Život}
\begin{itemize}
\item od mládí talentovaný
\item konzervatoř v Praze, oblíbený žák Dvořáka
\item vzal si za ženu Dvořákovu dceru Otylku
\item 40 let hrál druhé housle v České kvartetu (2 housle, viola a cello)
\end{itemize}

\paragraph{Tvorba}
\begin{itemize}
\item největší český lyrik
\item řazen mezi impresionisty
\end{itemize}

\paragraph{Dílo}
\begin{itemize}
\item symfonická hudba k Radůzovi a Mahuleně \textbf{pohádka}
\item klavírní skladby: \textbf{Píseň lásky}, \textbf{O matince} 
\item instrumentální skladby: \textbf{Pochod v nový život}, \textbf{Serenáda}
\end{itemize}

\subsection{Vítězslav Novák}
\paragraph{Život}
\begin{itemize}
\item gymnázium Kamenice nad Lipou
\item právnická fakulta v Praze
\item konzervatoř v Praze
\item učitel Antonín Dvořák
\item po Dvořákově smrti se stává ředitelem konzervatoře
\item učil mnoho skladatelů 	
\item vynikající horolezec, Tatry
\end{itemize}

\paragraph{Tvorba}
\begin{itemize}
\item opery nemají dobrá libreta \ra moc se nehrají
\item suita -- popisný hudební obrázek
\item největší český impresionista
\item "Život je neustálý boj, kde výhry a prohry se střídají, hlavní je fair play."
\end{itemize}

\paragraph{Dílo}
\begin{itemize}
\item symfonická básně \textbf{V Tatrách}, \textbf{Májová}
\item opery: \textbf{Zvíkovský rarášek}, \textbf{Dědův odkaz}
\item suity: \textbf{Slovácká suita} -- líčí neděli na Slovácku, \textbf{Jihočeská suita} -- ztvárňuje krajinu jižních Čech
\end{itemize}


\subsection{Leoš Janáček (1854--1928)} 
\paragraph{Život}
\begin{itemize}
\item narozen v Hukvaldech (dnes muzeum)
\item v 11 letech v cisterciánské škole na Mendelově náměstí
\item špatný žák, dobrý zpěvák, věnoval se hudbě
\item varhanická škola, vyhozen
\item učitel Křížkovský
\item učil na klavír Zdeničku, zamiloval se, vzali se (v jejích 16)
\item zakazoval jí styky s německou rodinou
\item obě děti zemřely
\item v 60 v Luhačovicích potkal Kamilu Stösslovou
\item vztah pouze platonický, v dopisech
\end{itemize}

\paragraph{Tvorba}
\begin{itemize}
\item předbíhá dobu, není romantik, spíše impresionista
\item 
\end{itemize}

\paragraph{Dílo:}
\begin{itemize}
\item 
\end{itemize}

\end{document}