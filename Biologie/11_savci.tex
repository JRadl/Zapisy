\title{Třída: Savci}
\documentclass[10pt,a4paper]{article}
\usepackage[utf8]{inputenc}
\usepackage[czech]{babel}
\usepackage{amsmath}
\usepackage{amsfonts}
\usepackage{amssymb}
\usepackage{chemfig}
\usepackage{geometry}
\usepackage{wrapfig}
\usepackage{graphicx}
\usepackage{floatflt}
\usepackage{hyperref}
\usepackage{fancyhdr}
\usepackage{tabularx}
\usepackage{makecell}
\usepackage{csquotes}
\usepackage{footnote}

\MakeOuterQuote{"}

\renewcommand{\labelitemii}{$\circ$}
\renewcommand{\labelitemiii}{--}
\newcommand{\ra}{$\rightarrow$ }
\newcommand{\x}{$\times$ }
\newcommand{\lp}[2]{#1 -- #2}
\newcommand{\timeline}{\input{timeline}}


\geometry{lmargin = 0.8in, rmargin = 0.8in, tmargin = 0.8in, bmargin = 0.8in}
\date{\today}
\author{Jakub Rádl}

\makeatletter
\let\thetitle\@title
\let\theauthor\@author
\makeatother

\hypersetup{
colorlinks=true,
linkcolor=black,
urlcolor=cyan,
}



\begin{document}
\maketitle
\tableofcontents
\begin{figure}[b]
Toto dílo \textit{\thetitle} podléhá licenci Creative Commons \href{https://creativecommons.org/licenses/by-nc/4.0/}{CC BY-NC 4.0}.\\ (creativecommons.org/licenses/by-nc/4.0/)
\end{figure}
\newpage

\section{Evoluce savců}
\begin{itemize}
\item druhohory -- savcotvární plazi
	\begin{itemize}
	\item na konci dopad meteoritu \ra \textbf{vymření dinosaurů} \ra možnost rozvoje savců
	\end{itemize}
\item třetihory -- zlatý věk savců
\end{itemize}

\paragraph{Charakteristika}
\begin{itemize}
\item vývojově nejrozmanitější skupina obratlovců
\item mláďata sají mateřské mléko z \textbf{mléčných žláz} matky
\item rodí \textbf{živá mláďata}, výživa zárodku zprostředkována \textbf{placentou}
\item malý počet mláďata, dlouhý a energeticky náročný odchov
	\begin{itemize}
	\item \textbf{k-stratég} -- málo potomků (člověk)
	\item \textbf{r-stratég} -- mnoho potomků (myš)
	\end{itemize}
\item stálá tělní teplota
\item povrch těla kryt srstí tvořenou chlupy, které mají odlišnou stavbu, než plazí šupiny, velké množství žláz (potní, mazové)
\item \textbf{sukcese} -- přizpůsobení se novému prostředí
\item \textbf{analogie} -- stejné uzpůsobení bez společného předka
\end{itemize}


\paragraph{Kostra}
\begin{itemize}
\item lebka připojena pomocí obratlů \textbf{atlas} a \textbf{axis}
\item páteř = osa, končetiny vybočují
\item heterodontní chrup -- řezáky, špičáky, třenové, stoličky \ra možné určit způsob stravování
\item modifikace končetin
\end{itemize}


\paragraph{Teplokrevnost}
\begin{itemize}
\item srst -- \textbf{pesík}(dlouhé chlupy, zbarvení), \textbf{podsada}(krátké chlupy, tepelná izolace)
\item rychlý metabolismus
\end{itemize}

\end{document}